\chapter{The CMS detector} % (fold)
\label{cha:the_cms_detector}
\begin{figure}[htbp]
  \centering
    \includegraphics[width=0.9\textwidth]{figures/CMS/cmsDetector.png}
  \caption{A perspective view of the C.M.S detector\cite{cms-sketchup}.}
  \label{fig:figures_CMS_cmsDetector}
\end{figure}

The \ac{cms} detector was designed and built to study proton-proton 
interactions at the \ac{lhc} with the aim of discovering the Higgs boson and 
searching for beyond the standard model physics signals. The detector is a 
traditional onion layer design with high precision tracking detectors nearest 
the interaction point and high energy resolution calorimetric detectors in the 
outer layers. Due to the predicted presence of missing energy \MET in new 
physics models, energy measurement over the full $\eta$ range is required. As 
alluded to in the name the possibilities of the new physics models containing 
muons and the ``golden'' Higgs decay channel 
\HepProcess{\PH\to\PZ\PZ\to\Pmu\Pmu\Pmu\Pmu}, and \PZprime models the detector 
was designed to accurately reconstruct muons with \PT of up to \unit{1}{\TeV}. 
Due to the prevalence of lepton signals and the ability to distinguish these 
leptons from the large amount of hadronic fragmentation caused by smashing two 
protons together at a centre of mass energy of \unit{14}{\TeV} the \ac{cms} 
Electromagnetic calorimeter was also designed to have precise energy 
measurement and fine grain spatial resolution, this design feature was motivated by the Higgs decay channel \HepProcess{\PH\to\Pphoton\Pphoton}.The 
other new feature of \ac{cms} is the use silicon detectors throughout for 
particle tracking these give precise track reconstruction abilities and the 
associated fine grain resolution in both position and momentum. All the 
sub-detectors save the muon system are contained with in the barrel of a 
\unit{4}{\tesla} superconducting solenoid. This magnet provides the particle 
track bending required for momentum and lepton charge measurement. In this 
section the key detector elements and their design parameters are discussed.

\section{The Silicon Tracker} % (fold)
\label{sec:the_silicon_tracker}
\begin{figure}[htbp]
  \centering
    \includegraphics[width=0.5\textwidth]{figures/CMS/tracker.png}
  \caption{The \ac{cms} tracking system, pixel detectors are situated at the 
  centre of the detector closest to the interaction point, surrounded by layers 
  of silicon strip detectors\cite{cms-sketchup}.}
  \label{fig:figures_CMS_tracker}
\end{figure}

The design goal for the \ac{cms} tracking system was to produce a system that 
can precisely and efficiently measure the trajectories of charged particles 
produced in \ac{lhc} collisions. The tracking system surrounds the collision 
point and has a length of \unit{5.8}{\meter} and a diameter of 
\unit{2.5}{\meter}. At the \ac{lhc}'s design instantaneous luminosity of 
\unit{10$^{34}$}{\lumiunits} an average of 1000 charged particles from more 
than 20 proton-proton interactions will be produced per \unit{25}{\nano\second} 
bunch crossing. This imposes the requirements of high granularity so that the 
individual particles and their trajectories can be distinguished and a fast 
response so that the hits can be assigned to the correct bunch crossing. The 
technical implications of these requirements imply both a large amount of 
on-detector electronics and their associated cooling equipment. This large 
amount of tracker material increases the multiple scattering, bremsstrahlung, 
photon conversion and nuclear interactions of the particles which are 
traversing the tracking system, this in turn limits the position and energy 
resolution of the calorimeter systems. A compromise between the material budget 
of the tracking system, the desired features and the requirement for the 
tracking system to have an expected life time of 10 years in a high radiation 
environment, this resulted in the construction of the tracking system using only solid state
silicon detector technology.

The \ac{cms} tracking system is formed of a pixel detector with three layers 
situated between \unit{4.4}{\cm} and \unit{10.2}{\cm} of the interaction point 
and a silicon strip tracker with 10 barrel layers extending outwards to a 
radius of \unit{1.1}{\meter} from the interaction point. The barrel layers of 
both the pixel and strip detectors are complimented by layers of either pixel 
or strip disks which extend the acceptance of the tracker to \mETA $<$ 2.5. The 
resulting structure is that of a detector with \unit{200}{\squaremetre} of 
active silicon which makes the \ac{cms} tracker the largest detector of its 
type ever constructed.

The individual pixels that comprise the pixel detector are 100 $\times$ 
150 $\micro\textrm{m}^{2}$ which corresponds to an occupancy of around 
10$^{-4}$ per bunch crossing. The detector cell size for the microstrip 
detectors at a radius between \unit{20-55}{\cm} is \unit{10}{\cm} $\times$ 
\unit{180}{\micro\meter} which leads to an average occupancy of 2-3$\%$. 
Further out at a radius greater than \unit{55}{\cm}  the minimum cell size is 
increased to \unit{25}{\cm} $\times$ \unit{180}{\micro\meter} with an occupancy 
of around 1$\%$. The tracker performance is extensively documented in 
\cite{cmstracking}.

% section the_silicon_tracker (end)

\section{The Electromagnetic Calorimeter (ECAL).} % (fold)
\label{sec:the_electromagnetic_calorimeter}
The \ac{cms} electromagnetic calorimeter\cite{ref:cmsecaltdr} is a hermetic 
homogeneous calorimeter constructed from lead tungstate (PbWO$_{4}$) crystals. 
The ECAL is split into two parts, a barrel covering $|\eta| < 1.479$ read out 
by avalanche photodiodes and the two end-caps covering $1.479 < |\eta| < 3.0$, 
read out by vacuum photo-triodes.

Lead tungstate crystals were chosen because of their short radiation length, 
fast scintillation and radiation hardness. During the research and development 
program it was shown that radiation damage does not affect the scintillation 
method or the uniformity of the emitted light yield along the crystal, it only 
effects the transparency through the creation of colour centres. This will be 
monitored throughout the lifetime of the ECAL via a light injection 
system\cite{Smith:2000p5245}.

The barrel crystals have a front face of \unit{22 $\times$ 22}{\mm\squared}, 
this corresponds to the Moliere radius of $\approx $ \unit{22}{\mm}; with a 
length of \unit{230}{\mm} giving a each crystal length of $25.8 X_{0}$ where 
$X_{0}$ is amount of material required for 68$\%$ of an electromagneticly 
interacting particle's energy to be radiated in the form of bremsstrahlung or 
pair production whilst traversing that material for lead tungstate this length 
is $X_{0}= \unit{0.89}{\cm}$. They are arranged in 36 super-modules, forming 
two half barrels. The crystal axes are skewed at \unit{3}{\degree} with respect 
to the vertex and each covers \unit{1}{\degree} in $\phi$ and $\eta$

The endcap crystals are arranged in two semi-circular `Dees', groups of 
$5\times 5$ crystals are canter-levered on a aluminium backing plate. The 
crystals are again skewed with regard to the interaction vertex, however they 
are arranged in an $x-y$ grid rather than a $\phi - \eta$ grid. The crystal 
dimensions are different to the barrel, in that the cross section is \unit{28.6 
$\times$ 28.6}{\mm\squared}, with a length of \unit{220}{\mm} corresponding to 
24.7 radiation lengths

The crystals are read out with avalanche photo-diodes in the barrel region 
which covers $\mETA < 1.479$ and with vacuum photo-triodes in the endcap 
region which covers $1.479 < \mETA < 3.0$.

In addition to the ECAL there is a pre-shower detector situated at $1.653 < 
\mETA < 2.6$, this provides identification of isolated elections against 
electrons produced in showers, each particle passes though the detector leaving 
a minimum ionising track, or hit, the number of these hits per area gives 
information on the isolation of the particle and improves the position 
measurement of electrons and photons in the ECAL endcaps.
\begin{figure}[ht]
  \centering
    \includegraphics[width=0.75\textwidth]{figures/CMS/laser_history_2011-12.png}
  \caption{Relative response to laser light (440 nm) measured by the ECAL laser 
monitoring system, averaged over all crystals in bins of pseudo-rapidity, for 
the 2011 and 2012 data taking periods 
The response change observed in the ECAL channels is of the order of a few 
percent in the barrel, while it reaches up to 25$\%$ in the most forward endcap 
regions used for electron and photon reconstruction. The response change is up 
to 60$\%$ in channels closest to the beam pipe. These measurements are used to 
correct the physics data. \\ The bottom plot shows the instantaneous \ac{lhc} 
luminosity delivered during this time period.\cite{CMS-DP-2012-015}}
  \label{fig:figures_CMS_laser_history_2011-12}
\end{figure}

Figure~\ref{fig:figures_CMS_laser_history_2011-12} shows the change in response 
of the ECAL during running, due to the formation of colour centres, this 
response is measured using laser light at 440 nm and is used to correct the 
energies recorded in data during each run to form a uniform response.
The ECAL performance in 2011 is extensively documented in \cite{CMS-DP-2012-002}.
% section the_electro_magnetic_calorimeter (end)

\section{The Hadronic Calorimeter (HCAL).} % (fold)
\label{sec:the_hadronic_calorimeter}
The design of the \ac{cms} HCAL is constrained by the size requirements of 
fitting the tracking system, ECAL and HCAL inside the solenoid magnet. The HCAL 
is situated between the ECAL which ends at a radius of \unit{1.77}{\meter} and 
the solenoid which starts at a radius of \unit{2.95}{\meter}. This constraint 
limits the amount of material which can be put in place to fully contain the 
hadronic showers. To overcome this an outer lay of instrumentation is placed 
outside of the magnet and cryogenic system to fully contain hadronic showers. 
The full containment of the hadronic objects is necessary for precise missing 
energy measurement, which is a key discriminatory feature for new physics 
models with undetectable final state particles.

The hadronic calorimeter is split in to three sub detectors, the hadronic 
barrel (HB) \mETA $< 1.3$ , hadronic endcaps (HE) $1.3 < \mETA < 3.0$ and a 
forward calorimeter (HF) $3.0 < \mETA < 5.5$.

The HB and HE are comprised of layered tiles of brass absorber plates 
interspaced with scintillator plates which are read out by wavelength shifting 
fibres. The total absorber thickness varies between 5.2 and 10.6 interaction 
lengths, the ECAL adds approximately one extra interaction length. The 
scintillators are segmented in to towers of area $\Delta\eta\times\Delta\phi = 
0.087\times0.087$ in the barrel and $\Delta\eta\times\Delta\phi = 
0.17\times0.17$ in the endcaps. The light produced and in the scintillators is 
merged in the wavelength shifting fibres and then read out using hybrid 
photo-diodes.The HF is constructed from radiation hard quartz fibres, this 
enables the detector to survive in the very forward regions of the detector 
where high levels of radiation are experienced.

The hadronic outer (HO) which is situated outside the solenoid used the 
solenoid coils as an extra absorber adding extra interaction lengths in the 
barrel region ensuring full containment of hadronic showers. The HO is 
constructed from layers of scintillator tiles, the light from which is then merged in the wavelength shifting fibres and read out using hybrid photo-diodes.
HCAL performance is documented in \cite{CMS-DP-2012-025}.
% section the_hadronic_calorimeter (end)


\section{The \ac{cms} Super-Conducting Solenoid} % (fold)
\label{sec:the_cms_super_conducting_solenoid}
The requirement for precise muon momentum measurements for muons with \PT $>$ 
\unit{1}{\TeV} infers the requirement of large bending power, this requirement 
forces the choice of a superconducting magnet. The \ac{cms} magnet is 
\unit{13}{\meter} long, has an inner diameter of \unit{6}{\meter} and provides 
a \unit{4}{\tesla} magnetic field, which gives a bending power of 
\unit{12}{\tesla\meter} before the muon bending angle is measured by the muon 
system. The bore of the solenoid contains the tracking and calorimeter systems.
The magnet is constructed from Niobium-Titanium superconductor embedded in an 
aluminium stabiliser is coiled in four layers resulting in \unit{220}{\ton} of 
cold mass. This is then cooled to around \unit{4}{\kelvin} using liquid helium 
and a current of $\approx$\unit{20}{\kilo\ampere} is applied to generate the 
magnetic field.
% section the_cms_super_conducting_solenoid (end)

\section{The \ac{cms} Muon system.} % (fold)
\label{sec:the_cms_muon_system}
The \ac{cms} muon system is designed to provide accurate muon \PT measurements 
($\sqrt{\PT}/\PT < 0.1$) for muons over a large \PT range. Due to the 
cylindrical nature of the solenoid, the muon system is also designed to be 
cylindrical. The muon system has three purposes the first is the identification 
of muons, secondly the momentum measurement of these muons and thirdly to 
provide information to the trigger system.
The muon system uses three types of gaseous detectors for particle tracking and 
identification, in the barrel region where the magnetic field is uniform, drift 
chambers (DT) are used, these cover the region $\mETA < 1.2$ are are 
interspaced between the layers of the magnetic flux return plates. In the end 
cap regions where the muon and background rates are higher and the magnetic 
field is non-uniform, cathode strip chambers (CSC) are are used due to their 
fast response, radiation hardness and fine grain segmentation, the CSCs cover 
$0.9 < \mETA < 2.4$. Due to the initial uncertainty on the background rates and 
the \unit{25}{\nano\second} bunch crossing intervals expected when the \ac{lhc} 
is running under design conditions a complementary dedicated muon triggering 
system consisting of \ac{rpc} was added in the range $\mETA < 1.6$, the 
\ac{rpc}s provide a fast, independent and fine grain system for 
which to trigger on muon objects in the harshest of running conditions.

Due to multiple scattering and the shear quantity of detector material before 
the first muon station momentum measurement using the muon system only is 
accurate to a level of $\approx10\%$ below \unit{200}{\GeV} and accurate to a 
level of $15-40\%$, \mETA dependant for \unit{1}{\TeV} muons, when including 
the tracker information in the muon momentum measurement the resolution is 
improved to $\approx1\%$ below \unit{200}{\GeV} and to about $5\%$ for 
\unit{1}{\TeV} muons. The performance of the \ac{cms} muon system is detailed 
in \cite{cmsMuPerf}.
% section the_cms_muon_system (end)

\section{The \Lone Trigger System.} % (fold)
\label{sec:the_level_one_trigger_system}
The \ac{cms} trigger system is designed in two levels, the first, the \Lone 
trigger is built using custom electronics and is designed to reduce the input 
rate of 40 million events per second to a manageable rate of 100 thousand 
events per second. Information only from the calorimeter and muon systems are 
considered as the time required to read out the tracking information is 
prohibitive at this level. Two separate trigger systems, one performing 
triggering on the calorimeter system, the other performing triggering on the 
muon systems are employed. For a detailed discussion of the calorimeter 
triggering algorithms and their performance see 
Chapter~\ref{cha:level_one_trigger}. The information from these two sub 
triggers is passed to the global trigger, where the decision to accept the 
event or not is made. 
The muon trigger considers information from each of the DT, CSC and \ac{rpc} 
muon systems. Tracks are created from the hits in each of the sub systems and 
fitted, the muon momentum is then calculated from the radius of curvature of 
these muons, the four highest \PT muon candidates are then passed to the global 
trigger.
Electron and photon candidates are created by the regional calorimeter trigger, 
the GCT creates jet candidates, energy and missing energy sums, the four 
highest \ET jet candidates of each type (central, tau and forward), the energy 
sums and the electron/photon candidates are then passed to the global trigger 
where the final trigger decisions are made. These decisions can require 
information from a single detector or can require cross objects such as the 
requirement of an energy sum value and a muon.
% section the_level_one_trigger_system (end)

\section{The High Level Trigger System.} % (fold)
\label{sec:the_high_level_trigger_system}
The high level trigger system is constructed from off the shelf components. 
The system is composed of two sets of machines, the first are the Event Builder 
(EB) units, these build raw data in to regional chunks from the data read out 
at the front end of the detector when a \Lone accept is received, this is then  
combined and transmitted to the Event Filter cluster. The EB units transfer the 
data to the (EF) units via a standard TCP/IP gigabit ethernet link.
The task of the HLT EF units is to run complex reconstruction algorithms using 
combined detector information to reduce the data rate accepted by the \Lone 
trigger to a manageable rate, the original design anticipated an output rate of 
100 events per second, however in 2012 running 1000 events per second were 
stored, half of this rate was assigned to the prompt reconstruction queue, 
the other half was stored for reconstruction during the long shut down of the 
\ac{lhc} in 2013-2015.

The EF farm is formed from standard rack-mounted PC units with a total of 
approximately 1000 computing cores operating at a clock frequency of around 
\unit{2}{\giga\hertz}. Upon receiving an event, each unit performs the \ac{cms} 
reconstruction, the same software framework is used as for offline analysis 
meaning that the objects used for trigger decisions are as close in performance 
to the offline objects as possible,  this increases the over all trigger 
efficiency, however the calibrations used at HLT level are not the final 
derived version. 
Trigger chains are designed so that full event reconstruction can be performed, 
however when constructing a trigger that requires full tracking reconstruction 
steps are made so that if a given requirement is failed at a lower stage of the 
reconstruction then time is not wasted performing the full reconstruction. The 
EF farm is connected to a large (several hundred terrabytes) storage area 
network, this acts as a temporary storage area for the events accepted by the 
HLT before the events are transferred to the tier-zero reconstruction farm 
which is located at the central CERN site.


% section the_high_level_trigger_system (end)

% chapter the_cms_detector (end)
