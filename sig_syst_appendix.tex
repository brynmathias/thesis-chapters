\newpage
\chapter{Signal efficiency\label{app:sig-eff}}

\section{CMSSM\label{app:sig-eff-cmssm}}

\begin{figure}[h!]
  \begin{center}
    \subfigure[\label{fig:sig-eff-cmssm-b0}$N_{\rm b-tag} = 0$.]{
      \includegraphics[width=0.45\textwidth]{figures/Analysis/eff_plots/cmssm-b0.pdf}
    }
    \subfigure[\label{fig:sig-eff-cmssm-b1}$N_{\rm b-tag} = 1$.]{
      \includegraphics[width=0.45\textwidth]{figures/Analysis/eff_plots/cmssm-b1.pdf}
    } \\
    \subfigure[\label{fig:sig-eff-cmssm-b2}$N_{\rm b-tag} = 2$.]{
      \includegraphics[width=0.45\textwidth]{figures/Analysis/eff_plots/cmssm-b2.pdf}
    }
    \subfigure[\label{fig:sig-eff-cmssm-b3}$N_{\rm b-tag} \ge 3$.]{
      \includegraphics[width=0.45\textwidth]{figures/Analysis/eff_plots/cmssm-b3.pdf}
    } \\
    \caption{\label{fig:sig-eff-cmssm}Signal efficiency in the
      ($m_{0},m_{1/2}$) plane of the CMSSM, of the full hadronic
      signal selection, integrating over all eight \HT bins and
      requiring (a) exactly zero, (b) exactly one, (c) exactly two,
      and (d) at least three b-tags per event.}
  \end{center}
\end{figure}

\newpage
\section{\texttt{T1}\label{app:sig-eff-t1}}

\begin{figure}[h!]
  \begin{center}
    \subfigure[\label{fig:sig-eff-t1-b0}$N_{\rm b-tag} = 0$.]{
      \includegraphics[width=0.45\textwidth]{figures/Analysis/eff_plots/sms-t1-b0.pdf}
    }
    \subfigure[\label{fig:sig-eff-t1-b1}$N_{\rm b-tag} = 1$.]{
      \includegraphics[width=0.45\textwidth]{figures/Analysis/eff_plots/sms-t1-b1.pdf}
    } \\
    \subfigure[\label{fig:sig-eff-t1-b2}$N_{\rm b-tag} = 2$.]{
      \includegraphics[width=0.45\textwidth]{figures/Analysis/eff_plots/sms-t1-b2.pdf}
    }
    \subfigure[\label{fig:sig-eff-t1-b3}$N_{\rm b-tag} \ge 3$.]{
      \includegraphics[width=0.45\textwidth]{figures/Analysis/eff_plots/sms-t1-b3.pdf}
    } \\
    \caption{\label{fig:sig-eff-t1}Signal efficiency in the ($m_{\rm
        gl},m_{\rm LSP}$) plane of the \texttt{T1} simplified model,
      of the full hadronic signal selection, integrating over all
      eight \HT bins and requiring (a) exactly zero, (b) exactly
      one, (c) exactly two, and (d) at least three b-tags per event.}
  \end{center}
\end{figure}

\newpage
\section{\texttt{T2}\label{app:sig-eff-t2}}

\begin{figure}[h!]
  \begin{center}
    \subfigure[\label{fig:sig-eff-t2-b0}$N_{\rm b-tag} = 0$.]{
      \includegraphics[width=0.45\textwidth]{figures/Analysis/eff_plots/sms-t2-b0.pdf}
    }
    \subfigure[\label{fig:sig-eff-t2-b1}$N_{\rm b-tag} = 1$.]{
      \includegraphics[width=0.45\textwidth]{figures/Analysis/eff_plots/sms-t2-b1.pdf}
    } \\
    \subfigure[\label{fig:sig-eff-t2-b2}$N_{\rm b-tag} = 2$.]{
      \includegraphics[width=0.45\textwidth]{figures/Analysis/eff_plots/sms-t2-b2.pdf}
    }
    \subfigure[\label{fig:sig-eff-t2-b3}$N_{\rm b-tag} \ge 3$.]{
      \includegraphics[width=0.45\textwidth]{figures/Analysis/eff_plots/sms-t2-b3.pdf}
    } \\
    \caption{\label{fig:sig-eff-t2}Signal efficiency in the ($m_{\rm
        sq},m_{\rm LSP}$) plane of the \texttt{T2} simplified model,
      of the full hadronic signal selection, integrating over all
      eight \HT bins and requiring (a) exactly zero, (b) exactly
      one, (c) exactly two, and (d) at least three b-tags per event.}
  \end{center}
\end{figure}

\newpage
\section{\texttt{T2tt}\label{app:sig-eff-t2tt}}

\begin{figure}[h!]
  \begin{center}
    \subfigure[\label{fig:sig-eff-t2tt-b0}$N_{\rm b-tag} = 0$.]{
      \includegraphics[width=0.45\textwidth]{figures/Analysis/eff_plots/sms-t2tt-b0.pdf}
    }
    \subfigure[\label{fig:sig-eff-t2tt-b1}$N_{\rm b-tag} = 1$.]{
      \includegraphics[width=0.45\textwidth]{figures/Analysis/eff_plots/sms-t2tt-b1.pdf}
    } \\
    \subfigure[\label{fig:sig-eff-t2tt-b2}$N_{\rm b-tag} = 2$.]{
      \includegraphics[width=0.45\textwidth]{figures/Analysis/eff_plots/sms-t2tt-b2.pdf}
    }
    \subfigure[\label{fig:sig-eff-t2tt-b3}$N_{\rm b-tag} \ge 3$.]{
      \includegraphics[width=0.45\textwidth]{figures/Analysis/eff_plots/sms-t2tt-b3.pdf}
    } \\
    \caption{\label{fig:sig-eff-t2tt}Signal efficiency in the ($m_{\rm
        sq},m_{\rm LSP}$) plane of the \texttt{T2tt} simplified model,
      of the full hadronic signal selection, integrating over all
      eight \HT bins and requiring (a) exactly zero, (b) exactly
      one, (c) exactly two, and (d) at least three b-tags per event.}
  \end{center}
\end{figure}

\newpage
\section{\texttt{T2bb}\label{app:sig-eff-t2bb}}

\begin{figure}[h!]
  \begin{center}
    \subfigure[\label{fig:sig-eff-t2bb-b0}$N_{\rm b-tag} = 0$.]{
      \includegraphics[width=0.45\textwidth]{figures/Analysis/eff_plots/sms-t2bb-b0.pdf}
    }
    \subfigure[\label{fig:sig-eff-t2bb-b1}$N_{\rm b-tag} = 1$.]{
      \includegraphics[width=0.45\textwidth]{figures/Analysis/eff_plots/sms-t2bb-b1.pdf}
    } \\
    \subfigure[\label{fig:sig-eff-t2bb-b2}$N_{\rm b-tag} = 2$.]{
      \includegraphics[width=0.45\textwidth]{figures/Analysis/eff_plots/sms-t2bb-b2.pdf}
    }
    \subfigure[\label{fig:sig-eff-t2bb-b3}$N_{\rm b-tag} \ge 3$.]{
      \includegraphics[width=0.45\textwidth]{figures/Analysis/eff_plots/sms-t2bb-b3.pdf}
    } \\
    \caption{\label{fig:sig-eff-t2bb}Signal efficiency in the ($m_{\rm
        sq},m_{\rm LSP}$) plane of the \texttt{T2bb} simplified model,
      of the full hadronic signal selection, integrating over all
      eight \HT bins and requiring (a) exactly zero, (b) exactly
      one, (c) exactly two, and (d) at least three b-tags per event.}
  \end{center}
\end{figure}

\newpage
\section{\texttt{T1tttt}\label{app:sig-eff-t1tttt}}

\begin{figure}[h!]
  \begin{center}
    \subfigure[\label{fig:sig-eff-t1tttt-b0}$N_{\rm b-tag} = 0$.]{
      \includegraphics[width=0.45\textwidth]{figures/Analysis/eff_plots/sms-t1tttt-b0.pdf}
    }
    \subfigure[\label{fig:sig-eff-t1tttt-b1}$N_{\rm b-tag} = 1$.]{
      \includegraphics[width=0.45\textwidth]{figures/Analysis/eff_plots/sms-t1tttt-b1.pdf}
    } \\
    \subfigure[\label{fig:sig-eff-t1tttt-b2}$N_{\rm b-tag} = 2$.]{
      \includegraphics[width=0.45\textwidth]{figures/Analysis/eff_plots/sms-t1tttt-b2.pdf}
    }
    \subfigure[\label{fig:sig-eff-t1tttt-b3}$N_{\rm b-tag} \ge 3$.]{
      \includegraphics[width=0.45\textwidth]{figures/Analysis/eff_plots/sms-t1tttt-b3.pdf}
    } \\
    \caption{\label{fig:sig-eff-t1tttt}Signal efficiency in the ($m_{\rm
        gl},m_{\rm LSP}$) plane of the \texttt{T1tttt} simplified model,
      of the full hadronic signal selection, integrating over all
      eight \HT bins and requiring (a) exactly zero, (b) exactly
      one, (c) exactly two, and (d) at least three b-tags per event.}
  \end{center}
\end{figure}

\newpage
\section{\texttt{T1bbbb}\label{app:sig-eff-t1bbbb}}

\begin{figure}[h!]
  \begin{center}
    \subfigure[\label{fig:sig-eff-t1bbbb-b0}$N_{\rm b-tag} = 0$.]{
      \includegraphics[width=0.45\textwidth]{figures/Analysis/eff_plots/sms-t1bbbb-b0.pdf}
    }
    \subfigure[\label{fig:sig-eff-t1bbbb-b1}$N_{\rm b-tag} = 1$.]{
      \includegraphics[width=0.45\textwidth]{figures/Analysis/eff_plots/sms-t1bbbb-b1.pdf}
    } \\
    \subfigure[\label{fig:sig-eff-t1bbbb-b2}$N_{\rm b-tag} = 2$.]{
      \includegraphics[width=0.45\textwidth]{figures/Analysis/eff_plots/sms-t1bbbb-b2.pdf}
    }
    \subfigure[\label{fig:sig-eff-t1bbbb-b3}$N_{\rm b-tag} \ge 3$.]{
      \includegraphics[width=0.45\textwidth]{figures/Analysis/eff_plots/sms-t1bbbb-b3.pdf}
    } \\
    \caption{\label{fig:sig-eff-t1bbbb}Signal efficiency in the ($m_{\rm
        gl},m_{\rm LSP}$) plane of the \texttt{T1bbbb} simplified model,
      of the full hadronic signal selection, integrating over all
      eight \HT bins and requiring (a) exactly zero, (b) exactly
      one, (c) exactly two, and (d) at least three b-tags per event.}
  \end{center}
\end{figure}

\newpage
\section{Signal contamination for \texttt{T1tttt}\label{app:sig-eff-contam}}

\begin{figure}[h!]
  \begin{center}
    \subfigure[\label{fig:sig-eff-sms-b1-t1tttt}Hadronic signal region, $N_{\rm b-tag} = 1$]{
      \includegraphics[width=0.45\textwidth]{figures/Analysis/eff_plots/sms-t1tttt-b1.pdf}
    } 
    \subfigure[\label{fig:sig-eff-sms-b1-t1tttt-mu}Muon control sample, $N_{\rm b-tag} = 1$]{
      \includegraphics[width=0.45\textwidth]{figures/Analysis/eff_plots/sms-t1tttt-mu-b1.pdf}
    } \\
    \subfigure[\label{fig:sig-eff-sms-b2-t1tttt}Hadronic signal region, $N_{\rm b-tag} = 2$]{
      \includegraphics[width=0.45\textwidth]{figures/Analysis/eff_plots/sms-t1tttt-b2.pdf}
    } 
    \subfigure[\label{fig:sig-eff-sms-b2-t1tttt-mu}Muon control sample, $N_{\rm b-tag} = 2$]{
      \includegraphics[width=0.45\textwidth]{figures/Analysis/eff_plots/sms-t1tttt-mu-b2.pdf}
    } \\
    \subfigure[\label{fig:sig-eff-sms-b3-t1tttt}Hadronic signal region, $N_{\rm b-tag} \ge 3$]{
      \includegraphics[width=0.45\textwidth]{figures/Analysis/eff_plots/sms-t1tttt-b3.pdf}
    } 
    \subfigure[\label{fig:sig-eff-sms-b3-t1tttt-mu}Muon control sample, $N_{\rm b-tag} \ge 3$]{
      \includegraphics[width=0.45\textwidth]{figures/Analysis/eff_plots/sms-t1tttt-mu-b3.pdf}
    } \\
    \caption{\label{fig:sig-eff-contam}Signal efficiency in the
      planes of simplified model \texttt{T1tttt}, of the (left)
      hadronic signal sample selection or (right) single muon control
      sample selection, integrating over all eight \HT bins and
      requiring (top) exactly one, (middle) exactly two, or (bottom)
      at least three b-tags per event.}
  \end{center}
\end{figure}


\newpage
\chapter{Experimental systematic uncertainties on signal efficiency
  times acceptance\label{app:sig-uncert}}

\section{Systematics due to jet energy scale uncertainties}

\begin{figure}[h!]
  \begin{center}
    \subfigure[\label{fig:cmssm-jes-up}Varying the event scale up.]{
      \includegraphics[width=0.45\textwidth]{figures/Analysis/syst_plots/JES_CMSSM_NEWSCAN-up.pdf}
    } 
    \subfigure[\label{fig:cmssm-jes-down}Varying the event scale down.]{
      \includegraphics[width=0.45\textwidth]{figures/Analysis/syst_plots/JES_CMSSM_NEWSCAN-down.pdf}
    } \\
    \subfigure[\label{fig:cmssm-jes-1d}Combined distribution of (a)
    and (b) above. Only points within a $\pm$\unit{60}{\GeV} band around
    observed limit are considered.]{
      \includegraphics[width=0.6\textwidth]{figures/Analysis/syst_plots/JES_CMSSM_NEWSCAN-1d.pdf}
    } \\
    \caption{\label{fig:cmssm-jes}The effect of jet energy scale
      variations on signal efficiency in the CMSSM plane. All plots
      show the relative change in efficiency. No requirement is made
      on the number of reconstructed b jets. The red shaded area is
      bounded by the 68$^{\textrm{th}}$ percentile.}
  \end{center}
\end{figure}

\newpage
\begin{figure}[h!]
  \begin{center}
    \subfigure[\label{fig:sms-jes-up-t1}\HepProcess{\PSgluino\PSgluino\to\Pquark\APquark\PSneutralino\Pquark\APquark\PSneutralino} (\texttt{T1})]{
      \includegraphics[width=0.45\textwidth]{figures/Analysis/syst_plots/JES_SMST1-up}
    }
    \subfigure[\label{fig:sms-jes-up-t2}\HepProcess{\Psquark\Psquark\to\Pquark\PSneutralino\APquark\PSneutralino} (\texttt{T2})]{ 
      \includegraphics[width=0.45\textwidth]{figures/Analysis/syst_plots/JES_SMST2-up}
    } \\
    \subfigure[\label{fig:sms-jes-up-t2tt}\HepProcess{\PStop\PStop\to\Ptop\PSneutralino\APtop\PSneutralino} (\texttt{T2tt})]{ 
      \includegraphics[width=0.45\textwidth]{figures/Analysis/syst_plots/JES_SMST2tt-up}
    } 
    \subfigure[\label{fig:sms-jes-up-t2bb}\HepProcess{\PSbottom\PSbottom\to\Pbottom\PSneutralino\APbottom\PSneutralino} (\texttt{T2bb})]{ 
      \includegraphics[width=0.45\textwidth]{figures/Analysis/syst_plots/JES_SMST2bb-up}
    } \\
    \subfigure[\label{fig:sms-jes-up-t1tttt}\HepProcess{\PSgluino\PSgluino\to\Pbottom\APbottom\PSneutralino\Pbottom\APbottom\PSneutralino} (\texttt{T1tttt})]{
      \includegraphics[width=0.45\textwidth]{figures/Analysis/syst_plots/JES_SMST1tttt-up}
    } 
    \subfigure[\label{fig:sms-jes-up-t1bbbb}\HepProcess{\PSgluino\PSgluino\to\Ptop\APtop\PSneutralino\Ptop\APtop\PSneutralino} (\texttt{T1bbbb})]{
      \includegraphics[width=0.45\textwidth]{figures/Analysis/syst_plots/JES_SMST1bbbb-up}
    } \\
    \caption{\label{fig:sms-jes-up}The fractional change in signal
      efficiency due to systematically increasing all jet energies,
      for various topologies. No requirement is made on the number of
      reconstructed b jets. }
  \end{center}
\end{figure}

\newpage
\begin{figure}[h!]
  \begin{center}
    \subfigure[\label{fig:sms-jes-down-t1}\HepProcess{\PSgluino\PSgluino\to\Pquark\APquark\PSneutralino\Pquark\APquark\PSneutralino} (\texttt{T1})]{
      \includegraphics[width=0.45\textwidth]{figures/Analysis/syst_plots/JES_SMST1-down}
    }
    \subfigure[\label{fig:sms-jes-down-t2}\HepProcess{\Psquark\Psquark\to\Pquark\PSneutralino\APquark\PSneutralino} (\texttt{T2})]{ 
      \includegraphics[width=0.45\textwidth]{figures/Analysis/syst_plots/JES_SMST2-down}
    } \\
    \subfigure[\label{fig:sms-jes-down-t2tt}\HepProcess{\PStop\PStop\to\Ptop\PSneutralino\APtop\PSneutralino} (\texttt{T2tt})]{ 
      \includegraphics[width=0.45\textwidth]{figures/Analysis/syst_plots/JES_SMST2tt-down}
    } 
    \subfigure[\label{fig:sms-jes-down-t2bb}\HepProcess{\PSbottom\PSbottom\to\Pbottom\PSneutralino\APbottom\PSneutralino} (\texttt{T2bb})]{ 
      \includegraphics[width=0.45\textwidth]{figures/Analysis/syst_plots/JES_SMST2bb-down}
    } \\
    \subfigure[\label{fig:sms-jes-down-t1tttt}\HepProcess{\PSgluino\PSgluino\to\Pbottom\APbottom\PSneutralino\Pbottom\APbottom\PSneutralino} (\texttt{T1tttt})]{
      \includegraphics[width=0.45\textwidth]{figures/Analysis/syst_plots/JES_SMST1tttt-down}
    } 
    \subfigure[\label{fig:sms-jes-down-t1bbbb}\HepProcess{\PSgluino\PSgluino\to\Ptop\APtop\PSneutralino\Ptop\APtop\PSneutralino} (\texttt{T1bbbb})]{
      \includegraphics[width=0.45\textwidth]{figures/Analysis/syst_plots/JES_SMST1bbbb-down}
    } \\
    \caption{\label{fig:sms-jes-down}The fractional change in signal
      efficiency due to systematically decreasing all jet energies,
      for various topologies. No requirement is made on the number of
      reconstructed b jets. }
  \end{center}
\end{figure}

\newpage
\begin{figure}[h!]
  \begin{center}
    \subfigure[\label{fig:sms-jes-near-t1}\HepProcess{\PSgluino\PSgluino\to\Pquark\APquark\PSneutralino\Pquark\APquark\PSneutralino} (\texttt{T1})]{
      \includegraphics[width=0.45\textwidth]{figures/Analysis/syst_plots/JES_SMST1-near}
    }
    \subfigure[\label{fig:sms-jes-near-t2}\HepProcess{\Psquark\Psquark\to\Pquark\PSneutralino\APquark\PSneutralino} (\texttt{T2})]{ 
      \includegraphics[width=0.45\textwidth]{figures/Analysis/syst_plots/JES_SMST2-near}
    } \\
    \subfigure[\label{fig:sms-jes-near-t2tt}\HepProcess{\PStop\PStop\to\Ptop\PSneutralino\APtop\PSneutralino} (\texttt{T2tt})]{ 
      \includegraphics[width=0.45\textwidth]{figures/Analysis/syst_plots/JES_SMST2tt-near}
    } 
    \subfigure[\label{fig:sms-jes-near-t2bb}\HepProcess{\PSbottom\PSbottom\to\Pbottom\PSneutralino\APbottom\PSneutralino} (\texttt{T2bb})]{ 
      \includegraphics[width=0.45\textwidth]{figures/Analysis/syst_plots/JES_SMST2bb-near}
    } \\
%     \subfigure[\label{fig:sms-jes-near-t1tttt}\HepProcess{\PSgluino\PSgluino\to\Pbottom\APbottom\PSneutralino\Pbottom\APbottom\PSneutralino} (\texttt{T1tttt})]{
%       \includegraphics[width=0.45\textwidth]{figures/Analysis/syst_plots/JES_SMST1tttt-near}
%     } 
    \subfigure[\label{fig:sms-jes-near-t1bbbb}\HepProcess{\PSgluino\PSgluino\to\Ptop\APtop\PSneutralino\Ptop\APtop\PSneutralino} (\texttt{T1bbbb})]{
      \includegraphics[width=0.45\textwidth]{figures/Analysis/syst_plots/JES_SMST1bbbb-near}
    } \\
    \caption{\label{fig:sms-jes-near}The fractional change in signal
      efficiency near to the diagonal due to systematically increasing
      or decreasing all jet energies, for various topologies. No
      requirement is made on the number of reconstructed b jets. }
  \end{center}
\end{figure}

\newpage
\begin{figure}[h!]
  \begin{center}
    \subfigure[\label{fig:sms-jes-far-t1}\HepProcess{\PSgluino\PSgluino\to\Pquark\APquark\PSneutralino\Pquark\APquark\PSneutralino} (\texttt{T1})]{
      \includegraphics[width=0.45\textwidth]{figures/Analysis/syst_plots/JES_SMST1-far}
    }
    \subfigure[\label{fig:sms-jes-far-t2}\HepProcess{\Psquark\Psquark\to\Pquark\PSneutralino\APquark\PSneutralino} (\texttt{T2})]{ 
      \includegraphics[width=0.45\textwidth]{figures/Analysis/syst_plots/JES_SMST2-far}
    } \\
    \subfigure[\label{fig:sms-jes-far-t2tt}\HepProcess{\PStop\PStop\to\Ptop\PSneutralino\APtop\PSneutralino} (\texttt{T2tt})]{ 
      \includegraphics[width=0.45\textwidth]{figures/Analysis/syst_plots/JES_SMST2tt-far}
    } 
    \subfigure[\label{fig:sms-jes-far-t2bb}\HepProcess{\PSbottom\PSbottom\to\Pbottom\PSneutralino\APbottom\PSneutralino} (\texttt{T2bb})]{ 
      \includegraphics[width=0.45\textwidth]{figures/Analysis/syst_plots/JES_SMST2bb-far}
    } \\
    \subfigure[\label{fig:sms-jes-far-t1tttt}\HepProcess{\PSgluino\PSgluino\to\Pbottom\APbottom\PSneutralino\Pbottom\APbottom\PSneutralino} (\texttt{T1tttt})]{
      \includegraphics[width=0.45\textwidth]{figures/Analysis/syst_plots/JES_SMST1tttt-far}
    } 
    \subfigure[\label{fig:sms-jes-far-t1bbbb}\HepProcess{\PSgluino\PSgluino\to\Ptop\APtop\PSneutralino\Ptop\APtop\PSneutralino} (\texttt{T1bbbb})]{
      \includegraphics[width=0.45\textwidth]{figures/Analysis/syst_plots/JES_SMST1bbbb-far}
    } \\
    \caption{\label{fig:sms-jes-far}The fractional change in signal
      efficiency far from the diagonal due to systematically increasing
      or decreasing all jet energies, for various topologies. No
      requirement is made on the number of reconstructed b jets. }
  \end{center}
\end{figure}

\newpage
\section{Systematics due to the MHT/MET cut}

\begin{figure}[h!]
  \begin{center}
    \subfigure[\label{fig:mht-met-eff}]{
      \includegraphics[width=0.45\textwidth]{figures/Analysis/syst_plots/MHTovMET-data-mc-eff}
    } 
    \subfigure[\label{fig:mht-met-ratio}]{
      \includegraphics[width=0.45\textwidth]{figures/Analysis/syst_plots/MHTovMET-data-mc-ratio}
    } \\
    \caption{\label{fig:modelling}(a) Comparison of the MHT/MET
      distributions from data and MC, and (b) the ratio of the
      distributions, data/MC, as a function of the MHT/MET cut value.}
  \end{center}
\end{figure}

\begin{figure}[h!]
  \begin{center}
    \subfigure[\label{fig:cmssm-mht-met-ineff}Fraction of expected signal yield
    rejected.]{
      \includegraphics[width=0.45\textwidth]{figures/Analysis/syst_plots/MHTovMET_tanB10-ineff.pdf}
    } 
    \subfigure[\label{fig:cmssm-mht-met-1d}Fraction of expected signal yield
    rejected in a $\pm$\unit{60}{\GeV} band around the observed limit.]{
      \includegraphics[width=0.45\textwidth]{figures/Analysis/syst_plots/MHTovMET_tanB10-1d.pdf}
    } \\
    \caption{\label{fig:cmssm-mht-met}The fraction of expected signal
      yield that is rejected by the MHT/MET cut in the CMSSM plane. No
      requirement is made on the number of reconstructed b jets. }
  \end{center}
\end{figure}

\begin{table}[h!]
  \caption{Conservative estimates of inefficiency (\%) for
    the MHT/MET cut when considering model points in the
    region near to the diagonal (\ie small mass splitting and
    compressed spectra) for various simplified models.} 
  \label{tab:sms-mht-met-ineff}
  \centering
  \footnotesize
  \begin{tabular}{ lcccccc }
    \hline
         & T1   & T2  & T2tt & T2bb & T1tttt & T1bbbb \\ [0.5ex]
    \hline  
    Near & 10.9 & 3.5 & 20.4 & 3.9  & -      & 10.8   \\
    Far  & 3.2  & 0.9 & 3.1  & 1.2  & 32.4   & 3.7    \\
    \hline
  \end{tabular}
\end{table}

\newpage
\begin{figure}[h!]
  \begin{center}
    \subfigure[\label{fig:sms-mht-met-ineff-t1}\HepProcess{\PSgluino\PSgluino\to\Pquark\APquark\PSneutralino\Pquark\APquark\PSneutralino} (\texttt{T1})]{
      \includegraphics[width=0.45\textwidth]{figures/Analysis/syst_plots/MHTovMET_SMST1-ineff}
    }
    \subfigure[\label{fig:sms-mht-met-ineff-t2}\HepProcess{\Psquark\Psquark\to\Pquark\PSneutralino\APquark\PSneutralino} (\texttt{T2})]{ 
      \includegraphics[width=0.45\textwidth]{figures/Analysis/syst_plots/MHTovMET_SMST2-ineff}
    } \\
    \subfigure[\label{fig:sms-mht-met-ineff-t2tt}\HepProcess{\PStop\PStop\to\Ptop\PSneutralino\APtop\PSneutralino} (\texttt{T2tt})]{ 
      \includegraphics[width=0.45\textwidth]{figures/Analysis/syst_plots/MHTovMET_SMST2tt-ineff}
    } 
    \subfigure[\label{fig:sms-mht-met-ineff-t2bb}\HepProcess{\PSbottom\PSbottom\to\Pbottom\PSneutralino\APbottom\PSneutralino} (\texttt{T2bb})]{ 
      \includegraphics[width=0.45\textwidth]{figures/Analysis/syst_plots/MHTovMET_SMST2bb-ineff}
    } \\
    \subfigure[\label{fig:sms-mht-met-ineff-t1tttt}\HepProcess{\PSgluino\PSgluino\to\Pbottom\APbottom\PSneutralino\Pbottom\APbottom\PSneutralino} (\texttt{T1tttt})]{
      \includegraphics[width=0.45\textwidth]{figures/Analysis/syst_plots/MHTovMET_SMST1tttt-ineff}
    } 
    \subfigure[\label{fig:sms-mht-met-ineff-t1bbbb}\HepProcess{\PSgluino\PSgluino\to\Ptop\APtop\PSneutralino\Ptop\APtop\PSneutralino} (\texttt{T1bbbb})]{
      \includegraphics[width=0.45\textwidth]{figures/Analysis/syst_plots/MHTovMET_SMST1bbbb-ineff}
    } \\
    \caption{\label{fig:sms-mht-met-ineff}The fraction of expected signal
      yield that is rejected by the MHT/MET cleaning cut, for various
      topologies. No requirement is made on the number of
      reconstructed b jets. }
  \end{center}
\end{figure}

\newpage
\begin{figure}[h!]
  \begin{center}
    \subfigure[\label{fig:sms-mht-met-near-t1}\HepProcess{\PSgluino\PSgluino\to\Pquark\APquark\PSneutralino\Pquark\APquark\PSneutralino} (\texttt{T1})]{
      \includegraphics[width=0.45\textwidth]{figures/Analysis/syst_plots/MHTovMET_SMST1-near}
    }
    \subfigure[\label{fig:sms-mht-met-near-t2}\HepProcess{\Psquark\Psquark\to\Pquark\PSneutralino\APquark\PSneutralino} (\texttt{T2})]{ 
      \includegraphics[width=0.45\textwidth]{figures/Analysis/syst_plots/MHTovMET_SMST2-near}
    } \\
    \subfigure[\label{fig:sms-mht-met-near-t2tt}\HepProcess{\PStop\PStop\to\Ptop\PSneutralino\APtop\PSneutralino} (\texttt{T2tt})]{ 
      \includegraphics[width=0.45\textwidth]{figures/Analysis/syst_plots/MHTovMET_SMST2tt-near}
    } 
    \subfigure[\label{fig:sms-mht-met-near-t2bb}\HepProcess{\PSbottom\PSbottom\to\Pbottom\PSneutralino\APbottom\PSneutralino} (\texttt{T2bb})]{ 
      \includegraphics[width=0.45\textwidth]{figures/Analysis/syst_plots/MHTovMET_SMST2bb-near}
    } \\
%     \subfigure[\label{fig:sms-mht-met-near-t1tttt}\HepProcess{\PSgluino\PSgluino\to\Pbottom\APbottom\PSneutralino\Pbottom\APbottom\PSneutralino} (\texttt{T1tttt})]{
%       \includegraphics[width=0.45\textwidth]{figures/Analysis/syst_plots/MHTovMET_SMST1tttt-near}
%     } 
    \subfigure[\label{fig:sms-mht-met-near-t1bbbb}\HepProcess{\PSgluino\PSgluino\to\Ptop\APtop\PSneutralino\Ptop\APtop\PSneutralino} (\texttt{T1bbbb})]{
      \includegraphics[width=0.45\textwidth]{figures/Analysis/syst_plots/MHTovMET_SMST1bbbb-near}
    } \\
    \caption{\label{fig:sms-mht-met-near}The fraction of expected
      signal yield that is rejected by the MHT/MET cleaning cut, near
      to the diagonal, for various topologies. No requirement is made
      on the number of reconstructed b jets. }
  \end{center}
\end{figure}

\newpage
\begin{figure}[h!]
  \begin{center}
    \subfigure[\label{fig:sms-mht-met-far-t1}\HepProcess{\PSgluino\PSgluino\to\Pquark\APquark\PSneutralino\Pquark\APquark\PSneutralino} (\texttt{T1})]{
      \includegraphics[width=0.45\textwidth]{figures/Analysis/syst_plots/MHTovMET_SMST1-far}
    }
    \subfigure[\label{fig:sms-mht-met-far-t2}\HepProcess{\Psquark\Psquark\to\Pquark\PSneutralino\APquark\PSneutralino} (\texttt{T2})]{ 
      \includegraphics[width=0.45\textwidth]{figures/Analysis/syst_plots/MHTovMET_SMST2-far}
    } \\
    \subfigure[\label{fig:sms-mht-met-far-t2tt}\HepProcess{\PStop\PStop\to\Ptop\PSneutralino\APtop\PSneutralino} (\texttt{T2tt})]{ 
      \includegraphics[width=0.45\textwidth]{figures/Analysis/syst_plots/MHTovMET_SMST2tt-far}
    } 
    \subfigure[\label{fig:sms-mht-met-far-t2bb}\HepProcess{\PSbottom\PSbottom\to\Pbottom\PSneutralino\APbottom\PSneutralino} (\texttt{T2bb})]{ 
      \includegraphics[width=0.45\textwidth]{figures/Analysis/syst_plots/MHTovMET_SMST2bb-far}
    } \\
    \subfigure[\label{fig:sms-mht-met-far-t1tttt}\HepProcess{\PSgluino\PSgluino\to\Pbottom\APbottom\PSneutralino\Pbottom\APbottom\PSneutralino} (\texttt{T1tttt})]{
      \includegraphics[width=0.45\textwidth]{figures/Analysis/syst_plots/MHTovMET_SMST1tttt-far}
    } 
    \subfigure[\label{fig:sms-mht-met-far-t1bbbb}\HepProcess{\PSgluino\PSgluino\to\Ptop\APtop\PSneutralino\Ptop\APtop\PSneutralino} (\texttt{T1bbbb})]{
      \includegraphics[width=0.45\textwidth]{figures/Analysis/syst_plots/MHTovMET_SMST1bbbb-far}
    } \\
    \caption{\label{fig:sms-mht-met-far}The fraction of expected
      signal yield that is rejected by the MHT/MET cleaning cut, far
      from the diagonal, for various topologies. No requirement is made
      on the number of reconstructed b jets. }
  \end{center}
\end{figure}

\newpage
\section{Systematics due to the dead ECAL cut}

\begin{figure}[h!]
  \begin{center}
    \subfigure[\label{fig:cmssm-dead-ecal-ineff}Fraction of expected signal yield
    rejected.]{
      \includegraphics[width=0.6\textwidth]{figures/Analysis/syst_plots/DECAL_tanB10-ineff.pdf}
    } \\
    \subfigure[\label{fig:cmssm-dead-ecal-1d}Fraction of expected signal yield
    rejected in a $\pm$\unit{60}{\GeV} band around the observed limit.]{
      \includegraphics[width=0.6\textwidth]{figures/Analysis/syst_plots/DECAL_tanB10-1d.pdf}
    } \\
    \caption{\label{fig:cmssm-dead-ecal}The fraction of expected signal
      yield that is rejected by the dead ECAL cut in the CMSSM
      plane. No requirement is made on the number of reconstructed
      b jets. }
  \end{center}
\end{figure}

\begin{table}[h!]
  \caption{Conservative estimates of inefficiency (\%) for
    the dead ECAL cut when considering model points in the
    region near to the diagonal (\ie small mass splitting and
    compressed spectra) for various simplified models.} 
  \label{tab:sms-dead-ecal-ineff}
  \centering
  \footnotesize
  \begin{tabular}{ lcccccc }
    \hline
         & T1   & T2  & T2tt & T2bb & T1tttt & T1bbbb \\ [0.5ex]
    \hline  
    Near & 13.3 & 6.4 & 23.7 & 7.5  & -      & 16.4   \\
    Far  & 13.6 & 5.8 & 9.4  & 6.2  & 27.6   & 13.9   \\
    \hline
  \end{tabular}
\end{table}

\newpage
\begin{figure}[h!]
  \begin{center}
    \subfigure[\label{fig:sms-dead-ecal-ineff-t1}\HepProcess{\PSgluino\PSgluino\to\Pquark\APquark\PSneutralino\Pquark\APquark\PSneutralino} (\texttt{T1})]{
      \includegraphics[width=0.45\textwidth]{figures/Analysis/syst_plots/DECAL_SMST1-ineff}
    }
    \subfigure[\label{fig:sms-dead-ecal-ineff-t2}\HepProcess{\Psquark\Psquark\to\Pquark\PSneutralino\APquark\PSneutralino} (\texttt{T2})]{ 
      \includegraphics[width=0.45\textwidth]{figures/Analysis/syst_plots/DECAL_SMST2-ineff}
    } \\
    \subfigure[\label{fig:sms-dead-ecal-ineff-t2tt}\HepProcess{\PStop\PStop\to\Ptop\PSneutralino\APtop\PSneutralino} (\texttt{T2tt})]{ 
      \includegraphics[width=0.45\textwidth]{figures/Analysis/syst_plots/DECAL_SMST2tt-ineff}
    } 
    \subfigure[\label{fig:sms-dead-ecal-ineff-t2bb}\HepProcess{\PSbottom\PSbottom\to\Pbottom\PSneutralino\APbottom\PSneutralino} (\texttt{T2bb})]{ 
      \includegraphics[width=0.45\textwidth]{figures/Analysis/syst_plots/DECAL_SMST2bb-ineff}
    } \\
    \subfigure[\label{fig:sms-dead-ecal-ineff-t1tttt}\HepProcess{\PSgluino\PSgluino\to\Pbottom\APbottom\PSneutralino\Pbottom\APbottom\PSneutralino} (\texttt{T1tttt})]{
      \includegraphics[width=0.45\textwidth]{figures/Analysis/syst_plots/DECAL_SMST1tttt-ineff}
    } 
    \subfigure[\label{fig:sms-dead-ecal-ineff-t1bbbb}\HepProcess{\PSgluino\PSgluino\to\Ptop\APtop\PSneutralino\Ptop\APtop\PSneutralino} (\texttt{T1bbbb})]{
      \includegraphics[width=0.45\textwidth]{figures/Analysis/syst_plots/DECAL_SMST1bbbb-ineff}
    } \\
    \caption{\label{fig:sms-dead-ecal-ineff}The fraction of expected signal
      yield that is rejected by the dead ECAL cleaning cut, for various
      topologies. No requirement is made on the number of
      reconstructed b jets. }
  \end{center}
\end{figure}

\newpage
\begin{figure}[h!]
  \begin{center}
    \subfigure[\label{fig:sms-dead-ecal-near-t1}\HepProcess{\PSgluino\PSgluino\to\Pquark\APquark\PSneutralino\Pquark\APquark\PSneutralino} (\texttt{T1})]{
      \includegraphics[width=0.45\textwidth]{figures/Analysis/syst_plots/DECAL_SMST1-near}
    }
    \subfigure[\label{fig:sms-dead-ecal-near-t2}\HepProcess{\Psquark\Psquark\to\Pquark\PSneutralino\APquark\PSneutralino} (\texttt{T2})]{ 
      \includegraphics[width=0.45\textwidth]{figures/Analysis/syst_plots/DECAL_SMST2-near}
    } \\
    \subfigure[\label{fig:sms-dead-ecal-near-t2tt}\HepProcess{\PStop\PStop\to\Ptop\PSneutralino\APtop\PSneutralino} (\texttt{T2tt})]{ 
      \includegraphics[width=0.45\textwidth]{figures/Analysis/syst_plots/DECAL_SMST2tt-near}
    } 
    \subfigure[\label{fig:sms-dead-ecal-near-t2bb}\HepProcess{\PSbottom\PSbottom\to\Pbottom\PSneutralino\APbottom\PSneutralino} (\texttt{T2bb})]{ 
      \includegraphics[width=0.45\textwidth]{figures/Analysis/syst_plots/DECAL_SMST2bb-near}
    } \\
%     \subfigure[\label{fig:sms-dead-ecal-near-t1tttt}\HepProcess{\PSgluino\PSgluino\to\Pbottom\APbottom\PSneutralino\Pbottom\APbottom\PSneutralino} (\texttt{T1tttt})]{
%       \includegraphics[width=0.45\textwidth]{figures/Analysis/syst_plots/DECAL_SMST1tttt-near}
%     } 
    \subfigure[\label{fig:sms-dead-ecal-near-t1bbbb}\HepProcess{\PSgluino\PSgluino\to\Ptop\APtop\PSneutralino\Ptop\APtop\PSneutralino} (\texttt{T1bbbb})]{
      \includegraphics[width=0.45\textwidth]{figures/Analysis/syst_plots/DECAL_SMST1bbbb-near}
    } \\
    \caption{\label{fig:sms-dead-ecal-near}The fraction of expected
      signal yield that is rejected by the dead ECAL cleaning cut,
      near to the diagonal, for various topologies. No requirement is
      made on the number of reconstructed b jets. }
  \end{center}
\end{figure}

\newpage
\begin{figure}[h!]
  \begin{center}
    \subfigure[\label{fig:sms-dead-ecal-far-t1}\HepProcess{\PSgluino\PSgluino\to\Pquark\APquark\PSneutralino\Pquark\APquark\PSneutralino} (\texttt{T1})]{
      \includegraphics[width=0.45\textwidth]{figures/Analysis/syst_plots/DECAL_SMST1-far}
    }
    \subfigure[\label{fig:sms-dead-ecal-far-t2}\HepProcess{\Psquark\Psquark\to\Pquark\PSneutralino\APquark\PSneutralino} (\texttt{T2})]{ 
      \includegraphics[width=0.45\textwidth]{figures/Analysis/syst_plots/DECAL_SMST2-far}
    } \\
    \subfigure[\label{fig:sms-dead-ecal-far-t2tt}\HepProcess{\PStop\PStop\to\Ptop\PSneutralino\APtop\PSneutralino} (\texttt{T2tt})]{ 
      \includegraphics[width=0.45\textwidth]{figures/Analysis/syst_plots/DECAL_SMST2tt-far}
    } 
    \subfigure[\label{fig:sms-dead-ecal-far-t2bb}\HepProcess{\PSbottom\PSbottom\to\Pbottom\PSneutralino\APbottom\PSneutralino} (\texttt{T2bb})]{ 
      \includegraphics[width=0.45\textwidth]{figures/Analysis/syst_plots/DECAL_SMST2bb-far}
    } \\
    \subfigure[\label{fig:sms-dead-ecal-far-t1tttt}\HepProcess{\PSgluino\PSgluino\to\Pbottom\APbottom\PSneutralino\Pbottom\APbottom\PSneutralino} (\texttt{T1tttt})]{
      \includegraphics[width=0.45\textwidth]{figures/Analysis/syst_plots/DECAL_SMST1tttt-far}
    } 
    \subfigure[\label{fig:sms-dead-ecal-far-t1bbbb}\HepProcess{\PSgluino\PSgluino\to\Ptop\APtop\PSneutralino\Ptop\APtop\PSneutralino} (\texttt{T1bbbb})]{
      \includegraphics[width=0.45\textwidth]{figures/Analysis/syst_plots/DECAL_SMST1bbbb-far}
    } \\
    \caption{\label{fig:sms-dead-ecal-far}The fraction of expected
      signal yield that is rejected by the dead ECAL cleaning cut, far
      from the diagonal, for various topologies. No requirement is
      made on the number of reconstructed b jets. }
  \end{center}
\end{figure}


\newpage
\section{Systematics due to the lepton and photon vetoes}

\begin{figure}[h!]
  \begin{center}
    \subfigure[\label{fig:cmssm-lepton-veto-ineff}Fraction of expected signal yield
    rejected.]{
      \includegraphics[width=0.6\textwidth]{figures/Analysis/syst_plots/LEPVETO_CMSSM-ineff.pdf}
    } \\
    \subfigure[\label{fig:cmssm-lepton-veto-1d}Fraction of expected signal yield
    rejected in a $\pm$\unit{60}{\GeV} band around the observed limit.]{
      \includegraphics[width=0.6\textwidth]{figures/Analysis/syst_plots/LEPVETO_CMSSM-1d.pdf}
    } \\
    \caption{\label{fig:cmssm-lepton-veto}The fraction of expected signal
      yield that is rejected by the dead ECAL cut in the CMSSM
      plane. No requirement is made on the number of reconstructed
      b jets. }
  \end{center}
\end{figure}

\newpage
\begin{figure}[h!]
  \begin{center}
    \subfigure[\label{fig:sms-lepton-veto-ineff-t2tt}\HepProcess{\PStop\PStop\to\Ptop\PSneutralino\APtop\PSneutralino} (\texttt{T2tt})]{ 
      \includegraphics[width=0.45\textwidth]{figures/Analysis/syst_plots/LepVeto_SMST2tt-ineff}
    } 
    \subfigure[\label{fig:sms-lepton-veto-ineff-t2bb}\HepProcess{\PSbottom\PSbottom\to\Pbottom\PSneutralino\APbottom\PSneutralino} (\texttt{T2bb})]{ 
      \includegraphics[width=0.45\textwidth]{figures/Analysis/syst_plots/LepVeto_SMST2bb-ineff}
    } \\
    \subfigure[\label{fig:sms-lepton-veto-ineff-t1tttt}\HepProcess{\PSgluino\PSgluino\to\Pbottom\APbottom\PSneutralino\Pbottom\APbottom\PSneutralino} (\texttt{T1tttt})]{
      \includegraphics[width=0.45\textwidth]{figures/Analysis/syst_plots/LepVeto_SMST1tttt-ineff}
    } 
    \subfigure[\label{fig:sms-lepton-veto-ineff-t1bbbb}\HepProcess{\PSgluino\PSgluino\to\Ptop\APtop\PSneutralino\Ptop\APtop\PSneutralino} (\texttt{T1bbbb})]{
      \includegraphics[width=0.45\textwidth]{figures/Analysis/syst_plots/LepVeto_SMST1bbbb-ineff}
    } \\
    \caption{\label{fig:sms-lepton-veto-ineff}The fraction of
      expected signal yield that is rejected by the lepton and photon
      vetoes, for various topologies. No requirement is made on the
      number of reconstructed b jets. }
  \end{center}
\end{figure}

\newpage
\begin{figure}[h!]
  \begin{center}
    \subfigure[\label{fig:sms-lepton-veto-near-t2tt}\HepProcess{\PStop\PStop\to\Ptop\PSneutralino\APtop\PSneutralino} (\texttt{T2tt})]{ 
      \includegraphics[width=0.45\textwidth]{figures/Analysis/syst_plots/LepVeto_SMST2tt-near}
    } 
    \subfigure[\label{fig:sms-lepton-veto-near-t2bb}\HepProcess{\PSbottom\PSbottom\to\Pbottom\PSneutralino\APbottom\PSneutralino} (\texttt{T2bb})]{ 
      \includegraphics[width=0.45\textwidth]{figures/Analysis/syst_plots/LepVeto_SMST2bb-near}
    } \\
    \subfigure[\label{fig:sms-lepton-veto-near-t1tttt}\HepProcess{\PSgluino\PSgluino\to\Pbottom\APbottom\PSneutralino\Pbottom\APbottom\PSneutralino} (\texttt{T1tttt})]{
%       \includegraphics[width=0.45\textwidth]{figures/Analysis/syst_plots/LepVeto_SMST1tttt-near}
%     } 
%     \subfigure[\label{fig:sms-lepton-veto-near-t1bbbb}\HepProcess{\PSgluino\PSgluino\to\Ptop\APtop\PSneutralino\Ptop\APtop\PSneutralino} (\texttt{T1bbbb})]{
      \includegraphics[width=0.45\textwidth]{figures/Analysis/syst_plots/LepVeto_SMST1bbbb-near}
    } \\
    \caption{\label{fig:sms-lepton-veto-near}The fraction of expected
      signal yield that is rejected by the lepton and photon vetoes,
      near to the diagonal, for various topologies. No requirement is
      made on the number of reconstructed b jets. }
  \end{center}
\end{figure}

\newpage
\begin{figure}[h!]
  \begin{center}
    \subfigure[\label{fig:sms-lepton-veto-far-t2tt}\HepProcess{\PStop\PStop\to\Ptop\PSneutralino\APtop\PSneutralino} (\texttt{T2tt})]{ 
      \includegraphics[width=0.45\textwidth]{figures/Analysis/syst_plots/LepVeto_SMST2tt-far}
    } 
    \subfigure[\label{fig:sms-lepton-veto-far-t2bb}\HepProcess{\PSbottom\PSbottom\to\Pbottom\PSneutralino\APbottom\PSneutralino} (\texttt{T2bb})]{ 
      \includegraphics[width=0.45\textwidth]{figures/Analysis/syst_plots/LepVeto_SMST2bb-far}
    } \\
    \subfigure[\label{fig:sms-lepton-veto-far-t1tttt}\HepProcess{\PSgluino\PSgluino\to\Pbottom\APbottom\PSneutralino\Pbottom\APbottom\PSneutralino} (\texttt{T1tttt})]{
      \includegraphics[width=0.45\textwidth]{figures/Analysis/syst_plots/LepVeto_SMST1tttt-far}
    } 
    \subfigure[\label{fig:sms-lepton-veto-far-t1bbbb}\HepProcess{\PSgluino\PSgluino\to\Ptop\APtop\PSneutralino\Ptop\APtop\PSneutralino} (\texttt{T1bbbb})]{
      \includegraphics[width=0.45\textwidth]{figures/Analysis/syst_plots/LepVeto_SMST1bbbb-far}
    } \\
    \caption{\label{fig:sms-lepton-veto-far}The fraction of expected
      signal yield that is rejected by the lepton and photon vetoes,
      far from the diagonal, for various topologies. No requirement is
      made on the number of reconstructed b jets. }
  \end{center}
\end{figure}


