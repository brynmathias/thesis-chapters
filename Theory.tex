\chapter{Theory} % (fold)
\label{cha:theory}
% The standard model of particle physics is the most complete model describing the particles that make up the visible universe and their interactions. The particles have half integer spin and carry combinations of various charges, the effects of these charges are mediated by the force carrying bosons. Figure~\ref{fig:figures_Theory_particle-interaction} shows the matter particles the three families of quarks and leptons and the force mediating bosons the photon, gluon, \PW, \PZ and the Higgs, the lines represent the allowed couplings between particles.
% 
% 

% 
% These forces can be described in terms of three symmetry groups, the force carrying bosons arise from breaking these symmetries via a local gauge transform. The strong force which acts between quarks and is mediated by the gluon is described by $SU(3)$, the weak isospin is described by $SU(2)$ and the weak hyper-charge is described by $U(1)$, $SU(2)\times U(1)$ provides the photon and the massive electro-weak bosons. The invariance under a local gauge transform infers that the electro-weak bosons are massless, the addition of a global scalar field to the Lagrangian breaks this mass symmetry and gives non-zero expectation values for the masses of the electro-weak bosons and the massive fermions.



% The \ac{sm} of particle physics which is fully described in \cite{} is a theoretical description of the fundamental particles observed in nature and their interactions. It describes the three fundamental forces: the strong force which is mediated by the gluon (\Pgluon) which carries a ``colour'' charge and interacts with quarks and has a self interaction; the weak force which is mediated by the electrically neutral \PZ boson and the electrically charged \PW, this weak force interacts with all known fundamental spin (1/2) particles; finally the electro-magnetic force is carried by the photon (\Pphoton) which mediates all electrical and magnetic interactions.
% One final mechanism grants the fundamental particles their masses, this is a scalar field who's manifestation is know as the Higgs boson (\PHiggs).
% The matter fermions and force carrying bosons are shown in Figure~\ref{fig:figures_Theory_particle-interaction}.
% 

% 
% The model it's self is a quantum field theory which is invariant under local gauge transform. This requirement determines the structure of the model. The full model is represented by the unitary product group of $SU(3) \bigotimes SU(2) \bigotimes U(1)$. Below we show that the requirement of the invariance under local gauge transform infers that all particles are massless:
% 
% 
% Consider two spin (1/2) fields $\psi_{1}$ and $\psi_{2}$, in natural units the free Lagrangian is then the sum of two Dirac Lagrangians:
% 
% \begin{equation}
%   \mathcal{L} = [i\bar{\psi}_{1}\gamma^{\mu}\partial_{\mu}\psi_{1} - m_{1}\bar{\psi}_{1}\psi_{1}] + [i\bar{\psi}_{2}\gamma^{\mu}\partial_{\mu}\psi_{2} - m_{2}\bar{\psi}_{2}\psi_{2}],
%   \label{eq:diracLagrangian}
% \end{equation}
% this can be simplified by combining $\psi_{1}$ and $\psi_{2}$ in to the column vector $\psi \equiv \colvec{2}{\psi_{1}}{\psi_{2}}$ where the adjoint spinor is $\bar{\psi} = \left(\bar{\psi}_{1}~\bar{\psi}_{2}\right)$, the simplified Lagrangian is then:
% 
% \begin{equation}
%   \mathcal{L} = i\bar{\psi}\gamma^{\mu}\partial_{\mu}\psi - m\bar{\psi}\psi
%   \label{eq:simpleDiracLagrangian}
% \end{equation}
% for the case where $m_1 = m_2$. $\psi$ is invariant under the transform
% \begin{equation}
%   \psi \rightarrow U\psi
% \end{equation}
% where $U$ is any $2\times2$ unitary matrix ie:
% \begin{equation}
%   U^{\dagger} U = 1,
% \end{equation}
% likewise
% \begin{equation}
%   \bar{\psi} \rightarrow \bar{\psi}U^{\dagger}
% \end{equation}
% is invariant meaning that $\bar{\psi}\psi$ is also invariant. $U$ may be written $U = e^{iH}$ where $H$ is hermitian and can be represented as 
% \begin{equation}
%   H = \theta\bm{1} + \bm{\tau}\cdot\bm{a},
% \end{equation}
% where $\bm{1}$ is the $2\times2$ unitary matrix and $\bm{\tau}$ represents the three Pauli matrices, hence any unitary $2\times2$ matrix can be expressed as
% \begin{equation}
%   U = e^{1\theta}e^{i\bm{\tau}\cdot\bm{a}}
% \end{equation}
% taking $\theta$ to be some global phase, the system is invariant under $\psi\rightarrow e^{i\theta}\psi$ and so only the effects of the global $SU(2)$ transform $\psi\rightarrow e^{i\bm{\tau}\cdot\bm{a}}$ are considered, as known the effects of a global transform do not effect the physics of the system, it is like changing the sea level on a contour map, the values all change but the gradients of the slopes do not. However the required invariance under a local gauge transform does not necessarily leave the system unchanged. To investigate this let $\bm{a}\rightarrow\bm{a}(x)$, we now have the transform:
% \begin{equation}
%   \psi\rightarrow e^{i\bm{\tau}\cdot\bm{a}(x)\psi}
%   \label{eq:localGaugeTransform}
% \end{equation}
% for simplicities sake let $S = e^{i\bm{\tau}\cdot\bm{a}(x)}$ which is a local $SU(2)$ transform. The Lagrangian $\mathcal{L}$ is not invariant under the transformation as there is an extra term due to the derivative of $S$, $\partial_\mu\psi\rightarrow S\partial_{\mu}\psi+\left(\partial_{\mu}S\right)\psi$, to over come this the derivative in $\mathcal{L}$ is replaced by a covariant derivative
% \begin{equation}
%   \mathcal{D}_{\mu} \equiv \partial_{\mu} + \bm{\tau} \cdot \bm{A}_{\mu}
% \end{equation}
% where $\bm{A}_{\mu}$ are the three gauge fields which transform such that
% \begin{equation}
%   \mathcal{D}_{\mu}\psi\rightarrow S\left(\mathcal{D}_{\mu}\psi\right)
% \end{equation}
\section{The Standard Model (SM)} % (fold)
\label{sec:the_standard_model_sm_}

% section the_standard_model_sm_ (end)
The \ac{sm} of particle physics seeks to describe all of the observed matter 
particles and their interactions. The theory is a quantum field theory which is 
both invariant under local gauge transform and renormilisable. The theory is 
constructed from the unitary product group $SU(3)\bigotimes SU(2)_{L} 
\bigotimes U(1)_{Y}$ where $SU(3)$ describes the colour charged strong nuclear 
force and $SU(2)_{L} \bigotimes U(1)_{Y}$ describes the electro-weak 
interactions. The electro-weak sector contains bosons which exhibit mass, which 
is contrary to what is inferred by the invariance under local gauge transform, 
that the gauge bosons should be massless. The $SU(2)_{L} \bigotimes U(1)_{Y}$ 
symmetry is thus seen to be broken, this is achieved by the addition of a 
scalar field with a non-zero vacuum expectation value, the extra degrees of 
freedom introducted by this scalar field allow the electro-weak bosons to gain 
mass with out breaking the local gauge invariance of the Lagrangian. The method 
was proposed by Englert, Brout, Higgs, Guralnik, Hagen and Kibble \cite{} but 
is shortened to the ``Higgs'' mechanism and the resulting mediator is known as 
the Higgs boson (\PHiggs).

The \ac{sm} matter particles have spin (1/2) and obey Fermi-Dirac statistics, 
they exist in three families of quark and lepton doublets. The force carrying 
particles have spin (1) except for the \PHiggs which is predicted to have spin 
(0), the bosons obey Bose-Einstein statistics. 
Figure~\ref{fig:figures_Theory_particle-interaction} shows the hierarchy of the 
standard model particles and the couplings between them.

\begin{figure}[htbp]
  \centering
    \includegraphics[width=0.5\textwidth]{figures/Theory/particle-interaction.jpeg}
  \caption{caption}
  \label{fig:figures_Theory_particle-interaction}
\end{figure}

\subsection{Gauge Invariance} % (fold)
\label{sub:gauge_invariance}
The example of adding a local gauge term to the free Dirac Lagrangian is chosen to show the effects of requiring a system to be invariant under local gauge transform. It is to be noted that natural units are used throughout ($c = \hbar = 1$). The free Dirac Lagrangian is written as
\begin{equation}
  \mathcal{L} = i\bar{\psi}\gamma^{\mu}\partial_{\mu}\psi - m\bar{\psi}\psi,
  \label{eq:freeDiracL}
\end{equation}
under a simple global phase transform $\psi\rightarrow e^{i\theta}\psi$ and $\bar{\psi}\rightarrow e^{-i\theta}\bar{\psi}$ the exponents cancel and we are left with an invariant system. However if $\theta\rightarrow\theta (x)$ there is a $\theta$ dependant term introduced when evaluating the derivative which destroys the invariance of the system:
\begin{equation}
  \partial_{\mu}\left(e^{i\theta}\psi\right) = i\left(\partial_{\mu}\theta\right)e^{i\theta}\psi + e^{i\theta}\partial_{\mu}\psi.
\end{equation} 

This infers that an extra term is added to the Lagrangian, i.e
\begin{equation}
  \mathcal{L}\rightarrow\mathcal{L}-\left(\partial_{\mu}\theta\right)\bar{\psi}\gamma^{\mu}\psi
\end{equation}
for convenience we set $\lambda(x) = -\frac{\theta(x)}{q}$ where $q$ is the charge/coupling of the particle we can re-express the Lagrangian as
\begin{equation}  \mathcal{L} \rightarrow \mathcal{L} + \left( q\bar{\psi}\gamma^{\mu} \psi \right) \partial_{\mu} \lambda(x)
  \label{eq:diracLplusLambda}
\end{equation}

when demanding that the entire Lagrangian is invariant under local transform one must add a term to cancel the additional term in Equation~\eqref{eq:diracLplusLambda}, we take
\begin{equation}
  \mathcal{L} = \left[i\bar{\psi}\gamma^{\mu}\partial_{\mu} - m \bar{\psi}\psi\right] - \left(q\bar{\psi}\gamma^{\mu}\psi\right)\bm{A}_{\mu}
\end{equation}
where $\bm{A}_{\mu}$ is some new field which transforms as
\begin{equation}
  \bm{A}_{\mu}\rightarrow \bm{A}_{\mu} + \partial_{\mu}\lambda(x)
  \label{eq:localGaugeInvar}
\end{equation}
This property of $\bm{A}_{\mu}$ means that the Lagrangian is now invariant to 
the additions of a local phase or gauge. However $\mathcal{L}$ must include a 
term for the free field $\bm{A}^{\mu}$
\begin{equation}
  \mathcal{L} = \frac{-1}{16\pi}\bm{F}^{\mu\nu}\bm{F}_{\mu\nu} + \frac{1}{8\pi}m_{A}^{2}\bm{A}^{\nu}\bm{A}_{\nu}
\end{equation}
for the Lagrangian to remain invariant $m_{A}$ must equal zero. Hence requiring 
that the Dirac Lagrangian be invariant under local phase or gauge transform 
requires the addition of a massless vector field $\bm{A}_{\mu}$ giving the full 
Lagrangian as
\begin{equation}
  \mathcal{L} = \left[i\bar{\psi}\gamma^{\mu}\partial_{\mu}\psi - m \bar{\psi}\psi\right] - \left[\frac{-1}{16\pi}\bm{F}^{\mu\nu}\bm{F}_{\mu\nu}\right] - \left(q\bar{\psi}\gamma^{\mu}\psi\right)\bm{A}_{\mu}.
\end{equation}
Equation~\eqref{eq:localGaugeInvar} shows the choice of gauge, in this case the 
electro-magnetic potential does not change the system. This addition of a local 
phase invariance to the free Dirac Lagrangian generates all electro dynamics.

The difference between the addition of a global phase and a local phase arrises 
from the calculation of the derivatives of the fields
\begin{equation}
  \partial_{\mu}\psi\rightarrow e^{-iq\lambda(x)}\left[\partial_{\mu}- iq\left(\partial_{\mu}\lambda(x)\right)\right]\psi
  \label{eq:needsCanceling}
\end{equation}
here rather than picking up a phase factor we pick up a term involving 
$\partial_{\mu}\lambda(x)$, this can be removed by replacing $\partial_{\mu}$ 
in the full Lagrangian by the covariant derivative
\begin{equation}
  \mathcal{D}_{\mu} \equiv \partial_{\mu} + iq\bm{A}_{\mu}
\end{equation}
this replacement cancels the extra term in Equation~\eqref{eq:needsCanceling}. 
This substitution of $\mathcal{D}_{\mu}$ is an elegant method for promoting a 
globally invariant Lagrangian to a locally invariant one.
% subsection gauge_invariance (end)

\subsection{Electro-weak symmetry and interactions} % (fold)
\label{sub:electro_weak_symmetry_and_interactions}
If we take the example show in Section~\ref{sub:gauge_invariance} which 
describes the effects of requiring that the Lagrangian for a free Dirac 
particle is invariant under local gauge transformation infers the existence of 
a massless gauge field, which is responsible for the photon and the 
electrodynamic interactions.

If we expand this so that rather than considering the Dirac equation for one 
free particle we consider two free particles, the requirement for invariance 
under local gauge transform describes not only the interaction with the gauge 
field but also the inter particle interactions. In this case the covariant 
derivative $\mathcal{D}_{\mu}$ is expressed as:
\begin{equation}
  \mathcal{D}_{\mu}\equiv \partial_{\mu} + iq \bm{\tau}\cdot\bm{A}_{\mu}
\end{equation}
this promotes us to the $SU(2)_{L} \bigotimes U(1)_{Y}$ regime where there are 
a total of four gauge fields, $\bm{\tau}$ represents the three Pauli matrices 
and $\bm{A}_{\mu}$ represents three gauge fields, these three gauge fields and 
the mixing with the gauge field seen in $U(1)_{Y}$ are responsible for the \PW 
and \PZ bosons and the photon. However as shown before the requirement for the 
whole Lagrangian including the terms for the free gauge fields to be invariant 
under local gauge transform forces these gauge fields to be massless.


\subsection{Quantum Chromo Dynamics} % (fold)
\label{sub:quantum_chromo_dynamics}
The component which the scribes the strong force is the $SU(3)$ term in 
$SU(3)\bigotimes SU(2)_{L} \bigotimes U(1)_{Y}$. A similar gauge invariance is 
required of the $SU(3)$ group as is required in the previous sections. However 
the gluon mass is already set to zero by nature so this is not a broken 
symmetry. The eight gauge fields represent the eight colour combinations of 
gluons. The tensor in the Lagrangian is given by
\begin{equation}
  F^{\mu\nu} \equiv \partial_{\mu}\bm{A}^{\nu} - \partial^{\nu}\bm{A}_{\mu} - 2q\left(\bm{A}^{\mu}\times\bm{A^{\nu}}\right)
\end{equation}
where the cross product is given by
\begin{equation}
  \left(\bm{B}\times\bm{C}\right)_{i} = \sum^{8}_{j,k=1}f_{ijk}\bm{B}_{j}\bm{C}_{k}
\end{equation}
this cross product contains the self interaction of the gluon.
% subsection quantum_chromo_dynamics (end)





\subsection{The Higgs and Electro-weak Symmetry Breaking} % (fold)
\label{sub:the_higgs_and_electro_weak_symmetry_breaking}
It has been shown that the choice of representing the \ac{sm} as a gauge invariant Lagrangian for the $SU(3)\bigotimes SU(2)_{L} \bigotimes U(1)_{Y}$ group product naturally gives the inter-particle couplings, interaction fields and the force carrying bosons. It has also been shown in \cite{Hooft1971167} that these gauge invariant theories are renormilisable. However as shown in Section~\ref{sub:gauge_invariance} the mass terms of the free fields are not locally gauge invariant, whilst this is not a problem for the photon or gluons as their masses are set to zero by nature, it is a problem for the \PW and \PZ bosons which are massive \cite{Arnison1983103,Aarnio1989539}. TO break the electro-weak symmetry and give the \PW and \PZ mass a scalar field with a non-zero vacuum expectation value is introduced in to the Lagrangian\cite{Higgs:1966cl,Higgs:1964eu}. This field is knows as the Higgs field, it breaks the symmetry of the $SU(2)\bigotimes U(1)$ group, introducing an extra degree of freedom which can be used to give the bosons mass, the breaking method also predicts the now observed\cite{Chatrchyan201230,Aad20121} Higgs boson.

% subsection the_higgs_and_electro_weak_symmetry_breaking (end)

% subsection electro_weak_symmetry_and_interactions (end)

\section{Beyond the Standard Mode.} % (fold)
\label{sec:beyond_the_standard_mode_}
The \ac{sm} does a good job of describing the fundamental particles, their 
interactions and the generation of fundamental mass. However it is not a full 
description of the observed physics in the universe.

Firstly whilst the theory is invariant under special relativity, it is not 
invariant under general relativity and as such provides no description of the 
gravitational interactions of the fundamental particles. 
Secondly the amount of \ac{cp} violation predicted by the \ac{sm} is not 
sufficient to account for the observed matter dominance in the universe.

There are also problems in the neutrino sector where the \ac{sm} predicts that 
the neutrino is massless, however due to the observation of flavour changing 
the neutrino mass must be non-zero.
The most glaring short coming other than the lack of a description of 
gravitational dynamics is the lack of a dark matter candidate. This dark matter 
candidate needs to be heavy, stable, chargeless, colourless and interact via 
the weak nuclear force only. Such particles are predicted by \ac{susy}\cite{} 
models which also provide a description of gravity. These models insert a 
further broken symmetry between bosons and fermions, i.e. for each \ac{sm} 
boson there is a \ac{susy} fermion and visa versa.


% section beyond_the_standard_mode_ (end)

\subsection{The Constrained Minimal Super Symmetric Model} % (fold)
\label{sec:the_constrained_minimal_super_symmetric_model}
Whilst there are a plethora of \ac{susy} models, in order to compare the 
\ac{cms} searches against previous and contemporary experiments the \ac{cmssm} 
is chosen as a bench mark, however there are compelling reasons to study the 
\ac{cmssm} it's self. \ac{susy} introduces 105 new parameters to the existing 
19 of the \ac{sm}, effectively sampling a space this large is prohibitive, 
instead the \ac{cmssm} reduces the number of free parameters to 5 these are: 
$tan \beta$, where $\beta$ is the ratio of the vacuum expectation values for 
the two Higgs fields; $m_{0}$ and $m_{1/2}$ which are the mass values for the 
boson and fermion states at the unification scale; $A_{0}$ which is the 
\ac{susy} breaking tri-linear coupling; finally the sign of $\mu$ which is the  
Higgs breaking parameter.

Through out the accessible regions of these values both electro-weak and strong 
production methods are predicted as well as a multitude of final states 
involving a stable dark matter candidate particle $\chi_{0}$ and standard model 
particles. For the interpretation of the analysis presented in this thesis, 
limits are set as a function of $m_{0}$ and $m_{1/2}$ with fixed values for the 
other variables of $tan \beta = 10$, $A_{0} = 0$ and $\mu > 0$.
Before \ac{lhc} start up the preferred values for $m_{0}$ and $m_{1/2}$ as 
given by \cite{mastercode} were well defined and with in reach of the early 
\ac{lhc} \ac{susy} searches.
% section the_constrained_minimal_super_symmetric_model (end)

\subsection{Simplified Models} % (fold)
\label{sub:simplified_models}
\cite{ArkaniHamed:2007fw}
An alternative to searching for a specific beyond the standard model Lagrangian and it's associated physical manifestation is to build a set of self consistent models of new physics which are characterised by production and decay topology, most generally the type of the pair produced particle (quark like or gluon like) and the mass splitting between this parent particle and the final state weekly interacting massive particle which provides a dark matter candidate.
This method has several advantages over searching for a specific model:
\begin{itemize}
\item If a discrepancy with the \ac{sm} is found the amount of collected data during early running of the \ac{lhc} will be insufficient to confirm or rule out specific models of new physics.
\item Generalised limits on decay topologies can be applied to many models and are useful for guiding model building.
\item Difficult to search for decay topologies such as those with small mass splittings can be investigated and analyses turned to these areas of kinematic phase space.
\end{itemize}



\begin{figure}[h!]
    \centering
    \subfigure[Production and decay of the \texttt{T1} simplified model with the process \HepProcess{\Pproton\Pproton\to\PSgluino\PSgluino\to\Pquark\APquark\PSneutralino\Pquark\APquark\PSneutralino}]{
          \label{fig:T1feyn}
          \includegraphics[width=0.45\textwidth]{figures/Theory/T1.pdf}
     }
    \subfigure[Production and decay of the \texttt{T2} simplified model with the process \HepProcess{\Pproton\Pproton\to\Psquark\Psquark\to\Pquark\PSneutralino\APquark\PSneutralino}]{
          \label{fig:T2feyn}
          \includegraphics[width=0.45\textwidth]{figures/Theory/T2.pdf}
     }
    \caption{Simplified model production and decay diagrams.}
    \label{fig:T1T2feyn}
\end{figure}



The models considered by this analysis require fully hadronic final states. Two 
production topologies are considered. Gluino-gluino production which is 
pictured in Figure~\ref{fig:T1feyn} is referred to as \texttt{T1} type models, 
where any suffix labels the flavour that the final state particles are forced 
to. \texttt{T1} decays to four light (\Pup,\Pdown or \Pcharm) quarks and two 
\PSneutralino particles.
Squark-squark production, pictured in Figure~\ref{fig:T2feyn} is referred to as \texttt{T2} type production, again 
any suffix denotes the flavour of the final state quarks. The decay topology 
involves two \PSneutralino particles and two final state quarks which hadronise 
to form jets.

% subsection simplified_models (end)


% chapter theory (end)






























