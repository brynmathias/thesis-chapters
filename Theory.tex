\chapter{Theoretical Overview} % (fold)
\label{cha:theory}
% The \ac{sm} of particle physics is the most complete model describing the particles that make up the visible universe and their interactions. The particles have half integer spin and carry combinations of various charges, the effects of these charges are mediated by the force carrying bosons. Figure~\ref{fig:figures_Theory_particle-interaction} shows the matter particles the three families of quarks and leptons and the force mediating bosons the photon, gluon, \PW, \PZ and the Higgs, the lines represent the allowed couplings between particles.
% 
% 

% 
% These forces can be described in terms of three symmetry groups, the force carrying bosons arise from breaking these symmetries via a local gauge transform. The strong force which acts between quarks and is mediated by the gluon is described by $SU(3)$, the weak isospin is described by $SU(2)$ and the weak hyper-charge is described by $U(1)$, $SU(2)\times U(1)$ provides the photon and the massive Electroweak bosons. The invariance under a local gauge transform infers that the Electroweak bosons are massless, the addition of a global scalar field to the Lagrangian breaks this mass symmetry and gives non-zero expectation values for the masses of the Electroweak bosons and the massive fermions.



% The \ac{sm} of particle physics which is fully described in \cite{} is a theoretical description of the fundamental particles observed in nature and their interactions. It describes the three fundamental forces: the strong force which is mediated by the gluon (\Pgluon) which carries a ``colour'' charge and interacts with quarks and has a self interaction; the weak force which is mediated by the electrically neutral \PZ boson and the electrically charged \PW, this weak force interacts with all known fundamental spin (1/2) particles; finally the electro-magnetic force is carried by the photon (\Pphoton) which mediates all electrical and magnetic interactions.
% One final mechanism grants the fundamental particles their masses, this is a scalar field who's manifestation is know as the Higgs boson (\PHiggs).
% The matter fermions and force carrying bosons are shown in Figure~\ref{fig:figures_Theory_particle-interaction}.
% 

% 
% The model it's self is a quantum field theory which is invariant under local gauge transform. This requirement determines the structure of the model. The full model is represented by the unitary product group of $SU(3) \bigotimes SU(2) \bigotimes U(1)$. Below we show that the requirement of the invariance under local gauge transform infers that all particles are massless:
% 
% 
% Consider two spin (1/2) fields $\psi_{1}$ and $\psi_{2}$, in natural units the free Lagrangian is then the sum of two Dirac Lagrangians:
% 
% \begin{equation}
%   \mathcal{L} = [i\bar{\psi}_{1}\gamma^{\mu}\partial_{\mu}\psi_{1} - m_{1}\bar{\psi}_{1}\psi_{1}] + [i\bar{\psi}_{2}\gamma^{\mu}\partial_{\mu}\psi_{2} - m_{2}\bar{\psi}_{2}\psi_{2}],
%   \label{eq:diracLagrangian}
% \end{equation}
% this can be simplified by combining $\psi_{1}$ and $\psi_{2}$ in to the column vector $\psi \equiv \colvec{2}{\psi_{1}}{\psi_{2}}$ where the adjoint spinor is $\bar{\psi} = \left(\bar{\psi}_{1}~\bar{\psi}_{2}\right)$, the simplified Lagrangian is then:
% 
% \begin{equation}
%   \mathcal{L} = i\bar{\psi}\gamma^{\mu}\partial_{\mu}\psi - m\bar{\psi}\psi
%   \label{eq:simpleDiracLagrangian}
% \end{equation}
% for the case where $m_1 = m_2$. $\psi$ is invariant under the transform
% \begin{equation}
%   \psi \rightarrow U\psi
% \end{equation}
% where $U$ is any $2\times2$ unitary matrix ie:
% \begin{equation}
%   U^{\dagger} U = 1,
% \end{equation}
% likewise
% \begin{equation}
%   \bar{\psi} \rightarrow \bar{\psi}U^{\dagger}
% \end{equation}
% is invariant meaning that $\bar{\psi}\psi$ is also invariant. $U$ may be written $U = e^{iH}$ where $H$ is hermitian and can be represented as 
% \begin{equation}
%   H = \theta\bm{1} + \bm{\tau}\cdot\bm{a},
% \end{equation}
% where $\bm{1}$ is the $2\times2$ unitary matrix and $\bm{\tau}$ represents the three Pauli matrices, hence any unitary $2\times2$ matrix can be expressed as
% \begin{equation}
%   U = e^{1\theta}e^{i\bm{\tau}\cdot\bm{a}}
% \end{equation}
% taking $\theta$ to be some global phase, the system is invariant under $\psi\rightarrow e^{i\theta}\psi$ and so only the effects of the global $SU(2)$ transform $\psi\rightarrow e^{i\bm{\tau}\cdot\bm{a}}$ are considered, as known the effects of a global transform do not effect the physics of the system, it is like changing the sea level on a contour map, the values all change but the gradients of the slopes do not. However the required invariance under a local gauge transform does not necessarily leave the system unchanged. To investigate this let $\bm{a}\rightarrow\bm{a}(x)$, we now have the transform:
% \begin{equation}
%   \psi\rightarrow e^{i\bm{\tau}\cdot\bm{a}(x)\psi}
%   \label{eq:localGaugeTransform}
% \end{equation}
% for simplicities sake let $S = e^{i\bm{\tau}\cdot\bm{a}(x)}$ which is a local $SU(2)$ transform. The Lagrangian $\mathcal{L}$ is not invariant under the transformation as there is an extra term due to the derivative of $S$, $\partial_\mu\psi\rightarrow S\partial_{\mu}\psi+\left(\partial_{\mu}S\right)\psi$, to over come this the derivative in $\mathcal{L}$ is replaced by a covariant derivative
% \begin{equation}
%   \mathcal{D}_{\mu} \equiv \partial_{\mu} + \bm{\tau} \cdot \bm{A}_{\mu}
% \end{equation}
% where $\bm{A}_{\mu}$ are the three gauge fields which transform such that
% \begin{equation}
%   \mathcal{D}_{\mu}\psi\rightarrow S\left(\mathcal{D}_{\mu}\psi\right)
% \end{equation}

In this chapter the \ac{sm} of particle physics is outlined. The 
\ac{sm} describes the particles and their interactions. Several limitations 
with this model are discussed, which motivate the need for a more complete 
theory. Some proposals for models beyond the \ac{sm} and their theoretical 
motivation and possible physical realisation are discussed.

In addition the theoretical uncertainties on already known processes are 
discussed in terms of their effects on the level of accuracy of the simulation 
of both known processes and possible new models.

The particles of the \ac{sm} are comprised of half integer spin matter 
particles, know as fermions, which exist in three families of chiral doublets 
for each of the leptons and quarks. The particles also have a charge conjugate partner which is their anti-particle.
The force carriers are comprised of integer spin bosons which arise from the 
gauge invariance of the \ac{sm}. Tables~\ref{tab:smparticles} and 
\ref{tab:smforces} list the names and observable quantities of the \ac{sm} 
particles. The majority of observable mass in the universe consists of 
particles in the first generation of fermions, as the heavier generations decay 
via the charged weak interaction to particles in the first generation.
In the next section the formalisation of the \ac{sm} is described.

\begin{table}
\begin{tabular}{lccccr}
\hline
Name & Type & Generation & Spin & Charge (e) & Mass\\
\hline
Electron (e) & lepton & 1 & 1/2 & -1 & \unit{511}{\MeV}\\
\hline
Electron Neutrino ($\nu_{e}$) & lepton & 1 & 1/2 & 0 & \unit{$>2.2$}{\eV}\\
\hline
Muon ($\mu$) & lepton & 2 & 1/2 & -1 & \unit{105.7}{\MeV}\\
\hline
Muon Neutrino ($\nu_{\mu}$) & lepton & 2 & 1/2 & 0 &-\\
\hline
Tauon ($\tau$) & lepton & 3 & 1/2 & -1 & \unit{1.77}{\GeV}\\
\hline
Tau Neutrino ($\nu_{\tau}$) & lepton & 3& 1/2 &0 &-\\
\hline
Up quark (\Pup) & quark & 1 & 1/2 & +2/3 & \unit{2.3$_{-0.5}^{+0.7}$}{\MeV} \\
\hline
Down quark (\Pdown) & quark & 1 & 1/2 & -1/3 & \unit{4.8$_{-0.3}^{+0.7}$}{\MeV} \\
\hline
Charm quark (\Pcharm) & quark & 2 & 1/2 & +2/3 & \unit{1.275 $\pm$ 0.025}{\GeV} \\
\hline
Strange quark (\Pstrange) & quark & 2 & 1/2 & -1/3 & \unit{95 $\pm$ 5}{\MeV} \\
\hline
Top quark (\Ptop) & quark & 3 & 1/2 & +2/3 & \unit{173.5 $\pm$ 0.6 $\pm$ 0.8}{\GeV} \\
\hline
Bottom quark (\Pbottom) & quark & 3 & 1/2 & -1/3 & \unit{4.65 $\pm$ 0.03}{\GeV} \\
\hline

\end{tabular}
\caption{The \ac{sm} matter fermions, their masses, spin and charge. Values taken from \cite{pdg}.}
\label{tab:smparticles}
\end{table}
\begin{table}
\begin{tabular}{lcccr}
  \hline
  Name & Spin & Charge & Mass & Force Carried \\
  \hline
  Photon (\Pphoton) & 1 & 0 & 0 & Electromagnetism \\
  \hline
  Gluon (\Pgluon) & 1 & 0 & 0 & Strong Nuclear Force  \\
  \hline
  W Boson (\PW) & 1 & -1 & \unit{80.385 $\pm$ 0.015}{\GeV} & Weak Nuclear Force\\
  \hline
  Z Boson (\PZ) & 1 & 0 & \unit{91.187 $\pm$ 0.002}{\GeV}  & Weak Nuclear Force\\
  \hline
  Higgs Boson (\PHiggs) & 0 & 0 & \unit{125.3 $\pm$ 0.4 $\pm$ 0.5}{\GeV}\cite{Chatrchyan201230} & Fundamental Mass \\
  \hline
\end{tabular}
  \caption{The \ac{sm} force carrying bosons, their masses, spins, charges and the force which they carry. Values from \cite{pdg}.}
  \label{tab:smforces}
\end{table}
\clearpage
\section{The \ac{sm}} % (fold)
\label{sec:the_standard_model_sm_}

% section the_standard_model_sm_ (end)
The \ac{sm} of particle physics 
\cite{Glashow1961579,PhysRevLett.19.1264,tHooft1972189} seeks to describe all 
of the observed matter particles and their interactions. The theory is a 
quantum field theory which is both invariant under local gauge transform and 
renormalisable. The theory is constructed from the unitary product group
$SU(3)\bigotimes SU(2)_{L} \bigotimes U(1)_{Y}$ where $SU(3)$ describes the
colour charged strong nuclear force and $SU(2)_{L} \bigotimes U(1)_{Y}$
describes the Electroweak interactions. The Electroweak sector contains 
bosons which exhibit mass, which is contrary to what is inferred by the
invariance under local gauge transform, which states that the gauge bosons 
should be massless. The $SU(2)_{L} \bigotimes U(1)_{Y}$ symmetry is thus seen 
to be broken. This is achieved by the addition of a scalar field with a 
non-zero vacuum expectation value, the extra degrees of freedom introduced by 
this scalar field allow the Electroweak bosons to gain mass with out breaking 
the local gauge invariance of the Lagrangian. The method was proposed by 
Englert, Brout, Higgs, Guralnik, Hagen and Kibble 
\cite{tHooft1972189,PhysRevLett.13.321,Higgs1964132,PhysRevLett.13.508,PhysRevLett.13.585,PhysRev.145.1156,PhysRev.155.1554} but 
is shortened to the ``Higgs'' mechanism. The resulting mediator is known as
the Higgs boson (\PHiggs).

The \ac{sm} matter particles have spin (1/2) and obey Fermi-Dirac statistics,
they exist in three families of quark and lepton doublets. The force carrying
particles have spin (1) except for the \PHiggs which is predicted to have spin
(0), they obey Bose-Einstein statistics and are collectively known as `bosons'.
Figure~\ref{fig:figures_Theory_particle-interaction} shows the hierarchy of the
\ac{sm} particles and the couplings between them.

\begin{figure}[htbp]
  \centering
    \includegraphics[width=0.5\textwidth]{figures/Theory/particle-interaction.jpeg}
  \caption{Diagram of the \ac{sm} particles, force carriers and their tree 
  level interactions. Black circles represent particles and the blue lines 
  represent the possible interactions.\cite{wikiParticleInteractions}}
  \label{fig:figures_Theory_particle-interaction}
\end{figure}

\subsection{Gauge Invariance} % (fold)
\label{sub:gauge_invariance}
Following the workings in \cite{griffiths}, the example of adding a local gauge 
term to the free Dirac Lagrangian is chosen to show the effects of requiring a 
system to be invariant under local gauge transform. It is to be noted that 
natural units are used throughout ($c = \hbar = 1$). The free Dirac Lagrangian 
is written as
\begin{equation}
  \mathcal{L} = i\bar{\psi}\gamma^{\mu}\partial_{\mu}\psi - m\bar{\psi}\psi,
  \label{eq:freeDiracL}
\end{equation}
under a simple global phase transform $\psi\rightarrow e^{i\theta}\psi$ and 
$\bar{\psi}\rightarrow e^{-i\theta}\bar{\psi}$ the exponents cancel and we are 
left with an invariant system. However if $\theta\rightarrow\theta (x)$ there 
is a $\theta$ dependant term introduced when evaluating the derivative which 
destroys the invariance of the system:
\begin{equation}
  \partial_{\mu}\left(e^{i\theta}\psi\right) = i\left(\partial_{\mu}\theta\right)e^{i\theta}\psi + e^{i\theta}\partial_{\mu}\psi.
\end{equation} 

This infers that an extra term is added to the Lagrangian, i.e
\begin{equation}
  \mathcal{L}\rightarrow\mathcal{L}-\left(\partial_{\mu}\theta\right)\bar{\psi}\gamma^{\mu}\psi
\end{equation}
for convenience we set $\lambda(x) = -\frac{\theta(x)}{q}$ where $q$ is the 
charge/coupling of the particle. We can re-express the Lagrangian as
\begin{equation}  \mathcal{L} \rightarrow \mathcal{L} + \left( q\bar{\psi}\gamma^{\mu} \psi \right) \partial_{\mu} \lambda(x)
  \label{eq:diracLplusLambda}
\end{equation}

when demanding that the entire Lagrangian is invariant under local transform 
one must add a term to cancel the additional term in 
Equation~\eqref{eq:diracLplusLambda}. We take
\begin{equation}
  \mathcal{L} = \left[i\bar{\psi}\gamma^{\mu}\partial_{\mu} - m \bar{\psi}\psi\right] - \left(q\bar{\psi}\gamma^{\mu}\psi\right)\bm{A}_{\mu}
\end{equation}
where $\bm{A}_{\mu}$ is some new field which transforms as
\begin{equation}
  \bm{A}_{\mu}\rightarrow \bm{A}_{\mu} + \partial_{\mu}\lambda(x).
  \label{eq:localGaugeInvar}
\end{equation}
This property of $\bm{A}_{\mu}$ means that the Lagrangian is now invariant to 
the additions of a local phase or gauge. However $\mathcal{L}$ must include a 
term for the free field $\bm{A}^{\mu}$
\begin{equation}
  \mathcal{L} = \frac{-1}{16\pi}\bm{F}^{\mu\nu}\bm{F}_{\mu\nu} + \frac{1}{8\pi}m_{A}^{2}\bm{A}^{\nu}\bm{A}_{\nu}
\end{equation}
for the Lagrangian to remain invariant $m_{A}$ must equal zero. The requirement 
that the Dirac Lagrangian be invariant under local phase or gauge transform 
necessitates the addition of a massless vector field $\bm{A}_{\mu}$. The full Lagrangian is then given by
\begin{equation}
  \mathcal{L} = \left[i\bar{\psi}\gamma^{\mu}\partial_{\mu}\psi - m \bar{\psi}\psi\right] - \left[\frac{-1}{16\pi}\bm{F}^{\mu\nu}\bm{F}_{\mu\nu}\right] - \left(q\bar{\psi}\gamma^{\mu}\psi\right)\bm{A}_{\mu}.
\end{equation}
Equation~\eqref{eq:localGaugeInvar} shows the choice of gauge, in this case the 
electro-magnetic potential does not change the system. This addition of a local 
phase invariance to the free Dirac Lagrangian generates all electro dynamics.

The difference between the addition of a global phase and a local phase arrises 
from the calculation of the derivatives of the fields
\begin{equation}
  \partial_{\mu}\psi\rightarrow e^{-iq\lambda(x)}\left[\partial_{\mu}- iq\left(\partial_{\mu}\lambda(x)\right)\right]\psi
  \label{eq:needsCanceling}
\end{equation}
here rather than picking up a phase factor we pick up a term involving 
$\partial_{\mu}\lambda(x)$, this can be removed by replacing $\partial_{\mu}$ 
in the full Lagrangian by the covariant derivative
\begin{equation}
  \mathcal{D}_{\mu} \equiv \partial_{\mu} + iq\bm{A}_{\mu}
\end{equation}
this replacement cancels the extra term in Equation~\eqref{eq:needsCanceling}. 
This substitution of $\mathcal{D}_{\mu}$ is an elegant method for promoting a 
globally invariant Lagrangian to a locally invariant one.
% subsection gauge_invariance (end)

\subsection{Electroweak symmetry and interactions} % (fold)
\label{sub:electro_weak_symmetry_and_interactions}
If we take the example show in Section~\ref{sub:gauge_invariance} which 
describes the effects of requiring that the Lagrangian for a free Dirac 
particle is invariant under local gauge transformation infers the existence of 
a massless gauge field, which is responsible for the photon and the 
electrodynamic interactions.

If we expand this so that rather than considering the Dirac equation for one 
free particle we consider two free particles, the requirement for invariance 
under local gauge transform describes not only the interaction with the gauge 
field but also the inter particle interactions. In this case the covariant 
derivative $\mathcal{D}_{\mu}$ is expressed as:
\begin{equation}
  \mathcal{D}_{\mu}\equiv \partial_{\mu} + iq \bm{\tau}\cdot\bm{A}_{\mu}
\end{equation}
this promotes us to the $SU(2)_{L} \bigotimes U(1)_{Y}$ regime where there are 
a total of four gauge fields, $\bm{\tau}$ represents the three Pauli matrices 
and $\bm{A}_{\mu}$ represents three gauge fields, these three gauge fields and 
the mixing with the gauge field seen in $U(1)_{Y}$ are responsible for the \PW 
and \PZ bosons and the photon. However as shown before, the requirement for the 
whole Lagrangian, including the terms for the free gauge fields, to be invariant under local gauge transform forces these gauge fields to be massless.


\subsection{Quantum Chromo Dynamics} % (fold)
\label{sub:quantum_chromo_dynamics}
The component which describes the strong force is the $SU(3)$ term in 
$SU(3)\bigotimes SU(2)_{L} \bigotimes U(1)_{Y}$. A similar gauge invariance is 
required of the $SU(3)$ group as is required in the previous sections. However 
the gluon mass is already set to zero by nature, so this is not a broken 
symmetry. The eight gauge fields represent the eight colour combinations of 
gluons. The tensor in the Lagrangian is given by
\begin{equation}
  F^{\mu\nu} \equiv \partial_{\mu}\bm{A}^{\nu} - \partial^{\nu}\bm{A}_{\mu} - 2q\left(\bm{A}^{\mu}\times\bm{A^{\nu}}\right)
\end{equation}
where the cross product is given by
\begin{equation}
  \left(\bm{B}\times\bm{C}\right)_{i} = \sum^{8}_{j,k=1}f_{ijk}\bm{B}_{j}\bm{C}_{k}
\end{equation}
this cross product contains the self interaction of the gluon.
% subsection quantum_chromo_dynamics (end)





\subsection{The Higgs and Electroweak Symmetry Breaking} % (fold)
\label{sub:the_higgs_and_electro_weak_symmetry_breaking}
It has been shown that the choice of representing the \ac{sm} as a gauge 
invariant Lagrangian for the $SU(3)\bigotimes SU(2)_{L} \bigotimes U(1)_{Y}$ 
group product naturally gives the inter-particle couplings, interaction fields 
and the force carrying bosons. It has also been shown in \cite{Hooft1971167} 
that these gauge invariant theories are renormalizable. However as shown in 
Section~\ref{sub:gauge_invariance} the mass terms of the free fields are not 
locally gauge invariant, whilst this is not a problem for the photon or gluons 
as their masses are set to zero by nature, it is a problem for the \PW and \PZ 
bosons which are massive \cite{Arnison1983103,Aarnio1989539}. To break the 
Electroweak symmetry and give the \PW and \PZ mass, a scalar field with a 
non-zero vacuum expectation value is introduced in to the 
Lagrangian\cite{Higgs:1966cl,Higgs:1964eu}. This field is knows as the Higgs 
field, it breaks the symmetry of the $SU(2)\bigotimes U(1)$ group, introducing 
an extra degree of freedom which can be used to give the bosons mass, the 
breaking method also predicts the recently observed 
\cite{Chatrchyan201230,Aad20121} Higgs boson.

% subsection the_higgs_and_electro_weak_symmetry_breaking (end)

% subsection electro_weak_symmetry_and_interactions (end)

\section{Beyond the \ac{sm}.} % (fold)
\label{sec:beyond_the_standard_mode_}
The \ac{sm} describes the fundamental particles, their 
interactions and the generation of fundamental mass to very high precision. 
 However it is not a full description of the observed physics in the universe.

Firstly whilst the theory is invariant under special relativity, it is not 
invariant under general relativity and as such provides no description of the 
gravitational interactions of the fundamental particles. 
Secondly the amount of \ac{cp} violation predicted by the \ac{sm} is not 
sufficient to account for the observed matter dominance in the universe.

There are also problems in the neutrino sector where the \ac{sm} predicts that 
the neutrino is massless, however due to the observation of flavour changing 
the neutrino mass must be non-zero.
The most glaring short coming other than the lack of a description of 
gravitational dynamics is the lack of a dark matter candidate. This dark matter 
candidate needs to be heavy, stable, chargeless, colourless and interact via 
the weak nuclear force only. Such particles are predicted by 
\ac{susy}\cite{ref:SUSY-1, ref:SUSY0, ref:SUSY1, ref:SUSY2, ref:SUSY3, 
ref:SUSY4, ref:hierarchy1, ref:hierarchy2} 
models, which also provide a description of gravity. These models insert a 
further broken symmetry between bosons and fermions, i.e. for each \ac{sm} 
boson there is a \ac{susy} fermion and visa versa. In the case of the 
\ac{cmssm} it is the inclusion of gravity that breaks the \ac{susy} symmetry 
giving rise to the mass difference between the super partners and their \ac{sm} 
counterparts. 
The formalisation of \ac{susy} allows a work around for the constraints imposed 
by the Coleman-Mandula\cite{coleman-mandula} theorem, which states that the 
space-time and internal symmetries of a quantum field theory can only be 
combined trivially. This is due to the ``supercharges'' introduced by \ac{susy}
theories being spinor doublets rather than scalar charges. This allows for the 
extra symmetries introduced by such models.

% section beyond_the_standard_mode_ (end)

\subsection{The Constrained Minimal Super Symmetric Model} % (fold)
\label{sec:the_constrained_minimal_super_symmetric_model}
Whilst there are a plethora of \ac{susy} models, in order to compare the 
\ac{cms} searches against previous and contemporary experiments the 
\ac{cmssm}\cite{ref:CMSSM} is chosen as a bench mark, however there are 
compelling reasons to study the \ac{cmssm} it's self. \ac{susy} introduces 105 
new parameters to the existing 19 of the \ac{sm}. Sampling a space 
this large with sufficient coverage is prohibitive, instead the \ac{cmssm} 
reduces the number of free parameters to 5, these are: $tan \beta$, where 
$\beta$ is the ratio of the vacuum expectation values for the two Higgs fields; 
$m_{0}$ and $m_{1/2}$ which are the mass values for the boson and fermion 
states at the unification scale; $A_{0}$ which is the \ac{susy} breaking 
tri-linear coupling; finally the sign of $\mu$ which is the Higgs breaking 
parameter. This is achieved by: enforcing the requirement for `R' parity, the  
total number of \ac{susy} and \ac{sm} particles at a vertex, to be conserved; 
Requiring that the two Higgs doublets produce electroweak symmetry breaking; 
The gauge couplings are required to unify; The masses of the gauginos and 
scalars unify at the \ac{gut} scale; Limits from experimental searches are also 
required to be satisfied, these include the requirement for the LSP to be a 
dark-matter candidate and that this LSP is both colour and electric charge 
neutral, the dark matter relic density is inline with astronomical observations 
and is inline with the measured age of the universe. Finally it is required 
that the theory does not need fine tuning of the particle masses to produce the 
current observed physics in the universe and to produce a stable theory.
Given these physical constraints the \ac{cmssm} is a sensible model to 
interpret physical results from the \ac{lhc} in terms of.


Throughout the accessible regions of these values, both Electroweak and 
strong production are predicted, as well as a multitude of final states 
involving a stable dark matter candidate particle $\chi_{0}$ and \ac{sm} 
particles. For the interpretation of the analysis presented in this thesis, 
limits are set as a function of $m_{0}$ and $m_{1/2}$ with fixed values for the 
other variables of $tan \beta = 10$, $A_{0} = 0$ and $\mu > 0$.
Before \ac{lhc} start up the preferred values for $m_{0}$ and $m_{1/2}$ as 
given by \cite{Buchmueller:2011tr} were well defined and within reach of the 
early \ac{lhc} \ac{susy} searches.
Table~\ref{tab:listofSUSYParticles} lists the \ac{sm} particles and their 
\ac{susy} partners, it is to be noted that there is not a one to one 
correspondence between the particles as the observable states of the \ac{susy} 
particles are mixed states of the directly symmetrized \ac{sm} particles. Also 
\ac{susy} introduces a second Higgs doublet, whilst these particles are not 
super partners of the \ac{sm} particles they are added in addition to the 
\ac{sm} Higgs boson.

\begin{table}
\caption{List of the SM particles and their super partners. The neutralinos are 
the mixed state of the super partners of the chargeless SM bosons, note that 
there are now two Higgs doublets. The charginos are a mixed state involving the 
super partner of the charged \PW bosons and the charged Higgs boson 
(\PHp) and the neutralinos are a mixed state of the neutral \ac{sm} bosons \cite{ref:CMSSM}.}
\label{tab:listofSUSYParticles}
\begin{tabular}{c|c|c|c}
\hline
Partners & Super Partners & Spin & Charge\\
\hline
\Pphoton, \PZz \PHz, \Phz   & \HepSusyParticle{\chi}{0}{0}, \PSgxzi, \PSgxzii, \HepSusyParticle{\chi}{3}{0} & 1/2 & 0 \\
\PW, \PHp &                 \HepSusyParticle{\chi}{0}{+},\HepSusyParticle{\chi}{1}{+} & 1/2 & $\pm$1 \\
\hline
\Pe, \Pgnl, \Pgm, \Pgngm, \Pgngt & \HepSusyParticle{\Pe}{\bar{R}}{}, \HepSusyParticle{\Pe}{\bar{L}}{}, \HepSusyParticle{\nu}{\Pe}{}, \HepSusyParticle{\mu}{\bar{R}}{}, \HepSusyParticle{\mu}{\bar{L}}{}, \HepSusyParticle{\nu}{\Pgm}{}, \HepSusyParticle{\nu}{\Ptau}{}& 1 & $\pm$ 1, 0 \\

\Ptau & \HepSusyParticle{\Ptau}{0}{}, \HepSusyParticle{\Ptau}{1}{} & 1 & $\pm$ 1 \\
\hline
\Pup, \Pdown, \Pcharm, \Pstrange & \HepSusyParticle{u}{R}{}, \HepSusyParticle{u}{L}{}, \HepSusyParticle{d}{R}{}, \HepSusyParticle{d}{L}{}, \HepSusyParticle{c}{R}{}, \HepSusyParticle{c}{L}{}, \HepSusyParticle{s}{R}{}, \HepSusyParticle{s}{L}{} & 1 & $\pm$ 1/3, $\pm$ 2/3 \\

\Pbottom & \HepSusyParticle{b}{0}{}, \HepSusyParticle{b}{1}{} & 1 &  $\pm$ 1/3, $\pm$ 2/3 \\
\Ptop & \HepSusyParticle{t}{0}{}, \HepSusyParticle{t}{1}{} & 1 &  $\pm$ 1/3, $\pm$ 2/3 \\
\hline
\Pg & \PSg & 1/2 & 0 \\
\hline
\end{tabular}
\end{table}


% section the_constrained_minimal_super_symmetric_model (end)

\subsection{Simplified Models} % (fold)
\label{sub:simplified_models}
The \ac{sms} models contain only one production process and a set decay 
topology, making the interpretation in these models simpler, however the 
individual models are not representative of some complete \ac{susy} model. 
These are presented as an alternative to searching for a specific beyond the 
\ac{sm} Lagrangian and it's associated physical manifestation. The 
approach is is to  instead build a set of self consistent models of new 
physics, which are characterised by production and decay 
topologies\cite{ArkaniHamed:2007fw}.
Most generally this characterisation is based on the type of the pair produced 
particle (quark like or gluon like) and the mass splitting between this parent 
particle and the final state weekly interacting massive particle which provides 
a dark matter candidate.
This method has several advantages over searching for a specific model:
\begin{itemize}
\item If a discrepancy with the \ac{sm} is found, the amount of data collected  
during early running of the \ac{lhc} will be insufficient to confirm or rule 
out most specific models of new physics;
\item Generalised limits on decay topologies can be applied to many models and 
are useful for guiding model building;
\item Difficult decay topologies such as those with small mass 
splittings can be investigated and analyses tuned to those areas of kinematic 
phase space.
\end{itemize}



\begin{figure}[h!]
    \centering
    \subfigure[Production and decay of the \texttt{T1} simplified model with the process \HepProcess{\Pproton\Pproton\to\PSgluino\PSgluino\to\Pquark\APquark\PSneutralino\Pquark\APquark\PSneutralino}]{
          \label{fig:T1feyn}
          \includegraphics[width=0.45\textwidth]{figures/Theory/T1.pdf}
     }
    \subfigure[Production and decay of the \texttt{T2} simplified model with the process \HepProcess{\Pproton\Pproton\to\Psquark\Psquark\to\Pquark\PSneutralino\APquark\PSneutralino}]{
          \label{fig:T2feyn}
          \includegraphics[width=0.45\textwidth]{figures/Theory/T2.pdf}
     }
    \caption{Simplified model production and decay diagrams.}
    \label{fig:T1T2feyn}
\end{figure}



The models considered by this analysis require fully hadronic final states. Two 
production topologies are considered. Gluino-gluino production which is 
pictured in Figure~\ref{fig:T1feyn} is referred to as \texttt{T1} type models, 
where any suffix labels the flavour that the final state particles are forced 
to. \texttt{T1} decays to four light (\Pup,\Pdown or \Pcharm) quarks and two 
\PSneutralino particles.
Squark-squark production, pictured in Figure~\ref{fig:T2feyn} is referred to as 
\texttt{T2} type production, again 
any suffix denotes the flavour of the final state quarks. The decay topology 
involves two \PSneutralino particles and two final state quarks which hadronise 
to form jets.

% subsection simplified_models (end)

\section{SUSY Searches at the LHC} % (fold)
\label{cha:motivating_the_alt_analysis}

As seen above, the particles produced by new physics models that provide a dark 
matter candidate should leave a signature of a large amount of undetected 
energy, in proton-proton collisions the initial momentum of each of the 
colliding partons is not known as the proton is a composite object. However it 
is known that the momentum in the plane transverse to the direction of the 
proton beams (\pt) is zero. The final states that distinguish this new 
predicted physics from \ac{sm} processes involve large amount of missing energy 
projected on to the transverse plane \ETm. These events also involve the 
production of \ac{sm} particles in the decay chain. The final states that are 
searched for in this analysis involve hadronic jets and missing energy.

However there are \ac{sm} processes that produce signatures with missing energy 
in association with hadronic jets. These background sources need to be 
predicted or measured from \ac{sm} processes, these predictions are then 
compared to the observed number of events in each of the signal regions, the 
compatibility of these results with the \ac{sm} allows us to discover or rule 
out new physics models.

There are theoretical uncertainties introduced by the modelling of \ac{sm} 
processes and their observable features at detector level. These apply equally 
to simulated beyond the \ac{sm} processes. The first of these stems from the 
nature of the proton-proton collisions at the \ac{lhc}, the other from the 
complexity of perturbative calculations, due to the self coupling of the gluon, 
on low energy QCD systems which occur during the formation of jets from single 
quarks or gluons due to colour confinement, this process is known as 
hadronization.

\subsection{Parton Density Functions of the Proton} % (fold)
\label{sub:parton_density_functions_of_the_proton}

\begin{figure}[htbp]
  \centering
    \includegraphics[width=0.75\textwidth]{figures/Theory/mstw2008nlo68cl_allpdfs.pdf}
  \caption{MSTW 2008 NLO PDFs at $Q^{2} = 10$ GeV$^{2}$ and $Q^{2} = 10^{4}$ GeV$^{2}$\cite{MSTW}. Where x is the fraction of the momentum carried by that particle.}
  \label{fig:mstwpdfs}
\end{figure}

The proton is a composite particle at it's simplest it is a combination of 
three quarks, two up type and one down type, each of these carry a third of the 
protons mass-energy. However this picture is overly simple. As seen from 
Table~\ref{tab:smparticles} the mass of the constituent quarks sums to 
$\approx$\unit{10}{\MeV}, however the mass of the proton is 
$\approx$\unit{1}{\GeV}, the majority of the mass comes from the binding forces 
inside the proton and from the virtual particles which arise from the dynamics 
of the strong force and quantum mechanics.

This has several implications, the first of which is that it is impossible to 
know which particles have interacted when a collision occurs and what fraction 
of the protons total momentum was being carried by that particle, hence the 
total energy of the system is unknown. However, it is an accurate approximation 
to assume that the energy in the plane transverse to the direction of travel is 
zero, hence at hadron colliders all observables are measured in terms of the 
observed energy deposited in the plane transverse to the beam direction.


The second major implication is that at different interaction energies the the 
components of the proton change, as shown in Figure~\ref{fig:mstwpdfs}, where x 
is the fraction of the proton's momentum carried by that particle. These 
\ac{pdf} have not been measured at very high Q$^{2}$, as seen at the 
\ac{lhc}. However, they have been extrapolated from the measurements at lower 
energies. When the production cross section depends on the energy of the 
interacting particles the uncertainty has a large effect on the calculation of 
the production cross sections of new physics processes.






% subsection parton_density_functions_of_the_proton (end)
\subsection{Hadronization Models} % (fold)
\label{sub:hadronization_models}
The ``true'' fragmentation and formation of colour neutral hadrons from single 
quarks or gluons is not yet understood from first principles, starting with the 
QCD Lagrangian. However, effective models have been produced that do a 
reasonable job of producing colour neutral final states for simulated Monte 
Carlo events. There are two general approaches, the first is used by 
Pythia\cite{pythia}, this method uses string fragmentation, which runs 
iteratively over the remaining free particles to produce a colour neutral final 
state. Madgraph\cite{madgraph} uses a matrix element method to perform the 
hadronization process. The analysis uses samples generated by both Magraph and 
Pythia to test any difference in the observed results. No major difference is 
observed due to the high \ET requirements on the jets in the final event 
selection.
% subsection hadronization_models (end)


% chapter theory (end)