\chapter{Level One Calorimeter Trigger} % (fold)
\label{cha:level_one_trigger}
\begin{figure}[ht]
  \centering
    \includegraphics[width=0.8\textwidth]{generated/LoneTrigger/level1trigger.pdf}
  \caption{The CMS \Lone Trigger system}
  \label{fig:figures_LoneTrigger_level1trigger}
\end{figure}

The CMS \Lone trigger system\cite{l1} is a pipelined dead-timeless system based 
on custom-built electronics.
The \Lone trigger is a combination of several sub systems, which are 
interconnected as depicted in 
Figure~\ref{fig:figures_LoneTrigger_level1trigger}.

Corse grain information from the electro-magnetic, hadronic and forward 
calorimeters is processed by the Regional Calorimeter Trigger (RCT), this is 
then passed to the Global Calorimeter Trigger (GCT) where the corse grain 
information is clustered in to physics objects, these objects are then passed 
to the Global Trigger where the \Lone accept decision is made. Due to the 
limited size of the pipe line this \Lone accept must be issued within 4.0 
$\mu$s.

The objects passed from the GCT to the GT include electro-magnetic objects, 
both electrons and photons as due to the lack of
tracking information at the \Lone trigger these objects are indistinguishable, 
jets and energy sums.

The RCT generates up to 72 isolated and non-isolated electro-magnetic objects, 
these are sorted by rank, which is equivalent to
transverse energy \ET. The four highest ranked electro-magnetic objects are 
then passed via the GCT to the GT at an equivalent data 
rate of 29 \Gbs per type.

Hadronic objects under go two clustering steps. First the transverse energy 
sums of the ECAL and corresponding HCAL towers are
calculated, the towers are then summed in to 4$\times$4 trigger regions, these 
are passed to the GCT at a data rate of 172.8 \Gbs.
These trigger regions are clustered in to jet candidates by the GCT and ranked. 
The jets are then sub-divided in the 
categories depending on their pseudo-rapidity and the result of $\tau$ 
identification. 

Energy sums come in two forms, the total transverse energy \ET which is the 
scalar sum of all transverse energies and the total jet transverse energy \HT 
which is calculated as the scalar sum of all jets above some programable 
threshold.

The missing energy equivalents of these \MET and \MHT are formed from the 
negative vector sum of the objects considered for the
transverse sums.



\section{Leve-1 Trigger Jet Algorithm} % (fold)
\label{sec:leve_1_trigger_jet_algorithm}

\textbf{FIXME: This is taken pretty much straight from \cite{gctcomm} might 
want to steal less??} 

The CMS detector can be un-rolled in the $\phi$ direction to form a rectangular 
grid of the 396 calorimeter regions, connected along the $\phi$ edge. The
rectangle is formed from 18 $\phi$ divisions (from $-180^{\circ} < \phi \leq 180^{\circ}$) and 22 $\eta$ divisions ( from $-5 < \eta < 5$). Each $\phi$ division corresponds to 20$^{\circ}$. The $\eta$ divisions correspond to $\Delta\eta$ $= 0.5$ in the forward calorimeters and to $\Delta\eta \approx 0.348$ in the barrel. A pictorial representation of this can be seen in 
figure~\ref{fig:figures_LoneTrigger_jetfindermappings}.

\begin{figure}[ht]
  \centering
    \includegraphics[width=0.8\textwidth]{figures/LoneTrigger/level1jetalgo.jpg}
  \caption{The 3 $\times$ 3 jet-finder window at \Lone. Each cell represents a 
  trigger tower, which is the sum of the HCAL and ECAL transverse energies. The 
  $\tau$-jet veto patterns are shown to the right.}
  \label{fig:figures_LoneTrigger_level1jetalgo}
\end{figure}


A jet candidate is created if the sum of the ECAL and HCAL energies of the 
central calorimeter region has an energy deposit larger than all of its 
neighbours, as shown in figure~\ref{fig:figures_LoneTrigger_level1jetalgo}
The jet is centered at this region where $p_{T}^{central} > p_{T}^{surrounding}$
and the transverse energies of the surrounding regions are summed in to the 
central region. The jet is then classified as a $\tau$ jet if \mETA $< 3.0$ and 
none of the $\tau$ veto bits are set. If any $\tau$ vetoes are set the jet is 
classified as a central jet. The jet is classified as forward if $ 3.0 < \mETA 
< 5.0$

The $\tau$-vetoes are set by the RCT depending on whether or not the energy 
depositions in up to four contiguous trigger towers are below a programmable 
fraction of the regional \ET as shown in 
Figure~\ref{fig:figures_LoneTrigger_level1jetalgo}.

It is possible to apply separate jet energy corrections to each of the sub 
categories of GCT jets, however at current the same \ET and $\eta$ dependant corrections are used for all three jet types.  These corrections are discussed in detail in 
Section~\ref{sec:lone_trigger_performance}

\begin{figure}[ht]
  \centering    
  \includegraphics[width=0.8\textwidth]{figures/LoneTrigger/jetfindermappings.png}
  \caption{The calorimeter map that the 3 $\times$ 3 jet-finder operates over
   is made up for 396 calorimeter regions, each jet finder is mapped on to an 
   RCT crate which is composed of an 11 $\times$ 2 strip of these regions. RCT
   crate labels are shown for negative $\eta$ only. }
  \label{fig:figures_LoneTrigger_jetfindermappings}
\end{figure}

In order to reduce the total data duplicated and shared between the jet finders
the GCT employs a pre-clustering algorithm, which involves 18 jet finders 
operating simultaneously over the whole detector. These jet finders then only
share information with neighbouring regions when the clustered jets are found.
Figure~\ref{fig:figures_LoneTrigger_jetfindermappings} shows the boundaries 
between which the jet finders operate, these map naturally on to one RCT crate 
per jet finder. A maximum of 3 jets can be found on each of the $\phi$ strips
acted on by the jet finders, this gives a maximum of 108 jets per event. In
order to preserve continuity across the $\eta = 0$ boundary, the two adjacent
trigger regions are shared between the jet finders.


\begin{figure}[ht]
  \centering
 \includegraphics[width=0.8\textwidth]{figures/LoneTrigger/jetfinderfunction.png}
  \caption{The \Lone jet clustering method, six cells in $\eta$ are shown. An
  example of overlapping jets is shown}
  \label{fig:figures_LoneTrigger_jetfinderfunction}
\end{figure}


An example of the jet finding is shown in
Figure~\ref{fig:figures_LoneTrigger_jetfinderfunction}. The first step is to 
create a $2 \times 3$ mini cluster around any local maxima found in the $12 
\times 2$ strip. Equality statements are imposed so that the central cell is 
greater than its neighbours in some directions and greater than or equal to the 
neighbours other directions to enforce a gap of at least one trigger region in 
both $\eta$ and $\phi$ between the centres of the clustered jets.

In the second step the jet finder transfers the three largest mini clusters on 
a given $\phi$ strip to the closest $\phi$ strip on the neighbouring jet finder.
These are then compared against the existing mini clusters in that $\phi$ strip,
those that are adjacent or diagonally adjacent to a larger mini cluster are 
removed. The inequalities statements are then reimposed to prevent problems
with clusters having the same energies. In the final stages the mini clusters
have their three adjacent regions summed in to produce a $3 \times 3$ jet
cluster. Finally the four highest ranked jets are corrected and passed to the 
GT.
% section leve_1_trigger_jet_algorithm (end)

\section{\Lone Trigger Performance} % (fold)

During the start of data taking in 2010, no Jet Energy Corrections (JEC's) were applied in the \Lone trigger. This gave a relatively slow turn on in therms of offline hadronic objects.
During the winter shut down of the LHC between the 2010 and 2011 running periods a set of \Lone JEC's were developed. These corrections used a peicewise cubic form for the interpolation function used to correct the jet energy dependant on it's uncorrected \ET and $\eta$ values. However as can be seen in
Figure~\ref{fig:figures_LoneTrigger_EnCorrelationNormalized} these corrections were only applied to jets with a raw energy below 130 \GeV.

\begin{figure}[htbp]
  \centering
    \includegraphics[width=0.8\textwidth]{figures/LoneTrigger/EnCorrelationNormalized.pdf}
  \caption{Correlation between offline corrected jet energy and \Lone corrected jet energy for matched jets. The discontinuity shows where the \Lone jet corrections do not alter the raw energy of the jet.}
  \label{fig:figures_LoneTrigger_EnCorrelationNormalized}
\end{figure}

To overcome this a new set of corrections were derived using a well established tool for producing offline corrections, \textbf{REFERENCE TO tapper-001 here} using the same functional form that was derived for correcting particle flow jets. \textbf{REFERENCE TO PF HERE}
In this section we discuss the performance of both sets of \Lone JEC's and the performance of the energy sum trigger \HT, \HTm, and \MET, the performance of which are not effected by the application of jet energy corrections at the \Lone trigger due to the quantities being built from the internal GCT jets before they pass though the corrections look up table.
The performance is studied under both low pile up conditions where the mean peak pile up $<PU>$ is 16 primary vertices and under high pile up conditions where $<PU>$ is 36 primary vertices.

To measure the performance of the \Lone single jet triggers we assume that the leading offline corrected \AK calorimeter jet is the jet that triggered the event. We then match this offline jet to the closest \Lone jet in $\Delta$R, where for there to be a match $\Delta$R $< 0.5$ is required. For this match central, $\tau$ and forward jets are considered. Events where the recorded \Lone energy is set to the overflow bit, meaning they have more than 254 \GeV of \ET measured at \Lone are ignored.

To collected an unbiased sample in which to measure the performance, two methods are used; the first is to require a Minimum Bias trigger, which is triggered by beam induced activity in the CMS detector.
However due to the nature of these events the number of interactions with high energy interactions is low and the prescale applied to this trigger further reduces the sample size. However this method does produce the least bias.
The second method is to trigger an object that does not deposit signifiant energy in the calorimeter systems, in this case we choose the trigger with the lowest un prescaled \PT threshold. The muon trigger is chosen with some loose isolation requirements to make sure it does not overlap with a jet, causing a discrepancy in the measure of the calorimetric energy. The sample has a higher number of events due to the large amount of bandwidth given to the single object muon triggers in CMS. The use of a muon trigger also serves to increase the precision of the measurement of the \Lone missing energy trigger as the muons are not seen by the calorimeter system the \MET sample is enriched.

\begin{figure}[htbp]
  \centering
    \includegraphics[width=0.8\textwidth]{figures/LoneTrigger/cf_correctionScheemes.pdf}
  \caption{Comparison of the performance of L1$\_$SingleJet16, L1$\_$SingleJet36 and L1$\_$SingleJet92, when using the piecewise cubic corrections and using the new correction scheme. The performance of the two is negligible above 36\GeV.}
  \label{fig:figures_LoneTrigger_cf_correctionScheemes}
\end{figure}

Figure~\ref{fig:figures_LoneTrigger_cf_correctionScheemes} shows the performance of the piecewise cubic corrections (PWC) and the performance of the new corrections. The data was taken with the PWC enabled in the GCT hardware. The updated corrections were emulated in the bitwise reproduction of the GCT. The made an event by event comparison possible. At low \ET the new corrections turn on before the PWC corrections, if the new corrections were applied on with no change to the trigger menu, the \Lone trigger rate would rise. At a threshold of 36 \GeV and higher the performance of the two correction schemes is very similar. Due to the small change in observed performance and the ability to correct raw energies above 130 \GeV, the new corrections were deployed online at the start of Run2011B and are still online at the end of data taking in 2012.

\begin{figure}[htbp]
  \centering
    \includegraphics[width=0.8\textwidth]{figures/LoneTrigger/SingleMu_JetTurnOns.pdf}
  \caption{Performance measurements of L1$\_$SingleJet16, L1$\_$SingleJet36 and L1$\_$SingleJet92, when using the new correction scheme deployed in the GCT hardware. The performance is slightly worse than that of the emulated triggers due to a change in pile up conditions between the two data taking periods.}
  \label{fig:figures_LoneTrigger_SingleMu_JetTurnOns}
\end{figure}

The performance of the updated corrections was then measured with data taken with the corrections applied in the GCT hardware. The reference sample was taken with the  \verb|HLT_IsoMu24_v*| high level muon trigger.
The performance of three example triggers is show in Figure~\ref{fig:figures_LoneTrigger_SingleMu_JetTurnOns}, these curves are fitted with an error function, which is not a perfect description of the turn on, but does give some guiding figures as to the performance of the triggers. The data collected and represented in Figure~\ref{fig:figures_LoneTrigger_SingleMu_JetTurnOns} has a peak mean pile up ($<PU>$) of 16 interactions, this is higher than the $<PU>$ of approximately 8-10 which were the running conditions in Run2011A, on which the previous comparison was performed.
The observed difference in the performance of the \Lone single jet triggers as a function of pile up is a case of concern when data taking is underway at the LHC's design luminosity, where $<PU>$ is $>$36 at an instantaneous luminosity of 1$\times 10 ^{34} cm^{-27}s^{-1}$.

The instantaneous luminosity in 2012 was predicted to be 5$\times 10 ^{33} cm^{-27}s^{-1}$, with $<PU> \approx$ 32. In order to study the effect on the trigger rate and efficiency a high pile up, low instantaneous luminosity, LHC fill was taken in 2011.

The \Lone single jet performance was studied in this run in terms of two offline object definitions. The first was the standard \AK calorimeter jet reconstruction, the second was a set of \AK calorimeter jets which were corrected for pile up using the fastjet correction algorithm, \textbf{cite fast jet} which is further detailed in Section~\ref{cha:offline_object_deffinitions}.
The fast-jet corrections remove the energy deposited in the secondary interactions from the objects which are expected to come from the primary hard interaction, thus removing energy from the offline jets.
The effect of these pile up corrections on the \Lone trigger performance is first studied under conditions with $<PU>$ of 16, the performance of which has already been measured with respect to non pile up corrected offline objects, as a sanity check. The results are shown in Figure~\ref{fig:figures_LoneTrigger_LowPUFastCompare} , the performance is measured with respect to \verb|HLT_IsoMu24_v*|, in terms of both pile up corrected and standard offline objects. As expected the performance in the two cases is very similar. The same comparison is shown for \HT in Figure~\ref{fig:figures_LoneTrigger_LowPUFastCompareHT}, where the effect of the fastjet corrections is more pronounced due to the sum over jets.
The difference between the turn on points for the two offline quantities is on the order of 10 \GeV under low pile up conditions.

Due to the high pile up fill being a specialised fill with low instantaneous luminosity, the high level trigger paths were disabled, instead \Lone trigger pass though paths were utilised to take the data. The \Lone single muon pass though trigger is used to collect the reference sample. Otherwise the same analysis method is common between the two data sets.
Figure~\ref{fig:figures_LoneTrigger_HighFastCompare} shows the difference in turn on for three example \Lone single jet triggers when using standard calorimeter jets and fastjet corrected calorimeter jets.
In the high pile up conditions the switch to offline jets that are corrected for pile up shifts the turn on point to lower vales of \ET, the magnitude of this effect reduces as the \Lone trigger trigger threshold raises. This implies that the same offline performance as seen in the low pile up conditions can be achieved by using the pile up corrected offline objects and raising the \Lone single jet trigger thresholds.

Figure~\ref{fig:figures_LoneTrigger_HighPUFastCompareHT} shows the same high pile up comparison, but for the \Lone \HT triggers. Due to the size of the sample the precision of this measurement is low. However the same trend of a shift to lower \HT values of the turn on point of the \Lone triggers when using pile up corrected offline objects is observed. This again implies that the \Lone \HT trigger thresholds can be raised whilst preserving the same offline performance as during the low pile up conditions.

\begin{figure}[htbp]
  \centering
    \includegraphics[width=0.8\textwidth]{figures/LoneTrigger/LowPUFastCompare.pdf}
  \caption{Comparison of the performance of  L1$\_$SingleJet16, L1$\_$SingleJet36 and  L1$\_$SingleJet92 triggers. Where $<PU>$ = 16. For two offline reconstruction methods, standard \AK calorimeter jets and pile up corrected \AK calorimeter jets.}
  \label{fig:figures_LoneTrigger_LowPUFastCompare}
\end{figure}


\begin{figure}[htbp]
  \centering
    \includegraphics[width=0.8\textwidth]{figures/LoneTrigger/LowPUFastCompareHT.pdf}
  \caption{Comparison of the performance of L1$\_$HTT75, L1$\_$HTT100 and  L1$\_$HTT150 triggers. Where $<PU>$ = 16. For two offline reconstruction methods, standard \AK calorimeter jets and pile up corrected \AK calorimeter jets.}
  \label{fig:figures_LoneTrigger_LowPUFastCompareHT}
\end{figure}


\begin{figure}[htbp]
  \centering
    \includegraphics[width=0.8\textwidth]{figures/LoneTrigger/HighFastCompare.pdf}
  \caption{Comparison of the performance of L1$\_$SingleJet16, L1$\_$SingleJet36 and L1$\_$SingleJet92 triggers. Where $<PU>$ = 36. For two offline reconstruction methods, standard \AK calorimeter jets and pile up corrected \AK calorimeter jets.}
  \label{fig:figures_LoneTrigger_HighFastCompare}
\end{figure}

\begin{figure}[htbp]
  \centering
    \includegraphics[width=0.8\textwidth]{figures/LoneTrigger/HighPUFastCompareHT.pdf}
  \caption{Comparison of the performance of  L1$\_$HTT75, L1$\_$HTT100 and  L1$\_$HTT150 triggers. Where $<PU>$ = 36. For two offline reconstruction methods, standard \AK calorimeter jets and pile up corrected \AK calorimeter jets.}
  \label{fig:figures_LoneTrigger_HighPUFastCompareHT}
\end{figure}

% section lone_trigger_performance (end)
\clearpage
\section{\Lone Trigger Pile-up Mitigation} % (fold)
\label{sub:lone_trigger_pile_up_mitigation}


Whilst we have seen that the offline performance of the \Lone hadronic triggers 
can be maintained when raising the trigger thresholds to deal with increased 
rate, when switching to pile up corrected offline objects. 
Figure~\ref{fig:figures_HTT150_pileup} shows the trigger cross section as a 
function of instantaneous luminosity for the \verb|L1_HTT150| trigger, which 
requires \HT $>$ 150 \GeV. However beyond a certain point raising thresholds 
causes a loss of performance.
In this section we look at a method to reduce the effects of pile up hadronic \Lone triggers, by making an addition to the \Lone jet finding algorithm.

In Section~\ref{sec:leve_1_trigger_jet_algorithm} the \Lone jet clustering 
algorithm was described. The proposed change was to add a requirement that the 
seeding region has a direct energy thresholds, in addition to the equality 
relations that are set up. The effects of applying a 2 \GeV or a 5 \GeV 
threshold are studied. This threshold is on the raw, uncorrected energy of the 
trigger towers and effects all \Lone jets. The impact will be seen in the \Lone 
jet triggers which use corrected energy and \Lone \HT and \HTm which are formed 
from uncorrected jets.
The aim is to remove the events from pile up, but not to remove physics events.

The triggers most effected by this change are the energy sum triggers as they 
sum many jets of low threshold, where as the single object triggers are already 
cutting on high \ET objects.

Figure~\ref{fig:GCTrankRAW} shows the internal GCT uncorrected jet energy spectrum in high pile up conditions, taken with the \verb|L1_SingleMu| pass though triggers. The three histograms are for; no application of jet seed threshold in black, where there are many low \ET jets; In red a 2 \GeV seed requirement is made, the effect is to cut out all jets below 2 \GeV and cut out jets with an energy up to approximately 35 \GeV of uncorrected energy; The blue histogram shows the jet energy spectrum after applying a 5 \GeV seed threshold, the effect is to remove all jets below 5 \GeV and to cut out jets with energy up to 55 \GeV. Figure~\ref{fig:GCTrankRatio} shows the efficiency with respect to the no seed sample for the two test seed thresholds. The removal of jets in the low energy region of the \ET spectrum is where the advantage of applying a seed threshold is seen over simply raising the trigger thresholds, or raising the threshold of jets to be included in the \Lone \HT calculation.


\begin{figure}[h!]
    \centering
    \subfigure[GCT internal uncorrected jet \ET distributions for the same 
    events with a 0, 2 or 5 \GeV seed requirement.]{
          \label{fig:GCTrankRAW}
          \includegraphics[width=0.45\textwidth]{figures/LoneTrigger/GCT_Jet_Rank_highPU.pdf}
     }
    \subfigure[Efficiency of applying a requirement of 2 or 5 \GeV with respect 
    to no requirement.]{
          \label{fig:GCTrankRatio}
          \includegraphics[width=0.45\textwidth]{figures/LoneTrigger/GCT_Jet_Rank_highPU_ratio.pdf}
     }
    \caption{Effect of requiring a jet seed threshold on GCT internal jets.}
    \label{fig:GCTrank}
\end{figure}



To quantify the effects of the addition of the jet seed a low pile up sample, where the effects are expected to be small, is studied in terms of rate reduction and efficiency change. The dedicated high pile up fill is then studied in terms of rate reduction, due to the limited sample size of the high pile up fill the change in efficiency on this sample is not studied. However due to the addition of energy from the secondary pile up interactions the change in efficiency in the low pile up sample is the worse case scenario.

Table~\ref{tab:lowpuratereduction} details the rate reduction with respect to the 0 \GeV seed threshold for seed thresholds of 2 \GeV and 5 \GeV for three example triggers, these are:

\begin{itemize}
\item \verb|L1_SingleJet50|, which requires at least one jet with \ET $>$ 50 \GeV with in |$\eta$| $<$ 3.0;
\item \verb|L1_QuadJet38|, which requires 4 jets with \ET $>$ 38 \GeV with in |$\eta$| < 3.0;
\item \verb|L1_HTT100|, which requires that  \Lone \HT $>$ 100 \GeV.
\end{itemize}

The rate of \verb|L1_SingleJet50| is not effected by the requirement of a 2 \GeV seed threshold and is reduced by 15$\%$ when a 5 \GeV seed requirement is made.
The \verb|L1_QuadJet38| trigger rate is reduced by the same amount as the single jet trigger, under low pile up conditions for both seed thresholds. \verb|L1_HTT100| seeds a 2$\%$ rate reduction when requiring a 2 \GeV seed threshold and a 3$\%$ reduction in rate when requiring a 5 \GeV seed.

\begin{table}
\caption{Summary of rate reduction during low pile up conditions.}
\begin{tabular}{c|c|c}

\hline
Trigger & $\%$ of rate taken with 2\GeV requirement & $\%$ of rate taken with 5\GeV requirement\\
\hline
L1$\_$HTT100 & $98.6 \pm 11.6\%$ & $97.9 \pm 11.6\%$\\
\hline
L1$\_$QuadJet38 & $100.0 \pm 0.0\%$ & $85.3 + 6.2 - 8.7\%$\\
\hline
L1$\_$Jet50 & $100.0 + 0.0 - 12.3\%$ & $85.7 + 9.1 - 15.8\%$\\
\hline
\end{tabular}
\label{tab:lowpuratereduction}
\end{table}



\begin{table}
\caption{Summary of rate reduction during high pile up conditions.}
  
\begin{tabular}{c|c|c}
\hline
Trigger & $\%$ of rate taken with 2\GeV requirement & $\%$ of rate taken with
 5\GeV requirement\\
\hline
L1$\_$HTT100 & $60.4 \pm 5.7\%$ & $0.67 +/- 0.67\%$\\
\hline
L1$\_$QuadJet38 & $71.4 + 18.2 - 25.9\%$ & $57.1 + 22.3 - 24.8\%$\\
\hline
L1$\_$Jet50 & $100.0 + 0.0 - 7.7\%$ & $73.9 + 9.8 - 12.3\%$\\
\hline

\end{tabular}
\label{tab:highpuratereduction}
\end{table}



% subsection High Pile Up (end)










% section Effects on Rate (end)
\subsection{Effect on trigger efficiency} % (fold)
\label{sec:Effects of requiring a jet seed on offline efficiency}
Section~\ref{sec:Effects on Rate} shows that requiring a jet seed threshold
substantially reduces the trigger acceptance rate at in high pile up conditions.

However the aim of requiring a jet seed is to reduce rate, but not at the cost 
of physics. In this section we look at the effects of requiring a seed 
threshold, whilst requiring some loose, generic offline selection on the 
hadronic objects.

The change in efficiency is measured under low pile up conditions where the 
least extra energy added to the event. This gives a worse case estimate of the 
effect of requiring a jet seed on the offline efficiency.

Each offline reconstructed calorimeter jet must adhere to the following quality 
criteria:
\begin{itemize}
\item Pass loose calorimeter ID 
\item \PT $\geq$ 30 \GeV.
\item $|\eta| \leq$ 3.0.
\item Matched to a \Lone jet with $\Delta R \leq 0.5$.
\end{itemize}
Where loose calorimeter ID is defined as; Electro-Magnetic fraction $> 0.01$, 
fraction of energy in the Hybrid Photo Diodes $< 
0.98$ and the number of n90hits $> 1$.

\begin{figure}[h!]
    \centering
    \subfigure[]{
          \label{fig:HTlowPURECO}
          \includegraphics[width=0.46\textwidth]{figures/LoneTrigger/HT_LowPU_RECO.pdf}
     }
    \subfigure[]{
          \label{fig:cenJetLowPURECO}
          \includegraphics[width=0.46\textwidth]{figures/LoneTrigger/Cen_Tau_LowPU_RECO.pdf}
     }
     \newline
    \subfigure[]{
          \label{fig:quadJetLowPURECO}
          \includegraphics[width=0.46\textwidth]{figures/LoneTrigger/QuadJet_38Cen_LowPU_RECO.pdf}
     }
    \caption{Efficiency reductions for various \Lone algorithms when applying a 
    2 or 5 \GeV seed tower requirement, in low pile up 
    conditions. Figure (a) shows the efficiency reduction for \HT triggers at 
    low pile up in cut steps of 25 \GeV. Figure (b) 
    shows the efficiency reduction for jets with in $|\eta| <3.$ and $\pt > 
    50$\GeV. Figure (c) show the efficiency reduction for a quad jet trigger, 
    with jet $|\eta| <3.$ and $\pt > 38$\GeV.}
    \label{fig:lowpuratereductionRECO}
\end{figure}



\paragraph{Efficiency of \HT Triggers} % (fold)
\label{par:Efficneicy of HT triggers}
Figure~\ref{fig:HTlowPURECO} shows the acceptance reduction after applying the 
two jet seed thresholds. The distribution is the cumulative number of events 
passing a cut of $L1 HT^{cut}$ in bins of 25 \GeV. Due to \HT being the scalar 
sum of the jet \PT's in the event the value of \Lone \HT is reduced as jets are 
removed from the calculation. To preserve efficiency the \Lone trigger 
threshold will have to be reduced. Comparing figures~\ref{fig:HTLowPU} and 
\ref{fig:HTlowPURECO}, if the trigger threshold is reduced to 75 \GeV an 
efficiency of $\geq 95\%$ can be maintained whilst reducing the trigger rate by 
$\approx 2\%$ when requiring a 2 \GeV seed threshold and reduced by $\approx 
3\%$ when requiring a 5 \GeV seed threshold. When comparing to the high pile up 
rate reduction in figure~\ref{fig:HTHighPU} it is shown that the trigger rate 
can be reduced by $\approx 20\%$ when requiring a 2 \GeV seed threshold and
reduced by $\geq 99\%$ when requiring a 5 \GeV seed threshold.
% subsection Efficiency of \HT triggers (end)


\paragraph{Efficiency of Jet Triggers} % (fold)
\label{par:Efficiency of Jet Triggers}
Figure~\ref{fig:cenJetLowPURECO} shows the change in acceptance of jets in low 
pile up conditions when the two seed thresholds are required. The effect is on 
the order of a few percent for each of the thresholds. Requiring a 2 \GeV seed 
reduces the efficiency for jets above 50 \GeV by $\approx 2.5\%$, whilst 
requiring a 5\GeV seed reduces the efficiency of the same jets by $\approx 4\%$.
% subsection Efficiency of Jet Triggers (end)

\paragraph{Efficiency of MultiJet Triggers} % (fold)
\label{par:Efficiency of MultiJet Triggers}
Figure~\ref{fig:quadJetLowPURECO} shows that the effect of requiring a seed 
threshold of 2 \GeV has no effect on the efficiency of the quad jet 38 \GeV 
trigger and requiring a seed threshold of 5 \GeV reduces the efficiency of the 
quad jet 38 trigger by $8\%$. The change in rate is dramatic in high pile up 
conditions where for a 2 \GeV seed threshold the rate is reduced by 
$\approx 30\%$ and by $\approx 40\%$ when requiring a 5 \GeV seed.
However it is to be noted that the sample where this measurement has been made 
is of limited size, inferring a reasonably large statistical uncertainty. 
% subsection Efficiency of MultiJet Triggers (end)


% section Effects of requiring a jet seed on offline efficiency (end)

\subsection{Summary} % (fold)
\label{sec:Summary}
The effects of requiring a jet seed have been studied using the \Lone trigger 
emulator on high and low pile-up samples. The studies show that requiring a jet 
seed of 5\GeV greatly reduces the rate of the \HT and Multi Jet triggers in 
high pile up conditions, whilst not adversely effecting the data taking 
efficiency of these triggers.

\begin{figure}[ht]
  \centering
  \includegraphics[width=0.75\textwidth]{figures/LoneTrigger/HTT150_pileup.pdf}
  \caption{Trigger cross section as a function of number of pile up 
  interactions. Showing that applying a 5 \GeV jet seed threshold dramatically 
  reduces the quadratic dependance of cross section on the number of pile up 
  interactions}
  \label{fig:figures_HTT150_pileup}
\end{figure}

The cross section of L1$\_$HTT150 has been measured with and with out the 
addition of a jet seed threshold of 5 \GeV as shown in  
Figure~\ref{fig:figures_HTT150_pileup}. Ideally the trigger cross section would 
be independent of the instantaneous luminosity and pile up, 
Figure~\ref{fig:figures_HTT150_pileup} shows that the addition of a 5 \GeV seed 
threshold reduces the dependance on instantaneous luminosity of the trigger 
cross section.

% subsection lone_trigger_pile_up_mitigation (end)

% chapter level_one_trigger (end)