\chapter{Reconstruction and Event selection} % (fold)
\label{cha:the_t_analysis}
In this chapter the event reconstruction and selection are defined, the search 
variable \alt is constructed both as an offline discriminator and as an 
analysis specific trigger, the performance of this trigger under different data 
taking conditions is measured with respect to the final event selection.

The first performance measurement is performed on the full \unit{5}{\invfb} of 
\unit{7}{\TeV} data collected in 2011, this data set is then used for the full 
analysis presented in this thesis. The performance of a suite of upgraded \alt 
triggers is then measured on \unit{11}{\invfb} of \unit{8}{\TeV} data which was 
collected during the 2012 \ac{lhc} run. These performance measurements were 
used in the analysis presented in \cite{Chatrchyan:2012wa}


\section{The \alt variable.} % (fold)
\label{sec:the_alpha___t_variable_}
\alt is inspired by Ref~\cite{Randall:2008dk} and was expanded to transverse 
multi jet topologies by members of the CMS collaboration in 
Refs~\cite{cms-pas-sus-08005,cms-pas-sus-09001}. The purpose is to provide a 
variable that can be cut on to eliminate QCD from the final selection. To do 
this the inherent balance of the QCD system is exploited.

For di-jet systems \alt is defined as:

\begin{equation}
  \alt = \frac{E_{T}^{j_{2}}}{M_{T}}
\end{equation}
where $E_{T}^{j_{2}}$ is the transverse energy of least energetic of the two 
jets and M$_{T}$ is defined as:

\begin{equation}
  M_{T} = \sqrt{\left(\sum^{2}_{i=1}E_{T}^{j_{i}}\right)^{2} - \left(\sum^{2}_{i=1}p_{x}^{j_{i}}\right)^{2} - \left(\sum^{2}_{i=1}p_{y}^{j_{i}}\right)^{2}}
\end{equation}

For a perfectly measured di-jet system with $E_{T}^{j_{1}} = E_{T}^{j_{2}}$, 
where the jets are opposite in $\phi$ \alt = 0.5, for events with back to back 
jets where one is miss-measured \alt $ < 0.5$.
However the majority of signals predict many jets in the final state.
\alt can be generalised to work with n-jets in the flowing way.
The variables \HT, \HTm and $\Delta$\HT are constructed,
\begin{equation}
  \HT = \sum_{i = 0}^{n~jets}E_{T}^{jet_{i}}, \\
  \HTm = \left|\sum_{i = 0}^{n~jets}\bar{p}_{T}^{jet_{i}}\right|,
\end{equation}
for jets above some predefined threshold \ET which is common for all jet based 
quantities. The multi jet system is reduced to a pseudo di-jet system by 
forming two large jets. The individual jet \ET's are summed, with the final 
configuration being chosen to have the minimum difference in energy 
($\Delta$\HT) between the pseudo jets. This simple clustering criteria provides 
the best separation between miss-measured events and those with real \MET.

\alt is then defined as:
\begin{equation}
  \alt = \frac{\HT - \Delta\HT}{2\sqrt{\HT^{2}-\HTm^{2}}}
\end{equation}


Figure~\ref{fig:figures_Analysis_AlphaT_all_375_upwards} shows the \alt 
distribution for both data and simulated background samples. The QCD multi jet 
background is negligible above an \alt value of 0.55, where as the standard 
model processes which involve real \MET exist at all possible values of \alt.
Values of \alt in the range $0.5 < \alt < 0.55$ arise in multi jet QCD due to 
jets falling below threshold or large stochastic fluctuations.
It is to be noted that the discrepancy between data and simulation for \alt 
$\leq 0.55$ is due to no trigger emulation being applied to the simulated 
background samples.
\begin{figure}[ht|]
  \centering  \includegraphics[width=0.8\textwidth]{figures/Analysis/AlphaT_all_375_upwards.pdf}
  \caption{\alt distribution for background and data for the region \HT $>$ 
  \unit{375}{\GeV}. Trigger emulation is not applied in the simulated 
  background which leads to the discrepancy in the region \alt $\leq 0.55$. The 
  QCD multi-jet background is reduced to less than one event.}
  \label{fig:figures_Analysis_AlphaT_all_375_upwards}
\end{figure}

% section the_alpha___t_variable_ (end)

\section{Event selection} % (fold)
\label{sec:event_selection}
In order to select events for the hadronic signal sample and the muon and 
photon control samples a common set of section cuts is defined. In this section 
the objects are defined as are the flow of the analysis cuts and filters.

\subsection{Preselection of hadronic objects} % (fold)
\label{sub:Preselection_of_hadronic_objects}
Hadronic showers are clustered in to objects which from now on will be referred 
to as jets. These jets are formed from the energy deposits recored in the 
calorimeter towers. Individual jets are clustered using the \AK 
algorithm\cite{Cacciari:2008ua}, for high \PT jets this produces cone like 
jets, for low \PT jets deviations from the circular geometry are allowed, as 
show in Figure~\ref{fig:figures_Analysis_ak5algo}. \ac{cms} uses two 
complimentary reconstruction methods for creating jets, the first uses only 
calorimeter information, these jets are referred to as \AK-Calorimeter jets, 
the second method adds information from the tracking system to improve the 
energy resolution and is not used in this analysis due to the requirements of 
using the same objects at both trigger level and offline to preserve data 
taking efficiency.

\begin{figure}[htbp]
  \centering
    \includegraphics[width=0.5\textwidth]{figures/Analysis/ak5algo.pdf}
  \caption{Simulated hadronic objects clustered in to Jets using the \AK jet 
  clustering algorithm, note the high \PT green object which causes the low \PT 
  jet in close proximity to be cut off.}
  \label{fig:figures_Analysis_ak5algo}
\end{figure}
The jets have their raw energies corrected based on their position and momentum 
to establish a uniform relative response in $\eta$ and a calibrated absolute 
response in transverse energy \ET, with an associated uncertainty of between 
2$\%$ and 4$\%$ dependant on \ET and $\eta$\cite{Chatrchyan:2011ds}. 
Figure~\ref{fig:figures_Analysis_caloJetUncertianty} shows the absolute jet 
energy scale uncertainty as a function of the jet's \PT for the jets considered 
in this analysis.

\begin{figure}[htbp]
  \centering
    \includegraphics[width=0.5\textwidth]{figures/Analysis/caloJetUncertianty.pdf}
  \caption{Absolute jet energy scale uncertainty as a function of jet \PT for \AK-Calorimeter jets\cite{1748-0221-6-11-P11002}.}
  \label{fig:figures_Analysis_caloJetUncertianty}
\end{figure}

It is also possible to correct the raw energies of the jets for pile up. Pile 
up causes energy from secondary interactions to be added to the event, for soft 
pile up interactions this energy is distributed uniformly though out the 
detector. Two forms of pile up corrections are used in \ac{cms} to correct the 
energies of calorimeter jets. The first is an offset correction, the raw energy 
has a constant energy removed though out the detector based on the number of 
primary vertices in the event. The second method is to calculate per event the 
average energy deposited in each calorimeter tower using the 
\texttt{FastJet}\cite{Cacciari:2011ma,hep-ph/0512210} jet clustering package, 
this quantity is referred to as $\rho$ and is also used to correct the 
isolation variables for elections, muons and photons.


The jets considered in the analysis are not corrected for pile up, each jet is 
required to have \ET $>$ \unit{50}{\GeV}, the highest \ET jet in the events is 
required to be within tracker acceptance ($|\eta| <$ 2.5) and the sub leading 
jet is required to have \ET $>$ \unit{100}{\GeV}. In the lowest two offline \HT 
bins the jet thresholds are scaled to preserve the jet multiplicity, for the 
bin \unit{275}{\GeV} $<$ \HT $<$ \unit{325}{\GeV}, the jet threshold is 
\unit{36.$\dot{6}$}{\GeV} and the sub leading jet threshold is 
\unit{73.$\dot{3}$}{\GeV}. In the bin \unit{325}{\GeV} $<$ \HT $<$ 
\unit{375}{\GeV}, the jet threshold is \unit{43.$\dot{3}$}{\GeV} and the sub 
leading jet threshold is \unit{86.$\dot{6}$}{\GeV}.

The quantities \HT and \HTm are then formed from these jets.
% paragraph definition_of_event_level_jets (end)

\subsection{Electrons} % (fold)
\label{sub:electrons}
Two complementary algorithms are used at the track seeding stage of the 
electron reconstruction. Tracker driven seeding which is more suitable for low 
\PT (\PT $<$ \unit{5}{\GeV}) electrons and ECAL driven seeding, which starts at 
the ECAL and works back towards the interaction point. The ECAL driven seeding 
starts by creating superclusters\cite{0954-3899-34-6-S01} which contain 
information on the lateral shower shape in the $\eta$ direction of the electron 
or photon shower and the spread of energy deposits in the $\phi$ plane for 
elections, due to the radiation of photons by the election in the \ac{cms} 
magnetic field. Track seeds from inner tracking layers and electron track are 
then build from these track seeds. The trajectories are reconstructed using a 
dedicated modelling of the electron energy loss and fitted with a Gaussian Sum 
Filter\cite{0954-3899-31-9-N01}.
For the purpose of the analysis an object is defined as an electron if it has 
\PT $>$ \unit{10}{\GeV}, \mETA $<$ 2.5 and passes ``VBTF'' working point 95 
quality criteria\cite{PAS-EGM-10-004}. Any event containing an identified 
electron is vetoed. 
% paragraph elections (end)

\subsection{Muons} % (fold)
\label{sub:muons}
Muon reconstruction at \ac{cms} is designed to be regional meaning that the 
full reconstruction sequence can be run at both the HLT and offline. Regional 
reconstruction only performs the reconstruction in a small part of the detector 
meaning that the time taken is short. As an example the amount of information 
required to reconstruct a muon in the silicon tracker is less than one percent 
of the entire tracker read out. The regions which are reconstructed at the HLT 
are seeded by the muon candidates produced by the \Lone trigger even if these 
muons did not cause the issue of the \Lone accept. Global muon reconstruction 
uses information from both the stand alone muon system and from the silicon 
tracker and performs a fit to the individual detector hits.
Figure~\ref{fig:figures_Analysis_MuonPTResolution} shows the muon \PT 
resolution for muons reconstructed using either the muon system only, the 
tracker only or the global muon reconstruction.
\begin{figure}[htbp]
  \centering
    \includegraphics[width=\textwidth]{figures/Analysis/MuonPTResolution.pdf}
  \caption{Resolution of (1/p) versus p for standalone, global and tracker-only reconstruction. a) Barrel, $\eta$ = 0.5; b) Endcap, $\eta$ = 1.5.\cite{0954-3899-34-6-S01}}
  \label{fig:figures_Analysis_MuonPTResolution}
\end{figure}

Signal events are vetoed if they contain a muon with \PT $>$ \unit{10}{\GeV} 
and \mETA $<$ 2.5 that passes the ``VBTF'' working point 95 for muons 
\cite{PAS-MUO-10-002}. Two separate muon control samples are defined, one 
requiring exactly one muon with \PT $>$ \unit{10}{\GeV}, the second requiring 
two muons, who's invariant mass sums to the \PZ mass. 
% subsection muons (end)


\subsection{Photons} % (fold)
\label{sub:photons}
Photons are reconstructed in the ECAL using the same methods as used to 
reconstruct elections. However an anti selection is performed on the tracking 
requirements. For photons which do not convert in the tracker a supercluster 
energy deposit is required with no associated electron track, for photons which 
undergo pair production in the tracker material the tracks associated to the 
super clusters do not originate from the interaction vertex, instead they 
appear in a layer of the tracker. For unconverted photons the ratio of the 
energy deposited in the central ECAL crystal to the energy of the surrounding  
$3\times3$ group of ECAL crystals is used as an identification variable. Signal 
events are vetoed if they contain a photon with \ET $>$ \unit{25}{\GeV}, 
\mETA $<$ 2.5 and that passes ID requirements\cite{PAS-EGM-10-006}. A 
requirement of exactly one photon with \ET $>$ \unit{150}{\GeV} with in \mETA 
$<$ 1.45 is made for the photon control sample.
% subsection photons (end)



\section{Common Analysis cuts} % (fold)
\label{sec:common_analysis_cuts}



\begin{itemize}
\item \textbf{Good run selection}, All detector subsystems on, \ac{cms} in 
``Physics Declared'' mode and all physics object groups have certified the runs 
and luminosity sections. This removes any events where the sub-detectors were 
in an error state or events from before the tracker was switched to high 
voltage mode.
\item \textbf{P.K.A.M (Previously Known As Monsters) filter}, these events are 
caused by beam-gas interactions close to \ac{cms}, which cause a shower of 
particles to enter the pixel detector along the beam line, resulting in a large 
proportion of the pixel detector to record hits.
\item \textbf{Vertex Selection} requires at least one non-fake vertex with at 
least four associated tracks, within a cylinder of radius \unit{2}{\cm} and 
length \unit{48}{\cm}, centred at Z = 0 of the \ac{cms} detector.
\item \textbf{Hadronic barrel and end-cap noise filter}, this filter removes 
events where strips of towers in the hadronic calorimeters record energy from 
electrical noise, mimicking large, unbalanced energy deposits.
% \item \textbf{Cleaned Rec hit filter}, removes events where the sum of cleaned energy from calorimeter towers is greater than \unit{30}{\GeV}.
\item \textbf{Vertex \PT/\HT $>$ 0.1}, removes events where the sum of the 
\PT of all tracks from all good vertices is less than 10$\%$ of the energy 
deposited by jets in the calorimeters. This cut is designed to remove events 
with tracking failure, which would otherwise pass the calorimeter only event 
quality requirements.
\item\textbf{Masked ECAL channel filter:} Approximately 1$\%$ of the ECAL 
crystals are masked, or have read out failure. To avoid selecting events with 
large energy miss measurement, a topological cut was devised. The first step is 
to calculate $\Delta\phi^{*}$ for each jet ($\vec{j}$) in the event, where:
\begin{equation}
  \Delta\phi^{*} = \Delta\phi\left(\vec{\slashed{E}}_{T}+\vec{j},\vec{j} \right).
  \label{eq:biasedDphi}
\end{equation}
Which gives a measure of the miss measurement of a jet, if $\Delta\phi^{*}$ is 
small, the missing energy points along the jet in the $\phi$ direction. By 
selecting the miss measured jet, full position information is preserved. If any 
jet has $\Delta\phi^{*} <$ 0.5, the number of masked ECAL crystals with in 
$\Delta R < $0.3 are summed, if there are more than 10 masked crystals adjacent 
to the jet, the event is vetoed.
\item \textbf{\Rmiss$ <$ 1.25:} The total hadronic energy in an event is 
required to be greater than \unit{275}{\GeV} which is well above the transverse 
energy threshold of \unit{50}{\GeV} for each jet. However several jets falling 
below this threshold can sum to a significant quantity of ignored energy. This 
is shown in Figure~\ref{fig:figures_Analysis_BabyMHT50-10}, here the missing 
energy calculated from jets in the range \unit{10}{\GeV} $<$ \ET $<$ 
\unit{50}{\GeV} is shown, whilst requiring that \MET $<$ \unit{20}{\GeV}. This 
shows that for a well balanced event the jets below threshold can carry greater 
than \unit{100}{\GeV} of ignored energy. R$_{miss}$ is defined as $\HTm \slash 
\MET$ and can be used to single out events where the inclusion of lower 
momentum jets does significantly improve the balance of the event. 
Figure~\ref{fig:MHTovMET_lowHT_highHT} shows for two \HT regions the R$_{miss}$ 
distribution after the application of the full cut flow, including \alt, QCD 
contamination is visible in the signal sample for R$_{miss} >$ 1.25. 
\end{itemize}


\begin{figure}[htbp]
  \centering
    \includegraphics[width=0.75\textwidth]{figures/Analysis/BabyMHT50-10.pdf}
  \caption{\HTm from jets with \unit{10}{\GeV} $<$ \ET $<$ \unit{50}{\GeV} in events with \HT $>$ \unit{350}{\GeV} and \MET $<$ \unit{20}{\GeV} in \unit{35}{\invpicobarn} of data.}
  \label{fig:figures_Analysis_BabyMHT50-10}
\end{figure}

\begin{figure}[htbp]
  \centering
  \subfigure[]{ \includegraphics[width=0.45\textwidth]{figures/Analysis/250_300GevcombinedMHTovMET_afterAlphaT_all.pdf}
  \label{fig:figures_Analysis_250_300GevcombinedMHTovMET_afterAlphaT_all}
}
 \subfigure[]{   \includegraphics[width=0.45\textwidth]{figures/Analysis/350Gev_afterDeadEcalcombinedMHTovMET_afterAlphaT_all.pdf}
\label{fig:figures_Analysis_350Gev_afterDeadEcalcombinedMHTovMET_afterAlphaT_all}
}
\caption{(a) R$_{miss}$ distribution for events in the \unit{250}{\GeV} 
$<$\HT$<$\unit{350}{\GeV} region, where due to the low \HT requirement QCD 
contamination is enriched. \\ (b) R$_{miss}$ distribution for events in the 
\HT$>$\unit{350}{\GeV} region, QCD contamination occurs at R$_{miss} >$ 1.25.}
\label{fig:MHTovMET_lowHT_highHT}
\end{figure}


This cut flow and set of object definitions define the common selection, on top 
of this an \alt cut is applied, the passing events are then binned in 8 \HT 
bins, these are \unit{275}{\GeV} - \unit{325}{\GeV}, \unit{325}{\GeV} - 
\unit{375}{\GeV}, \unit{375}{\GeV} - \unit{475}{\GeV}, \unit{475}{\GeV} - 
\unit{575}{\GeV}, \unit{575}{\GeV} - \unit{675}{\GeV}, \unit{675}{\GeV} - 
\unit{775}{\GeV}, \unit{875}{\GeV} - \unit{7}{\TeV}.
However the raw number of events in the signal region is meaningless with out 
an accurate back ground prediction, as discussed earlier taking the background 
estimation from simulation is not viable due to the requirement of high order 
theory calculations on the cross sections of \ac{sm} processes involving 
jets, these have many divergencies due to the nature of QCD. Instead the 
simulation is used to form a translation factor between two samples measured in 
data, a control sample which closely mimics the \ac{sm} processes 
producing missing energy, but has a visible muon or photon in the final state. 
The background estimation methods are described and the background predictions 
stated in Section~\ref{sec:electro_weak_background_prediciton}.
The next section describes the triggers used to collect the data samples
% section event_selection (end)

\section{High Level triggers for the \alt analysis} % (fold)
\label{sec:high_level_triggers_for_the_alt_analysis}
The CMS trigger system has been discussed in detail in 
Section~\ref{sec:the_high_level_trigger_system} and 
Chapter~\ref{cha:level_one_trigger}, however details of analysis specific 
trigger paths were not discussed. During 2011 the first \alt specific trigger 
was designed and deployed online. The trigger was then upgraded for the higher 
luminosity and energy conditions of the 2012 data taking period.


The trigger takes advantage of cutting on two variables, \HT and \alt at low 
\HT a high \alt value cuts the trigger rate, where as at high \HT where the
trigger rate is lower the \alt requirement can be loosened.

Due to the scaling of jet thresholds in the lowest offline \HT bins as detailed 
in Section~\ref{sec:event_selection} using a fixed jet threshold would cause 
inefficiency in the lowest offline \HT bins. To over come this the trigger 
level \alt calculation is performed iteratively for all jets above a predefined 
threshold. This raises the total number of accepted events whilst adding the 
benefit of being efficient for any offline jet threshold above the minimum 
trigger jet threshold.

Due to concerns on the time taken to perform the \dHT minimisation at the 
trigger and time constraints enforced on trigger menu development, the first 
implementation calculates \alt for the first 3 jets. For higher jet 
multiplicities the variable \bt is calculated.

\begin{equation}
  \beta_{T} = \frac{\HT}{2\sqrt{\HT^{2} - \MHT^{2}}}
  \label{eqn:betaT}
\end{equation}
this gives us the relation:
\begin{equation}
  \alt \leq \beta_{T}.
\end{equation}
The decision flow is shown in Figure~\ref{fig:figures_Analysis_flowChart} and 
explained in detail below.

\begin{figure}[ht|]
  \centering
    \includegraphics[width=0.7\textwidth]{figures/Analysis/flowChart.pdf}
  \caption{Flow chart representing the steps taken to make a trigger decision 
  using the \alt trigger algorithm.}
  \label{fig:figures_Analysis_flowChart}
\end{figure}


When a level one accept is issued the trigger bits that fired are checked, if 
the event fires a L1 muon trigger it is passed to the HLT muon triggers where 
only muon reconstruction is performed, reducing the reconstruction time. The 
\alt triggers are seeded on the lowest threshold unprescaled L1 \HT trigger, 
during 2011 this was L1$\_$HTT100. Any events issuing a L1 accept and passing 
L1$\_$HTT100 undergo calorimeter jet reconstruction, the reconstruction 
algorithm is detailed in Section~\ref{sub:Preselection_of_hadronic_objects}.

Once the jets have been formed the trigger filter is entered. Initially the 
first two jets ranked by \ET, are considered, \HT and \alt are calculated, if 
both pass the trigger thresholds the event is accepted and the full detector 
read out is performed. If either \HT or \alt is below threshold, the next jet 
in \ET order is added, if the jet collection contains more than 3 jets then the 
\bt approximation is used. All jets in the event are added until either the 
event is accepted, or there are no more jets to be added above \unit{40}{\GeV}.

The effect of switching to the \bt approximation is to accept events that have 
missing energy due to miss-measurement, when calculating \alt offline these 
events have values of \alt $< 0.5$. This introduces an impurity to the trigger 
and costs rate for events that will not be considered in the offline analysis.


% section common_analysis_cuts (end)
\subsection{Trigger efficiency measurement} % (fold)
\label{sub:trigger_efficiency_measurement}

The performance of the \alt trigger suit is measured with respect to a sample 
collected using the muon system. This allows the measurement of efficiency of 
both the level one seed trigger and the higher level trigger at the same time 
as different sub-systems are used to collect the reference and the signal 
triggers. This is due to the exclusive use of calorimeter jets in the \alt 
trigger, if more complicated reconstruction methods which produce an event 
hypothesis were used, muons would at HLT level only be considered as jets. 
Where as during calorimeter only reconstruction, muons are not considered and 
the \pt of any muons in an event is viewed as missing energy.

The selection for the trigger efficiency measurement is the same as listed in 
Section~\ref{sec:event_selection} with the requirement of exactly one well 
identified muon with \PT $>$ \unit{45}{\GeV}, the sum of the \MET in the event 
and the muon must add to the \PW mass and finally the muon must be separated by 
at least 0.5 in $\Delta R$ to the closest jet, to avoid the muon energy 
changing the energy of the jet offline when the cross cleaning is applied.

Due to the increase in luminosity over the running period the trigger 
thresholds were increased, to ensure constant rate though out the year and the 
trigger version numbers were increased each time the trigger menu was updated.
The list of triggers considered and their reference triggers are listed in 
Tables~\ref{tab:triggers1}~and~\ref{tab:triggers2}.

The efficiency of each trigger version is measured in the required \HT bins, 
the total efficiency for each \HT bin is then calculated by combining the 
individual efficiencies using a weighted sum based on the fraction of 
the total luminosity of the sample that each version carries. This accounts for 
the change in running conditions during the data taking period and the criteria 
that the trigger suite takes constant rate. The higher trigger versions and 
thresholds generally represent more integrated luminosity due to the evolution 
of running conditions with time.

The efficiency is measured as a function of the cumulative number of events, ie 
the efficiency at each point on the x axis is the measured efficiency if a cut 
were applied offline at that cut value.




\begin{table}
    \centering
\footnotesize
\begin{tabular}{|c|c|c|}
  
\hline
Offline \HT bin & Signal Trigger & Reference Trigger\\
\hline
\unit{275}{\GeV} $<$ \HT $<$ \unit{325}{\GeV} & \verb|HLT_HT250_AlphaT0p53_v2| & \verb|HLT_Mu15_HT200_v2|\\
\hline
\unit{275}{\GeV} $<$ \HT $<$ \unit{325}{\GeV} & \verb|HLT_HT250_AlphaT0p53_v3| & \verb|HLT_Mu15_HT200_v3|\\
\hline
\unit{275}{\GeV} $<$ \HT $<$ \unit{325}{\GeV} & \verb|HLT_HT250_AlphaT0p53_v4| & \verb|HLT_Mu15_HT200_v4|\\
\hline
\unit{275}{\GeV} $<$ \HT $<$ \unit{325}{\GeV} & \verb|HLT_HT250_AlphaT0p53_v5| & \verb|HLT_Mu30_HT200_v1|\\
\hline
\unit{275}{\GeV} $<$ \HT $<$ \unit{325}{\GeV} & \verb|HLT_HT250_AlphaT0p55_v1| & \verb|HLT_Mu5_HT200_v4|\\
\hline
\unit{275}{\GeV} $<$ \HT $<$ \unit{325}{\GeV} & \verb|HLT_HT250_AlphaT0p55_v2| & \verb|HLT_Mu40_HT200_v4|\\
\hline
\unit{275}{\GeV} $<$ \HT $<$ \unit{325}{\GeV} & \verb|HLT_HT250_AlphaT0p58_v3| & \verb|HLT_DoubleMu8_Mass8_HT200_v4|\\
\hline
\unit{275}{\GeV} $<$ \HT $<$ \unit{325}{\GeV} & \verb|HLT_HT250_AlphaT0p58_v3| & \verb|HLT_DoubleMu8_Mass8_HT200_v5| \\
\hline
\unit{275}{\GeV} $<$ \HT $<$ \unit{325}{\GeV} & \verb|HLT_HT250_AlphaT0p60_v3| & \verb|HLT_DoubleMu8_Mass8_HT200_v4|\\
\hline
\unit{275}{\GeV} $<$ \HT $<$ \unit{325}{\GeV} & \verb|HLT_HT250_AlphaT0p60_v3| & \verb|HLT_DoubleMu8_Mass8_HT200_v5|\\
\hline
\unit{325}{\GeV} $<$ \HT $<$ \unit{375}{\GeV} & \verb|HLT_HT300_AlphaT0p52_v1| & \verb|HLT_Mu5_HT200_v4|\\
\hline
\unit{325}{\GeV} $<$ \HT $<$ \unit{375}{\GeV} & \verb|HLT_HT300_AlphaT0p52_v2| & \verb|HLT_Mu8_HT200_v4|\\
\hline
\unit{325}{\GeV} $<$ \HT $<$ \unit{375}{\GeV} & \verb|HLT_HT300_AlphaT0p52_v3| & \verb|HLT_Mu15_HT200_v2|\\
\hline
\unit{325}{\GeV} $<$ \HT $<$ \unit{375}{\GeV} & \verb|HLT_HT300_AlphaT0p53_v3| & \verb|HLT_Mu15_HT200_v3|\\
\hline
\unit{325}{\GeV} $<$ \HT $<$ \unit{375}{\GeV} & \verb|HLT_HT300_AlphaT0p53_v4| & \verb|HLT_Mu15_HT200_v4| \\
\hline
\unit{325}{\GeV} $<$ \HT $<$ \unit{375}{\GeV} & \verb|HLT_HT300_AlphaT0p53_v5| & \verb|HLT_Mu30_HT200_v1| \\
\hline
\unit{325}{\GeV} $<$ \HT $<$ \unit{375}{\GeV} & \verb|HLT_HT300_AlphaT0p53_v6| & \verb|HLT_Mu40_HT200_v3| \\
\hline
\unit{325}{\GeV} $<$ \HT $<$ \unit{375}{\GeV} & \verb|HLT_HT300_AlphaT0p53_v6| & \verb|HLT_Mu40_HT200_v4| \\
\hline
\unit{325}{\GeV} $<$ \HT $<$ \unit{375}{\GeV} & \verb|HLT_HT300_AlphaT0p54_v5| & \verb|HLT_Mu40_HT300_v4| \\
\hline
\unit{325}{\GeV} $<$ \HT $<$ \unit{375}{\GeV} & \verb|HLT_HT300_AlphaT0p54_v5| & \verb|HLT_Mu40_HT300_v5| \\
\hline
\unit{325}{\GeV} $<$ \HT $<$ \unit{375}{\GeV} & \verb|HLT_HT300_AlphaT0p55_v3| & \verb|HLT_DoubleMu8_Mass8_HT200_v4| \\
\hline
\unit{325}{\GeV} $<$ \HT $<$ \unit{375}{\GeV} & \verb|HLT_HT300_AlphaT0p55_v3| & \verb|HLT_DoubleMu8_Mass8_HT200_v5| \\
\hline

\end{tabular}
\caption{List of \alt triggers used in the lowest two offline \HT bins and the triggers used to collect the reference sample.}
\label{tab:triggers1}

\end{table}

\begin{table}
    \centering
  \footnotesize
\begin{tabular}{|c|c|c|}
    \hline
Offline \HT bin &Signal Trigger & Reference Trigger\\
\hline
\unit{375}{\GeV} $<$ \HT $<$ \unit{475}{\GeV} & \verb|HLT_HT350_AlphaT0p51_v1| & \verb|HLT_Mu5_HT200_v4| \\
\hline
\unit{375}{\GeV} $<$ \HT $<$ \unit{475}{\GeV} & \verb|HLT_HT350_AlphaT0p51_v2| & \verb|HLT_Mu8_HT200_v4| \\
\hline
\unit{375}{\GeV} $<$ \HT $<$ \unit{475}{\GeV} & \verb|HLT_HT350_AlphaT0p51_v3| & \verb|HLT_Mu15_HT200_v2| \\
\hline
\unit{375}{\GeV} $<$ \HT $<$ \unit{475}{\GeV} & \verb|HLT_HT350_AlphaT0p51_v4| & \verb|HLT_Mu15_HT200_v3| \\
\hline
\unit{375}{\GeV} $<$ \HT $<$ \unit{475}{\GeV} & \verb|HLT_HT350_AlphaT0p51_v5| & \verb|HLT_Mu15_HT200_v4| \\
\hline
\unit{375}{\GeV} $<$ \HT $<$ \unit{475}{\GeV} & \verb|HLT_HT350_AlphaT0p52_v1| & \verb|HLT_Mu30_HT200_v1| \\
\hline
\unit{375}{\GeV} $<$ \HT $<$ \unit{475}{\GeV} & \verb|HLT_HT350_AlphaT0p52_v2| & \verb|HLT_Mu40_HT200_v3| \\
\hline
\unit{375}{\GeV} $<$ \HT $<$ \unit{475}{\GeV} & \verb|HLT_HT350_AlphaT0p52_v2| & \verb|HLT_Mu40_HT200_v4| \\
\hline
\unit{375}{\GeV} $<$ \HT $<$ \unit{475}{\GeV} & \verb|HLT_HT350_AlphaT0p53_v10| & \verb|HLT_Mu40_HT300_v4| \\
\hline
\unit{375}{\GeV} $<$ \HT $<$ \unit{475}{\GeV} & \verb|HLT_HT350_AlphaT0p53_v10| & \verb|HLT_Mu40_HT300_v5| \\
\hline
\unit{475}{\GeV} $<$ \HT $<$ \unit{7}{\TeV} & \verb|HLT_HT400_AlphaT0p51_v1| & \verb|HLT_Mu5_HT200_v4| \\
\hline
\unit{475}{\GeV} $<$ \HT $<$ \unit{7}{\TeV} & \verb|HLT_HT400_AlphaT0p51_v2| & \verb|HLT_Mu8_HT200_v4| \\
\hline
\unit{475}{\GeV} $<$ \HT $<$ \unit{7}{\TeV} & \verb|HLT_HT400_AlphaT0p51_v3| & \verb|HLT_Mu15_HT200_v2| \\
\hline
\unit{475}{\GeV} $<$ \HT $<$ \unit{7}{\TeV} & \verb|HLT_HT400_AlphaT0p51_v4| & \verb|HLT_Mu15_HT200_v3| \\
\hline
\unit{475}{\GeV} $<$ \HT $<$ \unit{7}{\TeV} & \verb|HLT_HT400_AlphaT0p51_v5| & \verb|HLT_Mu15_HT200_v4| \\
\hline
\unit{475}{\GeV} $<$ \HT $<$ \unit{7}{\TeV} & \verb|HLT_HT400_AlphaT0p51_v6| & \verb|HLT_Mu30_HT200_v1| \\
\hline
\unit{475}{\GeV} $<$ \HT $<$ \unit{7}{\TeV} & \verb|HLT_HT400_AlphaT0p51_v7| & \verb|HLT_Mu40_HT200_v3| \\
\hline
\unit{475}{\GeV} $<$ \HT $<$ \unit{7}{\TeV} & \verb|HLT_HT400_AlphaT0p51_v7| & \verb|HLT_Mu40_HT200_v4| \\
\hline
\unit{475}{\GeV} $<$ \HT $<$ \unit{7}{\TeV} & \verb|HLT_HT400_AlphaT0p51_v10| & \verb|HLT_Mu40_HT300_v4| \\
\hline
\unit{475}{\GeV} $<$ \HT $<$ \unit{7}{\TeV} & \verb|HLT_HT400_AlphaT0p51_v10| & \verb|HLT_Mu40_HT300_v5| \\
\hline
\unit{475}{\GeV} $<$ \HT $<$ \unit{7}{\TeV} & \verb|HLT_HT400_AlphaT0p52_v5| & \verb|HLT_Mu40_HT300_v4| \\
\hline
\unit{475}{\GeV} $<$ \HT $<$ \unit{7}{\TeV} & \verb|HLT_HT400_AlphaT0p52_v5| & \verb|HLT_Mu40_HT300_v5| \\
\hline

\end{tabular}
\caption{List of \alt triggers used in the \HT $>$ \unit{375}{\GeV} bins and the triggers used to collect the reference sample.}
\label{tab:triggers2}

\end{table}

As an example, the efficiency of each trigger used in the \unit{275}{\GeV} $<$ 
\HT $<$ \unit{325}{\GeV} bin is measured and then combined to give the total 
efficiency. The cumulative efficiency curves for each trigger seeding the 
lowest bin is show in Figures~\ref{fig:HT275BinTriggers1}, \ref{fig:HT275BinTriggers2} $\&$ \ref{fig:HT275BinTriggers3}. Note that some of the 
triggers are repeated, due to the reference trigger version incrementing and 
the signal trigger not.

\begin{figure}[ht]
  \centering
   \subfigure[HLT$\_$HT250$\_$AlphaT0p55$\_$v1]{ \includegraphics[width=0.4\textwidth]{figures/Analysis/HT275_HLT_HT250_AlphaT0p55_v1_HLT_Mu5_HT200_v4.pdf}
\label{fig:figures_Analysis_HT275_HLT_HT250_AlphaT0p55_v1_HLT_Mu5_HT200_v4}}
\subfigure[HLT$\_$HT250$\_$AlphaT0p53$\_$v2]{   \includegraphics[width=0.4\textwidth]{figures/Analysis/HT275_HLT_HT250_AlphaT0p53_v2_HLT_Mu15_HT200_v2.pdf}
\label{fig:figures_Analysis_HT275_HLT_HT250_AlphaT0p53_v2_HLT_Mu15_HT200_v2}}

\subfigure[HLT$\_$HT250$\_$AlphaT0p53$\_$v3]{\includegraphics[width=0.4\textwidth]{figures/Analysis/HT275_HLT_HT250_AlphaT0p53_v3_HLT_Mu15_HT200_v3.pdf}
\label{fig:figures_Analysis_HT275_HLT_HT250_AlphaT0p53_v3_HLT_Mu15_HT200_v3}}

\subfigure[HLT$\_$HT250$\_$AlphaT0p53$\_$v4]{ \includegraphics[width=0.4\textwidth]{figures/Analysis/HT275_HLT_HT250_AlphaT0p53_v4_HLT_Mu15_HT200_v4.pdf}
\label{fig:figures_Analysis_HT275_HLT_HT250_AlphaT0p53_v4_HLT_Mu15_HT200_v4}}

\caption{Turn on curves for the individual \alt triggers used to seed the 
\unit{275}{\GeV} $< \HT <$ \unit{325}{\GeV} bin.}
\label{fig:HT275BinTriggers1}
\end{figure}

\begin{figure}[ht]
  \centering
\subfigure[HLT$\_$HT250$\_$AlphaT0p53$\_$v5]{ \includegraphics[width=0.4\textwidth]{figures/Analysis/HT275_HLT_HT250_AlphaT0p53_v5_HLT_Mu30_HT200_v1.pdf}
\label{fig:figures_Analysis_HT275_HLT_HT250_AlphaT0p53_v5_HLT_Mu30_HT200_v1}}
  
\subfigure[HLT$\_$HT250$\_$AlphaT0p58$\_$v3]{ \includegraphics[width=0.4\textwidth]{figures/Analysis/DiMu_HT275_HLT_HT250_AlphaT0p58_v3_HLT_DoubleMu8_Mass8_HT200_v4.pdf}
\label{fig:figures_Analysis_DiMu_HT275_HLT_HT250_AlphaT0p58_v3_HLT_DoubleMu8_Mass8_HT200_v4}}

\subfigure[HLT$\_$HT250$\_$AlphaT0p58$\_$v3]{ \includegraphics[width=0.4\textwidth]{figures/Analysis/DiMu_HT275_HLT_HT250_AlphaT0p58_v3_HLT_DoubleMu8_Mass8_HT200_v5.pdf} 
\label{fig:figures_Analysis_DiMu_HT275_HLT_HT250_AlphaT0p58_v3_HLT_DoubleMu8_Mass8_HT200_v5}}

\subfigure[HLT$\_$HT250$\_$AlphaT0p60$\_$v3]{ \includegraphics[width=0.4\textwidth]{figures/Analysis/DiMu_HT275_HLT_HT250_AlphaT0p60_v3_HLT_DoubleMu8_Mass8_HT200_v4.pdf} 

\label{fig:figures_Analysis_DiMu_HT275_HLT_HT250_AlphaT0p60_v3_HLT_DoubleMu8_Mass8_HT200_v4}}
\caption{Turn on curves for the individual \alt triggers used to seed the 
\unit{275}{\GeV} $< \HT <$ \unit{325}{\GeV} bin.}
\label{fig:HT275BinTriggers2}
\end{figure}

\begin{figure}[ht]
  \centering
\subfigure[HLT$\_$HT250$\_$AlphaT0p60$\_$v3]{ \includegraphics[width=0.4\textwidth]{figures/Analysis/HT275_HLT_HT250_AlphaT0p55_v2_HLT_Mu40_HT200_v4.pdf}
\label{fig:figures_Analysis_HT275_HLT_HT250_AlphaT0p55_v2_HLT_Mu40_HT200_v4}}
\caption{Turn on curves for the individual \alt triggers used to seed the 
\unit{275}{\GeV} $< \HT <$ \unit{325}{\GeV} bin.}
\label{fig:HT275BinTriggers3}
\end{figure}

These are then combined to give 
Figure~\ref{fig:figures_Analysis_TotalCumulativeTurnonforHT275AlphaT} the 
efficiency at a cut of \altg is 83.3$\%$ + 0.5$\%$ - 0.6$\%$. The loss 
in efficiency comes from the disparity between the minimum thresholds for jets 
to enter the \HT and \alt calculations at the HLT and those used in the 
analysis. The trigger jet \ET threshold is \unit{40}{\GeV} where as 
the analysis jet \ET threshold in this bin is 
\unit{36.$\dot{6}$}{\GeV}. The triggers used to take data at the end of this 
running period also have an \alt threshold above the analysis cut of \altg, which again causes an inefficiency. The list of efficiencies for each \HT 
bin are shown in Table~\ref{tab:altEfficiencies7TeV}. The trigger efficiencies 
are measured to better than one percent and this information is used in the 
final likelihood model, to correct the expected numbers from simulation when 
calculating the translation factors.


\begin{figure}[ht]
  \centering 
\includegraphics[width=\textwidth]{figures/Analysis/TotalCumulativeTurnonforHT275AlphaT.pdf}
  \caption{Combined cumulative efficiency for the triggers seeding the \unit{275}{\GeV} $< \HT <$ \unit{325}{\GeV} offline analysis bin.}
  \label{fig:figures_Analysis_TotalCumulativeTurnonforHT275AlphaT}
\end{figure}

\begin{table}
  \centering
\begin{tabular}{ |c|c| }
\hline
\HT range (GeV) & Trigger efficiency ($\%$)  \\ [1.0ex]
\hline
275--325 & 83.3$^{+0.5}_{-0.6}$ \\ [1.0ex]
\hline
325--375 & 95.9$^{+0.7}_{-0.9}$ \\ [1.0ex] 
\hline
375--475 & 98.5$^{+0.5}_{-0.9}$ \\ [1.0ex]
\hline
475--$\infty$ & 100.0$^{+0.0}_{-4.8}$ \\ [1.0ex]
\hline
\end{tabular}

\caption{Efficiencies of the \alt triggers used in the \unit{7}{\TeV} \alt 
         analysis on \unit{5}{\invfb} of \ac{lhc} data.}
\label{tab:altEfficiencies7TeV}
\end{table}


The \alt triggers were upgraded for the increased instantaneous luminosity and 
pile up conditions expected during the 2012 data taking period. The first stage 
was to implement the full \alt calculation for each addition of a new jet, this 
change increases the purity of the trigger, meaning that for the same threshold 
the rate taken is lower than that of the previous algorithm. This allows the 
thresholds to be kept low, whilst not increasing the overall trigger rate.

The second upgrade was to switch to pile up corrected jets at the HLT, this 
change keeps the trigger cross section liner as a function of instantaneous 
luminosity.

The third choice was to design a trigger suite that could run for the entire 
2012 data taking period, with out changing the trigger thresholds, this was 
done to make the measurement of the efficiency simpler.

The trigger thresholds used are presented in 
Table~\ref{tab:2012triggerThresholds}, due to the constant thresholds in 
\HT and \alt though out the run, the versioning of the triggers is excluded 
from this table.
The efficiency of the triggers is measured for \unit{11.7}{\invfb} of \ac{lhc} data 
taken at a centre of mass energy of \unit{8}{\TeV} the efficiencies were 
measured for the analysis presented at HCP, documented in 
\cite{CMS-PAS-SUS-12-028}. A single muon trigger is used to collect the 
reference sample, the threshold of this trigger was unchanged during the data 
taking period, the path selected is \verb|HLT_IsoMu24_v*| which requires at 
least one muon with \PT $>$ \unit{24}{\GeV} that is not over lapping with any 
other object in the detector.
The same method is used as for the 2011 trigger efficiency measurement. However 
the offline jets are corrected for pile up using the fast jet corrections, to 
stay inline with the HLT object definitions.

Figure~\ref{fig:2012TriggersHT275} shows the efficiencies of the four 
individual triggers that seed the \unit{275}{\GeV}--\unit{325}{\GeV} \HT bin 
for \unit{11.7}{\invfb} of \unit{8}{\TeV} \ac{lhc} data, 
Figure~\ref{fig:figures_Analysis_8TeVTriggers_HT275AlphaT} shows the combined 
cumulative efficiency of the 2012 trigger suite.

\begin{table}

\begin{tabular}{|c|c|}
\hline
\HT range & Trigger \\
\hline
\unit{275}{\GeV}--\unit{325}{\GeV} & \verb|HLT_HT250_AlphaT0p55_v*| \\
\hline
\unit{325}{\GeV}--\unit{375}{\GeV} & \verb|HLT_HT300_AlphaT0p53_v*| \\
\hline
\unit{375}{\GeV}--\unit{475}{\GeV} & \verb|HLT_HT350_AlphaT0p52_v*| \\
\hline
\unit{475}{\GeV}--\unit{8}{\TeV}   & \verb|HLT_HT400_AlphaT0p51_v*| \\
\hline
\end{tabular}
\caption{Triggers used to seed the analysis \HT bins during 2012 data taking.}
\label{tab:2012triggerThresholds}
\end{table}



\begin{figure}[ht]
 \subfigure[HLT$\_$HT250$\_$AlphaT0p55$\_$v1]{\includegraphics[width=0.4\textwidth]{figures/Analysis/8TeVTriggers/HT275_HLT_HT250_AlphaT0p55_v1_HLT_IsoMu24_eta2p1_v11.pdf} \label{fig:figures_Analysis_8TeVTriggers_HT275_HLT_HT250_AlphaT0p55_v1_HLT_IsoMu24_eta2p1_v11}}

\subfigure[HLT$\_$HT250$\_$AlphaT0p55$\_$v2]{\includegraphics[width=0.4\textwidth]{figures/Analysis/8TeVTriggers/HT275_HLT_HT250_AlphaT0p55_v2_HLT_IsoMu24_eta2p1_v12.pdf}  \label{fig:figures_Analysis_8TeVTriggers_HT275_HLT_HT250_AlphaT0p55_v2_HLT_IsoMu24_eta2p1_v12}}
    \subfigure[HLT$\_$HT250$\_$AlphaT0p55$\_$v3]{\includegraphics[width=0.4\textwidth]{figures/Analysis/8TeVTriggers/HT275_HLT_HT250_AlphaT0p55_v3_HLT_IsoMu24_eta2p1_v12.pdf}  
\label{fig:figures_Analysis_8TeVTriggers_HT275_HLT_HT250_AlphaT0p55_v3_HLT_IsoMu24_eta2p1_v12}}

\subfigure[HLT$\_$HT250$\_$AlphaT0p55$\_$v4]{\includegraphics[width=0.4\textwidth]{figures/Analysis/8TeVTriggers/HT275_HLT_HT250_AlphaT0p55_v4_HLT_IsoMu24_v15.pdf} 
\label{fig:figures_Analysis_8TeVTriggers_HT275_HLT_HT250_AlphaT0p55_v4_HLT_IsoMu24_v15}}

\caption{Turn on curves for the individual \alt triggers used to seed the 
\unit{275}{\GeV} $< \HT <$ \unit{325}{\GeV} bin, during 2012 data taking.}
\label{fig:2012TriggersHT275}
\end{figure}

\begin{figure}[ht]
\centering    \includegraphics[width=0.75\textwidth]{figures/Analysis/8TeVTriggers/HT275AlphaT.pdf}
\caption{Combined cumulative efficiency for the triggers seeding the \unit{275}{\GeV} $< \HT <$ \unit{325}{\GeV} offline bin for \unit{11.7}{\invfb} 
  of \unit{8}{\TeV} \ac{lhc} data.}
\label{fig:figures_Analysis_8TeVTriggers_HT275AlphaT}
\end{figure}


\begin{table}
\begin{tabular}{|c|c|}
\hline
\HT range (GeV) & Trigger efficiency ($\%$)  \\ 
\hline
\unit{275}{\GeV} -- \unit{325}{\GeV} & 89.6$^{+0.4}_{-0.4}$  \\
\hline
\unit{325}{\GeV} -- \unit{375}{\GeV} & 98.6$^{+0.2}_{-0.3}$  \\
\hline
\unit{375}{\GeV} -- \unit{475}{\GeV} & 99.4$^{+0.2}_{-0.3}$  \\
\hline
\unit{475}{\GeV} -- \unit{8}{\TeV}   & 100.0$^{+0.0}_{-0.5}$ \\
\hline
\end{tabular}
\label{tab:2012Triggerefficencies}
\caption{Efficiencies of the \alt triggers at a centre of mass energy of 
         \unit{8}{\TeV} \alt measured in \unit{11.7}{\invfb} of \ac{lhc} data.}
\end{table}

The two trigger suites have very similar offline performance with 100$\%$ 
efficiency in the analysis bins above \unit{475}{\GeV} and high efficiency in 
the lower bins. To quantify the difference in performance between the two 
algorithms the purity is defined as the number of events passing the trigger 
that pass the offline \alt requirement at the same threshold as the trigger requirement divided by the total number of events 
accepted by the trigger.
The purity of each trigger algorithm is measured for an example trigger with a 
\HT threshold of \unit{350}{\GeV} and an \alt threshold of 0.52.
The 2011 trigger which only performs the full \alt calculation for jet 
multiplicities of less than four has a purity of 48$\%$, which means a quarter 
of all rate taken by the trigger is used in the offline analysis.
The 2012 trigger which performs the full \alt calculation for all jet 
multiplicities has a purity of 75$\%$. This 25$\%$ increase in purity 
translates in to a 25$\%$ rate reduction for the same trigger threshold when
changing to the full \alt calculation for all jet multiplicities, thus enabling 
the trigger thresholds to be kept at the same or lower thresholds in the 2012 
run as in the 2011 run.

% section high_level_triggers_for_the_alt_analysis (end)

\section{Extension to higher analysis dimensions. B-tagging} % (fold)
\label{sec:extension_to_higher_analysis_dimensions_}
\paragraph{Declaration} % (fold)
\label{par:declaration-btagging}
Whilst it should be noted that the author did not contribute towards the 
extension of the analysis to include a dimension based on the measurement of 
the number of jets containing \Pbottom quarks in the final state, the final 
analysis does include this extension, hence the \Pbottom-tagging procedure is 
explained in sufficient detail as to understand the process but not to elude to 
all the nuances of the method in the following section.
% paragraph declaration (end)





Jets are tagged as originating from a \Pbottom decay via the identification of 
a displaced vertex as it takes the \Pbottom a finite amount of time to decay, 
during this time the quark which is not produced at rest travels away from the 
initial interaction point, where upon decaying a secondary decay vertex is 
created. The algorithm used to classify these jets as containing a \Pbottom 
quark is the \ac{csv} tagger. The ``Medium'' working point is used in the 
analysis, which requires at cut of $>0.679$ on the \ac{csv} value. This results 
in a gluon/light quark miss tag rate of $1\%$ where ``light'' quarks are \Pup, 
\Pdown and \Pstrange types. The efficiency for tagging \Pbottom quark jets in 
the range $60-70\%$\cite{ref-b-tag-eff} is achieved, this efficiency depends on 
the jet \PT. The discriminator is supported by the \ac{cms} \Pbottom physics 
object group\cite{ref-b-pog-algo} and its performance is well understood. The 
Montecarlo miss tag and identification rates are re-weighted to agree with data 
observations, thus provide an accurate background prediction for the high 
multiplicity \Pbottom-tagged bins which are dominated by the miss tag of light 
quarks as \Pbottom quarks.



% section extension_to_higher_analysis_dimensions_ (end)
% section signal_models (end)
% section final_results (end)
% chapter the_t_analysis (end)