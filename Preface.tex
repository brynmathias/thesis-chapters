\begin{abstract}
A search for supersymmetry in the exclusive hadronic and missing energy channel 
is presented on \unit{5}{\invfb} of data collected using the CMS detector at 
the LHC. The data were produced at a center-of-mass energy of \unit{7}{\TeV}. 
The kinematic discriminator \alt is used to select signal events which are then 
binned in terms of the visible energy per event.
  The efficiency of the hadronic \Lone triggers is measured though-out the data taking period and a scheme to reduce the effects of multiple collisions per bunch crossing on the cross section of the trigger paths is studied, implemented and tested in situ. These efficiency measurements are considered in the development of an analysis specific trigger, the performance of which is measured in situ, with the final efficiencies taken into account in the presented analysis. A data driven background estimation method is used to predict the expected yield in the signal regions from Standard Model processes. In the absence of an observed excess, limits are set to the $95\%$ confidence level on the production cross section and masses of new particles. In the context of the Constrained Minimal Supersymmetric Model (CMSSM), squarks and gluinos with a mass of up to \unit{1}{\TeV} are excluded. In terms of simplified models with various light and heavy flavour final states, squarks and gluinos are excluded at a mass of $\approx\unit{1}{\TeV}$ for a Lightest Supersymmetric Particle (LSP) mass of up to $\approx\unit{500}{\GeV}$. Natural units ($\hbar = c = 1$) are used though-out.


\end{abstract}


\begin{declaration}
This thesis presents work from an official CMS supersymmetry search as well as 
the measurement of the performance of the \Lone trigger system, performed from 
2010 to 2012. The work presented in Chapters~\ref{cha:level_one_trigger}, 
\ref{cha:the_t_analysis} and \ref{sub:uncertainty_on_signal_efficiency} is mine 
unless otherwise referenced and was carried out within the context of the 
Imperial College London and CERN SUSY groups as a subsection of the greater CMS 
collaboration. Contributions from others are referenced throughout.

\vspace*{1cm}

\begin{flushright}
- Bryn Lugh Shorney-Mathias, March 2013
\end{flushright}

\end{declaration}


\begin{preface}
\begin{centering}To see a World in a Grain of Sand\\
And a Heaven in a Wild Flower,\\
Hold Infinity in the palm of your hand \\
And Eternity in an hour.\\
\end{centering}

\begin{flushright}
- William Blake.
\end{flushright}
\end{preface}

\begin{acknowledgements}
So the journey is coming to an end, working at CERN, analysing some of the 
first data produced at the LHC and recorded with the CMS detector and leaving 
a mark no matter how small on the scientific world is certainly the fulfilment 
of an ambition. However I could not have done this alone and so I would like to 
thank: Robert Bainbridge and Henning Flaecher for their help and direction, 
Alex Tapper for his guidance, supervision and finding me interesting things to 
work on.

I have to thank my contemporaries and friends, Martyn Jarvis, Nick Wardle, Sam 
Rogerson, Alex Shires, Chris Parkinson, Arlo Bryer and Alex Sparrow for many 
long and in-depth discussions, some of them were even relevant. The entire of 
the RA1 team, Edward Laird, Markus Stoye, Tanja Rommerskirchen, Burt Betchart, 
Darren Burton, Chris Lucas, Oliver Buchmuller and Paris Sphicas.

Lucie Gauthier, Alex Mott, Massimiliano Chiorboli and Roberto Rossin for their 
help and guidance with all things HLT. Jim Brooke and Greg Iles for their 
expertise with the \Lone trigger.

Finally a large thanks have to go to my family for continued support and 
encouragement thoughout, also to all of my previous teachers over the years 
for keeping me on the right track.
 % and to Miri for helping me to stay sane in Geneva. 

\end{acknowledgements}

