\chapter{Conclusion} % (fold)
\label{cha:conclusion}
A search for \ac{susy} in the all hadronic channel using \alt has been 
presented. The measurement of the \Lone trigger efficiencies relevant for an all hadronic \ac{susy} search have been presented. In the region of interest these triggers have efficiencies greater than $95\%$. A suite of high level triggers are presented. These triggers are analysis specific cutting on the \alt variable and the sum of the hadronic energy in the event, \HT. This allows data collection at very low \HT compared to other HLT trigger paths, whilst maintaining a low trigger rate. The low \HT region is important for \ac{susy} searches considering models with compressed spectra as the total visible energy in the event is small. The efficiencies of these triggers are measured to high precision and these efficiencies and the error on them are taken into account in the final search. Biases and errors due to theory, detector effects and analysis choices are quantified and the impact of these uncertainties is accounted for in the final interpretation.
The total \ac{sm} background is estimated from a data control sample and the 
final number of events in the signal selection does not deviate significantly 
from this prediction.



In the absence of a signal like excess the search is interpreted in two ways, 
firstly the compatibility with the \ac{sm} only prediction, a high level of 
compatibility was observed. With a null result limits on the production cross 
section and masses of new physics models which involve a dark matter candidate 
are set, these results are summarised in Table~\ref{tab:sms_summary} for the 
simplified models. In the case of the \ac{cmssm} squarks and gluinos with 
masses up to \unit{1}{\TeV} are excluded.


\begin{table}
  \caption{The first three columns define the production and decay
    modes for various simplified models. The last two columns
    indicate the search sensitivity for these models, where
    $m_{\Psquark(\PSgluino)}^{\textrm{best}}$ and
    $m_{\textrm{LSP}}^{\textrm{best}}$ represent the largest mass
    beyond which no limit can be set for squarks/gluinos and the LSP,
    respectively\mcite{Chatrchyan:2012wa}.
  }  
  \label{tab:sms_summary}
  \centering
  %\footnotesize
  \begin{tabular}{ llccc }
    \hline
    Model & Production and decay modes & Figure & 
    $m_{\Psquark(\PSgluino)}^{\textrm{best}}$~(GeV) & $m_{
    \textrm{LSP}}^{\textrm{best}}$~(GeV) \\ [0.5ex]
    \hline
    $A$ & $\textrm{pp}\rightarrow$ \HepProcess{\PSgluino\PSgluino\to\Pquark\APquark\PSneutralino\Pquark\APquark\PSneutralino}
    & \ref{fig:t1}  & $\approx$950 & $\approx$400 \\
    $B$ & $\textrm{pp}\rightarrow$\HepProcess{\Psquark\Psquark\to\Pquark\PSneutralino\APquark\PSneutralino}
    & \ref{fig:t2}  & $\approx$750 & $\approx$275 \\
    $C$ & $\textrm{pp}\rightarrow$\HepProcess{\PStop\PStop\to\Ptop\PSneutralino\APtop\PSneutralino} & \ref{fig:t2tt}%, \ref{fig:t2tt-mlsp50} 
    & $-$ & $-$ \\ %% & \multicolumn{2}{c}{(Caption.)} \\
    $D$ &$\textrm{pp}\rightarrow$\HepProcess{\PSbottom\PSbottom\to\Pbottom\PSneutralino\APbottom\PSneutralino}
    & \ref{fig:t2bb} & $\approx$500 & $\approx$175 \\
    $E$ & $\textrm{pp}\rightarrow$\HepProcess{\PSgluino\PSgluino\to\Ptop\APtop\PSneutralino\Ptop\APtop\PSneutralino} & \ref{fig:t1tttt}  & $\approx$850 & $\approx$250 \\
    $F$ &
    $\textrm{pp}\rightarrow$\HepProcess{\PSgluino\PSgluino\to\Pbottom\APbottom\PSneutralino\Pbottom\APbottom\PSneutralino}
    & \ref{fig:t1bbbb} & $\approx$1025 & $\approx$550 \\
    \hline
  \end{tabular}
\end{table}




% chapter conclusion (end)