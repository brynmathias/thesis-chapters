\chapter{Introduction} % (fold)
\label{cha:introduction}
\paragraph{The accelerator and detectors} % (fold)
\label{par:the_accelerator_and_detectors}
The Large Hadron Collider (LHC) \cite{Benedikt:823808} is a proton-proton collider which is situated in the Large Electron Positron
(LEP) tunnel approximately 100 m under the franco-swiss border. Design center of mass energy is 14 TeV with an instantaneous
luminosity of $ 1 \times 10$e$^{34}$cm$^{-2}$s$^{-1}$. However during 2011 the center of mass energy was 7 TeV and the maximum
luminosity was $ 5 \times 10$e$^{33}$ cm$^{-2}$s$^{-1}$.
To achieve this high energy and high beam current the LHC uses superconducting niobium-titanium magnets, cooled to a temperature of 
1.8 Kelvin, that produce a maximum field strength of 8.36 Tesla.

\todo[inline]{we might well need some more stuff about the LHC its self in here!}

Situated around the LHC ring are four detectors, two general detectors ATLAS \cite{Akesson:1999uv} and CMS (see
Chapter~\ref{cha:the_cms_detector} for a detailed discussion of the CMS detector) \cite{Friedl:1140134}\cite{Wulz:vf} which are  designed to measure the \ac{sm} to high precision and search for new physics. The LHC beauty experiment 
\cite{Rademacker:2005tx} is designed to study at previously unattainable precision the decays of heavy quark 
flavors, both to measure the \ac{sm} couplings and to search for beyond the \ac{sm} (BSM) physical processes. Finally
the ALICE \cite{Alessandro:2006ht} experiment is designed to run when the LHC is running in it's secondary mode where rather than
proton bunches, lead ions are collided, in an effort to study the quark-gluon plasma.

% paragraph the_accelerator_and_detectors (end)



\paragraph{New physics} % (fold)
\label{par:new_physics}
Whilst the theory of the \ac{sm} and of new physics models will be discussed in chapter~\ref{cha:theory} it is prudent to 
discuss the observable features of these models with regard to design requirements for the general purpose detectors.
% paragraph new_physics (end)
% chapter introduction (end)