\newpage
\section{Closure tests and systematic uncertainties}
\label{sec:Closure_tests_and_systematic_uncertainties}
\subsection{Defining muon samples without an \texorpdfstring{\alt}{alphaT} requirement}
\label{app:control-noalphat}

% In this appendix, the \mj and \mmj control samples are defined by
% selection criteria that do {\it not} include an \alphat cut, in order
% to increase their acceptance and predictive power. In this case, the
% samples are collected by the \muht triggers, which requires a higher
% muon \pt threshold of 45\gev and means that the samples can only be
% used to provide translation factors and predictions for the six
% highest \HT bins, in the region $\HT > 375\gev$. For the two
% lowest \HT bins, in the region $\HT < 375\gev$, the
% predictions are given by \mj and \mmj samples for which a \htalphat
% trigger is used to collect the data and an offline \alphat cut is
% applied as usual in the event selection, as described in
% Sec.~\ref{sec:def-control-samples}. The above information is
% summarized in Table~\ref{tab:triggers-mu-mumu}. The procedure of
% relaxing the \alphat cut is currently only used in the b-tag analysis.

\begin{figure}[h!]
  \begin{center}
    \subfigure[\label{fig:munoat-to-muat-classic}\mj sample.]{
      \includegraphics[width=0.4\textwidth,angle=0]{figures/Analysis/closure_plots/alphat/1.png}
    } 
    \subfigure[\label{fig:dimunoat-to-dimuat-classic}\mmj sample.]{
      \includegraphics[width=0.4\textwidth,angle=0]{figures/Analysis/closure_plots/alphat/4.png}
    } \\
    \subfigure[\label{fig:munoat-to-muat-btag}\mj sample with 1 b-tag.]{
      \includegraphics[width=0.4\textwidth,angle=0]{figures/Analysis/closure_plots/alphat/2.png}
    } 
    \subfigure[\label{fig:munoat-to-muat-btags}\mj sample with $>$1 b-tags.]{
      \includegraphics[width=0.4\textwidth,angle=0]{figures/Analysis/closure_plots/alphat/3.png}
    }
    \caption{Closure tests that demonstrate the MC modelling of the
      \alt acceptance. The closure tests are performed for both the
      inclusive analysis with (a) the \mj sample and (b) the \mmj
      control sample. Similar tests are performed for the b-tag
      analysis using (a) the \mj sample and a requirement of exactly
      one b-tag, and (b) the \mj sample and a requirement of at least
      two b-tags. The red lines indicate the constant best fit value
      across all \HT bins.\label{fig:closure-with-alphat1}}
  \end{center}
\end{figure}

\newpage
\subsection{Closure tests for inclusive analysis\label{app:closure-incl}}

\begin{figure}[!h]
  \begin{center}
    \subfigure[\label{fig:mu-to-mumu-low}(Note the different y-axis scale.)]{\includegraphics[width=0.4\textwidth,angle=0]{figures/Analysis/closure_plots/Closure_D2a.png}}
    \subfigure[\label{fig:mu-to-mumu-high}]{\includegraphics[width=0.4\textwidth,angle=0]{figures/Analysis/closure_plots/Closure_D2b.png}} \\
    \subfigure[\label{fig:photon-to-mumu-low}]{\includegraphics[width=0.4\textwidth,angle=0]{figures/Analysis/closure_plots/Closure_D2c.png}}
    \subfigure[\label{fig:photon-to-mumu-high}]{\includegraphics[width=0.4\textwidth,angle=0]{figures/Analysis/closure_plots/Closure_D2d.png}} 
%     \subfigure[\label{fig:photon-to-mu-low}]{\includegraphics[width=0.4\textwidth,angle=0]{figures/Analysis/closure_plots/low/2.png}}
%     \subfigure[\label{fig:photon-to-mu-high}]{\includegraphics[width=0.4\textwidth,angle=0]{figures/Analysis/closure_plots/high/2.png}} 
    \caption{Closure tests using yields from one control to predict
      yields in another sample. The two plots on the left show closure
      tests which use ``low stats'' muon samples recorded with the
      \htalphat triggers and defined by offline selection criteria
      that include an \alt requirement: (a) \mj sample
      $\rightarrow$ \mmj sample and (c) \gj sample $\rightarrow$ \mmj
      sample. Similarly, the plots on the right show the same closure
      tests but using ``high-stats'' muon samples recorded with \muht
      triggers and defined with no offline \alt requirement. The
      same tests are performed: (b) \mj sample $\rightarrow$ \mmj
      sample and (d) \gj sample $\rightarrow$ \mmj sample. These
      closure tests are only possible for the six highest \HT bins
      due to the trigger conditions. The red lines indicate the
      constant best fit value across all \HT bins.}
    \label{fig:closure-with-alphat2}
  \end{center}
\end{figure}

% \newpage
% \begin{figure}[!h]
%   \begin{center}
%     \subfigure[\label{fig:mu-to-mumu-low}]{\includegraphics[width=0.4\textwidth,angle=0]{figures/Analysis/closure_plots/high/1.png}}
%     \subfigure[\label{fig:mu-to-mumu-higher}]{\includegraphics[width=0.4\textwidth,angle=0]{figures/Analysis/closure_plots/high/4.png}}
%     \caption{Closure tests which test the modelling of the \alphat acceptance
%       in the \mj and \mmj control samples for both the inclusive (left) and b-tag analysis (right)
%       The plots show the relative consistency
%       between the predictions and observations, as a function of
%       \HT. The red lines indicate a constant best fit value across
%       all bins. In all plots, the constant and its uncertainty is
%       consistent with a systematic uncertainty of 20\%, as indicated
%       by the solid grey band.}
%     \label{fig:closure-with-alphat}
%   \end{center}
% \end{figure}

\newpage
\subsection{Closure tests for b-tag analysis}

\begin{figure}[!h]
  \begin{center}
    \subfigure[\label{fig:btag-0}0 b-tags predicting 1 b-tag.]{
      \includegraphics[width=0.4\textwidth,angle=0]{figures/Analysis/closure_plots/btag/1.png}
    } \\
    \subfigure[\label{fig:btag-1}1 b-tag predicting $>$1 b-tags.]{
      \includegraphics[width=0.4\textwidth,angle=0]{figures/Analysis/closure_plots/btag/2.png} 
    } \\
    \subfigure[\label{fig:btag-2}0 b-tags predicting $>$1 b-tags.]{
      \includegraphics[width=0.4\textwidth,angle=0]{figures/Analysis/closure_plots/btag/3.png}
    }
    \caption{Closure tests with the \mj sample that demonstrate the MC
      modelling of the b-tagging algorithm and of different sample
      compositions by for different b-tag multiplicities: (a) 0 b-tags
      $\rightarrow$ 1 b-tag, (b) 1 b-tags $\rightarrow$ $\geq$2
      b-tags, (c) 0 b-tags $\rightarrow$ $\geq$2 b-tags.}
  \label{fig:btag-multi-closure}
  \end{center}
\end{figure}

\newpage
\begin{figure}[!h]
  \begin{center}
    \subfigure[\label{fig:0btag-mu-to-mumu}0 b-tags.]{
      \includegraphics[width=0.4\textwidth,angle=0]{figures/Analysis/closure_plots/high/4.png}
    } \\
    \subfigure[\label{fig:1btag-mu-to-mumu}1 b-tag.]{
      \includegraphics[width=0.4\textwidth,angle=0]{figures/Analysis/closure_plots/btag/closure/1.png}
    }
    \caption{Closure tests using the \mj sample to predict the yields in
      a \mmj sample, for events with (a) exactly 0-b-tags and (b)
      exactly 1-b-tags.}
    \label{fig:btag-samples}
  \end{center}
\end{figure}

\newpage
\subsection{Closure tests concerning pile-up\label{closure-pileup}}

\begin{figure}[!h]
  \begin{center}
    \subfigure[\label{fig:alphat-without}]{
      \includegraphics[width=0.4\textwidth,angle=0]{figures/Analysis/closure_plots/pileup/without/1.png}
    } 
    \subfigure[\label{fig:alphat-with}]{
      \includegraphics[width=0.4\textwidth,angle=0]{figures/Analysis/closure_plots/pileup/with/1.png}
    } \\
    \subfigure[\label{fig:sample-without}]{
      \includegraphics[width=0.4\textwidth,angle=0]{figures/Analysis/closure_plots/pileup/without/2.png}
    } 
    \subfigure[\label{fig:sample-with}]{
      \includegraphics[width=0.4\textwidth,angle=0]{figures/Analysis/closure_plots/pileup/with/2.png}
    } \\
    \subfigure[\label{fig:composition-without}]{
      \includegraphics[width=0.4\textwidth,angle=0]{figures/Analysis/closure_plots/pileup/without/3.png}
    } 
    \subfigure[\label{fig:composition-with}]{
      \includegraphics[width=0.4\textwidth,angle=0]{figures/Analysis/closure_plots/pileup/with/3.png}
    } 
    \caption{Closure tests using yields from one control to predict
      yields in another sample. The three plots on the left show
      closure tests from the inclusive analysis, which uses jets that
      are not corrected for the effects of pile-up. On the right, the
      jets in the analysis {\it are} corrected for pile-up effects by
      applying the \texttt{L1Offset} jet energy correction. The three
      closure tests are: probing the MC modelling of the \alt
      acceptance with the \mj sample (a) without and (b) with
      \texttt{L1Offset} jet energy corrections; using the \mj sample
      to predict yields in the \mmj sample (a) without and (b) with
      \texttt{L1Offset} jet energy corrections; and using a 0 b-tagged
      \mj sample to predict yields in a 1 b-tagged \mj sample (a)
      without and (b) with \texttt{L1Offset} jet energy
      corrections. The red lines indicate the constant best fit value
      across all \HT bins.}
    \label{fig:closure-pileup}
  \end{center}
\end{figure}

% \newpage
% \subsection{Systematics from closure tests (weighted option)}
% \begin{figure}[!h]
%   \begin{center}
%     \subfigure[\label{fig:}Inclusive, with weights (6/20/39)]{
%       \includegraphics[width=0.75\textwidth,angle=0]{figures/Analysis/syst_plots/incl.pdf}
%     } \\
%     \subfigure[\label{fig:}b-tag, with weights (8/29/53)]{
%       \includegraphics[width=0.75\textwidth,angle=0]{figures/Analysis/syst_plots/btag.pdf}
%     } 
%     \caption{(Following text is for internal comsumption only.) Top
%       plot: closure tests for inclusive analysis (open symbols), plus
%       weighted means for each HT bin (solid circles), plus systematics
%       (grey bands) for different HT regions, cooked up by combining
%       all closure tests in given region. Weight is $1/e^2$ where $e$
%       is stat error on each measurement. Systematic is defined by 3
%       times the std. dev. of all the closure tests in a given region,
%       independently for the 3 different HT regions. A choice of 3sigma
%       is rather conservative, but necessary in order to adequately
%       cover the small biases. Bottom plot: same as above, but for
%       b-tag analysis. Numbers in parentheses below each plot are the
%       values of the systematics per HT region, given in \%. Arguably,
%       different HT regions could be defined (eg, two regions: bottom
%       five bins, top three bins).}
%     \label{fig:closure-summary}
%   \end{center}
% \end{figure}

% \newpage
% \subsection{Systematics from closure tests (unweighted option)}
% \begin{figure}[!h]
%   \begin{center}
%     \subfigure[\label{fig:}Inclusive, no weights (11/33/50)]{
%       \includegraphics[width=0.75\textwidth,angle=0]{figures/Analysis/syst_plots/incl-rms.pdf}
%     } \\
%     \subfigure[\label{fig:}b-tag, no weights (18/70/136)]{
%       \includegraphics[width=0.75\textwidth,angle=0]{figures/Analysis/syst_plots/btag-rms.pdf}
%     } 
%     \caption{(Following text is for internal comsumption only.) Top
%       plot: closure tests for inclusive analysis (open symbols), plus
%       unweighted averages for each HT bin (solid circles), plus
%       systematics (grey bands) for different HT regions, cooked up by
%       combining all closure tests in given region. Stat error of each
%       closure test is not considered. Instead, systematic is defined
%       by the rms of the central values of all closure tests in a given
%       HT region, independently for 3 different HT regions. Bottom
%       plot: same as above, but for b-tag analysis. Numbers in
%       parentheses below each plot are the values of the systematics
%       per HT region, given in \%.Arguably, different HT regions could
%       be defined (eg, two regions: bottom five bins, top three bins).}
%     \label{fig:closure-summary}
%   \end{center}
% \end{figure}






