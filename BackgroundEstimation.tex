\chapter{Background Prediction} % (fold)

The raw number of events in the signal region, where \altg and \HT $ > $ \unit{275}{\GeV}, is meaningless without an accurate background prediction. 
Due to the hadronic nature of the backgrounds, selected in turn by the choice of a purely hadronic signal region and the prevalence of multi jet topologies in the final states, the detector simulation of the stochastic missmeasurement of jet energies is not advanced enough, not due to lack of understanding of the system, however due to the non-Gaussian nature of these effects the simulation is inherently incorrect. Hence the simulation is used to form a translation factor  between two samples measured in data, a control sample which closely mimics the \ac{sm} processes producing real missing energy, but has a visible muon or photon in the final state. The background estimation methods are described and the background predictions stated in this chapter. The backgrounds fall into the three catogories:

By far the largest of these backgrounds arrises from QCD multi-jet events where 
fake missing energy is introduced either from failures in reconstruction, or 
stochastic fluctuations in the calorimeter systems.
However due to the theoretical errors on the QCD production cross section 
predicting the number QCD background events from Monte Carlo simulation is not 
possible. A secondary QCD background also exists, where due to the requirement 
of a jet \ET threshold, multiple jets fall under threshold by a few \GeV, this 
causes a balanced event to look unbalanced as the jets under threshold are no 
longer considered. It is these sources of QCD background that \alt is designed 
to remove, for a detailed explanation of the \alt variable see 
Section~\ref{sec:the_alpha___t_variable_}.

The second major background comes from \ac{sm} Electroweak decays and 
is irreducible, as the final states involve real missing energy from 
neutrinos. The Electroweak decays that form the back ground are 
\HepProcess{\PW\to\Ptau\Pnu} + Jets, where the $\tau$ is reconstructed as a 
jet, or the lepton fails the identification required for the dedicated lepton 
vetoes. \HepProcess{\PZ\to\Pnu\APnu} + Jets is completely irreducible. These 
are generally di-jet topologies. At higher jet multiplicities top quark 
production, followed by semi-leptonic top decay accounts for the largest 
background. These backgrounds are predicted using a well understood control 
sample this is fully explained in 
Section~\ref{sec:electro_weak_background_prediciton}.

The final background source is that introduced by detector failure or 
electronic noise.
Approximately 1$\%$ of the ECAL read out is not available in offline event 
reconstruction, this provides a source of fake missing energy. These generally add to the QCD background but are separated due to the cause being systematic due to hardware rather than intrinsic factors of the physical processes involved or the selection criteria applied.



\section{Electroweak Background Prediction} % (fold)
\label{sec:electro_weak_background_prediciton}

The requirement of an \alt cut on the signal sample removes multi-jet QCD 
events where a balanced event is counted as signal due to missmeasurements. 
The remaining background events in the signal region are due to Electroweak 
processes which produce real missing energy. Primarily these events are 
produced from \HepProcess{\PZ\to\Pnu\APnu}~+~Jets, 
\HepProcess{\PW\to\Pl\APnu}~+~Jets and \HepProcess{\Ptop\APtop} decay, with 
smaller contributions from Drell-Yan + Jets, single top production in the s, t 
channels and from di-boson + Jet events. To predict the number of these 
events contributing to the number of signal like events three control samples 
are defined and though the use of a Monte Carlo derived transfer factor the 
control samples are used to predict the number of \ac{sm} events expected in 
the signal region.


Figure~\ref{fig:figures_Analysis_MCexpectation}\mcite{CMS-AN-11-517} shows the 
expectation from simulation in all bins of the hadronic signal region, for the 
different background samples and an example CMSSM reference model RM1 where $m_0=\unit{320}{\GeV}$, $m_{1/2}=\unit{520}{\GeV}$, $A_0=0$ and $\tan\beta=10$. 
The expected composition of the backgrounds in the signal region, 
as a percentage of the total \ac{sm} background, are summarised in 
Table~\ref{tab:bg-compositions}. \HepProcess{\PZ\to\nu\nu} contributes 
$\approx$43$\%$ of the \ac{sm} background in the 
\unit{275}{\GeV}--\unit{325}{\GeV} \HT bin, rising to $\approx$53$\%$ in the 
\HT $>$ \unit{875}{\GeV} bin. Events entering the signal region due to \PZ or 
\PW decays where the lepton is missed by the lepton vetoes account for 
$\approx$25$\%$ at low \HT and $\approx$13$\%$ at high \HT. Events from 
hadronic \Ptau ($\tau_{h}$) decays have little \HT dependance, contributing 
$\approx$22$\%$ of the background at low \HT and $\approx$27$\%$ of the 
background at high \HT. Those events arising from \Ptau particles which decay leptonically ($\tau_{l}$) and are missed by the lepton vetoes account for $\approx$10$\%$ of the \ac{sm} background.

\begin{figure}[ht|]
  \centering
    \includegraphics[width=0.5\textwidth]{figures/Analysis/MCexpectation.pdf}
  \caption{Expectation from MC in all bins of the hadronic signal
      region for the following different background processes: \HepProcess{\PZ\to\nu\nu}\
      + jets, W + jets, \HepProcess{\Ptop\APtop}, single top + jets, di-boson production
      (WW/WZ/ZZ), and Drell-Yan. The total SM expectation is also
      shown, along with that for the CMSSM benchmark model RM1.}
  \label{fig:figures_Analysis_MCexpectation}
\end{figure}

\begin{table}[th|]
  \caption{Relative background composition as given by MC simulation
    in all bins of the hadronic signal region (expressed as a
    percentage of the total SM background).} 
  \label{tab:bg-compositions}
  \centering
  % \footnotesize
  \begin{tabular}{ |c|c|c|c|c|c|c| }
    \hline
    \HT GeV & N$_{\rm events}$ & \HepProcess{\PZ\to\nu\nu} + jets & \multicolumn{4}{c}{W + jets, \HepProcess{\Ptop\APtop}, single top, DY and di-boson}  \\ [0.5ex]
    \cline{4-7}
    &                & (\%)           & missed e, $\mu$ & $\tau_{h}$ ($\%$) & $\tau_{l}$ (\%) & $\tau_{h}$ matched \\
    &                &                & from W/Z (\%)   &               &               & to jet (\%)      \\ [0.5ex]
    \hline
    275--325      & 3938.0 & 43 & 24 & 22 & 11 & 7  \\ 
    325--375      & 1569.9 & 46 & 25 & 22 & 9  & 7  \\ 
    375--475      & 1104.2 & 48 & 20 & 23 & 10 & 7  \\ 
    475--575      & 396.0  & 48 & 17 & 24 & 11 & 10 \\ 
    575--675      & 142.4  & 51 & 17 & 23 & 10 & 11 \\ 
    675--775      & 55.5   & 44 & 19 & 31 & 7  & 17 \\ 
    775--875      & 18.7   & 55 & 17 & 22 & 4  & 9  \\ 
    875--7000     & 13.2   & 53 & 13 & 27 & 7  & 19 \\ 
    \hline
  \end{tabular}
\end{table}

To calculate the bin-by-bin translation factors the signal selection and the 
individual control selections are applied to the Monte Carlo simulated 
background samples. The ratio of the number of accepted events is then taken, 
the signal yield prediction is computed from applying the the control 
selections to real data and multiplying the event yield by the translation 
factor. The measurements form the control samples are considered simultaneously 
in a fit defined in Section~\ref{sec:likelihood_model}. This method can be 
extended to include the exclusive b-tagged bins. For example in 
\mcite{CMS-PAS-SUS-12-028} a secondary dimension involving the number 
of \HepProcess{\Pbottom} quarks in the event is studied. However the 
\HepProcess{\Pbottom} extension is not detailed in this thesis.


Equation~\eqref{eq:traslationfactor} describes the background prediction 
method, the number of predicted signal events per \HT bin is given as the 
number of events observed in the data control selection per \HT bin multiplied 
by the ratio of the number of events observed in the simulation when the 
hadronic selection is applied to the number of events observed in simulation 
when the control selection is applied, again per \HT bin.

\begin{equation}
  N^{sig}_{prediction}\left(\HT\right) = N_{obs}^{control}\left(\HT\right) \times \frac{N_{MC}^{sig}\left(\HT\right)}{N_{MC}^{control}\left(\HT\right)}
\label{eq:traslationfactor}
\end{equation}

The three control samples used are a \HepProcess{\PW\to\Pmu\APnu}~+~Jets 
sample, a \HepProcess{\PZ\to\Pmu\Pmu}~+~Jets sample and a 
\HepProcess{\Pphoton}~+~Jets sample.

The selection criteria for each of these control samples is kept as similar to 
the signal selection as possible, so as to not introduce systematic errors from 
incorrect modelling in the simulation. The use of the ratio of the number of 
observed events in the Monte Carlo cancels the systematic effects. A systematic 
is still assigned to each translation factor to account for theoretical 
uncertainties and acceptance and instrumental effects.

Additional kinematic cuts are applied in the two muon control samples to enrich 
the \HepProcess{\PW}~+~Jets \HepProcess{\Ptop\APtop} and 
\HepProcess{\PZ}~+~Jets components in the control samples. The samples are 
defined to maximise efficiency rather than purity. Any impurities are 
accounted for in the transfer factors as the yields from all Monte Carlo samples are used. This is valid under the assumption that the \ac{sm} 
Electroweak and Drell-Yan processes are well modelled by the simulation.
The possibility of \ac{susy}-like signal contamination in the control samples is accounted for in the final likelihood, after measuring the signal acceptance 
for the control samples on simulated \ac{susy} events.

The magnitude of the systematic uncertainties on the transfer factors is 
motivated by a set of closure tests between the control samples. A transfer 
factor is produced to predict each control sample from each of the other 
control samples. No assumed systematic is applied in these closure tests, 
instead the level of agreement after accounting for the statistical uncertainty is used to set the scale of the systematic error for each \HT bin.


\subsection{The $\mu$~+~Jets control sample} % (fold)
\label{par:the_mu_jets_control_sample}
The $\mu$~+~Jets control sample is designed to mimic the events appearing in 
the signal region due to \HepProcess{\PW}~+~Jets and \HepProcess{\Ptop\APtop} 
decays where the leptons are missed offline, either due to falling out of 
acceptance or being missed by the reconstruction algorithms. Hadronic tau 
decays from high \PT W bosons are also predicted from this sample.
The additional selection criteria for this sample are designed to select events 
containing the decay \HepProcess{\PW\to\mu\nu}~+~Jets in the same kinematic 
conditions as those events entering the signal selection. Offline the event 
level discriminators, \HT and \alt, are calculated using only the hadronic 
components of the event. In order to select the \HepProcess{\PW}, exactly one 
tightly identified, isolated muon within \mETA $<$ 2.5 with \PT $>$ 
\unit{10}{\GeV} is required. The transverse mass of the muon combined with the 
missing energy of the event $M_{T}(\mu,\MET)$ is required to be larger than 
\unit{30}{\GeV}. As shown in Figure~\ref{fig:figures_Analysis_Wmunu_pfMT} the 
transverse mass cut removes a large amount of QCD whilst preserving a high 
efficiency \HepProcess{\PW} selection. Events are vetoed if for any jet $\Delta 
R(\mu,\mathrm{jet}) < 0.5$, or if a second muon candidate exists that is either loose, non-isolated or outside of acceptance if the two muons have an invariant mass within $\pm$\unit{25}{\GeV} of the \PZ mass, to suppress \HepProcess{\PZ\to\mu\mu} events.


\begin{figure}[htbp]
  \centering
    \includegraphics[width=0.5\textwidth]{figures/Analysis/Wmunu_pfMT.pdf}
  \caption{Transverse mass between the selected muon and \MET in 
           \HepProcess{\PW\to\mu\nu} events.\mcite{CMS-AN-10-264}.}
  \label{fig:figures_Analysis_Wmunu_pfMT}
\end{figure}

% paragraph the_mu_jets_control_sample (end)

\subsection{The $\mu\mu$~+~Jets control sample} % (fold)
\label{par:the_mumu_jets_control_sample}
The $\mu\mu$~+~Jets control sample is used to measure the 
\HepProcess{\PZ\to\nu\nu}~+~Jets irreducible background in the signal region. 
The process \HepProcess{\PZ\to\mu\mu}~+~Jets is identical kinematically, 
however the acceptance and the branching ratio are both smaller. The branching 
ratio due to the possible decay into one of three neutrino flavour states 
versus the requirement of a particular lepton flavour. The acceptance 
difference is due to the \PT and identification quality requirements on the 
muons which alter the acceptance between the two processes. The following selection criteria are applied on top of the common selection: Exactly two tightly identified, isolated muons, within \mETA $<$ 2.5 with \PT $>$ \unit{10}{\GeV} are required; The invariant mass of the di-muon pair is required to be within $\pm$\unit{25}{\GeV} of the \PZ mass; Events are vetoed if for any muon and jet combination $\Delta R(\mu,\mathrm{jet}) < 0.5$. As in the single muon control sample all event level quantities are calculated from the hadronic objects alone. This control sample can be used in all of the offline \HT bins.
% paragraph the_mumu_jets_control_sample (end)

\subsection{The \Pphoton~+~Jets control sample} % (fold)
\label{par:the_pphoton_jets_control_sample}
The \Pphoton~+~Jets control sample can also be used to measure the 
\HepProcess{\PZ\to\nu\nu}~+~Jets background as the \Pphoton~+~Jets process is 
kinematically similar when the photon \ET $>$ 
$\approx$~\unit{100}{\GeV}\mcite{PAS-SUS-08-002,Bern:2011pa}, again when 
calculating the event level quantities only hadronic objects are considered.
The photon sample requires the following criteria on top of the common 
selection requirements: exactly one photon with \ET $>$ \unit{150}{\GeV} to 
ensure trigger efficiency, \mETA $<$ 1.45. Events are vetoed if for any jet 
$\Delta R(\Pphoton,\mathrm{jet}) < 1.0$. Given that due to the trigger requirements the photon \ET is required to be greater than \unit{150}{\GeV} and the photon is treated as missing energy, the \alt $>$ 0.55 requirement implies a minimum \HT threshold of $\approx$~\unit{350}{\GeV}, hence the \Pphoton~+~Jets control sample can only be used in the offline region where \HT $>$ \unit{375}{\GeV}.
% paragraph the_pphoton_jets_control_sample (end)

Table~\ref{tab:total-sm-pred-mu} gives the hadronic signal yields in each of 
the offline \HT bins along with the simple background estimate from the single 
muon plus jets control sample. The full background prediction is given from the 
results of the simultaneous fit to the separate background estimates.

\begin{table}[|h]
  \caption{Total SM prediction using the \Pmu~+~Jets sample only. These
    are illustrative only, as the final prediction is provided by the
    final simultaneous fit.}
\label{tab:total-sm-pred-mu}
\centering
\footnotesize
\begin{tabular}{ |c|c|c|c|c| }
\hline
\alt bin                       & 0.55--$\infty$         & 0.55--$\infty$         & 0.55--$\infty$        & 0.55--$\infty$       \\ [0.5ex]
\HT bin (GeV)                 & 275--325               & 325--375               & 375--475              & 475--575             \\ [0.5ex] 
\hline
Hadronic selection MC             & 2872.32  $\pm$  64.44  & 1384.22  $\pm$  51.46  & 1041.38  $\pm$  12.53 & 396.13  $\pm$  19.85 \\ 
$\mu +$ jets selection MC         & 1228.90  $\pm$  46.18  & 670.50  $\pm$  38.74   & 495.14  $\pm$  7.86   & 181.65  $\pm$  9.65  \\ 
Translation factor                & 2.34  $\pm$  0.10      & 2.06  $\pm$  0.14      & 2.10  $\pm$  0.04     & 2.18  $\pm$  0.16    \\ 
$\mu +$ jets selection yield data & 1421                   & 645                    & 517                   & 169                  \\ 
Total SM prediction               & 3321.30  $\pm$  169.97 & 1331.57  $\pm$  105.45 & 1087.36  $\pm$  52.50 & 368.56  $\pm$  39.09 \\ 
Hadronic yield data               & 3703                   & 1533                   & 1043                  & 346                  \\ 
\hline
\hline
\alt bin                       & 0.55--$\infty$         & 0.55--$\infty$         & 0.55--$\infty$        & 0.55--$\infty$       \\ [0.5ex]
\HT bin (GeV)                 & 575--675               & 675--775               & 775--875              & 875--7000        \\ [0.5ex] 
\hline
Hadronic selection MC             & 142.37  $\pm$  7.61    & 55.47  $\pm$  3.51     & 18.68  $\pm$  1.45    & 13.18  $\pm$  1.15   \\ 
$\mu +$ jets selection MC         & 70.84  $\pm$  4.36     & 22.64  $\pm$  1.82     & 7.54  $\pm$  0.80     & 5.19  $\pm$  0.67    \\ 
Translation factor                & 2.01  $\pm$  0.16      & 2.45  $\pm$  0.25      & 2.4  $\pm$  0.33     & 2.54  $\pm$  0.40    \\ 
$\mu +$ jets selection yield data & 52                     & 18                     & 8                     & 1                    \\ 
Total SM prediction               & 104.50  $\pm$  16.81   & 44.09  $\pm$  11.33    & 19.83  $\pm$  7.41    & 2.54  $\pm$  3.47    \\ 
Hadronic yield data               & 122                    & 44                     & 14                    & 6                    \\ 
\hline

\end{tabular}
\end{table}

\subsection{Muon control samples without an \alt cut} % (fold)
\label{par:muon_control_samples_with_out_an_alt_cut_}
  
The requirement of an \alt value above 0.55 in the previous control samples 
limits the event yield of each of the Monte Carlo samples, increasing the 
statistical error of the prediction. This is especially evident when splitting 
the analysis into more dimensions than the \HT binning. The requirement of an 
\alt cut on the control samples means that as the muon is not seen by the 
calorimeter systems the signal trigger can also be used to collect the both the 
single and di muon background samples.
The translation factor method can be used to create a prediction from any 
sample to any other sample if and only if the modelling of the event kinematics 
and acceptances of any cuts introduces no large systematic errors. We now show 
that the Monte Carlo simulation accurately reproduces the kinematics and 
acceptance of the \alt cut when applied to Electroweak background samples, 
enabling the removal of the \alt requirement for the muon control samples.


The preselection of events in the two muon control samples ensures samples 
with negligible QCD contamination, which are enriched with 
\HepProcess{\Ptop\APtop}, \PW~+~Jets and \PZ~+~Jet events. This is shown for 
the \Pmu~+~Jets sample in Figure~\ref{fig:singleMuControl} and for the 
\HepProcess{\Pmu\Pmu}~+~Jets sample in Figure~\ref{fig:diMuControl}. In 
both sets of plots the expected number of QCD events from Monte Carlo simulation is less than one event at any \alt value for \unit{5}{\invfb} of integrated luminosity. The requirement of tight isolation on each of the muons is largely responsible for the purity of the sample, the transverse mass and di-muon mass window cuts ensure the sample is rich in Electroweak events.The agreement between data and Monte Carlo in these plots is good and the simulation models the \alt acceptance well. The systematic assigned to this acceptance is detailed in Section~\ref{sec:systematic_uncertainties_on_the_electro_weak_background_model_}.


\begin{figure}[|h]
  \centering
  \subfigure[ \alt distribution (see main caption).]{
    \label{fig:figures_AlphaT_all-mj}
    \includegraphics[width=0.4\textwidth]{figures/Analysis/muon_plots/AlphaT_all_OneMuon_375_upwards.png}
  } 
  \subfigure[ Zoomed \alt distribution (see main caption).]{
    \label{fig:figures_AlphaT_zoomed_all-mj}
    \includegraphics[width=0.4\textwidth]{figures/Analysis/muon_plots/AlphaT_Zoomed_all_OneMuon_375_upwards.png}
  } \\
  \subfigure[\HT distribution.]{
    \label{fig:figures_HT_all-mj}
    \includegraphics[width=0.4\textwidth]{figures/Analysis/muon_plots/HT_after_alphaT_55_all_OneMuon_375_upwards.png}
  } 
  \subfigure[$N_{\textrm{jet}}$ distribution.]{
    \label{fig:JetMultiplicityAfterAlphaT_all-mj}
    \includegraphics[width=0.4\textwidth]{figures/Analysis/muon_plots/JetMultiplicityAfterAlphaT_55_all_OneMuon_375_upwards.png}
  } 
  \caption{Data--MC comparisons of key variables for the muon control
    sample, for the region $\HT > 375\GeV$ and $\altg$. Bands represent the
    uncertainties due to the limited size of th MC samples.{\it The discrepancy 
    in the \alt distributions for values $\alt < 0.55$  is due to the trigger
    not being simulated in the MC simulation.}}
      \label{fig:singleMuControl}
\end{figure}


\begin{figure}[|h]
  \centering
  \subfigure[ \alt distribution (see main caption).]{
    \label{fig:figures_AlphaT_all-mmj}
    \includegraphics[width=0.4\textwidth]{figures/Analysis/mumu_plots/AlphaT_all_DiMuon_375_upwards.png}
  } 
  \subfigure[ Zoomed \alt distribution (see main caption).]{
    \label{fig:figures_AlphaT_zoomed_all-mmj}
    \includegraphics[width=0.4\textwidth]{figures/Analysis/mumu_plots/AlphaT_Zoomed_all_DiMuon_375_upwards.png}
  } \\
  \subfigure[\HT distribution.]{
    \label{fig:figures_HT_all-mmj}
    \includegraphics[width=0.4\textwidth]{figures/Analysis/mumu_plots/HT_after_alphaT_55_all_DiMuon_375_upwards.png}
  } 
  \subfigure[$N_{\textrm{jet}}$ distribution.]{
    \label{fig:JetMultiplicityAfterAlphaT_all-mmj}
    \includegraphics[width=0.4\textwidth]{figures/Analysis/mumu_plots/JetMultiplicityAfterAlphaT_55_all_DiMuon_375_upwards.png}
  } 
  \caption{Data--MC comparisons of key variables for the di-muon control
    sample, for the region $\HT > 375\GeV$ and $\altg$. Bands represent the uncertainties due to the limited size
    of MC samples.{\it The discrepancy in the \alt
      distributions for values $\alt < 0.55$ is due to the trigger
      not being simulated in the MC simulation.}}
    \label{fig:diMuControl}
\end{figure}


Moving to a selection where there is no required \alt cut means that the \alt 
trigger suite cannot be used to collect the high event yield control samples. 
Instead a trigger requiring \HT and a muon in the final state (\muht) is used, 
due to the muon trigger threshold the \PT acceptance cut is raised to 
\unit{45}{\GeV} in these control samples. The \HT requirement on these triggers 
raises to \unit{300}{\GeV} so only the offline bins with \HT $>$ 
\unit{375}{\GeV} are able to benefit from the increased background estimation 
precision, due to the larger size of the predicting sample. The efficiency for 
triggering on a single muon at \unit{45}{\GeV} is measured to be 
91.3$\pm$0.1$\%$ throughout the data taking period. For the \HT~+~single muon triggers, the \HT component of the trigger is measured to be 100$\%$ efficient throughout. In the case of the di-muon sample, as both muons have to be above \unit{45}{\GeV} and either of them could have triggered the event, the efficiency is found to be \HT-dependent  in the range of 95-97$\%$.

The muon control samples in the \HT bins where \HT $<$ \unit{375}{\GeV} are 
collected with the \alt trigger suite and the measured efficiencies are the 
same as those measured for the hadronic sample. The details of the triggers 
used for each of the muon~+~Jets control samples are listed in 
Table~\ref{tab:triggers-mu-mumu}.

\begin{table}[ht!]
  \caption{List of triggers used for the larger \Pmu~+~Jets and $\mu\mu$~+~Jets samples.}
  \label{tab:triggers-mu-mumu}
  \centering
  \footnotesize
  \begin{tabular}{ lcccccccc }
    \hline
    \HT bin (GeV) & 275--325 & 325--375 & 375--475 & 475--575 &
    575--675 & 675--775 & 775--875 & $>$875 \\ [0.5ex]
    \hline
    \alt cut & 0.55 & 0.55 & None & None & None & None & None & None \\
    Muon \pt cut & 10 & 10 & 45 & 45 & 45 & 45 & 45 & 45 \\
    Trigger & \alt & \alt & \muht & \muht & \muht & \muht & \muht & \muht \\
    
    Thresholds & Table~\ref{tab:htalphat-triggers} & Table~\ref{tab:htalphat-triggers} & Table~\ref{tab:muht-triggers} & Table~\ref{tab:muht-triggers} & Table~\ref{tab:muht-triggers} & Table~\ref{tab:muht-triggers} & Table~\ref{tab:muht-triggers} & Table~\ref{tab:muht-triggers} \\
    \hline
  \end{tabular}
\end{table}


% paragraph muon_control_samples_with_out_an_alt_cut_ (end)
% section electro_weak_background_prediction (end)

\section{Systematic uncertainties on the Electroweak background model} 
% (fold)
\label{sec:systematic_uncertainties_on_the_electro_weak_background_model_} 
As previously discussed in Section~\ref{sec:electro_weak_background_prediciton} the final background prediction is given by the simultaneous fit to the yields in the signal and control samples and the translation factors obtained from MC. The fit has some freedom via the statistical and systematic uncertainties measured for each translation factor. The measurement of the systematic uncertainties on the translation factors are thus vital for the fitting procedure.

A set of closure tests were performed on data to identify any sources of 
systematic biases introduced by the background prediction method.
To do this the individual background samples are used to predict one another 
using the same translation factor method as for the prediction of the \ac{sm} 
missing energy sources in the hadronic signal regions. The level of agreement 
is quantified in terms of the ratio $\left(N_{obs} - N_{pred}\right)/N_{pred}$, 
the statistical error from the translation factor, based on the available 
Monte Carlo sample size is combined with the statistical error on the number of 
events in the predicting sample to give the error on the closure. A 
deviation in the ratio from zero gives the level of closure per analysis bin. 
This gives a measure of any biases introduced by the background estimation 
method.

The closure tests between the background samples are designed to test the 
Monte Carlo's ability to model kinematic effects: such as the \alt acceptance; 
\Pmu acceptance and \Pphoton acceptance, instrumental effects such as,
reconstruction efficiencies and the effects of \pu on isolation and finally 
the theoretical precision of the production and decay cross sections and their 
relative contributions to the \ac{sm} background. These individual components 
are not separable by the closure tests, which instead gives a total systematic 
error estimation.

As described in Section~\ref{par:muon_control_samples_with_out_an_alt_cut_} the 
control samples which do not require an \alt cut use a \muht cross object 
trigger to collect the data events. As shown in Table~\ref{tab:muht-triggers} 
there was a period of data taking where due to the increased trigger thresholds 
the \muht triggers are unsuitable for use in the region \HT $<$ 
\unit{375}{\GeV}. Thus in the closure tests between the control samples which 
use an \alt cut and those that do not, the integrated luminosity is limited to 
\unit{3.9}{\invfb}. This causes a loss of some statistical power in these cases.

The individual closure tests and fits to the \HT dependance of the ratio 
$\left(N_{obs} - N_{pred}\right)/N_{pred}$ are shown in 
Appendix~\ref{sec:Closure_tests_and_systematic_uncertainties}.
Figures~\ref{fig:closure-with-alphat1} show the closure of the prediction 
between \Pmu~+~Jets(no \alt) $\rightarrow$ \Pmu~+~Jets(\altg) and $\mu\mu$~+~Jets(no \alt) $\rightarrow$ 
$\mu\mu$~+~Jets(\altg) for two samples, one with no requirement on the number of \Pbottom-tagged jets ($n_{b}$), which increases the precision of the measurement, and 
one requiring $n_{b} = 1$, the red line is the result of a one parameter fit. 
The level of closure shows that the Monte Carlo accurately models the \alt 
acceptance, with no significant bias.

Figure~\ref{fig:closure-with-alphat2} shows the closure between \Pmu~+~Jets 
$\rightarrow$ $\mu\mu$~+~Jets and between \Pphoton+jets $\rightarrow$ $\mu\mu$~+~Jets over the full \HT range using only \unit{3.9}{\invfb} of integrated luminosity and for \HT $>$ \unit{375}{\GeV} using the full data set. Again the red lines are the result of fitting with a one parameter fit. The muon to di-muon closure tests the value for $\sigma_{Z}/\sigma_{W}$ and any acceptance effects due to the inclusion of real missing energy in the same. The di-muon to photon closure quantifies the level of certainty of the ratio of the photon + Jets production cross section to the \PZ + Jets production cross section as well as the acceptance difference between the muon and photon requirements.

Figure~\ref{fig:btag-multi-closure} tests the closure between samples with 
differing $n_{b}$ the three tests are \Pmu~+~Jets($n_b = 0$)$\rightarrow$\Pmu~+~Jets($n_b = 
1$,no \alt), \Pmu~+~Jets($n_b = 1$)$\rightarrow$\Pmu~+~Jets($n_b > 1$,no \alt) and \Pmu~+~Jets($n_b = 
0$)$\rightarrow$\Pmu~+~Jets($n_b > 1$,no \alt). Figure~\ref{fig:btag-samples} also 
tests the closure between samples with differing $n_{b}$, here the test is 
between \Pmu~+~Jets($n_b = 0$)$\rightarrow$$\mu\mu$~+~Jets($n_b = 0$) and \Pmu~+~Jets($n_b =  
1$)$\rightarrow$$\mu\mu$~+~Jets($n_b = 1$). These tests show that the re-weighting method applied to the \Pbottom tagging efficiency and fake rates is correct and that by using these values it is possible to translate between different \Pbottom tag multiplicities.

Finally any dependence on \pu is measured by comparing a subset of the 
individual closure tests between samples which have \pu subtracted jets and 
those that do not. The example closures are \Pmu~+~Jets(no \alt)$\rightarrow$\Pmu~+~Jets(\altg),
\Pmu~+~Jets(no \alt)$\rightarrow$$\mu\mu$~+~Jets(no \alt), and \Pmu~+~Jets($n_b = 0$)$\rightarrow$\Pmu~+~Jets($n_b 
= 1$,no \alt).

\section{Systematic errors and biases on the translation factors}
 % (fold)
\label{sub:motivating_the_combined_systematic_on_the_translation_factors}
The closure tests described in the previous section are combined to give a 
total systematic uncertainty. This uncertainty is binned into three \HT 
regions \unit{275}{\GeV}$\rightarrow$\unit{575}{\GeV}, 
\unit{575}{\GeV}$\rightarrow$\unit{775}{\GeV} and $>$\unit{775}{\GeV}. In each 
of these regions all of the individual closure tests are used to calculate a 
weighted mean and variance. The systematic is defined as $3\sigma$ of this 
variance, which is conservative but necessary to cover any biases. The 
systematics are treated as fully uncorrelated between the three regions, again 
this is the conservative approach. Figure~\ref{fig:closure-summary} shows the 
key example closures, the grey shaded region shows the systematic error. The 
values obtained for the error are 6$\%$, 20$\%$ and 39$\%$, these are rounded 
to 10$\%$, 20$\%$ and 40$\%$ and then used in the final background simultaneous 
fit.

\begin{figure}[!h]
  \begin{center}
    \includegraphics[width=0.75\textwidth,angle=0]{figures/Analysis/syst_plots/incl.pdf}
    \caption{A set of closure tests (open symbols) overlaid on top of
      grey bands that represent the systematic uncertainties used for
      three \HT regions in the final simultaneous fit. The solid
      circles and their errors represent the weighted mean and
      standard deviation for the five closure tests of each individual
      \HT bin.}
    \label{fig:closure-summary}
  \end{center}
\end{figure}

% subsection motivating_the_combined_systematic_on_the_translation_factors (end)

% section systematic_uncertainties_on_the_electro_weak_background_model_ (end)


\section{Estimating the residual QCD background 
component.} % (fold)
\label{sec:estimating_the_residual_qcd_background_component_}
The expected QCD contamination in the signal region where \HT $>$
\unit{275}{\GeV} and \altg from simulated background samples is 
negligible\mcite{CMS-AN-11-517}. Residual events are removed via the application 
of the detector failure and \Rmiss $<$ 1.25 filters. However due to the 
difficulty in simulating QCD multi-jet events accurately a conservative 
approach is taken where a term is inserted in the likelihood to model any 
residual QCD contamination.

The term is based on the ratio of the number of events above and below the \alt 
threshold of 0.55 in the individual \HT bins. The dependance of this ratio is 
modelled as an exponentially falling quantity:
\begin{equation}
  R_{\alt}\left(\HT\right)=\mathcal{A}_{n_{b}}e^{-\kqcd\HT}.
\end{equation}
Where $\mathcal{A}_{n_{b}}$ is the b-tag bin dependent  normalisation factor and 
\kqcd is the b-tag dependent  decay constant.

The exponential behaviour, which is shown in 
Figure~\ref{fig:side-band-fits-lin} is due to several features, the first of 
which is the improvement of the relative jet energy resolution with \HT due to the larger energies deposited in the calorimeter systems. Secondly for the region \HT $>$ \unit{375}{\GeV} the jet multiplicity rises slowly with \HT, which due to the combinatorics used in the \alt calculation, results in a narrower \alt distribution peaked at 0.5. Due to the signal region definition and the exponentially falling nature of the QCD, the QCD background component is reduced to zero above $\approx$~\unit{500}{\GeV}, thus the validity of the QCD background model above \unit{575}{\GeV} is not of consequence to the final analysis, however the model chosen is shown to be valid over the whole \HT region of the analysis as shown below.

\ac{ml} values for \kqcd and $\mathcal{A}_{n_{b}}$ are found by the final 
likelihood fit, however \kqcd is first constrained by a measurement in the 
background enriched side band regions where either the \alt cut is relaxed or 
the \Rmiss cut is inverted. Figure~\ref{fig:figures_qcd_side-bands} depicts the 
regions where \kqcd is measured, the signal region is as described before where 
\altg and \Rmiss $<$ 1.25 are required. Region B is defined by the inversion of 
the \alt cut. Region C is defined by inverting both the \alt requirement and 
the \Rmiss requirement, this region is further divided into three slices in 
\alt of $0.52 < \alt < 0.53$, $0.53 < \alt < 0.54$ and $0.54 < \alt < 0.55$, as 
the index of C$_{i}$ rises the expected amount of QCD in that control region 
increases. Finally region D has only the \Rmiss requirement inverted, region D 
is not used to constrain \kqcd, but instead to check the validity of the 
exponential model.
\begin{figure}[ht]
  \centering
    \includegraphics[width=0.5\textwidth]{figures/Analysis/qcd_plots/side-bands.pdf}
  \caption{QCD side-band regions, used for determination of \kqcd.}
  \label{fig:figures_qcd_side-bands}
\end{figure}
The fits to the individual side bands are shown in 
Appendix~\ref{sec:addition_information_on_background_estimation_methods}.
The best fit value for \kqcd of \unit{2.96~$\pm$~0.64~$\times 
10^{-2}$}{\GinveV} obtained from region $B$ is used as the central value of the 
constraint. The assumption that this method gives an unbiased estimate of \kqcd 
stems from the similarity in event kinematics in the two \alt regions.
The best fit values for the three $C_{i}$ regions are used to estimate the 
systematic uncertainty on the central value obtained from region $B$. The fit 
results show no dependance on the \alt region used to measure the number of 
events, supporting the assumption that region $B$ provides an unbiased estimate 
of \kqcd. The variation of the measured values for each $C_{i}$ slice are used 
to calculate the error on the central value, the weighted mean and standard 
deviation of the three slices in \alt are calculated to be 
\unit{1.31~$\pm$~0.26~$\times~10^{-2}$}{\GinveV}, the relative error on this 
value is 20$\%$ which is then applied to the central value to give an estimate 
of the systematic uncertainty.

The data side bands are used to provide a constrained value of \kqcd as an 
input to the final likelihood model which describes the expected number of 
background events in bins of \HT and the number of observed jets containing a 
\Pbottom quark. The value measured for \kqcd is 
\unit{2.96~$\pm$~0.61(stat)~$\pm$~0.46(sys)~$\times~10^{-2}$}{\GinveV}. The 
uncertainty values are used as penalty terms in the likelihood model which is 
described in Section~\ref{sec:likelihood_model}.

\begin{table}[!h]
  \caption{\label{tab:expo-qcd-fits} Best fit values for the
    parameters $k$ as obtained from the regions $B$, $C_1$, $C_2$, and
    $C_3$. The latter three measurements are used to calculate a
    weighted mean (identified as region $C$). Also quoted is the
    maximum likelihood value of the parameter $k$ given by the
    simultaneous fit using the sample defined by region $D$. Quoted
    errors are statistical only. From \mcite{CMS-AN-11-517}.}
  \centering
  \footnotesize
  \begin{tabular}{ ccc }
    \hline
    Side-band region     & \kqcd (\unit{$\times 10^{-2}$}{\GeV}$^{-1}$) & $p$-value \\ [0.5ex]
    \hline                            
    $B$                  & $2.96\pm0.64$                 & 0.24      \\ 
    $C_1$                & $1.19\pm0.45$                 & 0.93      \\ 
    $C_2$                & $1.47\pm0.37$                 & 0.42      \\ 
    $C_3$                & $1.17\pm0.55$                 & 0.98      \\ 
    \hline                            
    $C$ (weighted mean)  & $1.31\pm0.26$                 & -         \\ 
    $D$ (likelihood fit) & $1.31\pm0.09$                 & 0.57      \\
    \hline
  \end{tabular}
\end{table}

A final check is performed using region $D$, which requires \altg but has no 
\Rmiss cut, this introduces QCD background into the signal region. The 
likelihood fit is performed on this background enriched region and no 
constraint is applied on \kqcd which is then determined by the fit only. The 
fit is performed over the full \HT range used in the final analysis.
Figure~\ref{fig:qcd-fit} shows the resulting fit, the \ac{ml} value obtained 
for \kqcd is \unit{(1.31 $\pm$ 0.09)$\times 10^{-2}$}{\GinveV}, this value is 
in excellent agreement with the value found from the weighed mean of the 
regions $C_{i}$. The fit shows that the choice of exponential function 
used in the likelihood model is valid over the entire \HT range. This supports 
the assumption that region $B$ provides an unbiased estimate of \kqcd in the 
signal region \altg and \Rmiss $<$ 1.25.


\begin{figure}[!h]
  \begin{center}
    \includegraphics[width=0.7\textwidth,angle=0]{figures/Analysis/qcd_plots/qcd-fit.pdf}
    \caption{Comparison of the observed yields and SM expectations
      given by the simultaneous fit in bins of \HT for the
      side-band region $D$. No requirement on the number of b jets is
      made. Shown are the observed event yields in data (black dots
      with error bars representing the statistical uncertainties) and
      the expectations given by the simultaneous fit for the
      \HepProcess{\PZ\to\nu\nu}+jets process (orange dotted-dashed line); the 
      sum of all processes with genuine \MET, which are primarily 
      \HepProcess{\Ptop\APtop}, W+jets, and \HepProcess{\PZ\to\nu\nu}+jets
      (dark blue long-dashed line); and the sum of QCD and all aforementioned 
      SM processes (light blue solid line).\mcite{CMS-AN-11-517}}
    \label{fig:qcd-fit}
  \end{center}
\end{figure}

% subsection estimating_the_residual_qcd_background_component_ (end)



\section{Likelihood model} % (fold)
\label{sec:likelihood_model}
The likelihood model which is fully described in \mcite{ted} is used to gauge 
the agreement between the observed yields in the hadronic signal region and the 
predicted yields obtained from the control samples.
For the hadronic sample the likelihood is as follows, for $N$ bins in \HT, let 
$n_{i}$ represent the observed hadronic yield in \HT bin $i$, the likelihood 
for observing $n_{i}$ events is then
\begin{equation}
  L_{had} = \prod_{i}\mathrm{Pois}\left(n_{i}|b_{i}+s_{i}\right)
\label{eqn:l_had}
\end{equation}


where $b_{i}$ is the number of expected background events and $s_{i}$ the 
expected signal yield, which is dependent  on the signal model being considered.
Pois refers to the Poisson distribution of these values and is defined in Equation~\equationref{eq:pois}, $b_{i}$ is equal to 
the sum of Electroweak events expected plus the number of QCD events ($b_{i} = 
EWK_{i}+QCD_{i}$) expected per bin.

\begin{equation}
  Pois(\mathcal{X}|\lambda) = \frac{\lambda^{\mathcal{X}}e^{-\lambda}}{k!}
  \label{eq:pois}
\end{equation}


In Section~\ref{sec:estimating_the_residual_qcd_background_component_} it was 
shown that the QCD contribution can be modelled as an exponentially falling as 
a function of \HT.
\begin{equation}
  R_{\alt}(\HT) = A_{QCD}e^{-\kqcd \HT}
\end{equation}
Where $A_{QCD}$ and \kqcd have been constrained by measurements in the control regions. $QCD_{i}$ is then given as:
\begin{equation}
  QCD_{i} = m_{i}A_{QCD}e^{-\kqcd \left<\HT\right>^{i}}.
  \label{eqn:qcdi}
\end{equation}
Where $m_{i}$ is the number of events per \HT bin which fail \altg and $\left<\HT\right>^{i}$ is the mean \HT of the bin.


For the Electroweak background two components are considered, $f^{i}_{Zinv}$ which represents the fraction of the total Electroweak background contributed by \HepProcess{\PZ\to\Pnu\Pnu}~+~Jets events. This is modelled as a linear 
component:
\begin{equation}
  f^{i}_{Zinv} = f^{0}_{Zinv}+\frac{\left<\HT\right>^{i}-\left<\HT\right>^{0}}{\left<\HT\right>^{N-1}-\left<\HT\right>^{0}}\left(f^{N-1}_{Zinv}-f^{i}_{Zinv}\right),
\end{equation}
where $f^{i}_{Zinv}$ and $f^{N-1}_{Zinv}$ are floating parameters whose final values are limited between zero and one. The total number of \HepProcess{\PZ\to\Pnu\Pnu}~+~Jets events per bin is then given by:
\begin{equation}
  Z^{i}_{inv} = f^{i}_{Zinv} \times \mathrm{EWK}^{i}.
  \label{eq:fzinv}
\end{equation}
The remaining background from other Electroweak processes given by:
\begin{equation}
  Y=\left(1-f^{i}_{Zinv}\right)\times \mathrm{EWK}^{i}.
  \label{eq:noZinv}
\end{equation}

In each \HT bin $i$ there are three background measurements, $n_{\Pphoton}^{i}, 
n^{i}_{\mu}$ and $n^{i}_{\mu\mu}$, representing the event counts from the 
photon, muon and di-muon control samples respectively. Each of these yields has 
a Monte Carlo counter-part yield. From the Monte Carlo it is also possible to 
extract $MC^{i}_{Zinv}$ and $MC^{i}_{Y}$. After defining:
\begin{equation}
  r^{i}_{\Pphoton} = \frac{MC^{i}_{\Pphoton}}{MC^{i}_{Zinv}},~ r^{i}_{\mu\mu} = \frac{MC^{i}_{\mu\mu}}{MC^{i}_{Zinv}},~r^{i}_{\mu} = \frac{MC^{i}_{\mu}}{MC^{i}_{Y}}
\end{equation}
three likelihood functions are defined:
\begin{equation}
  L_{\Pphoton} = \prod_{i}Pois\left(n^{i}_\Pphoton|\rho^{j}_{\Pphoton\PZ} \cdot  r^{i}_{\Pphoton}\cdot  Z_{inv}^{i} \right),
  \label{eqn:l_photon}
\end{equation}
\begin{equation}
  L_{\mu\mu} = \prod_{i}Pois\left(n^{i}_{\mu\mu}|\rho^{j}_{\mu\mu\PZ}\cdot r^{i}_{\mu\mu}\cdot  Z^{i}_{inv}\right),
  \label{eqn:l_mumu}
\end{equation}
\begin{equation}
  L_{\mu} = \prod_{i}Pois\left(n^{i}_{\mu}|\rho^{j}_{\Pmu Y}\cdot  r^{i}_{\mu}\cdot  Y^{i}+s^{i}_{\mu}\right),
  \label{eqn:l_mu}
\end{equation}
Equation~\eqref{eqn:l_photon} is used to estimate the \ac{ml} values for 
$Z_{inv}$, which is the expectation of the number of 
\HepProcess{\PZ\to\Pnu\Pnu}~+~Jets events in the hadronic signal region, using 
the observations $n_{\Pphoton}^{i}$ in the photon control sample and the ratio 
$r^{i}_{\Pphoton}$. Similarly the number of \HepProcess{\PZ\to\Pnu\Pnu}~+~Jets 
events expected in the signal region, predicted by the 
\HepProcess{\PZ\to\Pmu\Pmu}~+~Jets are found using Equation~\eqref{eqn:l_mumu}. 
The non \HepProcess{\PZ\to\Pnu\Pnu} backgrounds are estimated by the \ac{ml} 
value for Equation~\eqref{eqn:l_mu}, in a similar way with the addition of 
$s^{i}_{\Pmu}$, which represents the signal contamination in the single muon 
control sample. The measurements and ratios are considered simultaneously 
though the relationships defined in Equations~\eqref{eqn:l_had}, \eqref{eq:fzinv} and \eqref{eq:noZinv}.
The ratios $r^{i}_{\Pphoton}, r^{i}_{\mu\mu}$ and $r^{i}_{\mu}$ are the inverse 
of the translation factors given by Equation~\eqref{eq:traslationfactor}. The 
parameters $\rho_{\Pphoton\PZ}, \rho_{\mu\mu\PZ}$ and $\rho_{\mu Y}$ 
are correction factors that account for the systematic uncertainty on each of 
the ratios and $\sigma_{\Pphoton\PZ}, \sigma_{\mu\mu\PZ}$ and 
$\sigma_{\mu Y}$. They represent the relative systematic uncertainties for 
the control sample constraints, these are accounted for in:
\begin{equation}
  L_{EWK~syst} = \prod_{j}Gaus\left(1.0|\rho_{\mu Y}^{j},\sigma_{\mu Y}^{j}\right)\times Gaus\left(1.0|\rho_{\mu\mu\PZ}^{j},\sigma_{\mu\mu\PZ}^{j}\right)\times Gaus\left(1.0|\rho_{\Pphoton\PZ}^{j},\sigma_{\Pphoton\PZ}^{j}\right)
\end{equation}
Three parameters per control sample are used to cover the eight \HT bins as 
show in Table~\ref{tab:systMap}.
\begin{table}\centering
\caption{The systematic parameters used in \HT bins.}
\label{tab:systMap}
\begin{tabular}{|l|cccccccc|}
\hline
\HT bin ($i$)         & 0 & 1 & 2 & 3 & 4 & 5 & 6 & 7 \\
\hline
syst. parameter ($j$) & 0 & 0 & 0 & 0 & 1 & 1 & 2 & 2 \\
\hline
\end{tabular}
\end{table}

Alternatively the single muon control sample can be used to constrain the total 
Electroweak background, we can define:
\begin{equation}
  r^{\prime i}_{\mu} = \frac{MC^{i}_{\mu}}{MC^{i}_{tot}}
\end{equation}
and
\begin{equation}
  L^{\prime}_{\mu} = \prod_{i}Pois\left(n^{i}_{\mu}|\rho_{\mu Y}\times r^{\prime i}_{\mu}\times EWK^{i} + s_{\mu}^{i}\right)
\end{equation}

\subsection{Signal Contamination} % (fold)
\label{par:signal_contamination}
The cross section for each model is represented by $x$ and $l$ represents the 
total recorded luminosity considered by the analysis in the signal region. The 
efficiency is defined as $\epsilon^{i}_{had}$ for the signal region and 
$\epsilon^{i}_{\mu}$ for the single muon control sample, $\delta$ represents 
the relative uncertainty on the signal as measured in the previous section, $\delta$ is taken to be fully correlated though the \HT bins. $\rho_{sig}$ is the correction factor to the signal yield which accommodates this uncertainty. 
$f$ represents an unknown multiplicative factor on the signal cross 
section, for which an allowed interval is computed.
The expected signal yield $s^{i}$ from Equation~\eqref{eqn:l_had} is defined as:
\begin{equation}
  s^{i}_{had} = f \rho_{sig}xl\epsilon_{had}^{i},
  \label{eqn:sihad}
\end{equation}
the signal contamination $s^{i}_{\mu}$ is given by:
\begin{equation}
  s^{i}_{\mu} = f \rho_{\mu}xl\epsilon_{\mu}^{i}.
  \label{eqn:simu}
\end{equation}
The systematic uncertainty on the signal is included by an additional term in the likelihood:
\begin{equation}
  L_{sig} = Gaus\left(1.0|\rho_{sig},\delta \right).
\end{equation}
% paragraph signal_contamination (end)
\subsection{The total likelihood} % (fold)
\label{par:the_total_likelihood}
The total likelihood for a given signal selection $k(\HT,n_b)$ is given by:
\begin{equation}
  L^{k} = L^{k}_{had}\times L^{k}_{\mu} \times L_{\Pphoton}^{k} \times L_{\mu\mu}^{k},
  \label{eqn:totallikelihood}
\end{equation}
each $k$ has $3+N$ nuisance parameters; $\mathcal{A}_{QCD}, f^{0}_{Zinv}, 
f^{N-1}_{Zinv}, \left\{\mathrm{EWK}^{i}\right\}^{N-1}_{i=0}$. The 11 parameters $\kqcd, 
\rho_{sig}$, and $\rho_{\Pphoton\PZ}^{k}, \rho_{\mu\mu}^{k}, \rho_{\mu}^{k}$ 
with $j = \left\{0,1,2 \right\}$ are shared between the selections, the total 
likelihood is then given as:
\begin{equation}
  L = L_{sig} \times L_{\mathrm{EWK} syst} \times \prod_{k} L^{k}_{had} \times L^{k}_{\mu} \times L^{k}_{\Pphoton} \times L^{k}_{\mu\mu}.
\end{equation}
% paragraph the_total_likelihood (end)
% section likelihood_model (end)

\section{Final Results} % (fold)
\label{sec:final_results}
The yields obtained in the hadronic signal regions are tested for their 
compatibility with the background predicted by the likelihood model both in 
terms of agreement with the \ac{sm} and interpretations in the forms of 
limits on new physics models.

To test the level of agreement with the \ac{sm} the signal terms are 
dropped from the likelihood, which is then maximised over all parameters using 
Roofit\mcite{ROOSTATS} and MINUIT\mcite{MINOS}. The individual yields and errors 
from the fits are shown in Appendix~\ref{app:enemble}. The total background and 
data yields and their errors as given by the \ac{ml} fit are show below in 
Table~\ref{tab:fit-summary}.

\begin{table}[ht!]
\caption{Comparison of the measured yields in the different \HT
  and b-jet multiplicity bins for the hadronic sample with the SM
  expectations and combined statistical and systematic uncertainties
  given by the simultaneous fit.}  
\label{tab:fit-summary}
\centering
\footnotesize
\begin{tabular}{ lllllllll }
\hline
\HT (GeV)        & 275--325             & 325--375             & 375--475             & 475--575             & 575--675             & 675--775             & 775--875             & 875--$\infty$       \\ [1.000000ex]
\hline
0 b jets SM          & $2933^{+56}_{-52}$   & $1139^{+17}_{-40}$   & $783^{+17}_{-27}$    & $261^{+14}_{-8}$     & $81.5^{+6.5}_{-6.5}$ & $34.2^{+4.0}_{-3.8}$ & $10.4^{+2.8}_{-1.8}$ & $5.3^{+1.7}_{-1.1}$ \\ 
0 b jets Data        & $2919$               & $1166$               & $769$                & $255$                & $91$                 & $31$                 & $10$                 & $4$                 \\ 
\hline
1 b jet SM          & $630^{+26}_{-25}$    & $271^{+10}_{-16}$    & $202^{+10}_{-6}$     & $78.0^{+6.9}_{-1.9}$ & $24.2^{+2.9}_{-2.0}$ & $10.6^{+1.7}_{-1.3}$ & $2.9^{+0.9}_{-0.5}$  & $2.2^{+0.7}_{-0.4}$ \\ 
1 b jet Data        & $614$                & $294$                & $214$                & $71$                 & $20$                 & $6$                  & $4$                  & $0$                 \\ 
\hline
2 b jets SM          & $162^{+13}_{-12}$    & $61.8^{+4.8}_{-6.3}$ & $58.8^{+4.8}_{-2.6}$ & $28.0^{+3.5}_{-1.1}$ & $9.0^{+1.4}_{-1.0}$  & $7.1^{+1.4}_{-1.0}$  & $0.6^{+0.3}_{-0.2}$  & $0.9^{+0.4}_{-0.2}$ \\ 
2 b jets Data        & $160$                & $68$                 & $52$                 & $19$                 & $11$                 & $7$                  & $0$                  & $2$                 \\ 
\hline
$\geq$ 3 b jets SM   & $10.5^{+3.5}_{-2.2}$ & $7.1^{+2.2}_{-1.8}$  & $5.8^{+1.4}_{-0.9}$  & $3.1^{+1.0}_{-0.7}$  & $1.7^{+0.5}_{-0.4}$  & $0.7^{+0.5}_{-0.4}$  & $0.1^{+0.1}_{-0.1}$  & $0.2^{+0.1}_{-0.1}$ \\ 
$\geq$ 3 b jets Data & $10$                 & $8$                  & $8$                  & $1$                  & $0$                  & $0$                  & $0$                  & $0$                 \\ 
\hline
\end{tabular}
\end{table}



Figures~\ref{fig:best-fit-0-btag},~\ref{fig:best-fit-1-btag},~\ref{fig:best-fit-2-btag}~and~\ref{fig:best-fit-3-btag} show comparisons of the observed yields 
and the \ac{sm} only expectations given by the simultaneous fit for 0, 
1, 2, $\ge3$ exclusive \Pbottom-tag bins. A good agreement with the standard 
model is observed in all \HT and \Pbottom-tag categories. Given the lack of a 
signal like observation, limits are set on the production masses and cross 
sections of beyond the \ac{sm} particles.   In the figures the two \ac{susy} models refer to points in the \ac{sms} plane. Model A has a heavy gluino and a relatively light neutralino, this gives a final state involving four or more jets and no preference to heavy quarks in the final state. Model D has a heavy stop and a light neutralino, given the smaller mass splitting the \HT bin that the signal falls in to is lower and due to the sbottom in the production method there are necessarily b-jets in the final state.

\begin{figure}[h!]
  \begin{center}
    \subfigure[Hadronic sample]{\includegraphics[width=0.48\textwidth]{figures/Analysis/stats_plots/v8/hadronic_signal_fit_logy55_0b}}
    \subfigure[$\mu$ + jets sample]{\includegraphics[width=0.48\textwidth]{figures/Analysis/stats_plots/v8/muon_control_fit_logy55_0b}} \\
    \subfigure[$\mu\mu$ + jets sample]{\includegraphics[width=0.48\textwidth]{figures/Analysis/stats_plots/v8/mumu_control_fit_logy55_0b}}
    \subfigure[$\gamma$ + jets sample]{\includegraphics[width=0.48\textwidth]{figures/Analysis/stats_plots/v8/photon_control_fit_logy55_0b}} \\
    \caption{\label{fig:best-fit-0-btag} Comparison of the observed
      yields and SM expectations given by the simultaneous fit in bins
      of \HT for the (a) hadronic, (b) \Pmu~+~Jets, (c)
      $\mu\mu$~+~Jets and (d) \Pphoton~+~Jets
      samples when requiring exactly zero reconstructed
      b-jets. The observed event yields in data (black dots) and the
      expectations and their uncertainties, as determined by the
      simultaneous fit, for all SM processes (light blue solid line
      with dark blue bands) are shown. For illustrative purposes only,
      an example signal model is superimposed on the SM expectation
      (magenta solid line). The expected signal contamination in the
      control samples is negligible.} 
  \end{center}
\end{figure}


\begin{figure}[h!]
  \begin{center}
    \subfigure[Hadronic sample]{\includegraphics[width=0.48\textwidth]{figures/Analysis/stats_plots/v8/hadronic_signal_fit_logy55_1b}}
    \subfigure[$\mu$ + jets sample]{\includegraphics[width=0.48\textwidth]{figures/Analysis/stats_plots/v8/muon_control_fit_logy55_1b}} \\
    \subfigure[$\mu\mu$ + jets sample]{\includegraphics[width=0.48\textwidth]{figures/Analysis/stats_plots/v8/mumu_control_fit_logy55_1b}}
    \subfigure[$\gamma$ + jets sample]{\includegraphics[width=0.48\textwidth]{figures/Analysis/stats_plots/v8/photon_control_fit_logy55_1b}} \\
    \caption{\label{fig:best-fit-1-btag} Comparison of the observed
      yields and SM expectations given by the simultaneous fit in bins
      of \HT for the (a) hadronic, (b) \Pmu~+~Jets, (c) $\mu\mu$~+~Jets and (d) \Pphoton~+~Jets
      samples when requiring exactly one reconstructed
      b-jet. The observed event yields in data (black dots) and the
      expectations and their uncertainties, as determined by the
      simultaneous fit, for all SM processes (light blue solid line
      with dark blue bands) are shown. For illustrative purposes only,
      an example signal model is superimposed on the SM expectation
      (magenta solid line). The expected signal contamination in the
      control samples is negligible.}
  \end{center}
\end{figure}


\begin{figure}[h!]
  \begin{center}
    \subfigure[Hadronic sample]{\includegraphics[width=0.48\textwidth]{figures/Analysis/stats_plots/v8/hadronic_signal_fit_logy55_2b}}
    \subfigure[$\mu$ + jets sample]{\includegraphics[width=0.48\textwidth]{figures/Analysis/stats_plots/v8/muon_control_fit_logy55_2b}} \\
    \subfigure[$\mu\mu$ + jets sample]{\includegraphics[width=0.48\textwidth]{figures/Analysis/stats_plots/v8/mumu_control_fit_logy55_2b}}
    \subfigure[$\gamma$ + jets sample]{\includegraphics[width=0.48\textwidth]{figures/Analysis/stats_plots/v8/photon_control_fit_logy55_2b}} \\
    \caption{\label{fig:best-fit-2-btag} Comparison of the observed
      yields and SM expectations given by the simultaneous fit in bins
      of \HT for the (a) hadronic, (b) \Pmu~+~Jets, (c) $\mu\mu$~+~Jets and (d) \Pphoton~+~Jets
      samples when requiring exactly two reconstructed
      b-jets. The observed event yields in data (black dots) and the
      expectations and their uncertainties, as determined by the
      simultaneous fit, for all SM processes (light blue solid line
      with dark blue bands) are shown. For illustrative purposes only,
      an example signal model is superimposed on the SM expectation
      (magenta solid line). The expected signal contamination in the
      control samples is negligible.}
  \end{center}
\end{figure}


\begin{figure}[h!]
  \begin{center}
    \subfigure[Hadronic sample]{\includegraphics[width=0.48\textwidth]{figures/Analysis/stats_plots/v8/hadronic_signal_fit_logy55_gt2b}}
    \subfigure[$\mu$ + jets sample]{\includegraphics[width=0.48\textwidth]{figures/Analysis/stats_plots/v8/muon_control_fit_logy55_gt2b}} \\
    \caption{\label{fig:best-fit-3-btag} Comparison of the observed
      yields and SM expectations given by the simultaneous fit in bins
      of \HT for the (a) hadronic and (b) \Pmu~+~Jets samples when
      requiring at least three reconstructed b-jets. The observed
      event yields in data (black dots) and the expectations and their
      uncertainties, as determined by the simultaneous fit, for all SM
      processes (light blue solid line with dark blue bands) are
      shown. For illustrative purposes only, an example signal model
      is superimposed on the SM expectation (magenta solid line). The
      expected signal contamination in the \Pmu~+~Jets control sample is
      negligible.}
  \end{center}
\end{figure}


% subsection agreement_with_the_standard_model (end)