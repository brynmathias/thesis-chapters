\chapter{The $\alpha_{T}$ analysis} % (fold)
\label{cha:the_t_analysis}
\paragraph{Declaration} % (fold)
\label{par:declaration}

% paragraph declaration (end)

In this chapter we discuss the main analysis performed as the subject of this 
thesis. For the theoretical motivations of this search please see 
Chapter~\ref{cha:theory}. The analysis is based on the full 2011 data set which is made up of \unit{5}{\invfb} of \unit{7}{\TeV} data. However \unit{5}{\invfb} of the 2012 \unit{8}{\TeV} is looked at to measure the performance of the upgraded \alt HLT paths.
\section{The Problem} % (fold)
\label{sec:the_problem}
If Supersymmetry or some other beyond the \ac{sm} theory is to provide a 
yet undiscovered dark matter candidate, it is predicted that this candidate 
will interact via the weak nuclear force only. This gives a decay topology 
involving missing energy in the form of the dark matter particle escaping the 
detector. Due to the nature of interactions at the L.H.C, these particles would 
be produced at the end of a decay chain of heavy particles that interact 
strongly, giving a final topology involving hadronic objects which are 
classified as jets for the purpose of analysis and missing energy.
There are several \ac{sm} processes that mimic this final state.

By far the largest of these backgrounds comes from QCD multi jet events where 
fake missing energy is introduced either from failures in reconstruction, or 
stochastic fluctuations in the calorimeter systems.
\todo[inline]{expand on this - E /sqrt E has non gaussian tails. Figures of 
jets falling below threshold, missed jets etc. probably from some jet-met 
paper.}
However due to the theoretical errors on the QCD production cross section 
predicting the number QCD background events from Montecarlo simulation is not 
possible.
A secondary QCD background also exists, where due to the requirement of a jet 
\ET threshold, multiple jets fall under threshold by a few \GeV, this causes a 
balanced event to look unbalanced as the jets under threshold are no longer 
considered. It is these events that \alt is designed to remove.

The second major background comes from \ac{sm} electro-weak decays and 
is irreducible as the final states involve real missing energy, from 
neutrinos. The electro-weak decays that form the back ground are 
\HepProcess{\PW\to\Ptau\Pnu} + Jets, where the $\tau$ is reconstructed as a 
jet, or the lepton fails the identification required for the dedicated lepton 
vetoes, \HepProcess{\PZ\to\Pnu\APnu} + Jets is completely irreducible. These 
are generally di-jet topologies. At higher jet multiplicities top quark 
production followed by semi-leptonic top decay accounts of the largest 
background. These backgrounds are predicted using a well understood control 
sample this is fully explained in 
Section~\ref{sec:electro_weak_background_prediction}.


The final background source is that introduced by detector failure or 
electronic noise induced by the movement of the L.H.C proton beam.
Approximately 1$\%$ of the ECAL read out is not available in offline event 
reconstruction, this provides a source of fake missing energy.
% section the_problem (end)

\section{The \alt variable.} % (fold)
\label{sec:the_alpha___t_variable_}
\alt is inspired by Ref~\cite{Randall:2008dk} and was expanded to transverse 
multi jet topologies by members of the CMS collaboration in 
Refs~\cite{cms-pas-sus-08005,cms-pas-sus-09001}. The purpose is to provide a 
variable that can be cut on to eliminate QCD from the final selection. To do 
this the inherent balance of the QCD system is exploited.

For di-jet systems \alt is defined as:

\begin{equation}
  \alt = \frac{\ET^{j_{2}}}{M_{T}}
\end{equation}
where \ET$^{j_{2}}$ is the transverse energy of least energetic of the two jets 
and M$_{T}$ is defined as:

\begin{equation}
  M_{T} = \sqrt{\left(\sum^{2}_{i=1}\ET^{j_{i}}\right)^{2} - \left(\sum^{2}_{i=1}p_{x}^{j_{i}}\right)^{2} - \left(\sum^{2}_{i=1}p_{y}^{j_{i}}\right)^{2}}
\end{equation}

For a perfectly measured di-jet system with \ET$^{j_{1}} = $\ET$^{j_{2}}$, 
where the jets are opposite in $\phi$ \alt = 0.5, for events with back to back 
jets where one is miss-measured \alt $ < 0.5$.
However the majority of signals predict many jets in the final state.
\alt can be generalised to work with n-jets in the flowing way.
The variables \HT, \HTm and $\Delta$\HT are constructed:
\begin{equation}
  \HT = \sum_{i = 0}^{n~jets}\ET^{jet_{i}} \\
  \HTm = \left|\sum_{i = 0}^{n~jets}\bar{p}_{T}^{jet_{i}}\right|
\end{equation}
for jets above some predefined threshold \ET which is common for all jet based 
quantities. The multi jet system is reduced to a pseudo di-jet system by 
forming two large jets. The individual jet \ET's are summed, with the final 
configuration being chosen to have the minimum difference in energy 
($\Delta$\HT) between the pseudo jets. This simple clustering criteria provides 
the best separation between miss-measured events and those with real \MET.

\alt is then defined as:
\begin{equation}
  \alt = \frac{\HT - \Delta\HT}{2\sqrt{\HT^{2}-\HTm^{2}}}
\end{equation}


Figure~\ref{fig:figures_Analysis_AlphaT_all_375_upwards} shows the \alt 
distribution for both data and simulated background samples. The QCD multi jet 
background is negligible above an \alt value of 0.55, where as the standard 
model processes which involve real \MET exist at all possible values of \alt.
Values of \alt in the range $0.5 < \alt < 0.55$ arise in multi jet QCD due to 
jets falling below threshold or large stochastic fluctuations.
It is to be noted that the discrepancy between data and simulation for \alt 
$\leq 0.55$ is due to no trigger emulation being applied to the simulated 
background samples.
\begin{figure}[ht|]
  \centering  \includegraphics[width=0.8\textwidth]{figures/Analysis/AlphaT_all_375_upwards.pdf}
  \caption{\alt distribution for background and data for the region \HT $>$ 
  \unit{375}{\GeV}. Trigger emulation is not applied in the simulated 
  background which leads to the discrepancy in the region \alt $\leq 0.55$. The 
  QCD multi-jet background is reduced to less than one event.}
  \label{fig:figures_Analysis_AlphaT_all_375_upwards}
\end{figure}

% section the_alpha___t_variable_ (end)

\section{Event selection} % (fold)
\label{sec:event_selection}
In order to select events for the hadronic signal sample and the muon and 
photon control samples a common set of section cuts is defined. In this section 
the objects are defined as are the flow of the analysis cuts and filters.

\paragraph{Preselection of hadronic objects} % (fold)
\label{par:Preselection_of_hadronic_objects}
Jets are created by running the \AK jet clustering 
algorithm\cite{Cacciari:2008ua} over the calorimeter towers. More detail of the 
jet reconstruction at \ac{cms} is given in Section~\ref{sec:hadronic_jets}. The 
jets have their raw energies corrected based on their position and momentum to 
establish a uniform relative response in $\eta$ and a calibrated absolute 
response in transverse energy \ET, with an associated uncertainty of between 
2$\%$ and 4$\%$ dependant on \ET and $\eta$\cite{Chatrchyan:2011ds}. Jets 
considered in the analysis are required to have \ET $>$ \unit{50}{\GeV}, the 
highest \ET jet in the events is required to be within tracker acceptance 
($|\eta| <$ 2.5) and the sub leading jet is required to have \ET $>$ 
\unit{100}{\GeV}. In the lowest two offline \HT bins the jet thresholds are scaled to preserve the jet multiplicity, for the bin \unit{275}{\GeV} $<$ \HT < \unit{325}{\GeV}, the jet threshold is \unit{36.$\dot{6}$}{\GeV} and the sub leading jet threshold is \unit{73.$\dot{3}$}{\GeV}. In the bin \unit{325}{\GeV} $<$ \HT < \unit{375}{\GeV}, the jet threshold is \unit{43.$\dot{3}$}{\GeV} and the sub leading jet threshold is \unit{86.$\dot{6}$}{\GeV}.

The quantities \HT and \HTm are then formed from these jets.
% paragraph definition_of_event_level_jets (end)

\paragraph{Elections} % (fold)
\label{par:elections}
Elections are reconstructed as described in Section~\ref{sec:electrons}. For 
the purpose of the analysis an object is defined as an electron if it has \PT 
$>$ \unit{10}{\GeV}, \mETA $<$ 2.5 and passes ``VBTF'' working point 95 quality 
criteria\cite{PAS-EGM-10-004}. Any event containing an identified electron is 
vetoed. 
% paragraph elections (end)

\paragraph{Muons} % (fold)
\label{par:muons}
Muon reconstruction at \ac{cms} is detailed in Section~\ref{sec:muons}. Signal 
events are vetoed if they contain a muon with \PT $>$ \unit{10}{\GeV} and \mETA 
$<$ 2.5 that passes the ``VBTF'' working point 95 for muons 
\cite{PAS-MUO-10-002}. Two separate muon control samples are defined, one 
requiring exactly one muon with \PT $>$ \unit{10}{\GeV}, the second requiring 
two muons, who's invariant mass sums to the \PZ mass. 
% paragraph muons (end)


\paragraph{Photons} % (fold)
\label{par:photons}
Photon reconstruction as \ac{cms} is detailed in Section~\ref{sec:photons}, 
signal events are vetoed if they contain a photon with \ET $>$ \unit{25}{\GeV}, 
\mETA $<$ 2.5 and that passes ID requirements\cite{PAS-EGM-10-006}. A 
requirement of exactly one photon with \ET $>$ \unit{150}{\GeV} with in \mETA 
$<$ 1.45 is made for the photon control sample.
% paragraph photons (end)


\todo[inline]{PUT SOME STUFF IN ABOUT THE CROSS CLEANING}

The use of these control samples will be discussed in Section~\ref{sec:electro_weak_background_prediction}. 

The common selection cuts and filters consist of:

\begin{itemize}
\item \textbf{Good run selection}, All detector subsystems on, \ac{cms} in 
``Physics Declared'' mode and all physics object groups have certified the runs 
and luminosity sections. This removes any events where the sub-detectors were 
in an error state or events from before the tracker was switched to high 
voltage mode.
\item \textbf{P.K.A.M (Previously Known As Monsters) filter}, these events are 
caused by beam-gas interactions close to \ac{cms}, which cause a shower of 
particles to enter the pixel detector along the beam line, resulting in a large 
proportion of the pixel detector to record hits.
\item \textbf{Vertex Selection} requires at least one non-fake vertex with at 
least four associated tracks, within a cylinder of radius \unit{2}{\cm} and 
length \unit{48}{\cm}, centred at Z = 0 of the \ac{cms} detector.
\item \textbf{Hadronic barrel and end-cap noise filter}, this filter removes 
events where strips of towers in the hadronic calorimeters record energy from 
electrical noise, mimicking large, unbalanced energy deposits.
% \item \textbf{Cleaned Rec hit filter}, removes events where the sum of cleaned energy from calorimeter towers is greater than \unit{30}{\GeV}.
\item \textbf{Vertex \PT/\HT $>$ 0.1}, removes events where the sum of the 
\PT of all tracks from all good vertices is less than 10$\%$ of the energy 
deposited by jets in the calorimeters. This cut is designed to remove events 
with tracking failure, which would otherwise pass the calorimeter only event 
quality requirements.
\item\textbf{Masked ECAL channel filter:} Approximately 1$\%$ of the ECAL 
crystals are masked, or have read out failure. To avoid selecting events with 
large energy miss measurement, a topological cut was devised. The first step is 
to calculate $\Delta\phi^{*}$ for each jet ($\vec{j}$) in the event, where:
\begin{equation}
  \Delta\phi^{*} = \Delta\phi\left(\vec{\slashed{E}}_{T}+\vec{j},\vec{j} \right).
  \label{eq:biasedDphi}
\end{equation}
Which gives a measure of the miss measurement of a jet, if $\Delta\phi^{*}$ is 
small, the missing energy points along the jet in the $\phi$ direction. By 
selecting the miss measured jet, full position information is preserved. If any 
jet has $\Delta\phi^{*} <$ 0.5, the number of masked ECAL crystals with in 
$\Delta R < $0.3 are summed, if there are more than 10 masked crystals adjacent 
to the jet, the event is vetoed.
\item \textbf{\Rmiss$ <$ 1.25:} The total hadronic energy in an event is 
required to be greater than \unit{275}{\GeV} which is well above the transverse 
energy threshold of \unit{50}{\GeV} for each jet. However several jets falling 
below this threshold can sum to a significant quantity of ignored energy. This 
is shown in Figure~\ref{fig:figures_Analysis_BabyMHT50-10}, here the missing 
energy calculated from jets in the range \unit{10}{\GeV} $<$ \ET $<$ 
\unit{50}{\GeV} is shown, whilst requiring that \MET $<$ \unit{20}{\GeV}. This 
shows that for a well balanced event the jets below threshold can carry greater 
than \unit{100}{\GeV} of ignored energy. R$_{miss}$ is defined as $\HTm \slash 
\MET$ and can be used to single out events where the inclusion of lower 
momentum jets does significantly improve the balance of the event. 
Figure~\ref{fig:MHTovMET_lowHT_highHT} shows for two \HT regions the R$_{miss}$ 
distribution after the application of the full cut flow, including \alt, QCD 
contamination is visible in the signal sample for R$_{miss} >$ 1.25. 
\end{itemize}


\begin{figure}[htbp]
  \centering
    \includegraphics[width=0.75\textwidth]{figures/Analysis/BabyMHT50-10.pdf}
  \caption{\HTm from jets with \unit{10}{\GeV} $<$ \ET $<$ \unit{50}{\GeV} in events with \HT $>$ \unit{350}{\GeV} and \MET $<$ \unit{20}{\GeV} in \unit{35}{\invpicobarn} of data.}
  \label{fig:figures_Analysis_BabyMHT50-10}
\end{figure}

\begin{figure}[htbp]
  \centering
  \subfigure[]{ \includegraphics[width=0.45\textwidth]{figures/Analysis/250_300GevcombinedMHTovMET_afterAlphaT_all.pdf}
  \label{fig:figures_Analysis_250_300GevcombinedMHTovMET_afterAlphaT_all}
}
 \subfigure[]{   \includegraphics[width=0.45\textwidth]{figures/Analysis/350Gev_afterDeadEcalcombinedMHTovMET_afterAlphaT_all.pdf}
\label{fig:figures_Analysis_350Gev_afterDeadEcalcombinedMHTovMET_afterAlphaT_all}
}
\caption{(a) R$_{miss}$ distribution for events in the \unit{250}{\GeV} 
$<$\HT$<$\unit{350}{\GeV} region, where due to the low \HT requirement QCD 
contamination is enriched. \\ (b) R$_{miss}$ distribution for events in the 
\HT$>$\unit{350}{\GeV} region, QCD contamination occurs at R$_{miss} >$ 1.25.}
\label{fig:MHTovMET_lowHT_highHT}
\end{figure}


This cut flow and set of object definitions define the common selection, on top 
of this an \alt cut is applied, the passing events are then binned in 8 \HT 
bins, these are \unit{275}{\GeV} - \unit{325}{\GeV}, \unit{325}{\GeV} - 
\unit{375}{\GeV}, \unit{375}{\GeV} - \unit{475}{\GeV}, \unit{475}{\GeV} - 
\unit{575}{\GeV}, \unit{575}{\GeV} - \unit{675}{\GeV}, \unit{675}{\GeV} - 
\unit{775}{\GeV}, \unit{875}{\GeV} - \unit{7}{\TeV}.
However the raw number of events in the signal region is meaningless with out 
an accurate back ground prediction, as discussed earlier taking the background 
estimation from simulation is not viable due to the requirement of high order 
theory calculations on the cross sections of \ac{sm} processes involving 
jets, these have many divergencies due to the nature of QCD. Instead the 
simulation is used to form a translation factor between two samples measured in 
data, a control sample which closely mimics the \ac{sm} processes 
producing missing energy, but has a visible muon or photon in the final state. 
The background estimation methods are described and the background predictions 
stated in Section~\ref{sec:electro_weak_background_prediction}.

% section event_selection (end)

\section{High Level triggers for the \alt analysis} % (fold)
\label{sec:high_level_triggers_for_the_alt_analysis}
The CMS trigger system has been discussed in detail in 
Section~\ref{sec:the_high_level_trigger_system} and 
Chapter~\ref{cha:level_one_trigger}, however details of analysis specific 
trigger paths were not discussed. During 2011 the first \alt specific trigger 
was designed and deployed online. The trigger was then upgraded for the higher 
luminosity and energy conditions of the 2012 data taking period.


The trigger takes advantage of cutting on two variables, \HT and \alt at low 
\HT a high \alt value cuts the trigger rate, where as at high \HT where the
trigger rate is lower the \alt requirement can be loosened.

Due to the scaling of jet thresholds in the lowest offline \HT bins as detailed 
in Section~\ref{sec:event_selection} using a fixed jet threshold would cause 
inefficiency in the lowest offline \HT bins. To over come this the trigger 
level \alt calculation is performed iteratively for all jets above a predefined 
threshold. This raises the total number of accepted events whilst adding the 
benefit of being efficient for any offline jet threshold above the minimum 
trigger jet threshold.

Due to concerns on the time taken to perform the \dHT minimisation at the 
trigger and time constraints enforced on trigger menu development, the first 
implementation calculates \alt for the first 3 jets. For higher jet 
multiplicities the variable \bt is calculated.

\begin{equation}
  \beta_{T} = \frac{\HT}{2\sqrt{\HT^{2} - \MHT^{2}}}
  \label{eqn:betaT}
\end{equation}
this gives us the relation:
\begin{equation}
  \alt \leq \beta_{T}.
\end{equation}
The decision flow is shown in Figure~\ref{fig:figures_Analysis_flowChart} and 
explained in detail below.

\begin{figure}[ht|]
  \centering
    \includegraphics[width=0.7\textwidth]{figures/Analysis/flowChart.pdf}
  \caption{Flow chart representing the steps taken to make a trigger decision 
  using the \alt trigger algorithm.}
  \label{fig:figures_Analysis_flowChart}
\end{figure}


When a level one accept is issued the trigger bits that fired are checked, if 
the event fires a L1 muon trigger it is passed to the HLT muon triggers where 
only muon reconstruction is performed, reducing the reconstruction time. The 
\alt triggers are seeded on the lowest threshold unprescaled L1 \HT trigger, 
during 2011 this was L1$\_$HTT100. Any events issuing a L1 accept and passing 
L1$\_$HTT100 undergo calorimeter jet reconstruction, the reconstruction 
algorithm is detailed in Section~\ref{sec:hadronic_jets}.

Once the jets have been formed the trigger filter is entered. Initially the 
first two jets ranked by \ET, are considered, \HT and \alt are calculated, if 
both pass the trigger thresholds the event is accepted and the full detector 
read out is performed. If either \HT or \alt is below threshold, the next jet 
in \ET order is added, if the jet collection contains more than 3 jets then the 
\bt approximation is used. All jets in the event are added until either the 
even is accepted, or there are no more jets to be added above \unit{40}{\GeV}.

The effect of switching to the \bt approximation is to accept events that have 
missing energy due to miss-measurement, when calculating \alt offline these 
events have values of \alt $< 0.5$. This introduces an impurity to the trigger 
and costs rate for events that will not be considered in the offline analysis.

\subsection{Trigger efficiency measurement} % (fold)
\label{sub:trigger_efficiency_measurement}

The performance of the \alt trigger suit is measured with respect to a sample 
collected using the muon system. This allows the measurement of efficiency of 
both the level one seed trigger and the higher level trigger at the same time 
as different sub-systems are used to collect the reference and the signal 
triggers. This is due to the exclusive use of calorimeter jets in the \alt 
trigger, if more complicated reconstruction methods which produce an event 
hypothesis were used, muons would at HLT level only be considered as jets. 
Where as during calorimeter only reconstruction, muons are not considered and 
the \pt of any muons in an event is viewed as missing energy.

The selection for the trigger efficiency measurement is the same as listed in 
Section~\ref{sec:event_selection} with the requirement of exactly one well 
identified muon with \PT $>$ \unit{45}{\GeV}, the sum of the \MET in the event 
and the muon must add to the \PW mass and finally the muon must be separated by 
at least 0.5 in $\Delta R$ to the closest jet, to avoid the muon energy 
changing the energy of the jet offline when the cross cleaning is applied.

Due to the increase in luminosity over the running period the trigger 
thresholds were increased, to ensure constant rate though out the year and the 
trigger version numbers were increased each time the trigger menu was updated.
The list of triggers considered and their reference triggers are listed in 
Tables~\ref{tab:triggers1}~and~\ref{tab:triggers2}.

The efficiency of each trigger version is measured in the required \HT bins, 
the total efficiency for each \HT bin is then calculated by combining the 
individual efficiencies using a weighted sum based on the fraction of 
the total luminosity of the sample that each version carries. This accounts for 
the change in running conditions during the data taking period and the criteria 
that the trigger suite takes constant rate over the length of then, the higher 
trigger versions and thresholds generally represent more integrated luminosity 
due to this.

The efficiency is measured as a function of the cumulative number of events, ie the efficiency at each point on the x axis is the measured efficiency if a cut were applied offline at that cut value.




\begin{table}
    \centering
\begin{tabular}{|c|c|c|}

\hline
Offline \HT bin & Signal Trigger & Reference Trigger\\
\hline
\unit{275}{\GeV} $<$ \HT $<$ \unit{325}{\GeV} & \verb|HLT_HT250_AlphaT0p53_v2| & \verb|HLT_Mu15_HT200_v2|\\
\hline
\unit{275}{\GeV} $<$ \HT $<$ \unit{325}{\GeV} & \verb|HLT_HT250_AlphaT0p53_v3| & \verb|HLT_Mu15_HT200_v3|\\
\hline
\unit{275}{\GeV} $<$ \HT $<$ \unit{325}{\GeV} & \verb|HLT_HT250_AlphaT0p53_v4| & \verb|HLT_Mu15_HT200_v4|\\
\hline
\unit{275}{\GeV} $<$ \HT $<$ \unit{325}{\GeV} & \verb|HLT_HT250_AlphaT0p53_v5| & \verb|HLT_Mu30_HT200_v1|\\
\hline
\unit{275}{\GeV} $<$ \HT $<$ \unit{325}{\GeV} & \verb|HLT_HT250_AlphaT0p55_v1| & \verb|HLT_Mu5_HT200_v4|\\
\hline
\unit{275}{\GeV} $<$ \HT $<$ \unit{325}{\GeV} & \verb|HLT_HT250_AlphaT0p55_v2| & \verb|HLT_Mu40_HT200_v4|\\
\hline
\unit{275}{\GeV} $<$ \HT $<$ \unit{325}{\GeV} & \verb|HLT_HT250_AlphaT0p58_v3| & \verb|HLT_DoubleMu8_Mass8_HT200_v4|\\
\hline
\unit{275}{\GeV} $<$ \HT $<$ \unit{325}{\GeV} & \verb|HLT_HT250_AlphaT0p58_v3| & \verb|HLT_DoubleMu8_Mass8_HT200_v5| \\
\hline
\unit{275}{\GeV} $<$ \HT $<$ \unit{325}{\GeV} & \verb|HLT_HT250_AlphaT0p60_v3| & \verb|HLT_DoubleMu8_Mass8_HT200_v4|\\
\hline
\unit{275}{\GeV} $<$ \HT $<$ \unit{325}{\GeV} & \verb|HLT_HT250_AlphaT0p60_v3| & \verb|HLT_DoubleMu8_Mass8_HT200_v5|\\
\hline
\unit{325}{\GeV} $<$ \HT $<$ \unit{375}{\GeV} & \verb|HLT_HT300_AlphaT0p52_v1| & \verb|HLT_Mu5_HT200_v4|\\
\hline
\unit{325}{\GeV} $<$ \HT $<$ \unit{375}{\GeV} & \verb|HLT_HT300_AlphaT0p52_v2| & \verb|HLT_Mu8_HT200_v4|\\
\hline
\unit{325}{\GeV} $<$ \HT $<$ \unit{375}{\GeV} & \verb|HLT_HT300_AlphaT0p52_v3| & \verb|HLT_Mu15_HT200_v2|\\
\hline
\unit{325}{\GeV} $<$ \HT $<$ \unit{375}{\GeV} & \verb|HLT_HT300_AlphaT0p53_v3| & \verb|HLT_Mu15_HT200_v3|\\
\hline
\unit{325}{\GeV} $<$ \HT $<$ \unit{375}{\GeV} & \verb|HLT_HT300_AlphaT0p53_v4| & \verb|HLT_Mu15_HT200_v4| \\
\hline
\unit{325}{\GeV} $<$ \HT $<$ \unit{375}{\GeV} & \verb|HLT_HT300_AlphaT0p53_v5| & \verb|HLT_Mu30_HT200_v1| \\
\hline
\unit{325}{\GeV} $<$ \HT $<$ \unit{375}{\GeV} & \verb|HLT_HT300_AlphaT0p53_v6| & \verb|HLT_Mu40_HT200_v3| \\
\hline
\unit{325}{\GeV} $<$ \HT $<$ \unit{375}{\GeV} & \verb|HLT_HT300_AlphaT0p53_v6| & \verb|HLT_Mu40_HT200_v4| \\
\hline
\unit{325}{\GeV} $<$ \HT $<$ \unit{375}{\GeV} & \verb|HLT_HT300_AlphaT0p54_v5| & \verb|HLT_Mu40_HT300_v4| \\
\hline
\unit{325}{\GeV} $<$ \HT $<$ \unit{375}{\GeV} & \verb|HLT_HT300_AlphaT0p54_v5| & \verb|HLT_Mu40_HT300_v5| \\
\hline
\unit{325}{\GeV} $<$ \HT $<$ \unit{375}{\GeV} & \verb|HLT_HT300_AlphaT0p55_v3| & \verb|HLT_DoubleMu8_Mass8_HT200_v4| \\
\hline
\unit{325}{\GeV} $<$ \HT $<$ \unit{375}{\GeV} & \verb|HLT_HT300_AlphaT0p55_v3| & \verb|HLT_DoubleMu8_Mass8_HT200_v5| \\
\hline

\end{tabular}
\caption{List of \alt triggers used in the lowest two offline \HT bins and the triggers used to collect the reference sample.}
\label{tab:triggers1}

\end{table}

\begin{table}
    \centering

\begin{tabular}{|c|c|c|}
    \hline
Offline \HT bin &Signal Trigger & Reference Trigger\\
\hline
\unit{375}{\GeV} $<$ \HT $<$ \unit{475}{\GeV} & \verb|HLT_HT350_AlphaT0p51_v1| & \verb|HLT_Mu5_HT200_v4| \\
\hline
\unit{375}{\GeV} $<$ \HT $<$ \unit{475}{\GeV} & \verb|HLT_HT350_AlphaT0p51_v2| & \verb|HLT_Mu8_HT200_v4| \\
\hline
\unit{375}{\GeV} $<$ \HT $<$ \unit{475}{\GeV} & \verb|HLT_HT350_AlphaT0p51_v3| & \verb|HLT_Mu15_HT200_v2| \\
\hline
\unit{375}{\GeV} $<$ \HT $<$ \unit{475}{\GeV} & \verb|HLT_HT350_AlphaT0p51_v4| & \verb|HLT_Mu15_HT200_v3| \\
\hline
\unit{375}{\GeV} $<$ \HT $<$ \unit{475}{\GeV} & \verb|HLT_HT350_AlphaT0p51_v5| & \verb|HLT_Mu15_HT200_v4| \\
\hline
\unit{375}{\GeV} $<$ \HT $<$ \unit{475}{\GeV} & \verb|HLT_HT350_AlphaT0p52_v1| & \verb|HLT_Mu30_HT200_v1| \\
\hline
\unit{375}{\GeV} $<$ \HT $<$ \unit{475}{\GeV} & \verb|HLT_HT350_AlphaT0p52_v2| & \verb|HLT_Mu40_HT200_v3| \\
\hline
\unit{375}{\GeV} $<$ \HT $<$ \unit{475}{\GeV} & \verb|HLT_HT350_AlphaT0p52_v2| & \verb|HLT_Mu40_HT200_v4| \\
\hline
\unit{375}{\GeV} $<$ \HT $<$ \unit{475}{\GeV} & \verb|HLT_HT350_AlphaT0p53_v10| & \verb|HLT_Mu40_HT300_v4| \\
\hline
\unit{375}{\GeV} $<$ \HT $<$ \unit{475}{\GeV} & \verb|HLT_HT350_AlphaT0p53_v10| & \verb|HLT_Mu40_HT300_v5| \\
\hline
\unit{475}{\GeV} $<$ \HT $<$ \unit{7}{\TeV} & \verb|HLT_HT400_AlphaT0p51_v1| & \verb|HLT_Mu5_HT200_v4| \\
\hline
\unit{475}{\GeV} $<$ \HT $<$ \unit{7}{\TeV} & \verb|HLT_HT400_AlphaT0p51_v2| & \verb|HLT_Mu8_HT200_v4| \\
\hline
\unit{475}{\GeV} $<$ \HT $<$ \unit{7}{\TeV} & \verb|HLT_HT400_AlphaT0p51_v3| & \verb|HLT_Mu15_HT200_v2| \\
\hline
\unit{475}{\GeV} $<$ \HT $<$ \unit{7}{\TeV} & \verb|HLT_HT400_AlphaT0p51_v4| & \verb|HLT_Mu15_HT200_v3| \\
\hline
\unit{475}{\GeV} $<$ \HT $<$ \unit{7}{\TeV} & \verb|HLT_HT400_AlphaT0p51_v5| & \verb|HLT_Mu15_HT200_v4| \\
\hline
\unit{475}{\GeV} $<$ \HT $<$ \unit{7}{\TeV} & \verb|HLT_HT400_AlphaT0p51_v6| & \verb|HLT_Mu30_HT200_v1| \\
\hline
\unit{475}{\GeV} $<$ \HT $<$ \unit{7}{\TeV} & \verb|HLT_HT400_AlphaT0p51_v7| & \verb|HLT_Mu40_HT200_v3| \\
\hline
\unit{475}{\GeV} $<$ \HT $<$ \unit{7}{\TeV} & \verb|HLT_HT400_AlphaT0p51_v7| & \verb|HLT_Mu40_HT200_v4| \\
\hline
\unit{475}{\GeV} $<$ \HT $<$ \unit{7}{\TeV} & \verb|HLT_HT400_AlphaT0p51_v10| & \verb|HLT_Mu40_HT300_v4| \\
\hline
\unit{475}{\GeV} $<$ \HT $<$ \unit{7}{\TeV} & \verb|HLT_HT400_AlphaT0p51_v10| & \verb|HLT_Mu40_HT300_v5| \\
\hline
\unit{475}{\GeV} $<$ \HT $<$ \unit{7}{\TeV} & \verb|HLT_HT400_AlphaT0p52_v5| & \verb|HLT_Mu40_HT300_v4| \\
\hline
\unit{475}{\GeV} $<$ \HT $<$ \unit{7}{\TeV} & \verb|HLT_HT400_AlphaT0p52_v5| & \verb|HLT_Mu40_HT300_v5| \\
\hline

\end{tabular}
\caption{List of \alt triggers used in the \HT $>$ \unit{375}{\GeV} bins and the triggers used to collect the reference sample.}
\label{tab:triggers2}

\end{table}

As an example, the efficiency of each trigger used in the \unit{275}{\GeV} $<$ 
\HT $<$ \unit{325}{\GeV} bin is measured and then combined to give the total 
efficiency. The cumulative efficiency curves for each trigger seeding the 
lowest bin is show in Figures~\ref{fig:HT275BinTriggers1}, \ref{fig:HT275BinTriggers2} $\&$ \ref{fig:HT275BinTriggers3}. Note that some of the 
triggers are repeated, due to the reference trigger version incrementing and 
the signal trigger not.

\begin{figure}[ht]
  \centering
   \subfigure[HLT$\_$HT250$\_$AlphaT0p55$\_$v1]{ \includegraphics[width=0.4\textwidth]{figures/Analysis/HT275_HLT_HT250_AlphaT0p55_v1_HLT_Mu5_HT200_v4.pdf}
\label{fig:figures_Analysis_HT275_HLT_HT250_AlphaT0p55_v1_HLT_Mu5_HT200_v4}}
\subfigure[HLT$\_$HT250$\_$AlphaT0p53$\_$v2]{   \includegraphics[width=0.4\textwidth]{figures/Analysis/HT275_HLT_HT250_AlphaT0p53_v2_HLT_Mu15_HT200_v2.pdf}
\label{fig:figures_Analysis_HT275_HLT_HT250_AlphaT0p53_v2_HLT_Mu15_HT200_v2}}

\subfigure[HLT$\_$HT250$\_$AlphaT0p53$\_$v3]{\includegraphics[width=0.4\textwidth]{figures/Analysis/HT275_HLT_HT250_AlphaT0p53_v3_HLT_Mu15_HT200_v3.pdf}
\label{fig:figures_Analysis_HT275_HLT_HT250_AlphaT0p53_v3_HLT_Mu15_HT200_v3}}

\subfigure[HLT$\_$HT250$\_$AlphaT0p53$\_$v4]{ \includegraphics[width=0.4\textwidth]{figures/Analysis/HT275_HLT_HT250_AlphaT0p53_v4_HLT_Mu15_HT200_v4.pdf}
\label{fig:figures_Analysis_HT275_HLT_HT250_AlphaT0p53_v4_HLT_Mu15_HT200_v4}}

\caption{Turn on curves for the individual \alt triggers used to seed the 
\unit{275}{\GeV} $< \HT <$ \unit{325}{\GeV} bin.}
\label{fig:HT275BinTriggers1}
\end{figure}

\begin{figure}[ht]
  \centering
\subfigure[HLT$\_$HT250$\_$AlphaT0p53$\_$v5]{ \includegraphics[width=0.4\textwidth]{figures/Analysis/HT275_HLT_HT250_AlphaT0p53_v5_HLT_Mu30_HT200_v1.pdf}
\label{fig:figures_Analysis_HT275_HLT_HT250_AlphaT0p53_v5_HLT_Mu30_HT200_v1}}
  
\subfigure[HLT$\_$HT250$\_$AlphaT0p58$\_$v3]{ \includegraphics[width=0.4\textwidth]{figures/Analysis/DiMu_HT275_HLT_HT250_AlphaT0p58_v3_HLT_DoubleMu8_Mass8_HT200_v4.pdf}
\label{fig:figures_Analysis_DiMu_HT275_HLT_HT250_AlphaT0p58_v3_HLT_DoubleMu8_Mass8_HT200_v4}}

\subfigure[HLT$\_$HT250$\_$AlphaT0p58$\_$v3]{ \includegraphics[width=0.4\textwidth]{figures/Analysis/DiMu_HT275_HLT_HT250_AlphaT0p58_v3_HLT_DoubleMu8_Mass8_HT200_v5.pdf} 
\label{fig:figures_Analysis_DiMu_HT275_HLT_HT250_AlphaT0p58_v3_HLT_DoubleMu8_Mass8_HT200_v5}}

\subfigure[HLT$\_$HT250$\_$AlphaT0p60$\_$v3]{ \includegraphics[width=0.4\textwidth]{figures/Analysis/DiMu_HT275_HLT_HT250_AlphaT0p60_v3_HLT_DoubleMu8_Mass8_HT200_v4.pdf} 

\label{fig:figures_Analysis_DiMu_HT275_HLT_HT250_AlphaT0p60_v3_HLT_DoubleMu8_Mass8_HT200_v4}}
\caption{Turn on curves for the individual \alt triggers used to seed the 
\unit{275}{\GeV} $< \HT <$ \unit{325}{\GeV} bin.}
\label{fig:HT275BinTriggers2}
\end{figure}

\begin{figure}[ht]
  \centering
\subfigure[HLT$\_$HT250$\_$AlphaT0p60$\_$v3]{ \includegraphics[width=0.4\textwidth]{figures/Analysis/HT275_HLT_HT250_AlphaT0p55_v2_HLT_Mu40_HT200_v4.pdf}
\label{fig:figures_Analysis_HT275_HLT_HT250_AlphaT0p55_v2_HLT_Mu40_HT200_v4}}
\caption{Turn on curves for the individual \alt triggers used to seed the 
\unit{275}{\GeV} $< \HT <$ \unit{325}{\GeV} bin.}
\label{fig:HT275BinTriggers3}
\end{figure}

These are then combined to give 
Figure~\ref{fig:figures_Analysis_TotalCumulativeTurnonforHT275AlphaT} the 
efficiency at a cut of \altg is 83.3$\%$ + 0.5$\%$ - 0.6$\%$. The loss 
in efficiency comes from the disparity between the minimum thresholds for jets 
to enter the \HT and \alt calculations at the HLT and those used in the 
analysis. The trigger jet \ET threshold is \unit{40}{\GeV} where as 
the analysis jet \ET threshold in this bin is 
\unit{36.$\dot{6}$}{\GeV}. The triggers used to take data at the end of this 
running period also have an \alt threshold above the analysis cut of \altg, which again causes an inefficiency. The list of efficiencies for each \HT 
bin are shown in Table~\ref{tab:altEfficiencies7TeV}. The trigger efficiencies 
are measured to better than one percent and this information is used in the 
final likelihood model, to correct the expected numbers from simulation when 
calculating the translation factors.


\begin{figure}[ht]
  \centering 
\includegraphics[width=\textwidth]{figures/Analysis/TotalCumulativeTurnonforHT275AlphaT.pdf}
  \caption{Combined cumulative efficiency for the triggers seeding the \unit{275}{\GeV} $< \HT <$ \unit{325}{\GeV} offline analysis bin.}
  \label{fig:figures_Analysis_TotalCumulativeTurnonforHT275AlphaT}
\end{figure}

\begin{table}
  \centering
\begin{tabular}{ |c|c| }
\hline
\HT range (GeV) & Trigger efficiency ($\%$)  \\ [1.0ex]
\hline
275--325 & 83.3$^{+0.5}_{-0.6}$ \\ [1.0ex]
\hline
325--375 & 95.9$^{+0.7}_{-0.9}$ \\ [1.0ex] 
\hline
375--475 & 98.5$^{+0.5}_{-0.9}$ \\ [1.0ex]
\hline
475--$\infty$ & 100.0$^{+0.0}_{-4.8}$ \\ [1.0ex]
\hline
\end{tabular}

\caption{Efficiencies of the \alt triggers used in the \unit{7}{\TeV} \alt 
         analysis on \unit{5}{\invfb} of LHC data.}
\label{tab:altEfficiencies7TeV}
\end{table}


The \alt triggers were upgraded for the increased instantaneous luminosity and 
pile up conditions expected during the 2012 data taking period. The first stage 
was to implement the full \alt calculation for each addition of a new jet, this 
change increases the purity of the trigger, meaning that for the same threshold 
the rate taken is lower than that of the previous algorithm. This allows the 
thresholds to be kept low, whilst not increasing the overall trigger rate.

The second upgrade was to switch to pile up corrected jets at the HLT, this 
change keeps the trigger cross section liner as a function of instantaneous 
luminosity.

The third choice was to design a trigger suite that could run for the entire 
2012 data taking period, with out changing the trigger thresholds, this was 
done to make the measurement of the efficiency simpler.

The trigger thresholds used are presented in 
Table~\ref{tab:2012triggerThresholds}, due to the constant thresholds in 
\HT and \alt though out the run, the versioning of the triggers is excluded 
from this table.
The efficiency of the triggers is measured for \unit{11.7}{\invfb} of LHC data 
taken at a centre of mass energy of \unit{8}{\TeV} the efficiencies were 
measured for the analysis presented at HCP, documented in 
\cite{CMS-PAS-SUS-12-028}. A single muon trigger is used to collect the 
reference sample, the threshold of this trigger was unchanged during the data 
taking period, the path selected is \verb|HLT_IsoMu24_v*| which requires at 
least one muon with \PT $>$ \unit{24}{\GeV} that is not over lapping with any 
other object in the detector.
The same method is used as for the 2011 trigger efficiency measurement. However 
the offline jets are corrected for pile up using the fast jet corrections, to 
stay inline with the HLT object definitions.

Figure~\ref{fig:2012TriggersHT275} shows the efficiencies of the four 
individual triggers that seed the \unit{275}{\GeV}--\unit{325}{\GeV} \HT bin 
for \unit{11.7}{\invfb} of \unit{8}{\TeV} LHC data, 
Figure~\ref{fig:figures_Analysis_8TeVTriggers_HT275AlphaT} shows the combined 
cumulative efficiency of the 2012 trigger suite.

\begin{table}
\begin{tabular}{|c|c|}
\hline
\HT range & Trigger \\
\hline
\unit{275}{\GeV}--\unit{325}{\GeV} & \verb|HLT_HT250_AlphaT0p55_v*| \\
\hline
\unit{325}{\GeV}--\unit{375}{\GeV} & \verb|HLT_HT300_AlphaT0p53_v*| \\
\hline
\unit{375}{\GeV}--\unit{475}{\GeV} & \verb|HLT_HT350_AlphaT0p52_v*| \\
\hline
\unit{475}{\GeV}--\unit{8}{\TeV}   & \verb|HLT_HT400_AlphaT0p51_v*| \\
\hline
\end{tabular}
\caption{Triggers used to seed the analysis \HT bins during 2012 data taking.}
\label{tab:2012triggerThresholds}
\end{table}



\begin{figure}[ht]
 \subfigure[HLT$\_$HT250$\_$AlphaT0p55$\_$v1]{\includegraphics[width=0.4\textwidth]{figures/Analysis/8TeVTriggers/HT275_HLT_HT250_AlphaT0p55_v1_HLT_IsoMu24_eta2p1_v11.pdf} \label{fig:figures_Analysis_8TeVTriggers_HT275_HLT_HT250_AlphaT0p55_v1_HLT_IsoMu24_eta2p1_v11}}

\subfigure[HLT$\_$HT250$\_$AlphaT0p55$\_$v2]{\includegraphics[width=0.4\textwidth]{figures/Analysis/8TeVTriggers/HT275_HLT_HT250_AlphaT0p55_v2_HLT_IsoMu24_eta2p1_v12.pdf}  \label{fig:figures_Analysis_8TeVTriggers_HT275_HLT_HT250_AlphaT0p55_v2_HLT_IsoMu24_eta2p1_v12}}
    \subfigure[HLT$\_$HT250$\_$AlphaT0p55$\_$v3]{\includegraphics[width=0.4\textwidth]{figures/Analysis/8TeVTriggers/HT275_HLT_HT250_AlphaT0p55_v3_HLT_IsoMu24_eta2p1_v12.pdf}  
\label{fig:figures_Analysis_8TeVTriggers_HT275_HLT_HT250_AlphaT0p55_v3_HLT_IsoMu24_eta2p1_v12}}

\subfigure[HLT$\_$HT250$\_$AlphaT0p55$\_$v4]{\includegraphics[width=0.4\textwidth]{figures/Analysis/8TeVTriggers/HT275_HLT_HT250_AlphaT0p55_v4_HLT_IsoMu24_v15.pdf} 
\label{fig:figures_Analysis_8TeVTriggers_HT275_HLT_HT250_AlphaT0p55_v4_HLT_IsoMu24_v15}}

\caption{Turn on curves for the individual \alt triggers used to seed the 
\unit{275}{\GeV} $< \HT <$ \unit{325}{\GeV} bin, during 2012 data taking.}
\label{fig:2012TriggersHT275}
\end{figure}

\begin{figure}[ht]
\centering    \includegraphics[width=0.75\textwidth]{figures/Analysis/8TeVTriggers/HT275AlphaT.pdf}
\caption{Combined cumulative efficiency for the triggers seeding the \unit{275}{\GeV} $< \HT <$ \unit{325}{\GeV} offline bin for \unit{11.7}{\invfb} 
  of \unit{8}{\TeV} LHC data.}
\label{fig:figures_Analysis_8TeVTriggers_HT275AlphaT}
\end{figure}


\begin{table}
\begin{tabular}{|c|c|}
\hline
\HT range (GeV) & Trigger efficiency ($\%$)  \\ 
\hline
\unit{275}{\GeV} -- \unit{325}{\GeV} & 89.6$^{+0.4}_{-0.4}$  \\
\hline
\unit{325}{\GeV} -- \unit{375}{\GeV} & 98.6$^{+0.2}_{-0.3}$  \\
\hline
\unit{375}{\GeV} -- \unit{475}{\GeV} & 99.4$^{+0.2}_{-0.3}$  \\
\hline
\unit{475}{\GeV} -- \unit{8}{\TeV}   & 100.0$^{+0.0}_{-0.5}$ \\
\hline
\end{tabular}
\label{tab:2012Triggerefficencies}
\caption{Efficiencies of the \alt triggers at a centre of mass energy of 
         \unit{8}{\TeV} \alt measured in \unit{11.7}{\invfb} of LHC data.}
\end{table}

The two trigger suites have very similar offline performance with 100$\%$ 
efficiency in the analysis bins above \unit{475}{\GeV} and high efficiency in 
the lower bins. To quantify the difference in performance between the two 
algorithms the purity is defined as the number of events passing the trigger 
that pass the offline \alt requirement at the same threshold as the trigger requirement divided by the total number of events 
accepted by the trigger.
The purity of each trigger algorithm is measured for an example trigger with a 
\HT threshold of \unit{350}{\GeV} and an \alt threshold of 0.52.
The 2011 trigger which only performs the full \alt calculation for jet 
multiplicities of less than four has a purity of 48$\%$, which means a quarter 
of all rate taken by the trigger is used in the offline analysis.
The 2012 trigger which performs the full \alt calculation for all jet 
multiplicities has a purity of 75$\%$. This 25$\%$ increase in purity 
translates in to a 25$\%$ rate reduction for the same trigger threshold when
changing to the full \alt calculation for all jet multiplicities, thus enabling 
the trigger thresholds to be kept at the same or lower thresholds in the 2012 
run as in the 2011 run.

% section high_level_triggers_for_the_alt_analysis (end)

\section{Extension to higher analysis dimensions.} % (fold)
\label{sec:extension_to_higher_analysis_dimensions_}
Whilst it should be noted that the author did not contribute towards the 
extension of the analysis to include a dimension based on the measurement of 
the number of jets containing \Pbottom quarks in the final state, the final 
analysis does include this extension, hence the \Pbottom-tagging procedure is 
explained in sufficient detail as to understand the process but not to elude to 
all the nuances of the method in the following section.
% section extension_to_higher_analysis_dimensions_ (end)

\clearpage
\section{Electro-Weak background prediction} % (fold)
\label{sec:electro_weak_background_prediction}
The requirement of an \alt cut on the signal sample removes multi-jet QCD 
events where a balanced event is counted as signal due to miss-measurements, 
the remaining background events in the signal region are due to electro-weak 
processes which produce real missing energy. Primarily these events are 
produced from \HepProcess{\PZ\to\Pnu\APnu}~+~Jets, 
\HepProcess{\PW\to\Pl\APnu}~+~Jets and \HepProcess{\Ptop\APtop} decay, with 
smaller contributions from Drell-Yan + Jets, single top production in the s,t 
and tW channels and from di-boson + Jet events. To predict the number of these 
events contribution to the number of signal events three control samples are 
defined and though the use of a montecarlo derived transfer factor the control 
samples are used to predict the number of \ac{sm} events expected in the 
signal region.


Figure~\ref{fig:figures_Analysis_MCexpectation}\cite{CMS-AN-11-517} shows the 
expectation from simulation in all bins of the hadronic signal region, for the 
different background samples and an example CMSSM reference model RM1. 
\todo[inline]{Expalin what RM1 is}
The expected composition of the the backgrounds in the hadronic signal region, 
as a percentage of the total \ac{sm} background, are summarised in 
Table~\ref{tab:bg-compositions}. \HepProcess{\PZ\to\nu\nu} contributes 
$\approx$43$\%$ of the \ac{sm} background in the 
\unit{275}{\GeV}--\unit{325}{\GeV} \HT bin, raising to $\approx$53$\%$ in the 
\HT $>$ \unit{875}{\GeV} bin. Events entering the signal region due to \PZ or 
\PW decays where the lepton is missed by the lepton vetoes account for 
$\approx$25$\%$ at low \HT and $\approx$13$\%$ at high \HT. Events from 
hadronicly decays \Ptau ($\tau_{h}$) have a low \HT dependance, contributing 
$\approx$22$\%$ of the background at low \HT and $\approx$27$\%$ of the 
background at high \HT. Those events arising from \Ptau particles which decay leptonically ($\tau_{l}$) and are missed by the lepton vetoes account for $\approx$10$\%$ of the \ac{sm} background.

\begin{figure}[ht|]
  \centering
    \includegraphics[width=0.5\textwidth]{figures/Analysis/MCexpectation.pdf}
  \caption{Expectation from MC in all bins of the hadronic signal
      region for the following different background processes: \HepProcess{\PZ\to\nu\nu}\
      + jets, W + jets, \HepProcess{\Ptop\APtop}, single top + jets, di-boson production
      (WW/WZ/ZZ), and Drell-Yan. The total SM expectation is also
      shown, along with that for the CMSSM benchmark model RM1.}
  \label{fig:figures_Analysis_MCexpectation}
\end{figure}

\begin{table}[th|]
  \caption{Relative background composition as given by MC simulation
    in all bins of the hadronic signal region (expressed as a
    percentage of the total SM background).} 
  \label{tab:bg-compositions}
  \centering
  % \footnotesize
  \begin{tabular}{ |c|c|c|c|c|c|c| }
    \hline
    \HT GeV & N$_{\rm events}$ & \HepProcess{\PZ\to\nu\nu} + jets & \multicolumn{4}{c}{W + jets, \HepProcess{\Ptop\APtop}, single top, DY and di-boson}  \\ [0.5ex]
    \cline{4-7}
    &                & (\%)           & missed e, $\mu$ & $\tau_{h}$ ($\%$) & $\tau_{l}$ (\%) & $\tau_{h}$ matched \\
    &                &                & from W/Z (\%)   &               &               & to jet (\%)      \\ [0.5ex]
    \hline
    275--325      & 3938.0 & 43 & 24 & 22 & 11 & 7  \\ 
    325--375      & 1569.9 & 46 & 25 & 22 & 9  & 7  \\ 
    375--475      & 1104.2 & 48 & 20 & 23 & 10 & 7  \\ 
    475--575      & 396.0  & 48 & 17 & 24 & 11 & 10 \\ 
    575--675      & 142.4  & 51 & 17 & 23 & 10 & 11 \\ 
    675--775      & 55.5   & 44 & 19 & 31 & 7  & 17 \\ 
    775--875      & 18.7   & 55 & 17 & 22 & 4  & 9  \\ 
    875--$\infty$ & 13.2   & 53 & 13 & 27 & 7  & 19 \\ 
    \hline
  \end{tabular}
\end{table}

To calculate the bin by bin translation factors the signal selection and the 
individual control selections are applied to the montecarlo simulated 
background samples. The ratio of the number of accepted events is then taken, 
the signal yield prediction is computed from applying the the control 
selections to real data and multiplying the event yield by the translation 
factor. The measurements form all control samples are considered simultaneously 
in a fit defined in Section~\ref{sec:likelihood_model}. This method can be 
extended to high dimensions of analysis bins, for example in 
\cite{CMS-PAS-SUS-12-028} a secondary dimension involving the number 
of \HepProcess{\PB} quarks in the event is studied. However the 
\HepProcess{\PB} extension is not detailed in this thesis.

\begin{equation}
  N^{sig}_{prediction}\left(\HT\right) = N_{obs}^{control}\left(\HT\right) \times \frac{N_{MC}^{sig}\left(\HT\right)}{N_{MC}^{control}\left(\HT\right)}
\label{eq:traslationfactor}
\end{equation}


The three control samples used are a \HepProcess{\PW\to\Pmu\APnu}~+~Jets 
sample, a \HepProcess{\PZ\to\Pmu\Pmu}~+~Jets sample and a 
\HepProcess{\Pphoton}~+~Jets sample.

The selection criteria for each of these control samples is kept as similar to 
the signal selection as possible, so as to not introduce systematic errors from 
incorrect modelling in the simulation. The use of the ratio of the number of 
observed events in the montecarlo cancels the systematic effects. A systematic 
is still assigned to each translation factor to account for theoretical 
uncertainties and acceptance and instrumental effects.

Additional kinematic cuts are applied in the two muon control samples to enrich 
the \HepProcess{\PW}~+~Jets \HepProcess{\Ptop\APtop} and 
\HepProcess{\PZ}~+~Jets components in the control samples. The samples are 
defined to maximised efficiency rather than purity, any impurities are 
accounted for in the transfer factors as the yields from all montecarlo samples 
are used. This is valid under the assumption that the \ac{sm} 
electro-weak and Drell-Yan processes are well modelled by the simulation.
The possibility of SUSY like signal contamination in the control samples is 
accounted for in the final likelihood, after measuring the signal acceptance 
for the control samples on simulated SUSY events.

The magnitude of the systematic uncertainties on the transfer factors is 
motivated by a set of closure tests between the control samples. A transfer 
factor is produced to predict each control sample from each of the other 
control samples. No assumed systematic is applied in these closure tests, 
instead the level of agreement with in statistical uncertainty is used to set 
the scale of the systematic error for each \HT bin.


\paragraph{The $\mu$~+~Jets control sample} % (fold)
\label{par:the_mu_jets_control_sample}
The $\mu$~+~Jets control sample is designed to mimic the events appearing in 
the signal region due to \HepProcess{\PW}~+~Jets and \HepProcess{\Ptop\APtop} 
decays where the leptons are missed offline, either due to falling out of 
acceptance or being missed by the reconstruction algorithms. Hadronic tau 
decays from high \PT W bosons are also predicted from this sample.
The additional selection criteria for this sample are designed to select events 
containing the decay \HepProcess{\PW\to\mu\nu}~+~Jets in the same kinematic 
conditions as those events entering the signal selection. Offline the event 
level discriminators, \HT and \alt, are calculated using only the hadronic 
components of the event. In order to select the \HepProcess{\PW} exactly one 
tightly identified, isolated muon with in \mETA $<$ 2.5 with \PT $>$ 
\unit{10}{\GeV} is required. The transverse mass of the muon combined with the 
missing energy of the event $M_{T}(\mu,\MET)$ is required to be larger than 
\unit{30}{\GeV}, as shown in Figure~\ref{fig:figures_Analysis_Wmunu_pfMT} the 
transverse mass cut removes a large amount of QCD whilst preserving a high 
efficiency \HepProcess{\PW} selection. Events are vetoed if for any jet $\Delta 
R(\mu,Jet) < 0.5$, or if a second muon candidate exists that is either loose, 
non-isolated or outside of acceptance if the two muons have an invariant mass 
with in $\pm$\unit{25}{\GeV} of the \PZ mass, to suppress 
\HepProcess{\PZ\to\mu\mu} events.


\begin{figure}[htbp]
  \centering
    \includegraphics[width=0.5\textwidth]{figures/Analysis/Wmunu_pfMT.pdf}
  \caption{Transverse mass between the selected muon and \MET in 
           \HepProcess{\PW\to\mu\nu} events.\cite{CMS-AN-10-264}.}
  \label{fig:figures_Analysis_Wmunu_pfMT}
\end{figure}

% paragraph the_mu_jets_control_sample (end)

\paragraph{The $\mu\mu$~+~Jets control sample} % (fold)
\label{par:the_mumu_jets_control_sample}
The $\mu\mu$~+~Jets control sample is used to measure the 
\HepProcess{\PZ\to\nu\nu}~+~Jets irreducible background in the signal region. 
The process \HepProcess{\PZ\to\mu\mu}~+~Jets is identical kinematically, 
however the acceptance and the branching ratio are both smaller the branching 
ratio due to the possible decay in to one of three neutrino flavour states 
verses the requirement of a particular lepton flavour, the acceptance 
difference is due to the \PT and identification quality requirements on the 
muons alter the acceptance between the two processes. The following selection 
criteria are applied on top of the common selection: Exactly two tightly 
identified, isolated muons, with in \mETA $<$ 2.5 with \PT $>$ \unit{10}{\GeV} 
are required; The invariant mass of the di-muon pair is required to be with in 
$\pm$\unit{25}{\GeV} of the \PZ mass; Events are vetoed if for any muon and jet 
combination $\Delta R(\mu,Jet) < 0.5$. As in the single muon control sample all 
event level quantities are calculated from the hadronic objects alone. This 
control sample can be used in all of the offline \HT bins.
% paragraph the_mumu_jets_control_sample (end)

\paragraph{The \Pphoton~+~Jets control sample} % (fold)
\label{par:the_pphoton_jets_control_sample}
The \Pphoton~+~Jets control sample can also be used to measure the 
\HepProcess{\PZ\to\nu\nu}~+~Jets background as the \Pphoton~+~Jets process is 
kinematically similar when the \Pphoton \ET $>$ 
$\approx$~\unit{100}{\GeV}\cite{PAS-SUS-08-002,Bern:2011pa}, again when 
calculating the event level quantities only hadronic objects are considered.
The photon sample requires the following criteria on top of the common 
selection requirements: Exactly one photon with \ET $>$ \unit{150}{\GeV} to 
ensure trigger efficiency, \mETA $<$ 1.45. Events are vetoed if for any jet 
$\Delta R(\Pphoton,jet) < 1.0$. Given that due to the trigger requirements the 
photon \ET is required to be greater than \unit{150}{\GeV} and the photon is 
treated as missing energy, the \alt $>$ 0.55 requirement implies a minimum \HT 
threshold of $\approx$~\unit{350}{\GeV}, hence the \Pphoton~+~Jets control 
sample can only be used in the offline region where \HT $>$ \unit{375}{\GeV}.
% paragraph the_pphoton_jets_control_sample (end)

Table~\ref{tab:total-sm-pred-mu} gives the hadronic signal yields in each of 
the offline \HT bins along with the simple background estimate from the single 
muon plus jets control sample. The full background prediction is given from the 
results of the simultaneous fit to the separate background estimates.

\begin{table}[|h]
  \caption{Total SM prediction using the \Pmu~+~Jets sample only. These
    are illustrative only, as the final prediction is provided by the
    final simultaneous fit.}
\label{tab:total-sm-pred-mu}
\centering
\footnotesize
\begin{tabular}{ |c|c|c|c|c| }
\hline
\alt bin                       & 0.55--$\infty$         & 0.55--$\infty$         & 0.55--$\infty$        & 0.55--$\infty$       \\ [0.5ex]
\HT bin (GeV)                 & 275--325               & 325--375               & 375--475              & 475--575             \\ [0.5ex] 
\hline
Hadronic selection MC             & 2872.32  $\pm$  64.44  & 1384.22  $\pm$  51.46  & 1041.38  $\pm$  12.53 & 396.13  $\pm$  19.85 \\ 
$\mu +$ jets selection MC         & 1228.90  $\pm$  46.18  & 670.50  $\pm$  38.74   & 495.14  $\pm$  7.86   & 181.65  $\pm$  9.65  \\ 
Translation factor                & 2.34  $\pm$  0.10      & 2.06  $\pm$  0.14      & 2.10  $\pm$  0.04     & 2.18  $\pm$  0.16    \\ 
$\mu +$ jets selection yield data & 1421                   & 645                    & 517                   & 169                  \\ 
Total SM prediction               & 3321.30  $\pm$  169.97 & 1331.57  $\pm$  105.45 & 1087.36  $\pm$  52.50 & 368.56  $\pm$  39.09 \\ 
Hadronic yield data               & 3703                   & 1533                   & 1043                  & 346                  \\ 
\hline
\hline
\alt bin                       & 0.55--$\infty$         & 0.55--$\infty$         & 0.55--$\infty$        & 0.55--$\infty$       \\ [0.5ex]
\HT bin (GeV)                 & 575--675               & 675--775               & 775--875              & 875--$\infty$        \\ [0.5ex] 
\hline
Hadronic selection MC             & 142.37  $\pm$  7.61    & 55.47  $\pm$  3.51     & 18.68  $\pm$  1.45    & 13.18  $\pm$  1.15   \\ 
$\mu +$ jets selection MC         & 70.84  $\pm$  4.36     & 22.64  $\pm$  1.82     & 7.54  $\pm$  0.80     & 5.19  $\pm$  0.67    \\ 
Translation factor                & 2.01  $\pm$  0.16      & 2.45  $\pm$  0.25      & 2.4  $\pm$  0.33     & 2.54  $\pm$  0.40    \\ 
$\mu +$ jets selection yield data & 52                     & 18                     & 8                     & 1                    \\ 
Total SM prediction               & 104.50  $\pm$  16.81   & 44.09  $\pm$  11.33    & 19.83  $\pm$  7.41    & 2.54  $\pm$  3.47    \\ 
Hadronic yield data               & 122                    & 44                     & 14                    & 6                    \\ 
\hline

\end{tabular}
\end{table}

\paragraph{Muon control samples with out an \alt cut.} % (fold)
\label{par:muon_control_samples_with_out_an_alt_cut_}
  
The requirement of an \alt value above 0.55 in the previous control samples 
limits the even yield of each of the montecarlo samples, increasing the 
statistical error of the prediction. This is especially evident when splitting 
the analysis in to more dimensions than the \HT binning. The requirement of an 
\alt cut on the control samples means that as the muon is not seen by the 
calorimeter systems the signal trigger can also be used to collect the both the 
single and di muon background samples.
The translation factor method can be used to create a prediction from any 
sample to any other sample if and only if the modelling of the even kinematics 
and acceptances of any cuts introduces no large systematic errors. We now show 
that the montecarlo simulation accurately reproduces the kinematics and 
acceptance of the \alt cut when applied to electro-weak background samples, 
enabling the removal of the \alt requirement for the muon control samples.


The preselection of events in the two muon control samples ensures samples 
with negligible QCD contamination, which are enriched with 
\HepProcess{\Ptop\APtop}, \PW~+~Jets and \PZ~+~Jet events. This is shown for 
the \Pmuon~+~Jets sample in Figure~\ref{fig:singleMuControl} and for the 
\HepProcess{\Pmuon\APmuon}~+~Jets sample in Figure~\ref{fig:diMuControl}, in 
both sets of plots the expected number of QCD events from montecarlo simulation 
is less than one even at any \alt value for \unit{5}{\invfb} of integrated 
luminosity. The requirement of tight isolation on each of the muons is largely 
responsible for the purity of the sample, the transverse mass and di-muon mass 
window cuts ensure the sample is rich in electro-weak events with visible muons.


\begin{figure}[|h]
  \centering
  \subfigure[ \alt distribution (see main caption).]{
    \label{fig:figures_AlphaT_all-mj}
    \includegraphics[width=0.4\textwidth]{figures/Analysis/muon_plots/AlphaT_all_OneMuon_375_upwards.png}
  } 
  \subfigure[ Zoomed \alt distribution (see main caption).]{
    \label{fig:figures_AlphaT_zoomed_all-mj}
    \includegraphics[width=0.4\textwidth]{figures/Analysis/muon_plots/AlphaT_Zoomed_all_OneMuon_375_upwards.png}
  } \\
  \subfigure[\HT distribution.]{
    \label{fig:figures_HT_all-mj}
    \includegraphics[width=0.4\textwidth]{figures/Analysis/muon_plots/HT_after_alphaT_55_all_OneMuon_375_upwards.png}
  } 
  \subfigure[$N_{\textrm{jet}}$ distribution.]{
    \label{fig:JetMultiplicityAfterAlphaT_all-mj}
    \includegraphics[width=0.4\textwidth]{figures/Analysis/muon_plots/JetMultiplicityAfterAlphaT_55_all_OneMuon_375_upwards.png}
  } 
  \caption{Data--MC comparisons of key variables for the muon control
    sample, for the region $\HT > 375\GeV$ and $\altg$. Bands represent the uncertainties due to the limited size
    of MC samples. No requirement is made on the number of b-tagged
    jets in an event. {\it The discrepancy in the \alt
      distributions for values $\altg$ is due to the trigger
      not being simulated in the MC simulation.}}
      \label{fig:singleMuControl}
\end{figure}


\begin{figure}[|h]
  \centering
  \subfigure[ \alt distribution (see main caption).]{
    \label{fig:figures_AlphaT_all-mmj}
    \includegraphics[width=0.4\textwidth]{figures/Analysis/mumu_plots/AlphaT_all_DiMuon_375_upwards.png}
  } 
  \subfigure[ Zoomed \alt distribution (see main caption).]{
    \label{fig:figures_AlphaT_zoomed_all-mmj}
    \includegraphics[width=0.4\textwidth]{figures/Analysis/mumu_plots/AlphaT_Zoomed_all_DiMuon_375_upwards.png}
  } \\
  \subfigure[\HT distribution.]{
    \label{fig:figures_HT_all-mmj}
    \includegraphics[width=0.4\textwidth]{figures/Analysis/mumu_plots/HT_after_alphaT_55_all_DiMuon_375_upwards.png}
  } 
  \subfigure[$N_{\textrm{jet}}$ distribution.]{
    \label{fig:JetMultiplicityAfterAlphaT_all-mmj}
    \includegraphics[width=0.4\textwidth]{figures/Analysis/mumu_plots/JetMultiplicityAfterAlphaT_55_all_DiMuon_375_upwards.png}
  } 
  \caption{Data--MC comparisons of key variables for the di-muon control
    sample, for the region $\HT > 375\GeV$ and $\altg$. Bands represent the uncertainties due to the limited size
    of MC samples. No requirement is made on the number of b-tagged
    jets in an event. {\it The discrepancy in the \alt
      distributions for values $\altg$ is due to the trigger
      not being simulated in the MC simulation.}}
    \label{fig:diMuControl}
\end{figure}


Moving to a selection where there is no required \alt cut means that the \alt 
trigger suite can not be used to collect the high event yield control samples. 
Instead a trigger requiring \HT and a muon in the final state (\muht) is used, 
due to the muon trigger threshold the \PT acceptance cut is raised to 
\unit{45}{\GeV} in these control samples. The \HT requirement on these triggers 
raises to \unit{300}{\GeV} so only the offline bins with \HT $>$ 
\unit{375}{\GeV} are able to benefit from the increased background estimation 
precision, due to the larger size of the predicting sample. The efficiency for 
triggering on a single muon at \unit{45}{\GeV} is measured to be 
91.3$\pm$0.1$\%$ though out the data taking period, measured for the 
\HT~+~single muon triggers, the \HT component of the trigger is measured to be 
100$\%$ efficient though out. In the case of the di-muon sample, as both muons 
have to be above \unit{45}{\GeV} and either of them could have triggered the 
event, the efficiency is found to be \HT-dependant in the range of 95-97$\%$.

The muon control samples in the \HT bins where \HT $<$ \unit{375}{\GeV} are 
collected with the \alt trigger suite and the measured efficiencies are the 
same as those measured for the hadronic sample. The details of the triggers 
used for each of the muon~+~Jets control samples are listed in 
Table~\ref{tab:triggers-mu-mumu}.

\begin{table}[ht!]
  \caption{List of triggers used for the larger \mj and \mmj samples.}
  \label{tab:triggers-mu-mumu}
  \centering
  \footnotesize
  \begin{tabular}{ lcccccccc }
    \hline
    \HT bin (GeV) & 275--325 & 325--375 & 375--475 & 475--575 &
    575--675 & 675--775 & 775--875 & $>$875 \\ [0.5ex]
    \hline
    \alt cut & 0.55 & 0.55 & None & None & None & None & None & None \\
    Muon \pt cut & 10 & 10 & 45 & 45 & 45 & 45 & 45 & 45 \\
    Trigger & \alt & \alt & \muht & \muht & \muht & \muht & \muht & \muht \\
    Dataset & \verb|HT| & \verb|HT| & \verb|MuHad| & \verb|MuHad| & \verb|MuHad| & \verb|MuHad| & \verb|MuHad| & \verb|MuHad| \\
    Thresholds & Table~\ref{tab:htalphat-triggers} & Table~\ref{tab:htalphat-triggers} & Table~\ref{tab:muht-triggers} & Table~\ref{tab:muht-triggers} & Table~\ref{tab:muht-triggers} & Table~\ref{tab:muht-triggers} & Table~\ref{tab:muht-triggers} & Table~\ref{tab:muht-triggers} \\
    \hline
  \end{tabular}
\end{table}


% paragraph muon_control_samples_with_out_an_alt_cut_ (end)
% section electro_weak_background_prediction (end)

\section{Estimating the residual QCD background 
component.\cite{CMS-AN-11-517}} % (fold)
\label{sec:estimating_the_residual_qcd_background_component_}
The expected QCD contamination in the signal region where \HT $>$
\unit{275}{\GeV} and \altg from simulated background samples is 
negligible, residual events are removed via the application of the detector 
failure and \Rmiss $<$ 1.25 filters. However due to the difficulty in 
simulating QCD multi jet events accurately a conservative approach is taken 
where a term is inserted in the likelihood to model any residual QCD 
contamination.

The term is based on the ratio of the number of events above and below the \alt 
threshold of 0.55 in the individual \HT bins. The dependance of this ratio is 
modelled as an exponentially falling quantity:
\begin{equation}
  R_{\alt}\left(\HT\right)=\mathcal{A}_{n_{b}}e^{-\kqcd\HT}
\end{equation}
Where $\mathcal{A}_{n_{b}}$ is the b-tag bin dependant normalisation factor and 
\kqcd is the b-tag dependant decay constant.

The exponential behaviour is due to several features, the first of which is the 
improvement of the jet energy resolution with \HT due to the larger energies 
deposited in the calorimeter systems. \todo[inline]{find CMS JET ENERGY 
RESOLUTION PLOT JETMET PAS???} Secondly for the region \HT $>$ \unit{375}{\GeV} 
the jet multiplicity rises slowly with \HT, which due to the combinatorics  
used in the \alt calculation, results in a narrower \alt distribution peaked at 
0.5. Due to the signal region definition and the exponentially falling nature 
of the QCD background component is reduced to zero above 
$\approx$~\unit{500}{\GeV}, thus the validity of the QCD background model above 
\unit{575}{\GeV} is not of consequence to the final analysis, however the model 
chosen is shown to be valid over the whole \HT region of the analysis as shown 
below.

\ac{ml} values for \kqcd and $\mathcal{A}_{n_{b}}$ are found by the final 
likelihood fit, however \kqcd is first constrained by a measurement in a 
background enriched side band regions where either the \alt cut is relaxed or 
the \Rmiss cut is inverted. Figure~\ref{fig:figures_qcd_side-bands} depicts the 
regions where \kqcd is measured, the signal region is as described before where 
\altg and \Rmiss $<$ 1.25 are required. Region B is defined by the inversion of 
the \alt cut. Region C is defined by inverting both the \alt requirement and 
the \Rmiss requirement, this region is further divided in to three slices in 
\alt of $0.52 < \alt < 0.53$, $0.53 < \alt < 0.54$ and $0.54 < \alt < 0.55$, as 
the index of C$_{i}$ rises the expected amount of QCD in that control region 
increases. Finally region D has only the \Rmiss requirement inverted, region D 
is not used to constrain \kqcd, but instead to check the validity of the 
exponential model.
\begin{figure}[ht]
  \centering
    \includegraphics[width=0.5\textwidth]{figures/Analysis/qcd_plots/side-bands.pdf}
  \caption{QCD side-band regions, used for determination of \kqcd.}
  \label{fig:figures_qcd_side-bands}
\end{figure}
The fits to the individual side bands are shown in 
Appendix~\ref{sec:addition_information_on_background_estimation_methods}.
The best fit value for \kqcd of \unit{2.96~$\pm$~0.64~$\times 
10^{-2}$}{\GinveV} obtained from region $B$ is used as the central value of the 
constraint. The assumption that this method gives an unbiased estimate of \kqcd 
stems from the similarity in event kinematics in the two \alt regions.
The best fit values for the three $C_{i}$ regions are used to estimate the 
systematic uncertainty on the central value obtained from region $B$. The fit 
results show now dependance on the \alt region used to measure the number of 
events, supporting the assumption that region $B$ provides an unbiased estimate 
of \kqcd. The variation of the measured values for each $C_{i}$ slice are used 
to calculate the error on the central value, the weighted mean and standard 
deviation of the three slices in \alt are calculated to be 
\unit{1.13~$\pm$~0.26~$\times~10^{-2}$}{\GinveV}, the relative error on this 
value is 20$\%$ which is then applied to the central value to give an estimate 
of the systematic uncertainty.

The data side bands are used to provide a constrained value of \kqcd as an 
input to the final likelihood model which describes the expected number of 
background events in bins of \HT and the number of observed jets containing a 
\Pbottom quark. The value measured for \kqcd is 
\unit{2.96~$\pm$~0.61(stat)~$\pm$~0.46(sys)~$\times~10^{-2}$}{\GinveV}. The 
uncertainty values are used as penalty timers in the likelihood model which is 
described in Section~\ref{sec:likelihood_model}

\begin{table}[!h]
  \caption{\label{tab:expo-qcd-fits} Best fit values for the
    parameters $k$ as obtained from the regions $B$, $C_1$, $C_2$, and
    $C_3$. The latter three measurements are used to calculate a
    weighted mean (identified as region $C$). Also quoted is the
    maximum likelihood value of the parameter $k$ given by the
    simultaneous fit using the sample defined by region $D$. Quoted
    errors are statistical only. From \cite{CMS-AN-11-517}.}
  \centering
  \footnotesize
  \begin{tabular}{ ccc }
    \hline
    Side-band region     & \kqcd (\unit{$\times 10^{-2}$}{\GeV}$^{-1}$) & $p$-value \\ [0.5ex]
    \hline                            
    $B$                  & $2.96\pm0.64$                 & 0.24      \\ 
    $C_1$                & $1.19\pm0.45$                 & 0.93      \\ 
    $C_2$                & $1.47\pm0.37$                 & 0.42      \\ 
    $C_3$                & $1.17\pm0.55$                 & 0.98      \\ 
    \hline                            
    $C$ (weighted mean)  & $1.31\pm0.26$                 & -         \\ 
    $D$ (likelihood fit) & $1.31\pm0.09$                 & 0.57      \\
    \hline
  \end{tabular}
\end{table}

A final check is performed using region $D$, which requires \altg but has no 
\Rmiss cut, this introduces QCD background in to the signal region. The 
likelihood fit is performed on this background enriched region and no 
constraint is applied on \kqcd which is then determined by the fit only. The 
fit is performed over the full \HT range used in the final analysis.
Figure~\ref{fig:qcd-fit} shows the resulting fit, the \ac{ml} value obtained 
for \kqcd is \unit{(1.31 $\pm$ 0.09)$\times 10^{-2}$}{\GinveV}, this value is 
in excellent agreement with the value found from the weighed mean of the 
regions $C_{i}$, secondly the fit shows that the choice of exponential function 
used in the likelihood model is valid over the entire \HT range. This supports 
the assumption that region $B$ provides an unbiased estimate of \kqcd in the 
signal region \altg and \Rmiss $<$ 1.25.


\begin{figure}[!h]
  \begin{center}
    \includegraphics[width=0.7\textwidth,angle=0]{figures/Analysis/qcd_plots/qcd-fit.pdf}
    \caption{Comparison of the observed yields and SM expectations
      given by the simultaneous fit in bins of \HT for the
      side-band region $D$. No requirement on the number of b jets is
      made. Shown are the observed event yields in data (black dots
      with error bars representing the statistical uncertainties) and
      the expectations given by the simultaneous fit for the
      \HepProcess{\PZ\to\nu\nu}+jets process (orange dotted-dashed line); the 
      sum of all processes with genuine \MET, which are primarily 
      \HepProcess{\Ptop\APtop}, W+jets, and \HepProcess{\PZ\to\nu\nu}+jets
      (dark blue long-dashed line); and the sum of QCD and all aforementioned 
      SM processes (light blue solid line).\cite{CMS-AN-11-517}}
    \label{fig:qcd-fit}
  \end{center}
\end{figure}


% subsection estimating_the_residual_qcd_background_component_ (end)


\section{Systematic uncertainties on the electro-weak background model.} 
% (fold)
\label{sec:systematic_uncertainties_on_the_electro_weak_background_model_}
As previously discussed in Section~\ref{sec:electro_weak_background_prediction} 
the final background prediction is given by the simultaneous fit to the yields 
in the signal and control samples and the translation factors obtained from MC. 
The fit has some freedom via the statistical and systematic uncertainties 
measured for each translation factor. This implies that the measurement of the 
systematic uncertainties on the translation factors are vital for the fitting 
procedure.

A set of closure tests were performed on data to identify any sources of 
systematic biased introduced by the background prediction method.
To do this the individual background samples are used to predict one another 
using the same translation factor method as for the prediction of \ac{sm} 
missing energy sources in the hadronic signal regions. The level of agreement 
is quantified in terms of the ratio $\left(N_{obs} - N_{pred}\right)/N_pred$, 
the statistical error from the translation factor, based on the available 
Montecarlo statists is combined with the statistical error on the number of 
events in the predicting sample to give the error on the closure, hence the 
deviation of the ratio from zero gives the level of closure per analysis bin. 
This gives a measure of any biases introduced by the background estimation 
method.

The closure tests between the background samples are designed to test the 
Montecarlo's ability to model kinematic effects; such as the \alt acceptance; 
\Pmuon acceptance and \Pphoton acceptance, instrumental effects such as; 
reconstruction efficiencies and the effects of pile up on isolation and finally 
the theoretical precision of the production and decay cross sections and their 
relative contributions to the \ac{sm} background. These individual components 
are not separable by the closure tests, which instead gives a total systematic 
error estimation.

As described in Section~\ref{par:muon_control_samples_with_out_an_alt_cut_} the 
control samples which do not require an \alt cut use a \muht cross object 
trigger to collect the data events. As shown in Table~\ref{tab:muht-triggers} 
there was a period of data taking where due to the increased trigger thresholds 
the \muht triggers are unsuitable for use in the region \HT $<$ 
\unit{375}{\GeV}. Thusly in the closure tests between the control samples which 
use an \alt cut and those that do not the integrated luminosity is limited to 
\unit{3.9}{\invfb}. This causes a loss of some statistical power in these cases.

The individual closure tests and fits to the \HT dependance of the ratio 
$\left(N_{obs} - N_{pred}\right)/N_pred$ are shown in 
Appendix~\ref{sec:Closure_tests_and_systematic_uncertainties}.
The Figures~\ref{fig:closure-with-alphat1} show the closure of the prediction 
between \mj(no \alt) $\rightarrow$ \mj(\altg) and \mmj(no \alt) $\rightarrow$ 
\mmj(\altg) for two samples, one with no requirement on the number of \Pbottom 
tagged jets ($n_{b}$), which increases the precision of the measurement, and 
one requiring $n_{b} = 1$, the read line is the result of a one parameter fit. 
The level of closure shows that the Montecarlo accurately models the \alt 
acceptance, with no significant bias.

Figure~\ref{fig:closure-with-alphat2} shows the closure between \mj 
$\rightarrow$ \mmj and between \Pphoton+jets $\rightarrow$ \mmj over both the 
full \HT range using only \unit{3.9}{\invfb} of integrated luminosity and for 
\HT $>$ \unit{375}{\GeV} using the full data set. Again the red lines are the 
result of fitting with a one parameter fit.

Figure~\ref{fig:btag-multi-closure} tests the closure between samples with 
differing $n_{b}$ the three tests are \mj($n_b = 0$)$\rightarrow$\mj($n_b = 
1$,no \alt), \mj($n_b = 1$)$\rightarrow$\mj($n_b > 1$,no \alt) and \mj($n_b = 
0$)$\rightarrow$\mj($n_b > 1$,no \alt). Figure~\ref{fig:btag-samples} also 
tests the closure between samples with differing $n_{b}$, here the test is 
between \mj($n_b = 0$)$\rightarrow$\mmj($n_b = 0$) and \mj($n_b =  
1$)$\rightarrow$\mmj($n_b = 1$).

Finally any dependance on pile up is measured by comparing a subset of the 
individual closure tests between samples which have pile up subtracted jets and 
those that do not. The example closures are \mj(no \alt)$\rightarrow$\mj(\altg),
\mj(no \alt)$\rightarrow$\mmj(no \alt), and \mj($n_b = 0$)$\rightarrow$\mj($n_b 
= 1$,no \alt).

\subsection{Motivating the combined systematic on the translation factors}
 % (fold)
\label{sub:motivating_the_combined_systematic_on_the_translation_factors}
The closure tests described in the previous section are combined to give a 
total systematic uncertainty. This uncertainty is binned in to three \HT 
regions \unit{275}{\GeV}$\rightarrow$\unit{575}{\GeV}, 
\unit{575}{\GeV}$\rightarrow$\unit{775}{\GeV} and \unit{775}{\GeV}+. In each of 
these regions all of the individual closure tests are used to calculate a 
weighted mean and variance. The systematic is defined as $3\sigma$ of this 
variance, which is conservative but necessary to cover any biases. The 
systematics are treated as fully uncorrelated between the three regions, again 
this is the conservative approach. Figure~\ref{fig:closure-summary} shows the 
key example closures, the grey shaded region shows the systematic error. The 
values obtained for the error are 6$\%$, 20$\%$ and 39$\%$, these are rounded 
to 10$\%$, 20$\%$ and 40$\%$ and then used in the final background simultaneous 
fit.

\begin{figure}[!h]
  \begin{center}
    \includegraphics[width=0.75\textwidth,angle=0]{figures/Analysis/syst_plots/incl.pdf}
    \caption{A set of closure tests (open symbols) overlaid on top of
      grey bands that represent the systematic uncertainties used for
      three \HT regions in the final simultaneous fit. The solid
      circles and their errors represent the weighted mean and
      standard deviation for the five closure tests of each individual
      \HT bin.}
    \label{fig:closure-summary}
  \end{center}
\end{figure}

% subsection motivating_the_combined_systematic_on_the_translation_factors (end)

% section systematic_uncertainties_on_the_electro_weak_background_model_ (end)


\section{Signal Models} % (fold)
\label{sec:signal_models}
The level of agreement of the analysis with the \ac{sm} give a handle on the 
production cross sections and masses of particles predicted by new physics 
models. The final results are presented in terms of a specific \ac{susy} model 
and a selection of generic final state topologies, known as \ac{sms}.

The \ac{susy} model considered is the \ac{cmssm}\cite{null} which is described 
in Section~\ref{sec:the_constrained_minimal_super_symmetric_model}, the 
parameters chosen are $\tan\beta =3, \mathcal{A}_{0} = , \mu = $ with the 
exclusion curve presented in the $m_{0}$, $m_{1/2}$ plane. This model combines 
many production and decay topologies and is common to results shown by previous 
and contemporary experiments \cite{} \todo[inline]{Reference ATLAS and Tevatron 
susy results}.
The results are also presented in terms of \ac{sms}\cite{} models, these are 
models with single production methods and a specified decay topology. The 
\ac{cms} terminology for the production methods are as follows: T1 models are 
gluino-gluino production, which then decay to four \ac{sm} hadronic jets and 
two neutrilinos. T2 models are squark-squark production with decays to two 
\ac{sm} hadronic jets and two neutrilinos.
These topologies can be further specialised by enforcing the squarks or gluinos 
to decay to heavy flavour \ac{sm} quarks, such as \Ptop, which promptly decay 
to jets containing \Pbottom quarks or direct decays to \Pbottom quarks. The 
limits on the \ac{sms} space are defined in terms of the mass splitting between 
the neutrilino and the pair produced \ac{susy} particle defining the model.
\subsection{Signal Efficiency} % (fold)
\label{sub:signal_efficiency}
\paragraph{\ac{cmssm}} % (fold)
\label{par:cmssm}
The \ac{cmssm} signal scan is composed of eight sub-processes which define the 
production and decay topologies. The dominant process varies with $m_{0}, 
m_{1/2}$ and at next to leading order the cross section for each of theses 
processes varies per point. The analysis is run over each sub process in turn 
with the final efficiency given by the weighted sum of the subprocess 
efficiencies. The yield per point is then given by 
$\epsilon\time\sigma\times\mathcal{L}$ for \unit{5}{\invfb} the total yields 
are show in the appendix in Figure~\ref{fig:sig-eff-cmssm} these yields are for 
the sum of the \HT analysis bins.
% paragraph cmssm (end)

\paragraph{Simplified Models} % (fold)
\label{par:simplified_models}
The \ac{sms} models contain only one production process and a set decay 
topology, making the interpretation in these models simpler, however the 
individual models are not representative of some complete \ac{susy} model. 
Instead these models allow the testing of specific facets of new physics 
models, with out the ambiguity of the relative contributions of each sub 
process at a point which is seen when testing full models. The efficiency is 
measured for each of the models, with the yield per point given by 
$\epsilon\times\sigma\times\mathcal{L}$ where $\sigma$ is given as a reference 
cross section only. The efficiencies for which are shown in the appendix, 
Table~\ref{tab:sms-eff} lists which figure corresponds to which model, it is to 
be noted that the total efficiency summed over \HT bins is shown.



\begin{table}[h!]
  \caption{Production and decay modes for various simplified models.}
  \label{tab:sms-eff}
  \centering
  \footnotesize
  \begin{tabular}{ llc }
    \hline
    Model & Production and decay modes & Figure showing efficiency \\ [0.5ex]
    \hline
    \texttt{T1} & \HepProcess{\PSgluino\PSgluino\to\Pquark\APquark\PSneutralino\Pquark\APquark\PSneutralino} & \ref{fig:sig-eff-t1} \\
    \texttt{T2} & \HepProcess{\Psquark\Psquark\to\Pquark\PSneutralino\APquark\PSneutralino} & \ref{fig:sig-eff-t2} \\
    \texttt{T2tt} & \HepProcess{\PStop\PStop\to\Ptop\PSneutralino\APtop\PSneutralino}
& \ref{fig:sig-eff-t2tt} \\
    \texttt{T2bb} & \HepProcess{\PSbottom\PSbottom\to\Pbottom\PSneutralino\APbottom\PSneutralino} & \ref{fig:sig-eff-t2bb} \\
    \texttt{T1tttt} & \HepProcess{\PSgluino\PSgluino\to\Ptop\APtop\PSneutralino\Ptop\APtop\PSneutralino} & \ref{fig:sig-eff-t1tttt} \\
    \texttt{T1bbbb} & \HepProcess{\PSgluino\PSgluino\to\Pbottom\APbottom\PSneutralino\Pbottom\APbottom\PSneutralino} & \ref{fig:sig-eff-t1bbbb} \\
    \hline
  \end{tabular}
\end{table}
% paragraph simplified_models (end)


\paragraph{Signal Efficiency for the Background Selection} % (fold)
\label{par:signal_efficiency_for_the_background_selection}
If the couplings of the \ac{susy} particles are analogous to the \ac{sm} 
particles then the final states may involve muons which in the presence of 
signal would infer an over estimation of the background from the control 
samples.
To measure this the background selection is applied to the signal models and 
the yields are taken into account in the final limit setting procedure. On 
average the background selection's efficiency on signal is $10\times$ lower 
than the efficiency in the hadronic signal region. This is shown for the model 
\texttt{T1tttt} in Figure~\ref{fig:sig-eff-contam} where the most muons of all 
the signal models are expected in the final state, the contamination in the 
$n_{b}$ bins that drive the limit is on the order of $10-20\%$.
% paragraph signal_efficiency_for_the_background_selection (end)
% subsection signal_efficiency (end)

\subsection{Uncertainty on Signal Efficiency} % (fold)
\label{sub:uncertainty_on_signal_efficiency}
The  systematic uncertainty on the signal models is due to:\ac{pdf}, the 
\ac{pdf} set used to generate the sample has effects on both the acceptance and 
the cross section; The measurement of the integrated luminosity is accounted 
for in the signal yield; Due to the signal models being created with \ac{cms} 
FastSim\cite{} rather than \ac{cms} FullSim\cite{} the acceptance differs 
between the signal and the background samples; The error on the jet energy 
scale is accounted for on the signal yield; Systematic errors from the 
cleanings cuts (\Rmiss and ECAL dead regions) and lepton/photon vetoes are also 
taken in to account; Finally corrections to the \Pbottom-tagging efficiency 
between the FullSim and the FastSim are applied as well as their errors.

Each of these uncertainties is expressed as a percentage change in the 
efficiency from the central value given by applying the full analysis to each 
signal model, the total systematic is given by summing the components in 
quadrature. In the following section the measurements of the error from each of 
these sources is detailed and summarised per considered signal model at the end 
of the section.

The uncertainties for the \ac{cmssm} are considered in a band of 
$\pm$\unit{60}{\GeV} in $m_{1/2}$ around the expected limit to confine the 
errors to the relevant part of the plane, in the very high $m_{0},m_{1/2}$ area 
the jet energy scale causes large fluctuations due to the small mass splitting 
between the \ac{susy} particles.
For the \ac{sms} models two regions are defined, one ``close'' to the diagonal, 
which has small mass splitting and thus the effects of jet energy scale and 
\ac{pdf} acceptance have a large impact on the analysis efficiency. A second 
``far'' region is defined with large mass splitting and thus a small change on 
the analysis efficiency due to jet energy scale and \ac{pdf} variations. The 
near and far regions are defined by:
\begin{equation}
  m_{sq}(m_{gl}) - m_{lsp} > \unit{350}{\GeV} \&\& m_{sq}(m_{gl}) > \unit{475}{\GeV}
\end{equation}
events passing these conditions are classified as being in the ``far'' region, 
those failing in the ``close'' region.
% subsection uncertainty_on_signal_efficiency (end)


% section signal_models (end)


\section{Likelihood model} % (fold)
\label{sec:likelihood_model}

% section likelihood_model (end)

% chapter the_t_analysis (end)