\chapter{The $\alpha_{T}$ analysis} % (fold)
\label{cha:the_t_analysis}
In this chapter we discuss the main analysis performed as the subject of this 
thesis. For the theoretical motivations of this search please see 
Chapter~\ref{cha:theory}. The analysis is based on the full 2011 data set which is made up of \unit{5}{\invfb} of \unit{7}{\TeV} data. However \unit{5}{\invfb} of the 2012 \unit{8}{\TeV} is looked at to measure the performance of the upgraded \alt HLT paths.
\section{The Problem} % (fold)
\label{sec:the_problem}
If Supersymmetry or some other beyond the standard model theory is to provide a 
yet undiscovered dark matter candidate, it is predicted that this candidate 
will interact via the weak nuclear force only. This gives a decay topology 
involving missing energy in the form of the dark matter particle escaping the 
detector. Due to the nature of interactions at the L.H.C, these particles would 
be produced at the end of a decay chain of heavy particles that interact 
strongly, giving a final topology involving hadronic objects which are 
classified as jets for the purpose of analysis and missing energy.
There are several standard model processes that mimic this final state.

By far the largest of these backgrounds comes from QCD multi jet events where 
fake missing energy is introduced either from failures in reconstruction, or 
stochastic fluctuations in the calorimeter systems.
\textbf{FIXME: expand on this - E /sqrt E has non gaussian tails. Figures of 
jets falling below threshold, missed jets etc. probably from some jet-met 
paper.}
However due to the theoretical errors on the QCD production cross section 
predicting the number QCD background events from Montecarlo simulation is not 
possible.
A secondary QCD background also exists, where due to the requirement of a jet 
\ET threshold, multiple jets fall under threshold by a few \GeV, this causes a 
balanced event to look unbalanced as the jets under threshold are no longer 
considered. It is these events that \alt is designed to remove.

The second major background comes from standard model electro-weak decays and 
is irreducible as the final states involve real missing energy, from 
neutrinos. The electro-weak decays that form the back ground are 
\HepProcess{\PW\to\Ptau\Pnu} + Jets, where the $\tau$ is reconstructed as a 
jet, or the lepton fails the identification required for the dedicated lepton 
vetoes, \HepProcess{\PZ\to\Pnu\APnu} + Jets is completely irreducible. These 
are generally di-jet topologies. At higher jet multiplicities top quark 
production followed by semi-leptonic top decay accounts of the largest 
background. These backgrounds are predicted using a well understood control 
sample this is fully explained in 
Section~\ref{sec:electro_weak_background_prediction}.


The final background source is that introduced by detector failure or 
electronic noise induced by the movement of the L.H.C proton beam.
Approximately 1$\%$ of the ECAL read out is not available in offline event 
reconstruction, this provides a source of fake missing energy.
% section the_problem (end)

\section{The \alt variable.} % (fold)
\label{sec:the_alpha___t_variable_}
\alt is inspired by Ref~\cite{Randall:2008dk} and was expanded to transverse 
multi jet topologies by members of the CMS collaboration in 
Refs~\cite{cms-pas-sus-08005,cms-pas-sus-09001}. The purpose is to provide a 
variable that can be cut on to eliminate QCD from the final selection. To do 
this the inherent balance of the QCD system is exploited.

For di-jet systems \alt is defined as:

\begin{equation}
  \alt = \frac{\ET^{j_{2}}}{M_{T}}
\end{equation}
where \ET$^{j_{2}}$ is the transverse energy of least energetic of the two jets 
and M$_{T}$ is defined as:

\begin{equation}
  M_{T} = \sqrt{\left(\sum^{2}_{i=1}\ET^{j_{i}}\right)^2 - \left(\sum^{2}_{i=1}p_{x}^{j_{i}}\right)^{2} - \left(\sum^{2}_{i=1}p_{y}^{j_{i}}\right)^{2}}
\end{equation}

For a perfectly measured di-jet system with \ET$^{j_{1}} = $\ET$^{j_{2}}$, 
where the jets are opposite in $\phi$ \alt = 0.5, for events with back to back 
jets where one is miss-measured \alt $ < 0.5$.
However the majority of signals predict many jets in the final state.
\alt can be generalised to work with n-jets in the flowing way.
The variables \HT, \HTm and $\Delta$\HT are constructed:
\begin{equation}
  \HT = \sum_{i = 0}^{n~jets}\ET^{jet_{i}} \\
  \HTm = \left|\sum_{i = 0}^{n~jets}\bar{p}_{T}^{jet_{i}}\right|
\end{equation}
for jets above some predefined threshold \ET which is common for all jet based 
quantities. The multi jet system is reduced to a pseudo di-jet system by 
forming two large jets. The individual jet \ET's are summed, with the final 
configuration being chosen to have the minimum difference in energy 
($\Delta$\HT) between the pseudo jets. This simple clustering criteria provides 
the best separation between miss-measured events and those with real \MET.

\alt is then defined as:
\begin{equation}
  \alt = \frac{\HT - \Delta\HT}{2\sqrt{\HT^{2}-\HTm^{2}}}
\end{equation}


Figure~\ref{fig:figures_Analysis_AlphaT_all_375_upwards} shows the \alt 
distribution for both data and simulated background samples. The QCD multi jet 
background is negligible above an \alt value of 0.55, where as the standard 
model processes which involve real \MET exist at all possible values of \alt.
Values of \alt in the range $0.5 < \alt < 0.55$ arise in multi jet QCD due to 
jets falling below threshold or large stochastic fluctuations.
It is to be noted that the discrepancy between data and simulation for \alt 
$\leq 0.55$ is due to no trigger emulation being applied to the simulated 
background samples.
\begin{figure}[ht!]
  \centering  \includegraphics[width=0.8\textwidth]{figures/Analysis/AlphaT_all_375_upwards.pdf}
  \caption{\alt distribution for background and data for the region \HT $>$ 
  \unit{375}{\GeV}. Trigger emulation is not applied in the simulated 
  background which leads to the discrepancy in the region \alt $\leq 0.55$. The 
  QCD multi-jet background is reduced to less than one event.}
  \label{fig:figures_Analysis_AlphaT_all_375_upwards}
\end{figure}

% section the_alpha___t_variable_ (end)

\section{Event selection} % (fold)
\label{sec:event_selection}
In order to select events for the hadronic signal sample and the muon and 
photon control samples a common set of section cuts is defined. In this section 
the objects are defined as are the flow of the analysis cuts and filters.

\paragraph{Preselection of hadronic objects} % (fold)
\label{par:Preselection_of_hadronic_objects}
Jets are created by running the \AK jet clustering 
algorithm\cite{Cacciari:2008ua} over the calorimeter towers. More detail of the 
jet reconstruction at \ac{cms} is given in Section~\ref{sec:hadronic_jets}. The 
jets have their raw energies corrected based on their position and momentum to 
establish a uniform relative response in $\eta$ and a calibrated absolute 
response in transverse energy \ET, with an associated uncertainty of between 
2$\%$ and 4$\%$ dependant on \ET and $\eta$\cite{Chatrchyan:2011ds}. Jets 
considered in the analysis are required to have \ET $>$ \unit{50}{\GeV}, the 
highest \ET jet in the events is required to be within tracker acceptance 
($|\eta| <$ 2.5) and the sub leading jet is required to have \ET $>$ 
\unit{100}{\GeV}. In the lowest two offline \HT bins the jet thresholds are scaled to preserve the jet multiplicity, for the bin \unit{275}{\GeV} $<$ \HT < \unit{325}{\GeV}, the jet threshold is \unit{36.$\dot{6}$}{\GeV} and the sub leading jet threshold is \unit{73.$\dot{3}$}{\GeV}. In the bin \unit{325}{\GeV} $<$ \HT < \unit{375}{\GeV}, the jet threshold is \unit{43.$\dot{3}$}{\GeV} and the sub leading jet threshold is \unit{86.$\dot{6}$}{\GeV}.

The quantities \HT and \HTm are then formed from these jets.
% paragraph definition_of_event_level_jets (end)

\paragraph{Elections} % (fold)
\label{par:elections}
Elections are reconstructed as described in Section~\ref{sec:electrons}. For 
the purpose of the analysis an object is defined as an electron if it has \PT 
$>$ \unit{10}{\GeV}, \mETA $<$ 2.5 and passes ``VBTF'' working point 95 quality 
criteria\cite{PAS-EGM-10-004}. Any event containing an identified electron is 
vetoed. 
% paragraph elections (end)

\paragraph{Muons} % (fold)
\label{par:muons}
Muon reconstruction at \ac{cms} is detailed in Section~\ref{sec:muons}. Signal 
events are vetoed if they contain a muon with \PT $>$ \unit{10}{\GeV} and \mETA 
$<$ 2.5 that passes the ``VBTF'' working point 95 for muons 
\cite{PAS-MUO-10-002}. Two separate muon control samples are defined, one 
requiring exactly one muon with \PT $>$ \unit{10}{\GeV}, the second requiring 
two muons, who's invariant mass sums to the \PZ mass. 
% paragraph muons (end)


\paragraph{Photons} % (fold)
\label{par:photons}
Photon reconstruction as \ac{cms} is detailed in Section~\ref{sec:photons}, 
signal events are vetoed if they contain a photon with \ET $>$ \unit{25}{\GeV}, 
\mETA $<$ 2.5 and that passes ID requirements\cite{PAS-EGM-10-006}. A 
requirement of exactly one photon with \ET $>$ \unit{150}{\GeV} with in \mETA 
$<$ 1.45 is made for the photon control sample.
% paragraph photons (end)


\textbf{PUT SOME STUFF IN ABOUT THE CROSS CLEANING}

The use of these control samples will be discussed in Section~\ref{sec:electro_weak_background_prediction}. 

The common selection cuts and filters consist of:

\begin{itemize}
\item \textbf{Good run selection}, All detector subsystems on, \ac{cms} in 
``Physics Declared'' mode and all physics object groups have certified the runs 
and luminosity sections. This removes any events where the sub-detectors were 
in an error state or events from before the tracker was switched to high 
voltage mode.
\item \textbf{P.K.A.M (Previously Known As Monsters) filter}, these events are 
caused by beam-gas interactions close to \ac{cms}, which cause a shower of 
particles to enter the pixel detector along the beam line, resulting in a large 
proportion of the pixel detector to record hits.
\item \textbf{Vertex Selection} requires at least one non-fake vertex with at 
least four associated tracks, within a cylinder of radius \unit{2}{\cm} and 
length \unit{48}{\cm}, centred at Z = 0 of the \ac{cms} detector.
\item \textbf{Hadronic barrel and end-cap noise filter}, this filter removes 
events where strips of towers in the hadronic calorimeters record energy from 
electrical noise, mimicking large, unbalanced energy deposits.
% \item \textbf{Cleaned Rec hit filter}, removes events where the sum of cleaned energy from calorimeter towers is greater than \unit{30}{\GeV}.
\item \textbf{Vertex \PT/\HT $>$ 0.1}, removes events where the sum of the 
\PT of all tracks from all good vertices is less than 10$\%$ of the energy 
deposited by jets in the calorimeters. This cut is designed to remove events 
with tracking failure, which would otherwise pass the calorimeter only event 
quality requirements.
\item\textbf{Masked ECAL channel filter:} Approximately 1$\%$ of the ECAL 
crystals are masked, or have read out failure. To avoid selecting events with 
large energy miss measurement, a topological cut was devised. The first step is 
to calculate $\Delta\phi^{*}$ for each jet ($\vec{j}$) in the event, where:
\begin{equation}
  \Delta\phi^{*} = \Delta\phi\left(\vec{\slashed{E}}_{T}+\vec{j},\vec{j} \right).
  \label{eq:biasedDphi}
\end{equation}
Which gives a measure of the miss measurement of a jet, if $\Delta\phi^{*}$ is 
small, the missing energy points along the jet in the $\phi$ direction. By 
selecting the miss measured jet, full position information is preserved. If any 
jet has $\Delta\phi^{*} <$ 0.5, the number of masked ECAL crystals with in 
$\Delta R < $0.3 are summed, if there are more than 10 masked crystals adjacent 
to the jet, the event is vetoed.
\item \textbf{R$_{miss} <$ 1.25:} The total hadronic energy in an event is 
required to be greater than \unit{275}{\GeV} which is well above the transverse 
energy threshold of \unit{50}{\GeV} for each jet. However several jets falling 
below this threshold can sum to a significant quantity of ignored energy. This 
is shown in Figure~\ref{fig:figures_Analysis_BabyMHT50-10}, here the missing 
energy calculated from jets in the range \unit{10}{\GeV} $<$ \ET $<$ 
\unit{50}{\GeV} is shown, whilst requiring that \MET $<$ \unit{20}{\GeV}. This 
shows that for a well balanced event the jets below threshold can carry greater 
than \unit{100}{\GeV} of ignored energy. R$_{miss}$ is defined as $\HTm \slash 
\MET$ and can be used to single out events where the inclusion of lower 
momentum jets does significantly improve the balance of the event. 
Figure~\ref{fig:MHTovMET_lowHT_highHT} shows for two \HT regions the R$_{miss}$ 
distribution after the application of the full cut flow, including \alt, QCD 
contamination is visible in the signal sample for R$_{miss} >$ 1.25. 
\end{itemize}


\begin{figure}[htbp]
  \centering
    \includegraphics[width=0.75\textwidth]{figures/Analysis/BabyMHT50-10.pdf}
  \caption{\HTm from jets with \unit{10}{\GeV} $<$ \ET $<$ \unit{50}{\GeV} in events with \HT $>$ \unit{350}{\GeV} and \MET $<$ \unit{20}{\GeV} in \unit{35}{\invpicobarn} of data.}
  \label{fig:figures_Analysis_BabyMHT50-10}
\end{figure}

\begin{figure}[htbp]
  \centering
  \subfigure[]{ \includegraphics[width=0.45\textwidth]{figures/Analysis/250_300GevcombinedMHTovMET_afterAlphaT_all.pdf}
  \label{fig:figures_Analysis_250_300GevcombinedMHTovMET_afterAlphaT_all}
}
 \subfigure[]{   \includegraphics[width=0.45\textwidth]{figures/Analysis/350Gev_afterDeadEcalcombinedMHTovMET_afterAlphaT_all.pdf}
\label{fig:figures_Analysis_350Gev_afterDeadEcalcombinedMHTovMET_afterAlphaT_all}
}
\caption{(a) R$_{miss}$ distribution for events in the \unit{250}{\GeV} 
$<$\HT$<$\unit{350}{\GeV} region, where due to the low \HT requirement QCD 
contamination is enriched. \\ (b) R$_{miss}$ distribution for events in the 
\HT$>$\unit{350}{\GeV} region, QCD contamination occurs at R$_{miss} >$ 1.25.}
\label{fig:MHTovMET_lowHT_highHT}
\end{figure}


This cut flow and set of object definitions define the common selection, on top 
of this an \alt cut is applied, the passing events are then binned in 8 \HT 
bins, these are \unit{275}{\GeV} - \unit{325}{\GeV}, \unit{325}{\GeV} - 
\unit{375}{\GeV}, \unit{375}{\GeV} - \unit{475}{\GeV}, \unit{475}{\GeV} - 
\unit{575}{\GeV}, \unit{575}{\GeV} - \unit{675}{\GeV}, \unit{675}{\GeV} - 
\unit{775}{\GeV}, \unit{875}{\GeV} - \unit{7}{\TeV}.
However the raw number of events in the signal region is meaningless with out 
an accurate back ground prediction, as discussed earlier taking the background 
estimation from simulation is not viable due to the requirement of high order 
theory calculations on the cross sections of standard model processes involving 
jets, these have many divergencies due to the nature of QCD. Instead the 
simulation is used to form a translation factor between two samples measured in 
data, a control sample which closely mimics the standard model processes 
producing missing energy, but has a visible muon or photon in the final state. 
The background estimation methods are described and the background predictions 
stated in Section~\ref{sec:electro_weak_background_prediction}.

% section event_selection (end)

\section{High Level triggers for the \alt analysis} % (fold)
\label{sec:high_level_triggers_for_the_alt_analysis}
The CMS trigger system has been discussed in detail in 
Section~\ref{sec:the_high_level_trigger_system} and 
Chapter~\ref{cha:level_one_trigger}, however details of analysis specific 
trigger paths were not discussed. During 2011 the first \alt specific trigger 
was designed and deployed online. The trigger was then upgraded for the higher 
luminosity and energy conditions of the 2012 data taking period.


The trigger takes advantage of cutting on two variables, \HT and \alt at low 
\HT a high \alt value cuts the trigger rate, where as at high \HT where the
trigger rate is lower the \alt requirement can be loosened.

Due to the scaling of jet thresholds in the lowest offline \HT bins as detailed 
in Section~\ref{sec:event_selection} using a fixed jet threshold would cause 
inefficiency in the lowest offline \HT bins. To over come this the trigger 
level \alt calculation is performed iteratively for all jets above a predefined 
threshold. This raises the total number of accepted events whilst adding the 
benefit of being efficient for any offline jet threshold above the minimum 
trigger jet threshold. The algorithm is show in 
Figure~\ref{fig:figures_Analysis_flowChart}.


\begin{figure}[ht!]
  \centering
    \includegraphics[width=0.75\textwidth]{figures/Analysis/flowChart.pdf}
  \caption{Flow chart representing the steps taken to make a trigger decision 
  using the \alt trigger algorithm.}
  \label{fig:figures_Analysis_flowChart}
\end{figure}


\section{2011 Trigger} % (fold)
\label{sec:2011_trigger}
Due to concerns on the time taken to perform the \dHT minimisation at the trigger and time constraints enforced on trigger menu development, the first implementation calculates \alt for the first 3 jets. For higher jet multiplicities the variable \bt is calculated.
\begin{equation}
  \beta_{T} = \frac{\HT}{2\sqrt{\HT^{2} - \MHT^{2}}}
  \label{eqn:betaT}
\end{equation}
this gives us the relation:
\begin{equation}
  \alt \leq \beta_{T}.
\end{equation}
The decision flow is shown in Figure~\ref{fig:figures_Analysis_flowChart} and explained in detail below.

When a level one accept is issued the trigger bits that fired are checked, if 
the event fires a L1 muon trigger it is passed to the HLT muon triggers where 
only muon reconstruction is performed, reducing the reconstruction time. The 
\alt triggers are seeded on the lowest threshold unprescaled L1 \HT trigger, 
during 2011 this was L1$\_$HTT100. Any events issuing a L1 accept and passing 
L1$\_$HTT100 undergo calorimeter jet reconstruction, the reconstruction 
algorithm is detailed in Section~\ref{sec:hadronic_jets}.

Once the jets have been formed the trigger filter is entered. Initially the 
first two jets ranked by \ET, are considered, \HT and \alt are calculated, if 
both pass the trigger thresholds the event is accepted and the full detector 
read out is performed. If either \HT or \alt is below threshold, the next jet 
in \ET order is added, if the jet collection contains more than 3 jets then the 
\bt approximation is used. All jets in the event are added until either the 
even is accepted, or there are no more jets to be added above \unit{40}{\GeV}.

The effect of switching to the \bt approximation is to accept events that have 
missing energy due to miss-measurement, when calculating \alt offline these 
events have values of \alt $< 0.5$. This introduces an impurity to the trigger 
and costs rate for events that will not be considered in the offline analysis.
% section 2011_trigger (end)


\subsection{Trigger efficiency measurement} % (fold)
\label{sub:trigger_efficiency_measurement}

The performance of the \alt trigger suit is measured with respect to a sample 
collected using the muon system. This allows the measurement of efficiency of 
both the level one seed trigger and the higher level trigger at the same time 
as different sub-systems are used to collect the reference and the signal 
triggers. This is due to the exclusive use of calorimeter jets in the \alt 
trigger, if more complicated reconstruction methods which produce an event 
hypothesis were used, muons would at HLT level only be considered as jets. 
Where as during calorimeter only reconstruction, muons are not considered and 
the \pt of any muons in an event is viewed as missing energy.

The selection for the trigger efficiency measurement is the same as listed in 
Section~\ref{sec:event_selection} with the requirement of exactly one well 
identified muon with \PT $>$ \unit{45}{\GeV}, the sum of the \MET in the event 
and the muon must add to the \PW mass and finally the muon must be separated by 
at least 0.5 in $\Delta R$ to the closest jet, to avoid the muon energy 
changing the energy of the jet offline when the cross cleaning is applied.

Due to the increase in luminosity over the running period the trigger 
thresholds were increased, to ensure constant rate though out the year and the 
trigger version numbers were increased each time the trigger menu was updated.
The list of triggers considered and their reference triggers are listed in 
Tables~\ref{tab:triggers1}~and~\ref{tab:triggers2}.

The efficiency of each trigger version is measured in the required \HT bins, 
the total efficiency for each \HT bin is then calculated by combining the 
individual efficiencies using a weighted sum based on the fraction of 
the total luminosity of the sample that each version carries. This accounts for 
the change in running conditions during the data taking period and the criteria 
that the trigger suite takes constant rate over the length of then, the higher 
trigger versions and thresholds generally represent more integrated luminosity 
due to this.

The efficiency is measured as a function of the cumulative number of events, ie the efficiency at each point on the x axis is the measured efficiency if a cut were applied offline at that cut value.




\begin{table}
    \centering
\begin{tabular}{|c|c|c|}

\hline
Offline \HT bin & Signal Trigger & Reference Trigger\\
\hline
\unit{275}{\GeV} $<$ \HT $<$ \unit{325}{\GeV} & \verb|HLT_HT250_AlphaT0p53_v2| & \verb|HLT_Mu15_HT200_v2|\\
\hline
\unit{275}{\GeV} $<$ \HT $<$ \unit{325}{\GeV} & \verb|HLT_HT250_AlphaT0p53_v3| & \verb|HLT_Mu15_HT200_v3|\\
\hline
\unit{275}{\GeV} $<$ \HT $<$ \unit{325}{\GeV} & \verb|HLT_HT250_AlphaT0p53_v4| & \verb|HLT_Mu15_HT200_v4|\\
\hline
\unit{275}{\GeV} $<$ \HT $<$ \unit{325}{\GeV} & \verb|HLT_HT250_AlphaT0p53_v5| & \verb|HLT_Mu30_HT200_v1|\\
\hline
\unit{275}{\GeV} $<$ \HT $<$ \unit{325}{\GeV} & \verb|HLT_HT250_AlphaT0p55_v1| & \verb|HLT_Mu5_HT200_v4|\\
\hline
\unit{275}{\GeV} $<$ \HT $<$ \unit{325}{\GeV} & \verb|HLT_HT250_AlphaT0p55_v2| & \verb|HLT_Mu40_HT200_v4|\\
\hline
\unit{275}{\GeV} $<$ \HT $<$ \unit{325}{\GeV} & \verb|HLT_HT250_AlphaT0p58_v3| & \verb|HLT_DoubleMu8_Mass8_HT200_v4|\\
\hline
\unit{275}{\GeV} $<$ \HT $<$ \unit{325}{\GeV} & \verb|HLT_HT250_AlphaT0p58_v3| & \verb|HLT_DoubleMu8_Mass8_HT200_v5| \\
\hline
\unit{275}{\GeV} $<$ \HT $<$ \unit{325}{\GeV} & \verb|HLT_HT250_AlphaT0p60_v3| & \verb|HLT_DoubleMu8_Mass8_HT200_v4|\\
\hline
\unit{275}{\GeV} $<$ \HT $<$ \unit{325}{\GeV} & \verb|HLT_HT250_AlphaT0p60_v3| & \verb|HLT_DoubleMu8_Mass8_HT200_v5|\\
\hline
\unit{325}{\GeV} $<$ \HT $<$ \unit{375}{\GeV} & \verb|HLT_HT300_AlphaT0p52_v1| & \verb|HLT_Mu5_HT200_v4|\\
\hline
\unit{325}{\GeV} $<$ \HT $<$ \unit{375}{\GeV} & \verb|HLT_HT300_AlphaT0p52_v2| & \verb|HLT_Mu8_HT200_v4|\\
\hline
\unit{325}{\GeV} $<$ \HT $<$ \unit{375}{\GeV} & \verb|HLT_HT300_AlphaT0p52_v3| & \verb|HLT_Mu15_HT200_v2|\\
\hline
\unit{325}{\GeV} $<$ \HT $<$ \unit{375}{\GeV} & \verb|HLT_HT300_AlphaT0p53_v3| & \verb|HLT_Mu15_HT200_v3|\\
\hline
\unit{325}{\GeV} $<$ \HT $<$ \unit{375}{\GeV} & \verb|HLT_HT300_AlphaT0p53_v4| & \verb|HLT_Mu15_HT200_v4| \\
\hline
\unit{325}{\GeV} $<$ \HT $<$ \unit{375}{\GeV} & \verb|HLT_HT300_AlphaT0p53_v5| & \verb|HLT_Mu30_HT200_v1| \\
\hline
\unit{325}{\GeV} $<$ \HT $<$ \unit{375}{\GeV} & \verb|HLT_HT300_AlphaT0p53_v6| & \verb|HLT_Mu40_HT200_v3| \\
\hline
\unit{325}{\GeV} $<$ \HT $<$ \unit{375}{\GeV} & \verb|HLT_HT300_AlphaT0p53_v6| & \verb|HLT_Mu40_HT200_v4| \\
\hline
\unit{325}{\GeV} $<$ \HT $<$ \unit{375}{\GeV} & \verb|HLT_HT300_AlphaT0p54_v5| & \verb|HLT_Mu40_HT300_v4| \\
\hline
\unit{325}{\GeV} $<$ \HT $<$ \unit{375}{\GeV} & \verb|HLT_HT300_AlphaT0p54_v5| & \verb|HLT_Mu40_HT300_v5| \\
\hline
\unit{325}{\GeV} $<$ \HT $<$ \unit{375}{\GeV} & \verb|HLT_HT300_AlphaT0p55_v3| & \verb|HLT_DoubleMu8_Mass8_HT200_v4| \\
\hline
\unit{325}{\GeV} $<$ \HT $<$ \unit{375}{\GeV} & \verb|HLT_HT300_AlphaT0p55_v3| & \verb|HLT_DoubleMu8_Mass8_HT200_v5| \\
\hline

\end{tabular}
\caption{List of \alt triggers used in the lowest two offline \HT bins and the triggers used to collect the reference sample.}
\label{tab:triggers1}

\end{table}

\begin{table}
    \centering

\begin{tabular}{|c|c|c|}
    \hline
Offline \HT bin &Signal Trigger & Reference Trigger\\
\hline
\unit{375}{\GeV} $<$ \HT $<$ \unit{475}{\GeV} & \verb|HLT_HT350_AlphaT0p51_v1| & \verb|HLT_Mu5_HT200_v4| \\
\hline
\unit{375}{\GeV} $<$ \HT $<$ \unit{475}{\GeV} & \verb|HLT_HT350_AlphaT0p51_v2| & \verb|HLT_Mu8_HT200_v4| \\
\hline
\unit{375}{\GeV} $<$ \HT $<$ \unit{475}{\GeV} & \verb|HLT_HT350_AlphaT0p51_v3| & \verb|HLT_Mu15_HT200_v2| \\
\hline
\unit{375}{\GeV} $<$ \HT $<$ \unit{475}{\GeV} & \verb|HLT_HT350_AlphaT0p51_v4| & \verb|HLT_Mu15_HT200_v3| \\
\hline
\unit{375}{\GeV} $<$ \HT $<$ \unit{475}{\GeV} & \verb|HLT_HT350_AlphaT0p51_v5| & \verb|HLT_Mu15_HT200_v4| \\
\hline
\unit{375}{\GeV} $<$ \HT $<$ \unit{475}{\GeV} & \verb|HLT_HT350_AlphaT0p52_v1| & \verb|HLT_Mu30_HT200_v1| \\
\hline
\unit{375}{\GeV} $<$ \HT $<$ \unit{475}{\GeV} & \verb|HLT_HT350_AlphaT0p52_v2| & \verb|HLT_Mu40_HT200_v3| \\
\hline
\unit{375}{\GeV} $<$ \HT $<$ \unit{475}{\GeV} & \verb|HLT_HT350_AlphaT0p52_v2| & \verb|HLT_Mu40_HT200_v4| \\
\hline
\unit{375}{\GeV} $<$ \HT $<$ \unit{475}{\GeV} & \verb|HLT_HT350_AlphaT0p53_v10| & \verb|HLT_Mu40_HT300_v4| \\
\hline
\unit{375}{\GeV} $<$ \HT $<$ \unit{475}{\GeV} & \verb|HLT_HT350_AlphaT0p53_v10| & \verb|HLT_Mu40_HT300_v5| \\
\hline
\unit{475}{\GeV} $<$ \HT $<$ \unit{7}{\TeV} & \verb|HLT_HT400_AlphaT0p51_v1| & \verb|HLT_Mu5_HT200_v4| \\
\hline
\unit{475}{\GeV} $<$ \HT $<$ \unit{7}{\TeV} & \verb|HLT_HT400_AlphaT0p51_v2| & \verb|HLT_Mu8_HT200_v4| \\
\hline
\unit{475}{\GeV} $<$ \HT $<$ \unit{7}{\TeV} & \verb|HLT_HT400_AlphaT0p51_v3| & \verb|HLT_Mu15_HT200_v2| \\
\hline
\unit{475}{\GeV} $<$ \HT $<$ \unit{7}{\TeV} & \verb|HLT_HT400_AlphaT0p51_v4| & \verb|HLT_Mu15_HT200_v3| \\
\hline
\unit{475}{\GeV} $<$ \HT $<$ \unit{7}{\TeV} & \verb|HLT_HT400_AlphaT0p51_v5| & \verb|HLT_Mu15_HT200_v4| \\
\hline
\unit{475}{\GeV} $<$ \HT $<$ \unit{7}{\TeV} & \verb|HLT_HT400_AlphaT0p51_v6| & \verb|HLT_Mu30_HT200_v1| \\
\hline
\unit{475}{\GeV} $<$ \HT $<$ \unit{7}{\TeV} & \verb|HLT_HT400_AlphaT0p51_v7| & \verb|HLT_Mu40_HT200_v3| \\
\hline
\unit{475}{\GeV} $<$ \HT $<$ \unit{7}{\TeV} & \verb|HLT_HT400_AlphaT0p51_v7| & \verb|HLT_Mu40_HT200_v4| \\
\hline
\unit{475}{\GeV} $<$ \HT $<$ \unit{7}{\TeV} & \verb|HLT_HT400_AlphaT0p51_v10| & \verb|HLT_Mu40_HT300_v4| \\
\hline
\unit{475}{\GeV} $<$ \HT $<$ \unit{7}{\TeV} & \verb|HLT_HT400_AlphaT0p51_v10| & \verb|HLT_Mu40_HT300_v5| \\
\hline
\unit{475}{\GeV} $<$ \HT $<$ \unit{7}{\TeV} & \verb|HLT_HT400_AlphaT0p52_v5| & \verb|HLT_Mu40_HT300_v4| \\
\hline
\unit{475}{\GeV} $<$ \HT $<$ \unit{7}{\TeV} & \verb|HLT_HT400_AlphaT0p52_v5| & \verb|HLT_Mu40_HT300_v5| \\
\hline

\end{tabular}
\caption{List of \alt triggers used in the \HT $>$ \unit{375}{\GeV} bins and the triggers used to collect the reference sample.}
\label{tab:triggers2}

\end{table}

As an example, the efficiency of each trigger used in the \unit{275}{\GeV} $<$ \HT $<$ \unit{325}{\GeV} bin is measured and then combined to give the total efficiency.

\begin{figure}[htbp]
  % \centering
   \subfigure[cap]{ \includegraphics[width=0.45\textwidth]{figures/Analysis/HT275_HLT_HT250_AlphaT0p55_v1_HLT_Mu5_HT200_v4.pdf}
  \label{fig:figures_Analysis_HT275_HLT_HT250_AlphaT0p55_v1_HLT_Mu5_HT200_v4}
}

\subfigure[cap]{   \includegraphics[width=0.45\textwidth]{figures/Analysis/HT275_HLT_HT250_AlphaT0p53_v2_HLT_Mu15_HT200_v2.pdf}
  \label{fig:figures_Analysis_HT275_HLT_HT250_AlphaT0p53_v2_HLT_Mu15_HT200_v2}
}
\\
\subfigure[cap]{\includegraphics[width=0.45\textwidth]{figures/Analysis/HT275_HLT_HT250_AlphaT0p53_v3_HLT_Mu15_HT200_v3.pdf}
  \label{fig:figures_Analysis_HT275_HLT_HT250_AlphaT0p53_v3_HLT_Mu15_HT200_v3}
}

\subfigure[cap]{ \includegraphics[width=0.45\textwidth]{figures/Analysis/HT275_HLT_HT250_AlphaT0p53_v4_HLT_Mu15_HT200_v4.pdf}
  \label{fig:figures_Analysis_HT275_HLT_HT250_AlphaT0p53_v4_HLT_Mu15_HT200_v4}
}
\\
\subfigure[cap]{ \includegraphics[width=0.45\textwidth]{figures/Analysis/HT275_HLT_HT250_AlphaT0p53_v5_HLT_Mu30_HT200_v1.pdf}
  \label{fig:figures_Analysis_HT275_HLT_HT250_AlphaT0p53_v5_HLT_Mu30_HT200_v1}
}

\subfigure[cap]{ \includegraphics[width=0.45\textwidth]{figures/Analysis/DiMu_HT275_HLT_HT250_AlphaT0p58_v3_HLT_DoubleMu8_Mass8_HT200_v4.pdf}
\label{fig:figures_Analysis_DiMu_HT275_HLT_HT250_AlphaT0p58_v3_HLT_DoubleMu8_Mass8_HT200_v4}
}
\\
\subfigure[cap]{ \includegraphics[width=0.45\textwidth]{figures/Analysis/DiMu_HT275_HLT_HT250_AlphaT0p58_v3_HLT_DoubleMu8_Mass8_HT200_v5.pdf} \label{fig:figures_Analysis_DiMu_HT275_HLT_HT250_AlphaT0p58_v3_HLT_DoubleMu8_Mass8_HT200_v5}
}

\subfigure[cap]{ \includegraphics[width=0.45\textwidth]{figures/Analysis/DiMu_HT275_HLT_HT250_AlphaT0p60_v3_HLT_DoubleMu8_Mass8_HT200_v4.pdf} \label{fig:figures_Analysis_DiMu_HT275_HLT_HT250_AlphaT0p60_v3_HLT_DoubleMu8_Mass8_HT200_v4}
}
% \\
% \subfigure[cap]{ \includegraphics[width=0.45\textwidth]{figures/Analysis/HT275_HLT_HT250_AlphaT0p55_v2_HLT_Mu40_HT200_v4.pdf}
%   \label{fig:figures_Analysis_HT275_HLT_HT250_AlphaT0p55_v2_HLT_Mu40_HT200_v4}
% }
\caption{somewords}
\label{fig:HT275BinTriggers}
\end{figure}


% subsection trigger_efficiency_measurement (end)








% section high_level_triggers_for_the_alt_analysis (end)


\section{Electro-Weak background prediction} % (fold)
\label{sec:electro_weak_background_prediction}

% section electro_weak_background_prediction (end)

% chapter the_t_analysis (end)