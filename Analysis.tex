\chapter{The $\alpha_{T}$ analysis} % (fold)
\label{cha:the_t_analysis}
In this chapter we discuss the main analysis performed as the subject of this 
thesis. For the theoretical motivations of this search please see 
Chapter~\ref{cha:theory}. The analysis is based on the full 2011 data set which is made up of 5\fbinv of 7 \TeV data. However 5\fbinv of the 2012 8 \TeV is looked at to measure the performance of the upgraded \alt HLT paths.
\section{The Problem} % (fold)
\label{sec:the_problem}
If Supersymmetry or some other beyond the standard model theory is to provide a 
yet undiscovered dark matter candidate, it is predicted that this candidate 
will interact via the weak nuclear force only. This gives a decay topology 
involving missing energy in the form of the dark matter particle escaping the 
detector. Due to the nature of interactions at the L.H.C, these particles would 
be produced at the end of a decay chain of heavy particles that interact 
strongly, giving a final topology involving hadronic objects which are 
classified as jets for the purpose of analysis and missing energy.
There are several standard model processes that mimic this final state.

By far the largest of these backgrounds comes from QCD multi jet events where 
fake missing energy is introduced either from failures in reconstruction, or 
stochastic fluctuations in the calorimeter systems.
\textbf{FIXME: expand on this - E /sqrt E has non gaussian tails. Figures of 
jets falling below threshold, missed jets etc. probably from some jet-met 
paper.}
However due to the theoretical errors on the QCD production cross section 
predicting the number QCD background events from Montecarlo simulation is not 
possible.
A secondary QCD background also exists, where due to the requirement of a jet 
\ET threshold, multiple jets fall under threshold by 1-5 \GeV, this causes a 
balanced event to look unbalanced as the jets under threshold are no longer 
considered. The solution is then to devise a kinematic cut that removes these 
events from the signal selection.

The second major background comes from standard model electro-weak decays and 
is irreducible. The electro-weak decays that form the back ground are 
\HepProcess{\PW\to\Ptau\Pnu} + Jets, where the $\tau$ is reconstructed as a 
jet, or the lepton fails the identification required for the dedicated lepton 
vetoes, \HepProcess{\PZ\to\Pnu\APnu} + Jets is completely irreducible. These 
are generally di-jet topologies. At higher jet multiplicities top quark 
production followed by semi-leptonic top decay accounts of the largest 
background. These backgrounds are predicted using a well understood control 
sample this is fully explained in 
Section~\ref{sec:electro_weak_background_prediction}.


The final background source is that introduced by detector failure or 
electronic noise induced by the movement of the L.H.C proton beam.
Approximately 1$\%$ of the ECAL read out is not available in offline event 
reconstruction, this provides a source of fake missing energy.
% section the_problem (end)

\section{The \alt variable.} % (fold)
\label{sec:the_alpha___t_variable_}
\alt is inspired by Ref~\cite{Randall:2008dk} and was expanded to transverse 
multi jet topologies by members of the CMS collaboration in 
Refs~\cite{cms-pas-sus-08005,cms-pas-sus-09001}. The purpose is to provide a 
variable that can be cut on to eliminate QCD from the final selection. To do 
this the inherent balance of the QCD system is exploited.

For di-jet systems \alt is defined as:

\begin{equation}
  \alt = \frac{\ET^{j_{2}}}{M_{T}}
\end{equation}
where \ET$^{j_{2}}$ is the transverse energy of least energetic of the two jets 
and M$_{T}$ is defined as:

\begin{equation}
  M_{T} = \sqrt{\left(\sum^{2}_{i=1}\ET^{j_{i}}\right)^2 - \left(\sum^{2}_{i=1}p_{x}^{j_{i}}\right)^{2} - \left(\sum^{2}_{i=1}p_{y}^{j_{i}}\right)^{2}}
\end{equation}

For a perfectly measured di-jet system with \ET$^{j_{1}} = $\ET$^{j_{2}}$, 
where the jets are opposite in $\phi$ \alt = 0.5, for events with back to back 
jets where one is miss-measured \alt $ < 0.5$.
However the majority of signals predict many jets in the final state.
\alt can be generalised to work with n-jets in the flowing way.
The variables \HT, \HTm and $\Delta$\HT are constructed:
\begin{equation}
  \HT = \sum_{i = 0}^{n~jets}\ET^{jet_{i}} \\
  \HTm = \left|\sum_{i = 0}^{n~jets}\bar{p}_{T}^{jet_{i}}\right|
\end{equation}
for jets above some predefined threshold \ET which is common for all jet based 
quantities. The multi jet system is reduced to a pseudo di-jet system by 
forming two large jets. The individual jet \ET's are summed, with the final 
configuration being chosen to have the minimum difference in energy 
($\Delta$\HT) between the pseudo jets. This simple clustering criteria provides 
the best separation between miss-measured events and those with real \MET.

\alt is then defined as:
\begin{equation}
  \alt = \frac{\HT - \Delta\HT}{2\sqrt{\HT^{2}-\HTm^{2}}}
\end{equation}


Figure~\ref{fig:figures_Analysis_AlphaT_all_375_upwards} shows the \alt 
distribution for both data and simulated background samples. The QCD multi jet 
background is negligible above an \alt value of 0.55, where as the standard 
model processes which involve real \MET exist at all possible values of \alt.
Values of \alt in the range $0.5 < \alt < 0.55$ arise in multi jet QCD due to 
jets falling below threshold or large stochastic fluctuations.
It is to be noted that the discrepancy between data and simulation for \alt 
$\leq 0.55$ is due to no trigger emulation being applied to the simulated 
background samples.
\begin{figure}[ht!]
  \centering  \includegraphics[width=0.8\textwidth]{figures/Analysis/AlphaT_all_375_upwards.pdf}
  \caption{\alt distribution for background and data. Trigger emulation is not 
  applied in the simulated background which leads to the discrepancy in the 
  region \alt $\leq 0.55$. The QCD multi-jet background is reduced to less than 
  one event.}
  \label{fig:figures_Analysis_AlphaT_all_375_upwards}
\end{figure}

% section the_alpha___t_variable_ (end)

\section{Event selection} % (fold)
\label{sec:event_selection}

% section event_selection (end)

\section{High Level triggers for the \alt analysis} % (fold)
\label{sec:high_level_triggers_for_the_alt_analysis}
The CMS trigger system has been discussed in detail in 
Section~\ref{sec:the_high_level_trigger_system} and 
Chapter~\ref{cha:level_one_trigger}, however details of analysis specific 
trigger paths were not discussed. During 2011 the first \alt specific trigger 
was designed and deployed online. The trigger was then upgraded for the higher 
luminosity and energy conditions of the 2012 data taking period.

The trigger takes advantage of cutting on two variables, \HT and \alt at low 
\HT a high \alt value cuts the trigger rate, where as at high \HT where the
trigger rate is lower the \alt requirement can be loosened.

Due to the scaling of jet thresholds in the lowest offline \HT bins as detailed 
in Section~\ref{sec:event_selection} using a fixed jet threshold would cause inefficiency in the lowest offline \HT bins. To over come this the trigger level \alt calculation is performed iteratively for all jets above a predefined threshold. This raises the total number of accepted events whilst adding the benefit of being efficient for any offline jet threshold above the minimum trigger jet threshold. The algorithm is show in Figure~\ref{fig:figures_Analysis_flowChart}.

\begin{figure}[ht!]
  \centering
    \includegraphics[width=0.75\textwidth]{figures/Analysis/flowChart.pdf}
  \caption{Flow chart representing the steps taken to make a trigger decision using the \alt trigger algorithm.}
  \label{fig:figures_Analysis_flowChart}
\end{figure}



% section high_level_triggers_for_the_alt_analysis (end)


\section{Electro-Weak background prediction} % (fold)
\label{sec:electro_weak_background_prediction}

% section electro_weak_background_prediction (end)

% chapter the_t_analysis (end)