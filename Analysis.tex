\chapter{The $\alpha_{T}$ analysis} % (fold)
\label{cha:the_t_analysis}
In this chapter we discuss the main analysis performed as the subject of this 
thesis. For the theoretical motivations of this search please see 
Chapter~\ref{cha:theory}
\section{The Problem} % (fold)
\label{sec:the_problem}
If Supersymmetry or some other beyond the standard model theory is to provide a 
yet undiscovered dark matter candidate, it is predicted that this candidate 
will interact via the weak nuclear force only. This gives a decay topology 
involving missing energy in the form of the dark matter particle escaping the 
detector. Due to the nature of interactions at the L.H.C, these particles would 
be produced at the end of a decay chain of heavy particles that interact 
strongly, giving a final topology involving hadronic objects which are 
classified as jets for the purpose of analysis and missing energy.
There are several standard model processes that mimic this final state.

By far the largest of these backgrounds comes from QCD multi jet events where 
fake missing energy is introduced either from failures in reconstruction, or 
stochastic fluctuations in the calorimeter systems.
\textbf{FIXME: expand on this - E /sqrt E has non gaussian tails. Figures of 
jets falling below threshold, missed jets etc. probably from some jet-met 
paper.}
However due to the theoretical errors on the QCD production cross section 
predicting the number QCD background events from Montecarlo simulation is not 
possible.
A secondary QCD background also exists, where due to the requirement of a jet 
\ET threshold, multiple jets fall under threshold by 1-5 \GeV, this causes a 
balanced event to look unbalanced as the jets under threshold are no longer 
considered.
The solution is then to devise a kinematic cut that removes these events from 
the signal selection.

The second major background comes from standard model electro-weak decays and 
is irreducible. The electro-weak decays that form the back ground are 
\HepProcess{\PW\to\Ptau\Pnu} + Jets, where the $\tau$ is reconstructed as a 
jet, or the lepton fails the identification required for the dedicated lepton 
vetoes, \HepProcess{\PZ\to\Pnu\APnu} + Jets is completely irreducible. These 
are generally di-jet topologies. At higher jet multiplicities top quark 
production followed by semi-leptonic top decay accounts of the largest 
background.


The final background source is that introduced by detector failure or 
electronic noise induced by the movement of the L.H.C proton beam.
Approximately 1$\%$ of the ECAL read out is not available in offline event 
reconstruction, this provides a source of fake missing energy.
% section the_problem (end)

\section{The \alt variable.} % (fold)
\label{sec:the_alpha___t_variable_}
\alt is inspired by Ref~\cite{Randall:2008dk} and was expanded to transverse 
multi jet topologies by members of the CMS collaboration in 
Refs~\cite{cms-pas-sus-08005,cms-pas-sus-09001}. The purpose is to provide a 
variable that can be cut on to eliminate QCD from the final selection. To do 
this the inherent balance of the QCD system is exploited.

For dijets \alt is defined as:

\begin{equation}
  \alt = \frac{\ET^{j_{2}}}{M_{T}}
\end{equation}
where \ET$^{j_{2}}$ is the transverse energy of least energetic of the two jets 
and M$_{T}$ is defined as:

\begin{equation}
  M_{T} = \sqrt{\left(\sum^{2}_{i=1}\ET^{j_{i}}\right)^2 - \left(\sum^{2}_{i=1}p_{x}^{j_{i}}\right)^{2} - \left(\sum^{2}_{i=1}p_{y}^{j_{i}}\right)^{2}}
\end{equation}





% section the_alpha___t_variable_ (end)



% chapter the_t_analysis (end)